\chapter{Conclusion}
\label{chap:conclusion}

Consectetur ullamco dolor pariatur ad id minim sunt do occaecat. Anim commodo consectetur proident pariatur dolor. Dolor laborum nisi id ipsum eiusmod ipsum exercitation consequat ullamco pariatur ex ut ullamco id.

\todo{Write a conclusion}


%This paper presents the results of a \acrlong{slr} of existing \acrlong{sc} vulnerability analysis and detection methods. The %motivation for this research was to provide a state-of-the-art overview of the current situation of the \acrshort{sc} vulnerability %detection. A total of 40 primary studies were selected based on predefined inclusion and exclusion criteria. A systematic analysis %and synthesis of the data were extracted from the papers, and comprehensive reviews were performed. Further, to the greatest %extent, this paper also identifies the current available cross-chain tools and methods. The cross-chain applicability for these %assets is investigated and analyzed.
%
%The findings from this study show that there are a number of methods and implemented tools readily available for vulnerability %analysis and detection. Several of these tools show great results. The most prevalent methods are static analysis tools, where %symbolic execution is among the most popular. Other methods such as syntax analysis, abstract interpretation, data flow analysis, %fuzzy testing, and machine learning are also readily used. In this paper, some potential cross-chain tools are highlighted and %discussed. Although they pose several limitations, they show significant potential for further development. Especially interesting %is the potential machine learning-based cross-chain detection methods.
%
%From this study, one can see that there is a significant lack of research on vulnerability detection on other blockchain platforms %than Ethereum. The hope is that the results from this study provide a starting point for future research on cross-chain analysis.
