\chapter{Related work}
\label{chap:related-work}

This chapter presents related research in the field of source code synthesis. This includes works related to dataset construction, model implementations, model inference guiding, and different approaches to code synthesis. 

First, various methods for source code synthesis are presented. Then follows a section on various model implementations, and   on the generation of datasets, and a section on the generation of models.



Make a literature review based on code synthesis / code auto completion. This would include both the main topic "code synthesis", as well as code comments for optimal code completion.

Code completion vs code synthesis vs automatic code generation vs intellisense.

Python: docstring  - structured comment first line under function  definition
C++: Doxygen
Java: javadoc
JS: JSDoc

TS=("code" AND ("completion" OR "generation" OR "synthesis" ))

TS=("code" AND ("auto-completion" OR "generation" OR "synthesis" ) AND comment)

spsp



\section{Language models}
\label{sec:language-models}
The problem of generating code is fundamentally a language modeling problem. Language modeling is the task of predicting the next word in a text given the previous words. This section begins with presenting some of the earlier techniques, followed by surveying more recent and state-of-the-art language models.


The first few language models came in the form of n-grams, a term first referenced by \textcite{shannon1948ngram}. An n-gram is a contiguous sequence of n items from a given sample of text. Most early approaches employed n-grams with smoothing to handle unseen n-grams \textcite{kneser1995improved}.



\subsection{Non-context models}
\subsection{Context aware models}



\subsection{Word embeddings}



Tomas Mikolov's Word2vec (google team),
Stanford University's GloVe
GN-GloVe (genderneutral) ->point tto thte bias problem off datasets -> link to insecure code on  github



\subsection{Neural language models}
N-grams

CoVe (Contextualized Word Vectors) needs "fixed" prettrained dataset :: Learned in Translation: Contextualized Word Vectors

ELMo biLM

Universal Language Model Fine-tuning (ULMFiT) :: introduced the concept of fine-tuning the language model.

OpenAI’s Gener- ative Pre-training Transformer (GPT)

GPT-2, improved version of GPT. Works without fine-tuning. (concerns of it being used to generate unintended or malicious content - delayed release)

Bidirectional Encoder Representations from Transformers (BERT) - > bidirectttional, in comparison to GPT

BERT is a bimodal Transformer with 12 layers, 768 dimentional hidden states, and 12 attention heads.

-----
GPT-3 (Codex)

GPT-J (Opensoursed)


\section{Code synthesis}
This section presents some of the various approaches to code synthesis. 

Code synthesis is ... 

One of the earlier classical works used a probabilistic \acrfull{pcfg} \cite{allamanis2015bimodal}.

\textcite{hindle2012natural} investigated whether code could be modeled by statistical language models. In particular, the authors used an n-gram model. They argue that "programs that real people actually write are mostly simple and rather repetitive, and thus they have usefully predictable statistical properties". They found that code is more predictable than natural languages. DeepCoder by \textcite{balog2017deepcoder} focused on solving programming competition-style problems. They trained a neural network for predicting properties of source code, which could be used for guiding program search.

\subsection{Code semantics}
Programs can also be synthesized by leveraging the semantics of the code. \textcite{alon2018code2vec} purposes a tool named code2vec. It is a neural network model for representing snippets of code as continuously distributed vectors, or "code embeddings". The authors leverage the semantic structure of code by passing serialized \acrfullpl{ast} into a neural network. Code2seq \cite{alon2018code2seq} builds on the works of \textcite{alon2018code2vec} which focuses on natural language sequence generation from code snippets. The authors use an encoder-decoder LSTM model and rely on \acrshortpl{ast} for code snippets. The model is trained on three Java corpuses small, medium, and large, achieving a \gls{f1} score of 50.64, 53.23, and 59.19, respectively. However, the model is limited to only considering the immediately surrounding context. Pythia by \textcite{svyatkovskiy2019pyhia} is able to generate ranked lists of method and API recommendations to be used by software developers at edit time. The code completion system is based on \acrshortpl{ast} and uses Word2vec for producing code embeddings of Python code. These code embeddings are then used to train a \gls{lstm} model. The model is evaluated on a dataset of 15.8 million method calls extracted from real-world source code, achieving an accuracy of 92\%.

\subsection{Transformers for code synthesis}
\label{sec:transformers-for-code-synthesis}
Inspired by the success of large natural language models such as ELMo, GPT, BERT, XLNet, and RoBERa (CITATION), large-scale Transformer models have been applied in the domains of code synthesis. \textcite{feng2020codebert} proposes a new approach to code synthesis by training the BERT transformer model on Python \gls{docstring} paired with functions. The resulting 125M parameter transformer model, named CodeBERT \cite{feng2020codebert}, achieves strong results on code-search and code-to-text generation. The authors also observe that models that leverage code semantics (\acrshortpl{ast}) can produce slightly better results. CodeGPT by \cite{lu2021codexglue} provides text-to-code generation by training several monolingual GPT-2 transformer models on Python functions and Java methods. For each programming language, one model was pre-trained from scratch, while another was fine-tuned on the code corpus, using the standard GPT-2 vocabulary and natural language understanding ability. The Java models was evaluated on the CONCODE \todo{citation} dataset, achieving a state-of-the-art performance \acrshort{bleu} score \cite{papineni2002bleu} of 28.69 for the model trained from scratch, and 32.79 for the fine-tuned version. Another model version based on GPT-2 is GPT-C by \textcite{svyatkovskiy2020intellicode}. The 366M parameter-sized model is trained on a code corpus consisting of 1.2 billion lines of source code in Python, C\#, JavaScript and TypeScript programming languages. The Python-only model reportedly achieves a \gls{rouge-l} precision of 0.80 and recall of 0.86. PyMT5 \textcite{colin2020pymt5} is based on the T5 model. The model can predict whole methods from natural language documentation strings (docstrings) and summarize code into docstrings of any common style. For method generation, PyMT5 achieves a \gls{bleu} score of 8.59 and a \gls{rouge-l} F-score of 24.8 on the CodeSearchNet \todo{cite} test set.

The model complexity of transformers has recently sky-rocketed, with model sizes growing to several tens of billions of parameters. GPT-J, a 6 billion parameter model trained on The Pile, an 825GB dataset. The Pile features many disparate domains, including books, GitHub repositories, webpages, chat logs, and medical, physics, math, computer science, and philosophy papers, making it one of the most extensive and diverse datasets available. The pre-trained version of GPT-J is also publicly available. Codex by \textcite{chen2021codex} is a 12 billion parameter model based on GPT. It was trained on 54 million GitHub repositories, and a production version of Codex powers GitHub Copilot \todo{cite}. The model solves 28.8\% of the problems in the HumanEval dataset \todo{cite}, while GPT-3 solves 0\% and GPT-J solves 11.4\%. Google DeepMind's AlphaCode is 41.4 billion parameters and is the first AI to reach a competitive level in programming competitions. AlphaCode was tested against challenges curated by Codeforces \todo{cite}, a competitive coding platform. It achieved an averaged ranking of 54.3\% across 10 contests. The authors found that repeated sampling on the same problem significantly increased the probability of a correct solution.



\newcommand*\emptycirc[1][1ex]{\tikz\draw (0,0) circle (#1);} 
\newcommand*\halfcirc[1][1ex]{%
  \begin{tikzpicture}
  \draw[fill] (0,0)-- (90:#1) arc (90:270:#1) -- cycle ;
  \draw (0,0) circle (#1);
  \end{tikzpicture}}
\newcommand*\fullcirc[1][1ex]{\tikz\fill (0,0) circle (#1);} 


%\begin{sidewaystable}
\begin{ThreePartTable}
    \todo{Add input, ouputt, meetric ,  language and model info to table}
    \newcolumntype{Y}{>{\centering\arraybackslash}X}
    \newcolumntype{R}{>{\raggedright\arraybackslash}X}
    \def\arraystretch{1.5}
    \setlength\tabcolsep{6pt} % <--- important, (default 6pt)
    \setlength{\LTleft}{-20cm plus -1fill}
    \setlength{\LTright}{\LTleft}
    \footnotesize
    \begin{center}
    \begin{TableNotes}
        \item[a] \label{tn:ml-name} Name of the tool or method. If no name exists, a short description or "--" is used.
    \end{TableNotes}
    \keepXColumns
    \begin{tabularx}{1\textwidth}{clRRRRR}
            \caption{Existing language models.}\label{tab:ml-tools}\\
            \toprule
            \textbf{Refs.} & \textbf{Year} & \textbf{Model\tnotex{tn:ml-name}} & \textbf{Metrics} & \textbf{Languages} &  \textbf{Input} & \textbf{Output}\\
            \hline
            \endfirsthead
            \caption{(\textit{Continued}) Existing static smart contract vulnerability detection tools.}\\
            \toprule
            \textbf{Refs.} & \textbf{Year} & \textbf{Model\tnotex{tn:ml-name}} & \textbf{Metrics} & \textbf{Languages} &  \textbf{Input} & \textbf{Output}\\
            \hline
        \endhead
            \midrule
            \multicolumn{7}{r}{\small(\textit{Continued on next page})}\\
        \endfoot
            \insertTableNotes\\
        \endlastfoot
        
        \cite{} & \citeyear{} & GPT & \acrshort{bleu} & Python  & Docstring & Code\\
        
        \bottomrule
    \end{tabularx}
    \end{center}

\end{ThreePartTable}
%\end{sidewaystable}

\section{Bias in language models}
Include security as a bias. Discuss for example gender bias (ex. male vs female jobs) due to datasets.... Same goes with vulnerablilities.. Include stats from gitthub securitty??



\section{Smart contract code generation}
Some old-school models? I know various methods for programming smart contracs exists.. Maybe some graphical solutions?

\section{Datasets}
CodeXGLUE - Microsoft

CodeNet - IBM




%\begin{sidewaystable}
\begin{ThreePartTable}
    \todo{Add input, ouputt, meetric ,  language and model info to table}
    \newcolumntype{Y}{>{\centering\arraybackslash}X}
    \newcolumntype{R}{>{\raggedright\arraybackslash}X}
    \def\arraystretch{1.5}
    \setlength\tabcolsep{6pt} % <--- important, (default 6pt)
    \setlength{\LTleft}{-20cm plus -1fill}
    \setlength{\LTright}{\LTleft}
    \footnotesize
    \begin{center}
    \begin{TableNotes}
        \item[a] \label{tn:ml-name} Name of the tool or method. If no name exists, a short description or "--" is used.
    \end{TableNotes}
    \keepXColumns
    \begin{tabularx}{1\textwidth}{clRRRRR}
            \caption{Existing code datasets.}\label{tab:ml-tools}\\
            \toprule
            \textbf{Refs.} & \textbf{Year} & \textbf{Model\tnotex{tn:ml-name}} & \textbf{Metrics} & \textbf{Languages} &  \textbf{Input} & \textbf{Output}\\
            \hline
            \endfirsthead
            \caption{(\textit{Continued}) Existing static smart contract vulnerability detection tools.}\\
            \toprule
            \textbf{Refs.} & \textbf{Year} & \textbf{Model\tnotex{tn:ml-name}} & \textbf{Metrics} & \textbf{Languages} &  \textbf{Input} & \textbf{Output}\\
            \hline
        \endhead
            \midrule
            \multicolumn{7}{r}{\small(\textit{Continued on next page})}\\
        \endfoot
            \insertTableNotes\\
        \endlastfoot
        
        \cite{} & \citeyear{} & GPT & \acrshort{bleu} & Python & Docstring & Code\\
        
        \bottomrule
    \end{tabularx}
    \end{center}

\end{ThreePartTable}
%\end{sidewaystable}


% vvvvvvvv VIKTIG vvvvvvvvvv

autoregressive generation tasks
see https://arxiv.org/pdf/2005.08025.pdf for structuring thesisl....!!!!

see docstring analysis 2.3 of https://arxiv.org/pdf/2010.03150.pdf for clustering comments....
%^^^^^^^^ VIKTIG ^^^^^^^^


\section{Code comment analysis}
\label{sec:code-comment-analysis}

Several papers on generating code comments from source code are available. The following is a list of the most popular papers.

However, to the best of my knowledge, there is no paper investigating how to best write comments for auto-generating code. There are however, 


\section{Vulnerability detection tools}
\label{sec:vulnerability-detection-tools}

\todo{Include vulnerability detection tools table from literature reeview}


SmartBugs

Most of the available tools only work on special versions of Solidity. ... Why use SoliDetector?


\section{Vulnerability detection}
\label{sec:vulnerability-detection}
\todo{Remove section}
Many tools and methods for vulnerability detection have been developed over recent years. This includes both static and dynamic vulnerability techniques, as well as tools based on \acrfull{ml}. These tools can be categorized in terms of their primary function. This includes symbolic execution, syntax analysis, abstract interpretation, data flow analysis, fuzzy testing, and machine learning. In the following sections, the identified vulnerability detection tools are summarized, compared, and analyzed in detail.

\subsection{Symbolic execution}
\label{sec:symbolic-execution}
Symbolic execution is a method for analyzing a computer program in order to determine what inputs cause each part of a program to execute. Symbolic execution requires the program to run. During the execution of the program, symbolic values are used instead of concrete values. The program execution arrives at expressions in terms of symbols for expressions and variables, as well as constraints expressed as symbols for each possible outcome of each conditional branch of the program. Finally, the possible inputs, expressed as symbols, that trigger a branch can be determined by solving the constraints.

\subsubsection{syntax analysis}
\label{sec:syntax-analysis}
Syntax analysis is a technique for analyzing computer programs by analyzing the syntactical features of a computer program. This usually involves some kind of pattern matching where the source code is first parsed into a tree structure. This tree is then analyzed by looking for vulnerable patterns while traversing the tree.

\subsubsection{Abstract interpretation}
\label{sec:abstract-interpretation}
Abstract interpretation is a method to analyze computer programs by soundly approximating the semantics of a computer program. This results in a superset of the concrete program semantics. Normally, this is then used to automatically extract information about the possible executions of computer programs.

\subsubsection{Data flow analysis}
\label{sec:data-flow-analysis}
Data flow analysis is a method for analyzing computer programs by gathering information about the flow of data through the source code. This is done by collecting all the possible set of values calculated at different points through a computer program. This method is able to analyze large programs, compared to, for example, symbolic execution.

\subsubsection{Fuzzy testing}
\label{sec:fuzzy-testing}
Fuzzing is an automated testing technique for analyzing computer programs. The technique involves supplying invalid, unexpected, or random data inputs to a program in order to uncover bugs. The program is then monitored during execution for unexpected behavior such as crashes, errors, or failing built-in code assertions.

