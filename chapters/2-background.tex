\chapter{Background}
\label{chap:background}
This chapter introduces the necessary background information for this study. First, a brief introduction to blockchain technology is provided in \cref{sec:blockchain} and then the concept of \acrfullpl{sc} is introduced in \cref{sec:smart-contract}. Finally, in \cref{sec:smart-contract-vulnerabilities}, the most popular \acrshort{sc} vulnerabilities are described.

%\section{Transfer learning}
%\label{sec:transfer-learning}

\section{Natural language processing}
\label{sec:natural-language-processing}

\subsection{Text preprocessing}
\label{sec:textt-preprocessing}

\subsection{Vector space model}
\label{sec:vector-space-model}

\subsubsection{Word2Vec}
\label{sec:vord2vec}
Or word embedding? as secttion.
https://ai.stackexchange.com/questions/26739/what-is-the-difference-between-a-language-model-and-a-word-embedding

Word Embeddings does not consider context, Language Models does.
For e.g Word2Vec, GloVe, or fastText, there exists one fixed vector per word.


The sentence:

Kill him, don't wait for me.

and

Don't kill him, wait for me.

If one averages the word embeddings, they would  produce the same vector. However,  in reality thheir meaning  (semantic) is very  diffferent.

\subsubsection{Term frequency-inverse document frequency}
\label{sec:tf-idf}


\section{Machine Learning}
\label{sec:machine-learning}
\todo{Rewrite}
\acrfull{ml} is the study of computer algorithms that can automatically improve through the use of data. \acrshort{ml} algorithms create a model based on training data. This model is then used to predict the outcome of new unseen data without being explicitly programmed to do so.

\subsection{Machine learning models}
\label{sec:machine-learning-models}

\subsection{Linear models}
\label{sec:linear-models}

\subsubsection{Logistic regression}
\label{sec:logistic-regression}

\subsubsection{Support vector machines}
\label{sec:support-vector-machines}

\subsubsection{Artificial neural networks}
\label{sec:artificial-neural-networks}

\subsection{Deep learning}
\label{sec:deep-learning}


\section{Cluster Analysis}
\label{sec:cluster-analysis}


\section{Transformer}
\label{sec:transformer}

\section{Data parallelism}
\label{sec:data-parallelism}

\section{Performance metric}
\label{sec:performance-metric}

\subsection{Precision}
\label{sec:precision}

\subsection{Recall}
\label{sec:recall}

\subsection{F1-score}
\label{sec:f1-score}

\section{Machine translation metrics}
\label{sec:machine-translation-metrics}

\subsection{\textsc{Bleu}}
\label{sec:blue-score}
\acrfullr{bleu} by \textcite{papineni2002bleu} is a metric for automatically evaluating machine-translated text. \acrshort{bleu} scores are between 0 and 1. A value of 0 means there is no overlap with the reference translation, while a value of 1 means that the translation perfectly overlaps. A score of 0.6 or 0.7 is considered the best you can achieve. The method is based on n-gram matching, where n-grams in the reference translation are matched against n-grams in the translation. The matches are position-independent. The more matches, the higher the score.\\

\noindent For example, consider the following two translations:\\

\indent Candidate: \underline{on} \underline{the} \underline{mat} \underline{the} \underline{cat} sat.\\
\indent Reference: \underline{The} \underline{cat} is \underline{on} \underline{the} \underline{mat}.\\

\noindent The unigram precision \(\left(p_1\right) = 5/6\)\\

\noindent However, machine translations tend to generate an abundance of reasonable words, which could result in an inaccurately high precision. To combat this, \acrshort{bleu} uses something called modified precision. The modification consists of clipping the occurrence of an n-gram to the maximum number the n-gram occurs in the reference. These clipped precision scores \(\left(p_n\right)\) are then calculated for n-grams up to length \(N\), normally 1-grams through 4-grams. They are then combined by computing the geometric average precision. In addition, positive weights \(w_n\) are used, normally set to \(w_n = 1/N\).

\begin{equation}
    \label{eq:geometric-average-precision}
    \text{Geometric Average Precision $\left(N\right)$} = \exp \left( \sum_{n=1}^{N} w_n \log{p_n} \right)
\end{equation}

\noindent \acrshort{bleu} also introduces a brevity penalty for penalizing translations that are shorter than the reference.

\begin{equation}
    \label{eqn:brevity-penalty}
    \text{Brevity Penalty} = 
    \begin{cases}
        1 & \text{if } c > r\\
        e^{\left(1-r/c \right)} & \text{if } c \le r
    \end{cases}
\end{equation}

\noindent The final \acrshort{bleu} score is then computed as:

\begin{equation}
    \label{eqn:bleu}
    \textsc{Bleu} = \text{Brevity Penalty} \cdot \text{Geometric Average Precision Scores $\left(N\right)$}
\end{equation}



\subsection{\textsc{CodeBleu}}
\label{sec:code-bleu}
https://arxiv.org/pdf/2009.10297.pdf

Maybe implement this for Solidity??


\subsection{\textsc{ROUGE}}
\label{sec:rouge-score}

\section{String metric}
\label{sec:string-metric}

\subsection{Jaccard index}
\label{sec:jaccard-index}


\section{Blockchain}
\label{sec:blockchain}
A blockchain is a growing list of records that are linked together by a cryptographic hash. Each record is called a block. The blocks contain a cryptographic hash of the previous block, a timestamp, and transactional data. By time-stamping a block, this proves that the transaction data existed when the block was published in order to get into its hash. Since all blocks contains the hash of the previous block, they end up forming a chain. In order to tamper with a block in the chain, this also requires altering all subsequent blocks. Blockchains are therefore resistant to modification. The longer the chain, the more secure it is.

Typically, blockchains are managed by a peer-to-peer network, resulting in a publicly distributed ledger. The network is composed of nodes that are connected to each other. The nodes collectively adhere to a protocol in order to communicate and validate new blocks. Blockchain records are possible to alter through a \gls{fork}. However, blockchains can be considered secure by design and presents a distributed computing system with high Byzantine fault tolerance \cite{sankar2017survey}.

The blockchain technology was popularized by Bitcoin in 2008. Satoshi Nakamoto introduced the formal idea of blockchain as a peer-to-peer electronic cash system. It enabled users to conduct transactions without the need for a central authority. From Bitcoin sprang several other cryptocurrencies and blockchain platforms such as Ethereum, Litescoin, and Ripple. \cref{tab:blockchain-platforms} shows an overview of the different blockchain platforms, including the different consensus protocols, programming languages, and execution environments used. It also shows the different types of blockchains, including public, private, and hybrid. If the platform also supplies a native currency (cryptocurrency), this is also shown.


\subsection{Ethereum blockchain}
\label{sec:ethereum}


\section{Smart Contract}
\label{sec:smart-contract}
\todo{Rewrite and adapt to vulnerabilities SoliDetector can detect.}

The term "\acrlong{sc}" was introduced with the Ethereum platform in 2014. A \acrfull{sc} is a program that is executed on a blockchain, enabling non-trusting parties to create an \textit{agreement}. \acrshortpl{sc} have enabled several interesting new concepts, such as \acrfull{nft} and entirely new business models. Since Ethereum's introduction of \acrshortpl{sc}, the platform has kept its market share as the most popular \acrshort{sc} blockchain platform. Ethereum is a open, decentralized platform that allows users to create, store, and transfer digital assets. Solidity is a programming language that is used to write smart contracts in Ethereum. Solidity is compiled down to bytecode, which is then deployed and stored on the blockchain. Ethereum also introduces the concept of gas. Ethereum describes gas as follows: \textquote{It is the fuel that allows it to operate, in the same way that a car needs gasoline to run.} \cite{ethereum2021gas}. The gas is used to pay for the cost of running the smart contract. This protects against malicious actors spamming the network \cite{ethereum2021gas}. The gas is paid in Wei, which is the smallest unit of Ethereum. Due to the immutable nature of blockchain technology, once a smart contract is deployed, it cannot be changed. This can have serious security implications, as vulnerable contracts can not be updated.

\subsection{Security Vulnerabilities}
\label{sec:smart-contract-vulnerabilities}
There are many vulnerabilities in \acrfullpl{sc} that can be exploited by malicious actors. Throughout the last years, an increase in the use of the Ethereum network has led to the development of \acrshortpl{sc} that are vulnerable to attacks. Due to the nature of blockchain technology, the attack surface of \acrshortpl{sc} is somewhat different from that of traditional computing systems. The Smart Contract Weakness Classification (SWC) Registry \footnote{\url{https://swcregistry.io}} collects information about various vulnerabilities. Following is a list of the most common vulnerabilities in \acrlongpl{sc}:

\subsubsection{Integer Overflow and Underflow}
Integer overflow and underflows happen when an arithmetic operation reaches the maximum or minimum size of a certain data type. In particular, multiplying or adding two integers may result in a value that is unexpectedly small, and subtracting from a small integer may cause a wrap to be an unexpectedly large positive value. For example, an 8-bit integer addition 255 + 2 might result in 1.

\subsubsection{Transaction-Ordering Dependence}
In blockchain systems, there is no guarantee on the execution order of transactions. A miner can influence the outcome of a transaction due to its own reordering criteria. For example, a transaction that is dependent on another transaction to be executed first may not be executed. This can be exploited by malicious actors. 

\subsubsection{Broken Access Control}
Access Control issues are common in most systems, not just smart contracts. However, due to the monetary nature and openness of most \acrshortpl{sc}, properly enforcing access controls are essential. Broken access control can, for example, occur due to wrong visibility settings, giving attackers a relatively straightforward way to access contracts' private assets. However, the bypass methods are sometimes more subtle. For example, in Solidity, reckless use of \lstinline[language=Solidity]!delegatecall! in proxy libraries, or the use of the deprecated \lstinline[language=Solidity]!tx.origin! might result in broken access control. \cref{lst:broken-access-control} shows a simple Solidity contract where anyone is able to trigger the contract's self-destruct.

\begin{lstlisting}[
    caption={Access control vulnerable Solidity \acrlong{sc} code},
    label=lst:broken-access-control,
    language=Solidity]s
contract SimpleSuicide {
    function sudicideAnyone() {
        selfdestruct(msg.sender);
    }
}
\end{lstlisting}

\subsubsection{Timestamp Dependency}
If a \acrlong{sc} is dependent on the timestamp of a transaction, it is vulnerable to attacks. A miner has control over the execution environment for the executing \acrshort{sc}. If the \acrshort{sc} platform allows for \acrshortpl{sc} to use the time defined by the execution environment, this can result in a vulnerability. An example vulnerable use is a timestamp used as part of the conditions to perform a critical operation (e.g., sending ether) or as the source of entropy to generate random numbers. Hence, if the miner holds a stake in a contract, he could gain an advantage by choosing a suitable timestamp for a block he is mining. \cref{lst:timestamp-dependency} shows an example Solidity \acrshort{sc} code that contains this vulnerability. Here, the timestamp (the \lstinline[language=Solidity]!now! keyword on line 10) is used as a source of entropy to generate a random number.

\begin{lstlisting}[
    caption={Timestamp Dependency vulnerable Solidity \acrlong{sc} code},
    label=lst:timestamp-dependency,
    language=Solidity]
contract Roulette {
    uint public prevBlockTime; // One bet per block
    constructor() external payable {} // Initially fund contract
    
    // Fallback function used to make a bet
    function () external payable {
        require(msg.value == 5 ether); // Require 5 ether to play
        require(now != prevBlockTime); // Only 1 transaction per block
        prevBlockTime = now;
        if(now % 15 == 0) { // winner
            msg.sender.transfer(this.balance);
        }
    }
}
\end{lstlisting}

\subsubsection{Reentrancy}
Reentrancy is a vulnerability that occurs when a \acrshort{sc} calls external contracts. Most blockchain platforms that implement \acrshort{sc} provide a way to make external contract calls. In Ethereum, an attacker may carefully construct a \acrshort{sc} at an external address that contains malicious code in its fallback function. Then, when a contract sends funds to the address, it will invoke the malicious code. Usually, the malicious code triggers a function in the vulnerable contract, performing operations not expected by the developer. It is called "reentrancy" since the external malicious contract calls a function on the vulnerable contract and the code execution then "reenters" it. \cref{lst:reentrancy} shows a Solidity \acrshort{sc} function where a user is able to withdraw all the user's funds from a contract. If a malicious actor carefully crafts a contract that calls the withdrawal function several times before completing, the actor would successfully withdraw more funds than the current available balance. This vulnerability could be eliminated by updating the balance (line 4) before transferring the funds (line 3).

\begin{lstlisting}[
    caption={Reentrancy vulnerable Solidity \acrlong{sc} code},
    label=lst:reentrancy,
    language=Solidity]
function withdraw() external {
    uint256 amount = balances[msg.sender];
    require(msg.sender.call.value(amount)());
    balances[msg.sender] = 0;
}   
\end{lstlisting}


\section{Vulnerability detection}
\label{sec:vulnerability-detection}
\todo{Remove section}
Many tools and methods for vulnerability detection have been developed over recent years. This includes both static and dynamic vulnerability techniques, as well as tools based on \acrfull{ml}. These tools can be categorized in terms of their primary function. This includes symbolic execution, syntax analysis, abstract interpretation, data flow analysis, fuzzy testing, and machine learning. In the following sections, the identified vulnerability detection tools are summarized, compared, and analyzed in detail.

\subsection{Symbolic execution}
\label{sec:symbolic-execution}
Symbolic execution is a method for analyzing a computer program in order to determine what inputs cause each part of a program to execute. Symbolic execution requires the program to run. During the execution of the program, symbolic values are used instead of concrete values. The program execution arrives at expressions in terms of symbols for expressions and variables, as well as constraints expressed as symbols for each possible outcome of each conditional branch of the program. Finally, the possible inputs, expressed as symbols, that trigger a branch can be determined by solving the constraints.

\subsubsection{syntax analysis}
\label{sec:syntax-analysis}
Syntax analysis is a technique for analyzing computer programs by analyzing the syntactical features of a computer program. This usually involves some kind of pattern matching where the source code is first parsed into a tree structure. This tree is then analyzed by looking for vulnerable patterns while traversing the tree.

\subsubsection{Abstract interpretation}
\label{sec:abstract-interpretation}
Abstract interpretation is a method to analyze computer programs by soundly approximating the semantics of a computer program. This results in a superset of the concrete program semantics. Normally, this is then used to automatically extract information about the possible executions of computer programs.

\subsubsection{Data flow analysis}
\label{sec:data-flow-analysis}
Data flow analysis is a method for analyzing computer programs by gathering information about the flow of data through the source code. This is done by collecting all the possible set of values calculated at different points through a computer program. This method is able to analyze large programs, compared to, for example, symbolic execution.

\subsubsection{Fuzzy testing}
\label{sec:fuzzy-testing}
Fuzzing is an automated testing technique for analyzing computer programs. The technique involves supplying invalid, unexpected, or random data inputs to a program in order to uncover bugs. The program is then monitored during execution for unexpected behavior such as crashes, errors, or failing built-in code assertions.

