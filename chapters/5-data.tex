\chapter{Data}
\label{chap:datasets}
This chapter introduces the necessary background information for this study. First, a brief introduction to blockchain technology is provided in \cref{sec:blockchain} and then the concept of \acrfullpl{sc} is introduced in \cref{sec:smart-contract}. Finally, in \cref{sec:smart-contract-vulnerabilities}, the most popular \acrshort{sc} vulnerabilities are described.

\section{Smart contract downloader}
\label{sec:smart-contract-downloader}
\url{https://github.com/andstor/smart-contract-downloader}


The largest provider of verified \acrshortpl{sc} is Etherscan. This website provides a list of all verified \acrshortpl{sc} on the blockchain. More on their service...... Etherscan provides a API for downloading verified Smart Contracts. The API is available at \url{https://api.etherscan.io/api}.

In order to download the \acrshortpl{sc} from Etherscan, a tool we need to provide the \acrshortpl{sc} address. The address is the first part of the \acrshortpl{sc} code. The address is the first part of the \acrshortpl{sc} code.

The following code snippet is a Google BigQuery query. It will select all \acrshortpl{sc} addresses on the Ethereum blockchain that has at least one transaction. This query was run on the 1st of April 2022, and the result was downloaded as a CSV file, and is available at \url{https://huggingface.co/datasets/andstor/smart_contracts/blob/main/contract_addresses.csv}. The CSV file is then used to download the \acrshortpl{sc} from Etherscan.

\begin{lstlisting}[
    caption={Google BigQuery query for selecting all \acrlong{sc} addresses on Ethereum that has at least one transaction.},
    label=lst:reentrancy,
    language=SQL]
SELECT contracts.address, COUNT(1) AS tx_count
FROM `bigquery-public-data.crypto_ethereum.contracts` AS contracts
JOIN `bigquery-public-data.crypto_ethereum.transactions` AS transactions 
      ON (transactions.to_address = contracts.address)
GROUP BY contracts.address
ORDER BY tx_count DESC
}
\end{lstlisting}

\todo{Include img of the processing script output}
Saved to file for simple reestarrting, multiprocessing and parallelization.

The total number of files generated by the downloading program was 5,810,042. In order to efficiently process these, all files were combined into a tarfile. A processing script was then created for filtering out all "empty" files. These correspond to a contract address on Ethereum that has not been verified on Etherscan.io. A total of 3,592,350 files were empty, making the source code of 38,17\% of the deployed contracts on Ethereum available. Each non-empty file is then parsed and the contract data is extracted. This extraction process is rather complicated, as smart contract sources come in a wide variety of flavors and formats.

\subsection{Normalization}
The most common is a contract written the Solidity language with  a single contract "entry"  \todo{Find a better name for contract keyword}. However, a single contrract file can contain multtiple contracts, making use of properties like inheritance etc.. The source code contracts can also be split over multiple files, a formmat rreefered to as "Multi file". When compiling thtese, the source code files aree "flattened" into a single contract file before compiliattion. Anotther flavour is hte JSON format, which is a language that is used to describe the \acrshortpl{sc}. Here the sourcecode is structured in tthe in the JSON code. Smart contracts can also be vritten in the Vyper language. Vyper is .... \todo{explain vyper}.


\begin{lstlisting}[
    caption={Solidity standard JSON Input format.},
    label=lst:flattened-dataset-cmd,
    language=JSON]
{
    "sources": {/* ... */},
    "settings": {
        "optimizer": {/* ... */},
        "evmVersion": "<VERSION>"
    }
}
\end{lstlisting}

All of the above formats are processed by the processing script, normalizing the contract source code to a single "flattened" contract file. The source code, along with the contract metadata, is then saved across multiple Parquet files, each consisting of 30000 "flattened" contracts. A total of 2,217,692 smart contracts were successfully parsed and normalized.

\subsection{Duplication filtering}
\label{sec:duplication-filtering}
A large quantity of Smart Contracts contains duplicated code. Primarily, this is due to the frequent use of library code, such as Safemath and ... \todo{Reference libraries}. Etherscan requires the library code used in a contract to be embedded in the source code. Filtering is applied to produce a dataset with a mostly unique contract source code to mitigate this. This filtering is done by calculating the string distance between the source code. Due to the rather large amount of contracts (\~2 million), the comparison is only made within groups of contracts. These groups are defined by grouping on the "contract\_name" for the \textit{flattened} dataset, and by "file\_name" for the \textit{inflated} dataset. These datasets will be discssed in detail in the following sections.

The actual code filtering is done by applying a token-based similarity algorithm named Jacard Index. The algorithm is computationally efficient and can be used to filter out \acrshortpl{sc} that are not similar to the query. The Jacard Index is a measure of the similarity between two sets. The Jacard Index is defined as the ratio of the size of the intersection to the size of the union of the two sets.


\section{Datasets}
This section describes the datasets used and created in this study.

Describe the  PILE...  It consists of among others, a lot of data from GitHub. HHowever, only x\% of the data is smart contracts (Solidity). Hence there is a need for a dataset made up of smart contracts. --> existing datasets....

\subsection{Verified Smart Contracts}
\label{sec:verified-smart-contracts}
\url{https://github.com/andstor/verified-smart-contracts}
\url{https://huggingface.co/datasets/andstor/smart_contracts}

The Verified Smart Contracts dataset is a dataset consisting of verified Smart Contracts from Etherscan.io. This is real smmart contracts that are deployed to the Ethereum blockchain. A set of 100,000 to 200,000 contracts are provided, containing both Solidity and Vyper code.

\cref{tab:verified-smart-contracts-metrics} shows the metrics of the various (sub)datasets.

\begin{table}
    %\newcolumntype{Y}{>{\centering\arraybackslash}X}
    \def\arraystretch{1.5}
    \small
    \centering
    \caption{Verified Smart Contracts Metrics}
    \label{tab:verified-smart-contracts-metrics}
    \begin{tabularx}{\textwidth}{XXXX}
        \toprule
        \textbf{Component} & \textbf{Size} &  \textbf{Num rows} & \textbf{LoC*}\\
        \midrule
        Raw & 0.80 GiB & 2,217,692 & 839,665,295\\
        Flattened & 1.16 GiB & 136,969 & 97,529,473\\
        Inflated & 0.76 GiB & 186,397 & 53,843,305\\
        Parsed & 4.44 GiB & 4,434,014 & 29,965,185\\
        \bottomrule
    \end{tabularx}
\end{table}

LoC refers to the lines of source\_code. The Parsed dataset counts lines of func\_code + func\_documentation.

\subsubsection{Raw}
\label{sec:verified-smart-contracts-raw}
The raw dataset contains mostly the raw data from Etherscan, downloaded with the smart-contract-downlader tool, as described in \cref{sec:smart-contract-downloader}. All different contract formats (JSON, multi-file, etc.) are normalized to a flattened source code structure. 

\todo{Add stats on the raw dataset}

\subsubsection{Flattened}
\label{sec:verified-smart-contracts-flattened}

The flattened dataset is a filtered version  of the Raw dataset\cref{sec:verified-smart-contracts-raw}. It contains smart contracts, where every contract contains all required library code. Each "file" is marked in the source code with a comment stating the original file path: //File: path/to/file.sol. These are then filtered for uniqueness with a similarity threshold of 0.9. This means that all contracts whose code shares more than 90\% of the tokens will be discarded. The low uniqueness requirement is due to the often large amount of embedded library code. If the requirement is set to high, the actual contract code will be negligible compared to the library code. Most contracts will be discarded, and the resulting dataset would contain mostly unique library code. However, the dataset as a whole will have a large amount of duplicated libray code. From the 2,217,692 contracts, 2,080,723 duplications are found, giving a duplication percentage of 93.82\%. The resulting dataset consists of 136,969 contracts.

%Processing: 100%|██████| 74/74 [20:22<00:00, 16.51s/it, dupes=2081081/2217692 (93.84%)]

The following command prooduces the flattened dataset:

\lstinline[language=Python]!python script/filter_data.py -s parquet -o data/flattened --threshold 0.9!


\begin{lstlisting}[
    caption={Solidity standard JSON Input format.},
    label=lst:flattened-dataset-cmd,
    language=JSON]
{
  'contract_name': 'MiaKhalifaDAO',
  'contract_address': '0xb3862ca215d5ed2de22734ed001d701adf0a30b4',
  'language': 'Solidity',
  'source_code': '// File: @openzeppelin/contracts/utils/Strings.sol\r\n\r\n\r\n// OpenZeppelin Contracts v4.4.1 (utils/Strings.sol)\r\n\r\npragma solidity ^0.8.0;\r\n\r\n/**\r\n * @dev String operations.\r\n */\r\nlibrary Strings {\r\n...',
  'abi': '[{"inputs":[{"internalType":"uint256","name":"maxBatchSize_","type":"uint256"}...]',
  'compiler_version': 'v0.8.7+commit.e28d00a7',
  'optimization_used': False,
  'runs': 200,
  'constructor_arguments': '000000000000000000000000000000000000000000000000000000000000000a000...',
  'evm_version': 'Default',
  'library': '',
  'license_type': 'MIT',
  'proxy': False,
  'implementation': '',
  'swarm_source': 'ipfs://e490df69bd9ca50e1831a1ac82177e826fee459b0b085a00bd7a727c80d74089'
}
\end{lstlisting}



\subsubsection{Inflated}
\label{sec:verified-smart-contracts-inflated}
The inflated dataset is also based on the raw dataset. Each contract file in the dataset is split into its original representative files. This mitigates a lot of the problems of the flattened dataset in terms of duplicated library code. The library code would, along with other imported contract files, be split into separate contract records. The 2,217,692 "raw" smart contracts are inflated to a total of 5,403,136 separate contract files. These are then grouped by "file\_name" and filtered for uniqueness with a similarity threshold of 0.9. This should produce a dataset with a large amount of unique source code, with low quantities of library code. A total of 5,216,739 duplications are found, giving a duplication percentage of 96.56\%. The resulting dataset consists of 186,397 contracts.

%Processing: 100%|██████| 74/74 [22:50<00:00, 18.52s/it, dupes=5217191/5403136 (96.56%)]


\lstinline[language=Python]!python script/filter_data.py -s parquet -o data/inflated --split-files --threshold 0.9!
dupes=5217191/5403136 (96.56%)


\begin{lstlisting}[
    caption={Solidity standard JSON Input format.},
    label=lst:flattened-dataset-cmd,
    language=JSON]
    {
        'contract_name': 'PinkLemonade',
        'file_path': 'PinkLemonade.sol',
        'contract_address': '0x9a5be3cc368f01a0566a613aad7183783cff7eec',
        'language': 'Solidity',
        'source_code': '/**\r\n\r\nt.me/pinklemonadecoin\r\n*/\r\n\r\n// SPDX-License-Identifier: MIT\r\npragma solidity ^0.8.0;\r\n\r\n\r\n/*\r\n * @dev Provides information about the current execution context, including the\r\n * sender of the transaction and its data. While these are generally available...',
        'abi': '[{"inputs":[],"stateMutability":"nonpayable","type":"constructor"}...]',
        'compiler_version': 'v0.8.4+commit.c7e474f2',
        'optimization_used': False,
        'runs': 200,
        'constructor_arguments': '',
        'evm_version': 'Default',
        'library': '',
        'license_type': 'MIT',
        'proxy': False,
        'implementation': '',
        'swarm_source': 'ipfs://eb0ac9491a04e7a196280fd27ce355a85d79b34c7b0a83ab606d27972a06050c'
      }
      
      
\end{lstlisting}

\begin{figure}[ht]
    \centering
    %% Creator: Matplotlib, PGF backend
%%
%% To include the figure in your LaTeX document, write
%%   \input{<filename>.pgf}
%%
%% Make sure the required packages are loaded in your preamble
%%   \usepackage{pgf}
%%
%% and, on pdftex
%%   \usepackage[utf8]{inputenc}\DeclareUnicodeCharacter{2212}{-}
%%
%% or, on luatex and xetex
%%   \usepackage{unicode-math}
%%
%% Figures using additional raster images can only be included by \input if
%% they are in the same directory as the main LaTeX file. For loading figures
%% from other directories you can use the `import` package
%%   \usepackage{import}
%%
%% and then include the figures with
%%   \import{<path to file>}{<filename>.pgf}
%%
%% Matplotlib used the following preamble
%%
\begingroup%
\makeatletter%
\begin{pgfpicture}%
\pgfpathrectangle{\pgfpointorigin}{\pgfqpoint{3.449272in}{3.773420in}}%
\pgfusepath{use as bounding box, clip}%
\begin{pgfscope}%
\pgfsetbuttcap%
\pgfsetmiterjoin%
\definecolor{currentfill}{rgb}{1.000000,1.000000,1.000000}%
\pgfsetfillcolor{currentfill}%
\pgfsetlinewidth{0.000000pt}%
\definecolor{currentstroke}{rgb}{1.000000,1.000000,1.000000}%
\pgfsetstrokecolor{currentstroke}%
\pgfsetdash{}{0pt}%
\pgfpathmoveto{\pgfqpoint{0.000000in}{-0.000000in}}%
\pgfpathlineto{\pgfqpoint{3.449272in}{-0.000000in}}%
\pgfpathlineto{\pgfqpoint{3.449272in}{3.773420in}}%
\pgfpathlineto{\pgfqpoint{0.000000in}{3.773420in}}%
\pgfpathclose%
\pgfusepath{fill}%
\end{pgfscope}%
\begin{pgfscope}%
\pgfsetbuttcap%
\pgfsetmiterjoin%
\definecolor{currentfill}{rgb}{0.475341,0.000000,0.943137}%
\pgfsetfillcolor{currentfill}%
\pgfsetfillopacity{0.700000}%
\pgfsetlinewidth{0.000000pt}%
\definecolor{currentstroke}{rgb}{0.000000,0.000000,0.000000}%
\pgfsetstrokecolor{currentstroke}%
\pgfsetstrokeopacity{0.700000}%
\pgfsetdash{}{0pt}%
\pgfpathmoveto{\pgfqpoint{2.651052in}{2.128968in}}%
\pgfpathcurveto{\pgfqpoint{2.651052in}{2.133588in}}{\pgfqpoint{2.651020in}{2.138209in}}{\pgfqpoint{2.650958in}{2.142829in}}%
\pgfpathcurveto{\pgfqpoint{2.650895in}{2.147449in}}{\pgfqpoint{2.650801in}{2.152068in}}{\pgfqpoint{2.650675in}{2.156687in}}%
\pgfpathlineto{\pgfqpoint{2.140693in}{2.142827in}}%
\pgfpathcurveto{\pgfqpoint{2.140756in}{2.140518in}}{\pgfqpoint{2.140803in}{2.138208in}}{\pgfqpoint{2.140834in}{2.135898in}}%
\pgfpathcurveto{\pgfqpoint{2.140866in}{2.133588in}}{\pgfqpoint{2.140881in}{2.131278in}}{\pgfqpoint{2.140881in}{2.128968in}}%
\pgfpathlineto{\pgfqpoint{2.651052in}{2.128968in}}%
\pgfpathclose%
\pgfusepath{fill}%
\end{pgfscope}%
\begin{pgfscope}%
\pgfsetbuttcap%
\pgfsetmiterjoin%
\definecolor{currentfill}{rgb}{0.882745,0.000000,0.111414}%
\pgfsetfillcolor{currentfill}%
\pgfsetfillopacity{0.700000}%
\pgfsetlinewidth{0.000000pt}%
\definecolor{currentstroke}{rgb}{0.000000,0.000000,0.000000}%
\pgfsetstrokecolor{currentstroke}%
\pgfsetstrokeopacity{0.700000}%
\pgfsetdash{}{0pt}%
\pgfpathmoveto{\pgfqpoint{2.650675in}{2.156687in}}%
\pgfpathcurveto{\pgfqpoint{2.643613in}{2.416544in}}{\pgfqpoint{2.537505in}{2.664127in}}{\pgfqpoint{2.354202in}{2.848452in}}%
\pgfpathcurveto{\pgfqpoint{2.170898in}{3.032777in}}{\pgfqpoint{1.923908in}{3.140258in}}{\pgfqpoint{1.664093in}{3.148763in}}%
\pgfpathcurveto{\pgfqpoint{1.404279in}{3.157268in}}{\pgfqpoint{1.150788in}{3.066169in}}{\pgfqpoint{0.955822in}{2.894227in}}%
\pgfpathcurveto{\pgfqpoint{0.760856in}{2.722285in}}{\pgfqpoint{0.638784in}{2.482171in}}{\pgfqpoint{0.614743in}{2.223332in}}%
\pgfpathlineto{\pgfqpoint{1.122727in}{2.176150in}}%
\pgfpathcurveto{\pgfqpoint{1.134747in}{2.305570in}}{\pgfqpoint{1.195784in}{2.425626in}}{\pgfqpoint{1.293267in}{2.511598in}}%
\pgfpathcurveto{\pgfqpoint{1.390750in}{2.597569in}}{\pgfqpoint{1.517495in}{2.643118in}}{\pgfqpoint{1.647402in}{2.638866in}}%
\pgfpathcurveto{\pgfqpoint{1.777309in}{2.634613in}}{\pgfqpoint{1.900804in}{2.580872in}}{\pgfqpoint{1.992456in}{2.488710in}}%
\pgfpathcurveto{\pgfqpoint{2.084108in}{2.396548in}}{\pgfqpoint{2.137162in}{2.272756in}}{\pgfqpoint{2.140693in}{2.142827in}}%
\pgfpathlineto{\pgfqpoint{2.650675in}{2.156687in}}%
\pgfpathclose%
\pgfusepath{fill}%
\end{pgfscope}%
\begin{pgfscope}%
\pgfsetbuttcap%
\pgfsetmiterjoin%
\definecolor{currentfill}{rgb}{0.000000,0.786667,0.312353}%
\pgfsetfillcolor{currentfill}%
\pgfsetfillopacity{0.700000}%
\pgfsetlinewidth{0.000000pt}%
\definecolor{currentstroke}{rgb}{0.000000,0.000000,0.000000}%
\pgfsetstrokecolor{currentstroke}%
\pgfsetstrokeopacity{0.700000}%
\pgfsetdash{}{0pt}%
\pgfpathmoveto{\pgfqpoint{0.614743in}{2.223332in}}%
\pgfpathcurveto{\pgfqpoint{0.611982in}{2.193606in}}{\pgfqpoint{0.610527in}{2.163774in}}{\pgfqpoint{0.610382in}{2.133921in}}%
\pgfpathcurveto{\pgfqpoint{0.610237in}{2.104068in}}{\pgfqpoint{0.611402in}{2.074223in}}{\pgfqpoint{0.613874in}{2.044472in}}%
\pgfpathlineto{\pgfqpoint{1.122293in}{2.086720in}}%
\pgfpathcurveto{\pgfqpoint{1.121057in}{2.101596in}}{\pgfqpoint{1.120474in}{2.116518in}}{\pgfqpoint{1.120546in}{2.131445in}}%
\pgfpathcurveto{\pgfqpoint{1.120619in}{2.146371in}}{\pgfqpoint{1.121346in}{2.161287in}}{\pgfqpoint{1.122727in}{2.176150in}}%
\pgfpathlineto{\pgfqpoint{0.614743in}{2.223332in}}%
\pgfpathclose%
\pgfusepath{fill}%
\end{pgfscope}%
\begin{pgfscope}%
\pgfsetbuttcap%
\pgfsetmiterjoin%
\definecolor{currentfill}{rgb}{0.930307,0.285685,0.029301}%
\pgfsetfillcolor{currentfill}%
\pgfsetfillopacity{0.700000}%
\pgfsetlinewidth{0.000000pt}%
\definecolor{currentstroke}{rgb}{0.000000,0.000000,0.000000}%
\pgfsetstrokecolor{currentstroke}%
\pgfsetstrokeopacity{0.700000}%
\pgfsetdash{}{0pt}%
\pgfpathmoveto{\pgfqpoint{0.613874in}{2.044472in}}%
\pgfpathcurveto{\pgfqpoint{0.633145in}{1.812567in}}{\pgfqpoint{0.731182in}{1.594089in}}{\pgfqpoint{0.891613in}{1.425527in}}%
\pgfpathcurveto{\pgfqpoint{1.052044in}{1.256964in}}{\pgfqpoint{1.265411in}{1.148252in}}{\pgfqpoint{1.496081in}{1.117548in}}%
\pgfpathlineto{\pgfqpoint{1.563396in}{1.623258in}}%
\pgfpathcurveto{\pgfqpoint{1.448061in}{1.638610in}}{\pgfqpoint{1.341377in}{1.692966in}}{\pgfqpoint{1.261162in}{1.777247in}}%
\pgfpathcurveto{\pgfqpoint{1.180946in}{1.861529in}}{\pgfqpoint{1.131928in}{1.970768in}}{\pgfqpoint{1.122293in}{2.086720in}}%
\pgfpathlineto{\pgfqpoint{0.613874in}{2.044472in}}%
\pgfpathclose%
\pgfusepath{fill}%
\end{pgfscope}%
\begin{pgfscope}%
\pgfsetbuttcap%
\pgfsetmiterjoin%
\definecolor{currentfill}{rgb}{0.063700,0.382199,0.851986}%
\pgfsetfillcolor{currentfill}%
\pgfsetfillopacity{0.700000}%
\pgfsetlinewidth{0.000000pt}%
\definecolor{currentstroke}{rgb}{0.000000,0.000000,0.000000}%
\pgfsetstrokecolor{currentstroke}%
\pgfsetstrokeopacity{0.700000}%
\pgfsetdash{}{0pt}%
\pgfpathmoveto{\pgfqpoint{1.496081in}{1.117548in}}%
\pgfpathcurveto{\pgfqpoint{1.640169in}{1.098369in}}{\pgfqpoint{1.786706in}{1.110199in}}{\pgfqpoint{1.925851in}{1.152245in}}%
\pgfpathcurveto{\pgfqpoint{2.064996in}{1.194291in}}{\pgfqpoint{2.193564in}{1.265590in}}{\pgfqpoint{2.302919in}{1.361354in}}%
\pgfpathcurveto{\pgfqpoint{2.412274in}{1.457117in}}{\pgfqpoint{2.499914in}{1.575153in}}{\pgfqpoint{2.559952in}{1.707534in}}%
\pgfpathcurveto{\pgfqpoint{2.619990in}{1.839914in}}{\pgfqpoint{2.651052in}{1.983609in}}{\pgfqpoint{2.651052in}{2.128968in}}%
\pgfpathlineto{\pgfqpoint{2.140881in}{2.128968in}}%
\pgfpathcurveto{\pgfqpoint{2.140881in}{2.056289in}}{\pgfqpoint{2.125350in}{1.984441in}}{\pgfqpoint{2.095331in}{1.918251in}}%
\pgfpathcurveto{\pgfqpoint{2.065312in}{1.852060in}}{\pgfqpoint{2.021493in}{1.793043in}}{\pgfqpoint{1.966815in}{1.745161in}}%
\pgfpathcurveto{\pgfqpoint{1.912137in}{1.697279in}}{\pgfqpoint{1.847854in}{1.661630in}}{\pgfqpoint{1.778281in}{1.640607in}}%
\pgfpathcurveto{\pgfqpoint{1.708709in}{1.619584in}}{\pgfqpoint{1.635440in}{1.613668in}}{\pgfqpoint{1.563396in}{1.623258in}}%
\pgfpathlineto{\pgfqpoint{1.496081in}{1.117548in}}%
\pgfpathclose%
\pgfusepath{fill}%
\end{pgfscope}%
\begin{pgfscope}%
\pgfsetbuttcap%
\pgfsetmiterjoin%
\definecolor{currentfill}{rgb}{0.475341,0.000000,0.943137}%
\pgfsetfillcolor{currentfill}%
\pgfsetfillopacity{0.700000}%
\pgfsetlinewidth{0.000000pt}%
\definecolor{currentstroke}{rgb}{0.000000,0.000000,0.000000}%
\pgfsetstrokecolor{currentstroke}%
\pgfsetstrokeopacity{0.700000}%
\pgfsetdash{}{0pt}%
\pgfpathmoveto{\pgfqpoint{3.161222in}{2.128968in}}%
\pgfpathcurveto{\pgfqpoint{3.161222in}{2.135899in}}{\pgfqpoint{3.161175in}{2.142829in}}{\pgfqpoint{3.161081in}{2.149759in}}%
\pgfpathcurveto{\pgfqpoint{3.160987in}{2.156689in}}{\pgfqpoint{3.160846in}{2.163618in}}{\pgfqpoint{3.160658in}{2.170546in}}%
\pgfpathlineto{\pgfqpoint{2.650675in}{2.156687in}}%
\pgfpathcurveto{\pgfqpoint{2.650801in}{2.152068in}}{\pgfqpoint{2.650895in}{2.147449in}}{\pgfqpoint{2.650958in}{2.142829in}}%
\pgfpathcurveto{\pgfqpoint{2.651020in}{2.138209in}}{\pgfqpoint{2.651052in}{2.133588in}}{\pgfqpoint{2.651052in}{2.128968in}}%
\pgfpathlineto{\pgfqpoint{3.161222in}{2.128968in}}%
\pgfpathclose%
\pgfusepath{fill}%
\end{pgfscope}%
\begin{pgfscope}%
\pgfsetbuttcap%
\pgfsetmiterjoin%
\definecolor{currentfill}{rgb}{0.882745,0.000000,0.111414}%
\pgfsetfillcolor{currentfill}%
\pgfsetfillopacity{0.700000}%
\pgfsetlinewidth{0.000000pt}%
\definecolor{currentstroke}{rgb}{0.000000,0.000000,0.000000}%
\pgfsetstrokecolor{currentstroke}%
\pgfsetstrokeopacity{0.700000}%
\pgfsetdash{}{0pt}%
\pgfpathmoveto{\pgfqpoint{3.160658in}{2.170546in}}%
\pgfpathcurveto{\pgfqpoint{3.159046in}{2.229862in}}{\pgfqpoint{3.153985in}{2.289035in}}{\pgfqpoint{3.145503in}{2.347763in}}%
\pgfpathcurveto{\pgfqpoint{3.137020in}{2.406492in}}{\pgfqpoint{3.125129in}{2.464678in}}{\pgfqpoint{3.109890in}{2.522026in}}%
\pgfpathlineto{\pgfqpoint{2.616830in}{2.391007in}}%
\pgfpathcurveto{\pgfqpoint{2.626990in}{2.352775in}}{\pgfqpoint{2.634917in}{2.313984in}}{\pgfqpoint{2.640572in}{2.274832in}}%
\pgfpathcurveto{\pgfqpoint{2.646227in}{2.235679in}}{\pgfqpoint{2.649601in}{2.196231in}}{\pgfqpoint{2.650675in}{2.156687in}}%
\pgfpathlineto{\pgfqpoint{3.160658in}{2.170546in}}%
\pgfpathclose%
\pgfusepath{fill}%
\end{pgfscope}%
\begin{pgfscope}%
\pgfsetbuttcap%
\pgfsetmiterjoin%
\definecolor{currentfill}{rgb}{0.917292,0.307374,0.323876}%
\pgfsetfillcolor{currentfill}%
\pgfsetfillopacity{0.700000}%
\pgfsetlinewidth{0.000000pt}%
\definecolor{currentstroke}{rgb}{0.000000,0.000000,0.000000}%
\pgfsetstrokecolor{currentstroke}%
\pgfsetstrokeopacity{0.700000}%
\pgfsetdash{}{0pt}%
\pgfpathmoveto{\pgfqpoint{3.109890in}{2.522026in}}%
\pgfpathcurveto{\pgfqpoint{3.091225in}{2.592268in}}{\pgfqpoint{3.067580in}{2.661092in}}{\pgfqpoint{3.039134in}{2.727974in}}%
\pgfpathcurveto{\pgfqpoint{3.010689in}{2.794857in}}{\pgfqpoint{2.977516in}{2.859628in}}{\pgfqpoint{2.939867in}{2.921797in}}%
\pgfpathlineto{\pgfqpoint{2.503481in}{2.657520in}}%
\pgfpathcurveto{\pgfqpoint{2.528581in}{2.616075in}}{\pgfqpoint{2.550696in}{2.572894in}}{\pgfqpoint{2.569660in}{2.528306in}}%
\pgfpathcurveto{\pgfqpoint{2.588623in}{2.483717in}}{\pgfqpoint{2.604387in}{2.437835in}}{\pgfqpoint{2.616830in}{2.391007in}}%
\pgfpathlineto{\pgfqpoint{3.109890in}{2.522026in}}%
\pgfpathclose%
\pgfusepath{fill}%
\end{pgfscope}%
\begin{pgfscope}%
\pgfsetbuttcap%
\pgfsetmiterjoin%
\definecolor{currentfill}{rgb}{0.989063,0.592703,0.504860}%
\pgfsetfillcolor{currentfill}%
\pgfsetfillopacity{0.700000}%
\pgfsetlinewidth{0.000000pt}%
\definecolor{currentstroke}{rgb}{0.000000,0.000000,0.000000}%
\pgfsetstrokecolor{currentstroke}%
\pgfsetstrokeopacity{0.700000}%
\pgfsetdash{}{0pt}%
\pgfpathmoveto{\pgfqpoint{2.939867in}{2.921797in}}%
\pgfpathcurveto{\pgfqpoint{2.822564in}{3.115492in}}{\pgfqpoint{2.663723in}{3.280762in}}{\pgfqpoint{2.474832in}{3.405653in}}%
\pgfpathcurveto{\pgfqpoint{2.285940in}{3.530545in}}{\pgfqpoint{2.071665in}{3.611973in}}{\pgfqpoint{1.847502in}{3.644048in}}%
\pgfpathlineto{\pgfqpoint{1.775238in}{3.139021in}}%
\pgfpathcurveto{\pgfqpoint{1.924680in}{3.117638in}}{\pgfqpoint{2.067531in}{3.063353in}}{\pgfqpoint{2.193458in}{2.980092in}}%
\pgfpathcurveto{\pgfqpoint{2.319386in}{2.896831in}}{\pgfqpoint{2.425280in}{2.786651in}}{\pgfqpoint{2.503481in}{2.657520in}}%
\pgfpathlineto{\pgfqpoint{2.939867in}{2.921797in}}%
\pgfpathclose%
\pgfusepath{fill}%
\end{pgfscope}%
\begin{pgfscope}%
\pgfsetbuttcap%
\pgfsetmiterjoin%
\definecolor{currentfill}{rgb}{0.991894,0.815723,0.745126}%
\pgfsetfillcolor{currentfill}%
\pgfsetfillopacity{0.700000}%
\pgfsetlinewidth{0.000000pt}%
\definecolor{currentstroke}{rgb}{0.000000,0.000000,0.000000}%
\pgfsetstrokecolor{currentstroke}%
\pgfsetstrokeopacity{0.700000}%
\pgfsetdash{}{0pt}%
\pgfpathmoveto{\pgfqpoint{1.847502in}{3.644048in}}%
\pgfpathcurveto{\pgfqpoint{1.642231in}{3.673420in}}{\pgfqpoint{1.433115in}{3.660771in}}{\pgfqpoint{1.232881in}{3.606871in}}%
\pgfpathcurveto{\pgfqpoint{1.032647in}{3.552971in}}{\pgfqpoint{0.845439in}{3.458936in}}{\pgfqpoint{0.682651in}{3.330487in}}%
\pgfpathcurveto{\pgfqpoint{0.519863in}{3.202039in}}{\pgfqpoint{0.384864in}{3.041837in}}{\pgfqpoint{0.285870in}{2.859631in}}%
\pgfpathcurveto{\pgfqpoint{0.186876in}{2.677425in}}{\pgfqpoint{0.125936in}{2.476986in}}{\pgfqpoint{0.106759in}{2.270513in}}%
\pgfpathlineto{\pgfqpoint{0.614743in}{2.223332in}}%
\pgfpathcurveto{\pgfqpoint{0.627528in}{2.360980in}}{\pgfqpoint{0.668154in}{2.494606in}}{\pgfqpoint{0.734150in}{2.616077in}}%
\pgfpathcurveto{\pgfqpoint{0.800146in}{2.737548in}}{\pgfqpoint{0.890146in}{2.844349in}}{\pgfqpoint{0.998671in}{2.929981in}}%
\pgfpathcurveto{\pgfqpoint{1.107197in}{3.015613in}}{\pgfqpoint{1.232001in}{3.078303in}}{\pgfqpoint{1.365491in}{3.114237in}}%
\pgfpathcurveto{\pgfqpoint{1.498980in}{3.150170in}}{\pgfqpoint{1.638391in}{3.158603in}}{\pgfqpoint{1.775238in}{3.139021in}}%
\pgfpathlineto{\pgfqpoint{1.847502in}{3.644048in}}%
\pgfpathclose%
\pgfusepath{fill}%
\end{pgfscope}%
\begin{pgfscope}%
\pgfsetbuttcap%
\pgfsetmiterjoin%
\definecolor{currentfill}{rgb}{0.000000,0.786667,0.312353}%
\pgfsetfillcolor{currentfill}%
\pgfsetfillopacity{0.700000}%
\pgfsetlinewidth{0.000000pt}%
\definecolor{currentstroke}{rgb}{0.000000,0.000000,0.000000}%
\pgfsetstrokecolor{currentstroke}%
\pgfsetstrokeopacity{0.700000}%
\pgfsetdash{}{0pt}%
\pgfpathmoveto{\pgfqpoint{0.106759in}{2.270513in}}%
\pgfpathcurveto{\pgfqpoint{0.102617in}{2.225925in}}{\pgfqpoint{0.100435in}{2.181177in}}{\pgfqpoint{0.100217in}{2.136397in}}%
\pgfpathcurveto{\pgfqpoint{0.100000in}{2.091618in}}{\pgfqpoint{0.101748in}{2.046851in}}{\pgfqpoint{0.105456in}{2.002225in}}%
\pgfpathlineto{\pgfqpoint{0.613874in}{2.044472in}}%
\pgfpathcurveto{\pgfqpoint{0.611402in}{2.074223in}}{\pgfqpoint{0.610237in}{2.104068in}}{\pgfqpoint{0.610382in}{2.133921in}}%
\pgfpathcurveto{\pgfqpoint{0.610527in}{2.163774in}}{\pgfqpoint{0.611982in}{2.193606in}}{\pgfqpoint{0.614743in}{2.223332in}}%
\pgfpathlineto{\pgfqpoint{0.106759in}{2.270513in}}%
\pgfpathclose%
\pgfusepath{fill}%
\end{pgfscope}%
\begin{pgfscope}%
\pgfsetbuttcap%
\pgfsetmiterjoin%
\definecolor{currentfill}{rgb}{0.930307,0.285685,0.029301}%
\pgfsetfillcolor{currentfill}%
\pgfsetfillopacity{0.700000}%
\pgfsetlinewidth{0.000000pt}%
\definecolor{currentstroke}{rgb}{0.000000,0.000000,0.000000}%
\pgfsetstrokecolor{currentstroke}%
\pgfsetstrokeopacity{0.700000}%
\pgfsetdash{}{0pt}%
\pgfpathmoveto{\pgfqpoint{0.105456in}{2.002225in}}%
\pgfpathcurveto{\pgfqpoint{0.124422in}{1.773989in}}{\pgfqpoint{0.194377in}{1.552894in}}{\pgfqpoint{0.310149in}{1.355287in}}%
\pgfpathcurveto{\pgfqpoint{0.425921in}{1.157681in}}{\pgfqpoint{0.584584in}{0.988558in}}{\pgfqpoint{0.774407in}{0.860422in}}%
\pgfpathlineto{\pgfqpoint{1.059841in}{1.283271in}}%
\pgfpathcurveto{\pgfqpoint{0.933293in}{1.368695in}}{\pgfqpoint{0.827518in}{1.481444in}}{\pgfqpoint{0.750336in}{1.613181in}}%
\pgfpathcurveto{\pgfqpoint{0.673155in}{1.744918in}}{\pgfqpoint{0.626518in}{1.892315in}}{\pgfqpoint{0.613874in}{2.044472in}}%
\pgfpathlineto{\pgfqpoint{0.105456in}{2.002225in}}%
\pgfpathclose%
\pgfusepath{fill}%
\end{pgfscope}%
\begin{pgfscope}%
\pgfsetbuttcap%
\pgfsetmiterjoin%
\definecolor{currentfill}{rgb}{0.995815,0.464325,0.203521}%
\pgfsetfillcolor{currentfill}%
\pgfsetfillopacity{0.700000}%
\pgfsetlinewidth{0.000000pt}%
\definecolor{currentstroke}{rgb}{0.000000,0.000000,0.000000}%
\pgfsetstrokecolor{currentstroke}%
\pgfsetstrokeopacity{0.700000}%
\pgfsetdash{}{0pt}%
\pgfpathmoveto{\pgfqpoint{0.774407in}{0.860422in}}%
\pgfpathcurveto{\pgfqpoint{0.872074in}{0.794494in}}{\pgfqpoint{0.977029in}{0.740065in}}{\pgfqpoint{1.087184in}{0.698219in}}%
\pgfpathcurveto{\pgfqpoint{1.197340in}{0.656372in}}{\pgfqpoint{1.311960in}{0.627386in}}{\pgfqpoint{1.428766in}{0.611838in}}%
\pgfpathlineto{\pgfqpoint{1.496081in}{1.117548in}}%
\pgfpathcurveto{\pgfqpoint{1.418210in}{1.127913in}}{\pgfqpoint{1.341797in}{1.147237in}}{\pgfqpoint{1.268360in}{1.175135in}}%
\pgfpathcurveto{\pgfqpoint{1.194923in}{1.203033in}}{\pgfqpoint{1.124953in}{1.239319in}}{\pgfqpoint{1.059841in}{1.283271in}}%
\pgfpathlineto{\pgfqpoint{0.774407in}{0.860422in}}%
\pgfpathclose%
\pgfusepath{fill}%
\end{pgfscope}%
\begin{pgfscope}%
\pgfsetbuttcap%
\pgfsetmiterjoin%
\definecolor{currentfill}{rgb}{0.063700,0.382199,0.851986}%
\pgfsetfillcolor{currentfill}%
\pgfsetfillopacity{0.700000}%
\pgfsetlinewidth{0.000000pt}%
\definecolor{currentstroke}{rgb}{0.000000,0.000000,0.000000}%
\pgfsetstrokecolor{currentstroke}%
\pgfsetstrokeopacity{0.700000}%
\pgfsetdash{}{0pt}%
\pgfpathmoveto{\pgfqpoint{1.428766in}{0.611838in}}%
\pgfpathcurveto{\pgfqpoint{1.644898in}{0.583069in}}{\pgfqpoint{1.864704in}{0.600815in}}{\pgfqpoint{2.073422in}{0.663884in}}%
\pgfpathcurveto{\pgfqpoint{2.282139in}{0.726953in}}{\pgfqpoint{2.474990in}{0.833901in}}{\pgfqpoint{2.639023in}{0.977546in}}%
\pgfpathcurveto{\pgfqpoint{2.803056in}{1.121192in}}{\pgfqpoint{2.934515in}{1.298245in}}{\pgfqpoint{3.024572in}{1.496816in}}%
\pgfpathcurveto{\pgfqpoint{3.114629in}{1.695387in}}{\pgfqpoint{3.161222in}{1.910930in}}{\pgfqpoint{3.161222in}{2.128968in}}%
\pgfpathlineto{\pgfqpoint{2.651052in}{2.128968in}}%
\pgfpathcurveto{\pgfqpoint{2.651052in}{1.983609in}}{\pgfqpoint{2.619990in}{1.839914in}}{\pgfqpoint{2.559952in}{1.707534in}}%
\pgfpathcurveto{\pgfqpoint{2.499914in}{1.575153in}}{\pgfqpoint{2.412274in}{1.457117in}}{\pgfqpoint{2.302919in}{1.361354in}}%
\pgfpathcurveto{\pgfqpoint{2.193564in}{1.265590in}}{\pgfqpoint{2.064996in}{1.194291in}}{\pgfqpoint{1.925851in}{1.152245in}}%
\pgfpathcurveto{\pgfqpoint{1.786706in}{1.110199in}}{\pgfqpoint{1.640169in}{1.098369in}}{\pgfqpoint{1.496081in}{1.117548in}}%
\pgfpathlineto{\pgfqpoint{1.428766in}{0.611838in}}%
\pgfpathclose%
\pgfusepath{fill}%
\end{pgfscope}%
\begin{pgfscope}%
\definecolor{textcolor}{rgb}{0.000000,0.000000,0.000000}%
\pgfsetstrokecolor{textcolor}%
\pgfsetfillcolor{textcolor}%
\pgftext[x=2.344884in,y=2.138671in,left,]{\color{textcolor}\rmfamily\fontsize{10.000000}{12.000000}\selectfont E}%
\end{pgfscope}%
\begin{pgfscope}%
\definecolor{textcolor}{rgb}{0.000000,0.000000,0.000000}%
\pgfsetstrokecolor{textcolor}%
\pgfsetfillcolor{textcolor}%
\pgftext[x=1.654078in,y=2.842825in,left,]{\color{textcolor}\rmfamily\fontsize{10.000000}{12.000000}\selectfont H}%
\end{pgfscope}%
\begin{pgfscope}%
\definecolor{textcolor}{rgb}{0.000000,0.000000,0.000000}%
\pgfsetstrokecolor{textcolor}%
\pgfsetfillcolor{textcolor}%
\pgftext[x=0.916481in,y=2.132435in,right,]{\color{textcolor}\rmfamily\fontsize{10.000000}{12.000000}\selectfont L}%
\end{pgfscope}%
\begin{pgfscope}%
\definecolor{textcolor}{rgb}{0.000000,0.000000,0.000000}%
\pgfsetstrokecolor{textcolor}%
\pgfsetfillcolor{textcolor}%
\pgftext[x=1.113342in,y=1.636559in,right,]{\color{textcolor}\rmfamily\fontsize{10.000000}{12.000000}\selectfont M}%
\end{pgfscope}%
\begin{pgfscope}%
\definecolor{textcolor}{rgb}{0.000000,0.000000,0.000000}%
\pgfsetstrokecolor{textcolor}%
\pgfsetfillcolor{textcolor}%
\pgftext[x=2.101257in,y=1.591638in,left,]{\color{textcolor}\rmfamily\fontsize{10.000000}{12.000000}\selectfont S}%
\end{pgfscope}%
\begin{pgfscope}%
\definecolor{textcolor}{rgb}{0.000000,0.000000,0.000000}%
\pgfsetstrokecolor{textcolor}%
\pgfsetfillcolor{textcolor}%
\pgftext[x=2.838003in,y=2.145370in,left,]{\color{textcolor}\rmfamily\fontsize{10.000000}{12.000000}\selectfont E}%
\end{pgfscope}%
\begin{pgfscope}%
\definecolor{textcolor}{rgb}{0.000000,0.000000,0.000000}%
\pgfsetstrokecolor{textcolor}%
\pgfsetfillcolor{textcolor}%
\pgftext[x=2.825713in,y=2.301573in,left,]{\color{textcolor}\rmfamily\fontsize{10.000000}{12.000000}\selectfont H}%
\end{pgfscope}%
\begin{pgfscope}%
\definecolor{textcolor}{rgb}{0.000000,0.000000,0.000000}%
\pgfsetstrokecolor{textcolor}%
\pgfsetfillcolor{textcolor}%
\pgftext[x=2.741801in,y=2.601518in,left,]{\color{textcolor}\rmfamily\fontsize{10.000000}{12.000000}\selectfont HL}%
\end{pgfscope}%
\begin{pgfscope}%
\definecolor{textcolor}{rgb}{0.000000,0.000000,0.000000}%
\pgfsetstrokecolor{textcolor}%
\pgfsetfillcolor{textcolor}%
\pgftext[x=2.296628in,y=3.136131in,left,]{\color{textcolor}\rmfamily\fontsize{10.000000}{12.000000}\selectfont HM}%
\end{pgfscope}%
\begin{pgfscope}%
\definecolor{textcolor}{rgb}{0.000000,0.000000,0.000000}%
\pgfsetstrokecolor{textcolor}%
\pgfsetfillcolor{textcolor}%
\pgftext[x=0.882797in,y=3.076833in,right,]{\color{textcolor}\rmfamily\fontsize{10.000000}{12.000000}\selectfont HML}%
\end{pgfscope}%
\begin{pgfscope}%
\definecolor{textcolor}{rgb}{0.000000,0.000000,0.000000}%
\pgfsetstrokecolor{textcolor}%
\pgfsetfillcolor{textcolor}%
\pgftext[x=0.423322in,y=2.134829in,right,]{\color{textcolor}\rmfamily\fontsize{10.000000}{12.000000}\selectfont L}%
\end{pgfscope}%
\begin{pgfscope}%
\definecolor{textcolor}{rgb}{0.000000,0.000000,0.000000}%
\pgfsetstrokecolor{textcolor}%
\pgfsetfillcolor{textcolor}%
\pgftext[x=0.588934in,y=1.518620in,right,]{\color{textcolor}\rmfamily\fontsize{10.000000}{12.000000}\selectfont M}%
\end{pgfscope}%
\begin{pgfscope}%
\definecolor{textcolor}{rgb}{0.000000,0.000000,0.000000}%
\pgfsetstrokecolor{textcolor}%
\pgfsetfillcolor{textcolor}%
\pgftext[x=1.201929in,y=1.000266in,right,]{\color{textcolor}\rmfamily\fontsize{10.000000}{12.000000}\selectfont ML}%
\end{pgfscope}%
\begin{pgfscope}%
\definecolor{textcolor}{rgb}{0.000000,0.000000,0.000000}%
\pgfsetstrokecolor{textcolor}%
\pgfsetfillcolor{textcolor}%
\pgftext[x=2.426157in,y=1.220624in,left,]{\color{textcolor}\rmfamily\fontsize{10.000000}{12.000000}\selectfont S}%
\end{pgfscope}%
\begin{pgfscope}%
\pgfsetbuttcap%
\pgfsetmiterjoin%
\definecolor{currentfill}{rgb}{0.475341,0.000000,0.943137}%
\pgfsetfillcolor{currentfill}%
\pgfsetfillopacity{0.700000}%
\pgfsetlinewidth{1.003750pt}%
\definecolor{currentstroke}{rgb}{0.475341,0.000000,0.943137}%
\pgfsetstrokecolor{currentstroke}%
\pgfsetstrokeopacity{0.700000}%
\pgfsetdash{}{0pt}%
\pgfpathmoveto{\pgfqpoint{0.352742in}{0.376234in}}%
\pgfpathlineto{\pgfqpoint{0.630520in}{0.376234in}}%
\pgfpathlineto{\pgfqpoint{0.630520in}{0.473457in}}%
\pgfpathlineto{\pgfqpoint{0.352742in}{0.473457in}}%
\pgfpathclose%
\pgfusepath{stroke,fill}%
\end{pgfscope}%
\begin{pgfscope}%
\definecolor{textcolor}{rgb}{0.000000,0.000000,0.000000}%
\pgfsetstrokecolor{textcolor}%
\pgfsetfillcolor{textcolor}%
\pgftext[x=0.741631in,y=0.376234in,left,base]{\color{textcolor}\rmfamily\fontsize{10.000000}{12.000000}\selectfont Error}%
\end{pgfscope}%
\begin{pgfscope}%
\pgfsetbuttcap%
\pgfsetmiterjoin%
\definecolor{currentfill}{rgb}{0.882745,0.000000,0.111414}%
\pgfsetfillcolor{currentfill}%
\pgfsetfillopacity{0.700000}%
\pgfsetlinewidth{1.003750pt}%
\definecolor{currentstroke}{rgb}{0.882745,0.000000,0.111414}%
\pgfsetstrokecolor{currentstroke}%
\pgfsetstrokeopacity{0.700000}%
\pgfsetdash{}{0pt}%
\pgfpathmoveto{\pgfqpoint{0.352742in}{0.182562in}}%
\pgfpathlineto{\pgfqpoint{0.630520in}{0.182562in}}%
\pgfpathlineto{\pgfqpoint{0.630520in}{0.279784in}}%
\pgfpathlineto{\pgfqpoint{0.352742in}{0.279784in}}%
\pgfpathclose%
\pgfusepath{stroke,fill}%
\end{pgfscope}%
\begin{pgfscope}%
\definecolor{textcolor}{rgb}{0.000000,0.000000,0.000000}%
\pgfsetstrokecolor{textcolor}%
\pgfsetfillcolor{textcolor}%
\pgftext[x=0.741631in,y=0.182562in,left,base]{\color{textcolor}\rmfamily\fontsize{10.000000}{12.000000}\selectfont High}%
\end{pgfscope}%
\begin{pgfscope}%
\pgfsetbuttcap%
\pgfsetmiterjoin%
\definecolor{currentfill}{rgb}{0.000000,0.786667,0.312353}%
\pgfsetfillcolor{currentfill}%
\pgfsetfillopacity{0.700000}%
\pgfsetlinewidth{1.003750pt}%
\definecolor{currentstroke}{rgb}{0.000000,0.786667,0.312353}%
\pgfsetstrokecolor{currentstroke}%
\pgfsetstrokeopacity{0.700000}%
\pgfsetdash{}{0pt}%
\pgfpathmoveto{\pgfqpoint{1.346570in}{0.376234in}}%
\pgfpathlineto{\pgfqpoint{1.624348in}{0.376234in}}%
\pgfpathlineto{\pgfqpoint{1.624348in}{0.473457in}}%
\pgfpathlineto{\pgfqpoint{1.346570in}{0.473457in}}%
\pgfpathclose%
\pgfusepath{stroke,fill}%
\end{pgfscope}%
\begin{pgfscope}%
\definecolor{textcolor}{rgb}{0.000000,0.000000,0.000000}%
\pgfsetstrokecolor{textcolor}%
\pgfsetfillcolor{textcolor}%
\pgftext[x=1.735459in,y=0.376234in,left,base]{\color{textcolor}\rmfamily\fontsize{10.000000}{12.000000}\selectfont Low}%
\end{pgfscope}%
\begin{pgfscope}%
\pgfsetbuttcap%
\pgfsetmiterjoin%
\definecolor{currentfill}{rgb}{0.930307,0.285685,0.029301}%
\pgfsetfillcolor{currentfill}%
\pgfsetfillopacity{0.700000}%
\pgfsetlinewidth{1.003750pt}%
\definecolor{currentstroke}{rgb}{0.930307,0.285685,0.029301}%
\pgfsetstrokecolor{currentstroke}%
\pgfsetstrokeopacity{0.700000}%
\pgfsetdash{}{0pt}%
\pgfpathmoveto{\pgfqpoint{1.346570in}{0.182562in}}%
\pgfpathlineto{\pgfqpoint{1.624348in}{0.182562in}}%
\pgfpathlineto{\pgfqpoint{1.624348in}{0.279784in}}%
\pgfpathlineto{\pgfqpoint{1.346570in}{0.279784in}}%
\pgfpathclose%
\pgfusepath{stroke,fill}%
\end{pgfscope}%
\begin{pgfscope}%
\definecolor{textcolor}{rgb}{0.000000,0.000000,0.000000}%
\pgfsetstrokecolor{textcolor}%
\pgfsetfillcolor{textcolor}%
\pgftext[x=1.735459in,y=0.182562in,left,base]{\color{textcolor}\rmfamily\fontsize{10.000000}{12.000000}\selectfont Medium}%
\end{pgfscope}%
\begin{pgfscope}%
\pgfsetbuttcap%
\pgfsetmiterjoin%
\definecolor{currentfill}{rgb}{0.063700,0.382199,0.851986}%
\pgfsetfillcolor{currentfill}%
\pgfsetfillopacity{0.700000}%
\pgfsetlinewidth{1.003750pt}%
\definecolor{currentstroke}{rgb}{0.063700,0.382199,0.851986}%
\pgfsetstrokecolor{currentstroke}%
\pgfsetstrokeopacity{0.700000}%
\pgfsetdash{}{0pt}%
\pgfpathmoveto{\pgfqpoint{2.510923in}{0.376234in}}%
\pgfpathlineto{\pgfqpoint{2.788701in}{0.376234in}}%
\pgfpathlineto{\pgfqpoint{2.788701in}{0.473457in}}%
\pgfpathlineto{\pgfqpoint{2.510923in}{0.473457in}}%
\pgfpathclose%
\pgfusepath{stroke,fill}%
\end{pgfscope}%
\begin{pgfscope}%
\definecolor{textcolor}{rgb}{0.000000,0.000000,0.000000}%
\pgfsetstrokecolor{textcolor}%
\pgfsetfillcolor{textcolor}%
\pgftext[x=2.899812in,y=0.376234in,left,base]{\color{textcolor}\rmfamily\fontsize{10.000000}{12.000000}\selectfont Secure}%
\end{pgfscope}%
\end{pgfpicture}%
\makeatother%
\endgroup%

    \caption{Doughnut chart over security levels, where each level occurs at least once in the \acrshort{sc}. The inner ring shows the distribution of the occurrences of each level. The outer ring shows the additional security levels for each contract. For example, "HML" means that the contract has at least three vulnerabilities with the corresponding "High", "Medium", and "Low" security levels.}
\end{figure}


\begin{figure}[ht]
    \centering
    %% Creator: Matplotlib, PGF backend
%%
%% To include the figure in your LaTeX document, write
%%   \input{<filename>.pgf}
%%
%% Make sure the required packages are loaded in your preamble
%%   \usepackage{pgf}
%%
%% and, on pdftex
%%   \usepackage[utf8]{inputenc}\DeclareUnicodeCharacter{2212}{-}
%%
%% or, on luatex and xetex
%%   \usepackage{unicode-math}
%%
%% Figures using additional raster images can only be included by \input if
%% they are in the same directory as the main LaTeX file. For loading figures
%% from other directories you can use the `import` package
%%   \usepackage{import}
%%
%% and then include the figures with
%%   \import{<path to file>}{<filename>.pgf}
%%
%% Matplotlib used the following preamble
%%
\begingroup%
\makeatletter%
\begin{pgfpicture}%
\pgfpathrectangle{\pgfpointorigin}{\pgfqpoint{5.034351in}{3.885574in}}%
\pgfusepath{use as bounding box, clip}%
\begin{pgfscope}%
\pgfsetbuttcap%
\pgfsetmiterjoin%
\definecolor{currentfill}{rgb}{1.000000,1.000000,1.000000}%
\pgfsetfillcolor{currentfill}%
\pgfsetlinewidth{0.000000pt}%
\definecolor{currentstroke}{rgb}{1.000000,1.000000,1.000000}%
\pgfsetstrokecolor{currentstroke}%
\pgfsetdash{}{0pt}%
\pgfpathmoveto{\pgfqpoint{0.000000in}{0.000000in}}%
\pgfpathlineto{\pgfqpoint{5.034351in}{0.000000in}}%
\pgfpathlineto{\pgfqpoint{5.034351in}{3.885574in}}%
\pgfpathlineto{\pgfqpoint{0.000000in}{3.885574in}}%
\pgfpathclose%
\pgfusepath{fill}%
\end{pgfscope}%
\begin{pgfscope}%
\pgfsetbuttcap%
\pgfsetmiterjoin%
\definecolor{currentfill}{rgb}{1.000000,1.000000,1.000000}%
\pgfsetfillcolor{currentfill}%
\pgfsetlinewidth{0.000000pt}%
\definecolor{currentstroke}{rgb}{0.000000,0.000000,0.000000}%
\pgfsetstrokecolor{currentstroke}%
\pgfsetstrokeopacity{0.000000}%
\pgfsetdash{}{0pt}%
\pgfpathmoveto{\pgfqpoint{2.069525in}{0.499691in}}%
\pgfpathlineto{\pgfqpoint{4.926105in}{0.499691in}}%
\pgfpathlineto{\pgfqpoint{4.926105in}{3.417367in}}%
\pgfpathlineto{\pgfqpoint{2.069525in}{3.417367in}}%
\pgfpathclose%
\pgfusepath{fill}%
\end{pgfscope}%
\begin{pgfscope}%
\pgfpathrectangle{\pgfqpoint{2.069525in}{0.499691in}}{\pgfqpoint{2.856580in}{2.917676in}}%
\pgfusepath{clip}%
\pgfsetrectcap%
\pgfsetroundjoin%
\pgfsetlinewidth{0.803000pt}%
\definecolor{currentstroke}{rgb}{0.690196,0.690196,0.690196}%
\pgfsetstrokecolor{currentstroke}%
\pgfsetdash{}{0pt}%
\pgfpathmoveto{\pgfqpoint{2.069525in}{0.499691in}}%
\pgfpathlineto{\pgfqpoint{2.069525in}{3.417367in}}%
\pgfusepath{stroke}%
\end{pgfscope}%
\begin{pgfscope}%
\pgfsetbuttcap%
\pgfsetroundjoin%
\definecolor{currentfill}{rgb}{0.000000,0.000000,0.000000}%
\pgfsetfillcolor{currentfill}%
\pgfsetlinewidth{0.803000pt}%
\definecolor{currentstroke}{rgb}{0.000000,0.000000,0.000000}%
\pgfsetstrokecolor{currentstroke}%
\pgfsetdash{}{0pt}%
\pgfsys@defobject{currentmarker}{\pgfqpoint{0.000000in}{-0.048611in}}{\pgfqpoint{0.000000in}{0.000000in}}{%
\pgfpathmoveto{\pgfqpoint{0.000000in}{0.000000in}}%
\pgfpathlineto{\pgfqpoint{0.000000in}{-0.048611in}}%
\pgfusepath{stroke,fill}%
}%
\begin{pgfscope}%
\pgfsys@transformshift{2.069525in}{0.499691in}%
\pgfsys@useobject{currentmarker}{}%
\end{pgfscope}%
\end{pgfscope}%
\begin{pgfscope}%
\definecolor{textcolor}{rgb}{0.000000,0.000000,0.000000}%
\pgfsetstrokecolor{textcolor}%
\pgfsetfillcolor{textcolor}%
\pgftext[x=2.069525in,y=0.402469in,,top]{\color{textcolor}\rmfamily\fontsize{10.000000}{12.000000}\selectfont \(\displaystyle {10^{2}}\)}%
\end{pgfscope}%
\begin{pgfscope}%
\pgfpathrectangle{\pgfqpoint{2.069525in}{0.499691in}}{\pgfqpoint{2.856580in}{2.917676in}}%
\pgfusepath{clip}%
\pgfsetrectcap%
\pgfsetroundjoin%
\pgfsetlinewidth{0.803000pt}%
\definecolor{currentstroke}{rgb}{0.690196,0.690196,0.690196}%
\pgfsetstrokecolor{currentstroke}%
\pgfsetdash{}{0pt}%
\pgfpathmoveto{\pgfqpoint{3.056794in}{0.499691in}}%
\pgfpathlineto{\pgfqpoint{3.056794in}{3.417367in}}%
\pgfusepath{stroke}%
\end{pgfscope}%
\begin{pgfscope}%
\pgfsetbuttcap%
\pgfsetroundjoin%
\definecolor{currentfill}{rgb}{0.000000,0.000000,0.000000}%
\pgfsetfillcolor{currentfill}%
\pgfsetlinewidth{0.803000pt}%
\definecolor{currentstroke}{rgb}{0.000000,0.000000,0.000000}%
\pgfsetstrokecolor{currentstroke}%
\pgfsetdash{}{0pt}%
\pgfsys@defobject{currentmarker}{\pgfqpoint{0.000000in}{-0.048611in}}{\pgfqpoint{0.000000in}{0.000000in}}{%
\pgfpathmoveto{\pgfqpoint{0.000000in}{0.000000in}}%
\pgfpathlineto{\pgfqpoint{0.000000in}{-0.048611in}}%
\pgfusepath{stroke,fill}%
}%
\begin{pgfscope}%
\pgfsys@transformshift{3.056794in}{0.499691in}%
\pgfsys@useobject{currentmarker}{}%
\end{pgfscope}%
\end{pgfscope}%
\begin{pgfscope}%
\definecolor{textcolor}{rgb}{0.000000,0.000000,0.000000}%
\pgfsetstrokecolor{textcolor}%
\pgfsetfillcolor{textcolor}%
\pgftext[x=3.056794in,y=0.402469in,,top]{\color{textcolor}\rmfamily\fontsize{10.000000}{12.000000}\selectfont \(\displaystyle {10^{3}}\)}%
\end{pgfscope}%
\begin{pgfscope}%
\pgfpathrectangle{\pgfqpoint{2.069525in}{0.499691in}}{\pgfqpoint{2.856580in}{2.917676in}}%
\pgfusepath{clip}%
\pgfsetrectcap%
\pgfsetroundjoin%
\pgfsetlinewidth{0.803000pt}%
\definecolor{currentstroke}{rgb}{0.690196,0.690196,0.690196}%
\pgfsetstrokecolor{currentstroke}%
\pgfsetdash{}{0pt}%
\pgfpathmoveto{\pgfqpoint{4.044063in}{0.499691in}}%
\pgfpathlineto{\pgfqpoint{4.044063in}{3.417367in}}%
\pgfusepath{stroke}%
\end{pgfscope}%
\begin{pgfscope}%
\pgfsetbuttcap%
\pgfsetroundjoin%
\definecolor{currentfill}{rgb}{0.000000,0.000000,0.000000}%
\pgfsetfillcolor{currentfill}%
\pgfsetlinewidth{0.803000pt}%
\definecolor{currentstroke}{rgb}{0.000000,0.000000,0.000000}%
\pgfsetstrokecolor{currentstroke}%
\pgfsetdash{}{0pt}%
\pgfsys@defobject{currentmarker}{\pgfqpoint{0.000000in}{-0.048611in}}{\pgfqpoint{0.000000in}{0.000000in}}{%
\pgfpathmoveto{\pgfqpoint{0.000000in}{0.000000in}}%
\pgfpathlineto{\pgfqpoint{0.000000in}{-0.048611in}}%
\pgfusepath{stroke,fill}%
}%
\begin{pgfscope}%
\pgfsys@transformshift{4.044063in}{0.499691in}%
\pgfsys@useobject{currentmarker}{}%
\end{pgfscope}%
\end{pgfscope}%
\begin{pgfscope}%
\definecolor{textcolor}{rgb}{0.000000,0.000000,0.000000}%
\pgfsetstrokecolor{textcolor}%
\pgfsetfillcolor{textcolor}%
\pgftext[x=4.044063in,y=0.402469in,,top]{\color{textcolor}\rmfamily\fontsize{10.000000}{12.000000}\selectfont \(\displaystyle {10^{4}}\)}%
\end{pgfscope}%
\begin{pgfscope}%
\definecolor{textcolor}{rgb}{0.000000,0.000000,0.000000}%
\pgfsetstrokecolor{textcolor}%
\pgfsetfillcolor{textcolor}%
\pgftext[x=3.497815in,y=0.223457in,,top]{\color{textcolor}\rmfamily\fontsize{10.000000}{12.000000}\selectfont Count}%
\end{pgfscope}%
\begin{pgfscope}%
\pgfsetbuttcap%
\pgfsetroundjoin%
\definecolor{currentfill}{rgb}{0.000000,0.000000,0.000000}%
\pgfsetfillcolor{currentfill}%
\pgfsetlinewidth{0.803000pt}%
\definecolor{currentstroke}{rgb}{0.000000,0.000000,0.000000}%
\pgfsetstrokecolor{currentstroke}%
\pgfsetdash{}{0pt}%
\pgfsys@defobject{currentmarker}{\pgfqpoint{-0.048611in}{0.000000in}}{\pgfqpoint{0.000000in}{0.000000in}}{%
\pgfpathmoveto{\pgfqpoint{0.000000in}{0.000000in}}%
\pgfpathlineto{\pgfqpoint{-0.048611in}{0.000000in}}%
\pgfusepath{stroke,fill}%
}%
\begin{pgfscope}%
\pgfsys@transformshift{2.069525in}{0.680981in}%
\pgfsys@useobject{currentmarker}{}%
\end{pgfscope}%
\end{pgfscope}%
\begin{pgfscope}%
\definecolor{textcolor}{rgb}{0.000000,0.000000,0.000000}%
\pgfsetstrokecolor{textcolor}%
\pgfsetfillcolor{textcolor}%
\pgftext[x=1.010882in, y=0.632756in, left, base]{\color{textcolor}\rmfamily\fontsize{10.000000}{12.000000}\selectfont Unchecked send}%
\end{pgfscope}%
\begin{pgfscope}%
\pgfsetbuttcap%
\pgfsetroundjoin%
\definecolor{currentfill}{rgb}{0.000000,0.000000,0.000000}%
\pgfsetfillcolor{currentfill}%
\pgfsetlinewidth{0.803000pt}%
\definecolor{currentstroke}{rgb}{0.000000,0.000000,0.000000}%
\pgfsetstrokecolor{currentstroke}%
\pgfsetdash{}{0pt}%
\pgfsys@defobject{currentmarker}{\pgfqpoint{-0.048611in}{0.000000in}}{\pgfqpoint{0.000000in}{0.000000in}}{%
\pgfpathmoveto{\pgfqpoint{0.000000in}{0.000000in}}%
\pgfpathlineto{\pgfqpoint{-0.048611in}{0.000000in}}%
\pgfusepath{stroke,fill}%
}%
\begin{pgfscope}%
\pgfsys@transformshift{2.069525in}{0.802652in}%
\pgfsys@useobject{currentmarker}{}%
\end{pgfscope}%
\end{pgfscope}%
\begin{pgfscope}%
\definecolor{textcolor}{rgb}{0.000000,0.000000,0.000000}%
\pgfsetstrokecolor{textcolor}%
\pgfsetfillcolor{textcolor}%
\pgftext[x=0.741204in, y=0.754427in, left, base]{\color{textcolor}\rmfamily\fontsize{10.000000}{12.000000}\selectfont Unprotected Suicide}%
\end{pgfscope}%
\begin{pgfscope}%
\pgfsetbuttcap%
\pgfsetroundjoin%
\definecolor{currentfill}{rgb}{0.000000,0.000000,0.000000}%
\pgfsetfillcolor{currentfill}%
\pgfsetlinewidth{0.803000pt}%
\definecolor{currentstroke}{rgb}{0.000000,0.000000,0.000000}%
\pgfsetstrokecolor{currentstroke}%
\pgfsetdash{}{0pt}%
\pgfsys@defobject{currentmarker}{\pgfqpoint{-0.048611in}{0.000000in}}{\pgfqpoint{0.000000in}{0.000000in}}{%
\pgfpathmoveto{\pgfqpoint{0.000000in}{0.000000in}}%
\pgfpathlineto{\pgfqpoint{-0.048611in}{0.000000in}}%
\pgfusepath{stroke,fill}%
}%
\begin{pgfscope}%
\pgfsys@transformshift{2.069525in}{0.924324in}%
\pgfsys@useobject{currentmarker}{}%
\end{pgfscope}%
\end{pgfscope}%
\begin{pgfscope}%
\definecolor{textcolor}{rgb}{0.000000,0.000000,0.000000}%
\pgfsetstrokecolor{textcolor}%
\pgfsetfillcolor{textcolor}%
\pgftext[x=1.283258in, y=0.876098in, left, base]{\color{textcolor}\rmfamily\fontsize{10.000000}{12.000000}\selectfont Reentrancy}%
\end{pgfscope}%
\begin{pgfscope}%
\pgfsetbuttcap%
\pgfsetroundjoin%
\definecolor{currentfill}{rgb}{0.000000,0.000000,0.000000}%
\pgfsetfillcolor{currentfill}%
\pgfsetlinewidth{0.803000pt}%
\definecolor{currentstroke}{rgb}{0.000000,0.000000,0.000000}%
\pgfsetstrokecolor{currentstroke}%
\pgfsetdash{}{0pt}%
\pgfsys@defobject{currentmarker}{\pgfqpoint{-0.048611in}{0.000000in}}{\pgfqpoint{0.000000in}{0.000000in}}{%
\pgfpathmoveto{\pgfqpoint{0.000000in}{0.000000in}}%
\pgfpathlineto{\pgfqpoint{-0.048611in}{0.000000in}}%
\pgfusepath{stroke,fill}%
}%
\begin{pgfscope}%
\pgfsys@transformshift{2.069525in}{1.045995in}%
\pgfsys@useobject{currentmarker}{}%
\end{pgfscope}%
\end{pgfscope}%
\begin{pgfscope}%
\definecolor{textcolor}{rgb}{0.000000,0.000000,0.000000}%
\pgfsetstrokecolor{textcolor}%
\pgfsetfillcolor{textcolor}%
\pgftext[x=0.100000in, y=0.997770in, left, base]{\color{textcolor}\rmfamily\fontsize{10.000000}{12.000000}\selectfont Integer overflow and underflow}%
\end{pgfscope}%
\begin{pgfscope}%
\pgfsetbuttcap%
\pgfsetroundjoin%
\definecolor{currentfill}{rgb}{0.000000,0.000000,0.000000}%
\pgfsetfillcolor{currentfill}%
\pgfsetlinewidth{0.803000pt}%
\definecolor{currentstroke}{rgb}{0.000000,0.000000,0.000000}%
\pgfsetstrokecolor{currentstroke}%
\pgfsetdash{}{0pt}%
\pgfsys@defobject{currentmarker}{\pgfqpoint{-0.048611in}{0.000000in}}{\pgfqpoint{0.000000in}{0.000000in}}{%
\pgfpathmoveto{\pgfqpoint{0.000000in}{0.000000in}}%
\pgfpathlineto{\pgfqpoint{-0.048611in}{0.000000in}}%
\pgfusepath{stroke,fill}%
}%
\begin{pgfscope}%
\pgfsys@transformshift{2.069525in}{1.167666in}%
\pgfsys@useobject{currentmarker}{}%
\end{pgfscope}%
\end{pgfscope}%
\begin{pgfscope}%
\definecolor{textcolor}{rgb}{0.000000,0.000000,0.000000}%
\pgfsetstrokecolor{textcolor}%
\pgfsetfillcolor{textcolor}%
\pgftext[x=1.202625in, y=1.119441in, left, base]{\color{textcolor}\rmfamily\fontsize{10.000000}{12.000000}\selectfont DelegateCall}%
\end{pgfscope}%
\begin{pgfscope}%
\pgfsetbuttcap%
\pgfsetroundjoin%
\definecolor{currentfill}{rgb}{0.000000,0.000000,0.000000}%
\pgfsetfillcolor{currentfill}%
\pgfsetlinewidth{0.803000pt}%
\definecolor{currentstroke}{rgb}{0.000000,0.000000,0.000000}%
\pgfsetstrokecolor{currentstroke}%
\pgfsetdash{}{0pt}%
\pgfsys@defobject{currentmarker}{\pgfqpoint{-0.048611in}{0.000000in}}{\pgfqpoint{0.000000in}{0.000000in}}{%
\pgfpathmoveto{\pgfqpoint{0.000000in}{0.000000in}}%
\pgfpathlineto{\pgfqpoint{-0.048611in}{0.000000in}}%
\pgfusepath{stroke,fill}%
}%
\begin{pgfscope}%
\pgfsys@transformshift{2.069525in}{1.289337in}%
\pgfsys@useobject{currentmarker}{}%
\end{pgfscope}%
\end{pgfscope}%
\begin{pgfscope}%
\definecolor{textcolor}{rgb}{0.000000,0.000000,0.000000}%
\pgfsetstrokecolor{textcolor}%
\pgfsetfillcolor{textcolor}%
\pgftext[x=1.404400in, y=1.241112in, left, base]{\color{textcolor}\rmfamily\fontsize{10.000000}{12.000000}\selectfont Nest Call}%
\end{pgfscope}%
\begin{pgfscope}%
\pgfsetbuttcap%
\pgfsetroundjoin%
\definecolor{currentfill}{rgb}{0.000000,0.000000,0.000000}%
\pgfsetfillcolor{currentfill}%
\pgfsetlinewidth{0.803000pt}%
\definecolor{currentstroke}{rgb}{0.000000,0.000000,0.000000}%
\pgfsetstrokecolor{currentstroke}%
\pgfsetdash{}{0pt}%
\pgfsys@defobject{currentmarker}{\pgfqpoint{-0.048611in}{0.000000in}}{\pgfqpoint{0.000000in}{0.000000in}}{%
\pgfpathmoveto{\pgfqpoint{0.000000in}{0.000000in}}%
\pgfpathlineto{\pgfqpoint{-0.048611in}{0.000000in}}%
\pgfusepath{stroke,fill}%
}%
\begin{pgfscope}%
\pgfsys@transformshift{2.069525in}{1.532680in}%
\pgfsys@useobject{currentmarker}{}%
\end{pgfscope}%
\end{pgfscope}%
\begin{pgfscope}%
\definecolor{textcolor}{rgb}{0.000000,0.000000,0.000000}%
\pgfsetstrokecolor{textcolor}%
\pgfsetfillcolor{textcolor}%
\pgftext[x=0.254321in, y=1.484454in, left, base]{\color{textcolor}\rmfamily\fontsize{10.000000}{12.000000}\selectfont Erroneous constructor name}%
\end{pgfscope}%
\begin{pgfscope}%
\pgfsetbuttcap%
\pgfsetroundjoin%
\definecolor{currentfill}{rgb}{0.000000,0.000000,0.000000}%
\pgfsetfillcolor{currentfill}%
\pgfsetlinewidth{0.803000pt}%
\definecolor{currentstroke}{rgb}{0.000000,0.000000,0.000000}%
\pgfsetstrokecolor{currentstroke}%
\pgfsetdash{}{0pt}%
\pgfsys@defobject{currentmarker}{\pgfqpoint{-0.048611in}{0.000000in}}{\pgfqpoint{0.000000in}{0.000000in}}{%
\pgfpathmoveto{\pgfqpoint{0.000000in}{0.000000in}}%
\pgfpathlineto{\pgfqpoint{-0.048611in}{0.000000in}}%
\pgfusepath{stroke,fill}%
}%
\begin{pgfscope}%
\pgfsys@transformshift{2.069525in}{1.654351in}%
\pgfsys@useobject{currentmarker}{}%
\end{pgfscope}%
\end{pgfscope}%
\begin{pgfscope}%
\definecolor{textcolor}{rgb}{0.000000,0.000000,0.000000}%
\pgfsetstrokecolor{textcolor}%
\pgfsetfillcolor{textcolor}%
\pgftext[x=0.293287in, y=1.606126in, left, base]{\color{textcolor}\rmfamily\fontsize{10.000000}{12.000000}\selectfont Leaking to arbitary address}%
\end{pgfscope}%
\begin{pgfscope}%
\pgfsetbuttcap%
\pgfsetroundjoin%
\definecolor{currentfill}{rgb}{0.000000,0.000000,0.000000}%
\pgfsetfillcolor{currentfill}%
\pgfsetlinewidth{0.803000pt}%
\definecolor{currentstroke}{rgb}{0.000000,0.000000,0.000000}%
\pgfsetstrokecolor{currentstroke}%
\pgfsetdash{}{0pt}%
\pgfsys@defobject{currentmarker}{\pgfqpoint{-0.048611in}{0.000000in}}{\pgfqpoint{0.000000in}{0.000000in}}{%
\pgfpathmoveto{\pgfqpoint{0.000000in}{0.000000in}}%
\pgfpathlineto{\pgfqpoint{-0.048611in}{0.000000in}}%
\pgfusepath{stroke,fill}%
}%
\begin{pgfscope}%
\pgfsys@transformshift{2.069525in}{1.776022in}%
\pgfsys@useobject{currentmarker}{}%
\end{pgfscope}%
\end{pgfscope}%
\begin{pgfscope}%
\definecolor{textcolor}{rgb}{0.000000,0.000000,0.000000}%
\pgfsetstrokecolor{textcolor}%
\pgfsetfillcolor{textcolor}%
\pgftext[x=0.969214in, y=1.727797in, left, base]{\color{textcolor}\rmfamily\fontsize{10.000000}{12.000000}\selectfont Bad randomness}%
\end{pgfscope}%
\begin{pgfscope}%
\pgfsetbuttcap%
\pgfsetroundjoin%
\definecolor{currentfill}{rgb}{0.000000,0.000000,0.000000}%
\pgfsetfillcolor{currentfill}%
\pgfsetlinewidth{0.803000pt}%
\definecolor{currentstroke}{rgb}{0.000000,0.000000,0.000000}%
\pgfsetstrokecolor{currentstroke}%
\pgfsetdash{}{0pt}%
\pgfsys@defobject{currentmarker}{\pgfqpoint{-0.048611in}{0.000000in}}{\pgfqpoint{0.000000in}{0.000000in}}{%
\pgfpathmoveto{\pgfqpoint{0.000000in}{0.000000in}}%
\pgfpathlineto{\pgfqpoint{-0.048611in}{0.000000in}}%
\pgfusepath{stroke,fill}%
}%
\begin{pgfscope}%
\pgfsys@transformshift{2.069525in}{1.897693in}%
\pgfsys@useobject{currentmarker}{}%
\end{pgfscope}%
\end{pgfscope}%
\begin{pgfscope}%
\definecolor{textcolor}{rgb}{0.000000,0.000000,0.000000}%
\pgfsetstrokecolor{textcolor}%
\pgfsetfillcolor{textcolor}%
\pgftext[x=0.137808in, y=1.849468in, left, base]{\color{textcolor}\rmfamily\fontsize{10.000000}{12.000000}\selectfont Transaction order dependency}%
\end{pgfscope}%
\begin{pgfscope}%
\pgfsetbuttcap%
\pgfsetroundjoin%
\definecolor{currentfill}{rgb}{0.000000,0.000000,0.000000}%
\pgfsetfillcolor{currentfill}%
\pgfsetlinewidth{0.803000pt}%
\definecolor{currentstroke}{rgb}{0.000000,0.000000,0.000000}%
\pgfsetstrokecolor{currentstroke}%
\pgfsetdash{}{0pt}%
\pgfsys@defobject{currentmarker}{\pgfqpoint{-0.048611in}{0.000000in}}{\pgfqpoint{0.000000in}{0.000000in}}{%
\pgfpathmoveto{\pgfqpoint{0.000000in}{0.000000in}}%
\pgfpathlineto{\pgfqpoint{-0.048611in}{0.000000in}}%
\pgfusepath{stroke,fill}%
}%
\begin{pgfscope}%
\pgfsys@transformshift{2.069525in}{2.019365in}%
\pgfsys@useobject{currentmarker}{}%
\end{pgfscope}%
\end{pgfscope}%
\begin{pgfscope}%
\definecolor{textcolor}{rgb}{0.000000,0.000000,0.000000}%
\pgfsetstrokecolor{textcolor}%
\pgfsetfillcolor{textcolor}%
\pgftext[x=0.647067in, y=1.971139in, left, base]{\color{textcolor}\rmfamily\fontsize{10.000000}{12.000000}\selectfont Balance manipulation}%
\end{pgfscope}%
\begin{pgfscope}%
\pgfsetbuttcap%
\pgfsetroundjoin%
\definecolor{currentfill}{rgb}{0.000000,0.000000,0.000000}%
\pgfsetfillcolor{currentfill}%
\pgfsetlinewidth{0.803000pt}%
\definecolor{currentstroke}{rgb}{0.000000,0.000000,0.000000}%
\pgfsetstrokecolor{currentstroke}%
\pgfsetdash{}{0pt}%
\pgfsys@defobject{currentmarker}{\pgfqpoint{-0.048611in}{0.000000in}}{\pgfqpoint{0.000000in}{0.000000in}}{%
\pgfpathmoveto{\pgfqpoint{0.000000in}{0.000000in}}%
\pgfpathlineto{\pgfqpoint{-0.048611in}{0.000000in}}%
\pgfusepath{stroke,fill}%
}%
\begin{pgfscope}%
\pgfsys@transformshift{2.069525in}{2.141036in}%
\pgfsys@useobject{currentmarker}{}%
\end{pgfscope}%
\end{pgfscope}%
\begin{pgfscope}%
\definecolor{textcolor}{rgb}{0.000000,0.000000,0.000000}%
\pgfsetstrokecolor{textcolor}%
\pgfsetfillcolor{textcolor}%
\pgftext[x=0.717284in, y=2.092810in, left, base]{\color{textcolor}\rmfamily\fontsize{10.000000}{12.000000}\selectfont Uninitialized storage}%
\end{pgfscope}%
\begin{pgfscope}%
\pgfsetbuttcap%
\pgfsetroundjoin%
\definecolor{currentfill}{rgb}{0.000000,0.000000,0.000000}%
\pgfsetfillcolor{currentfill}%
\pgfsetlinewidth{0.803000pt}%
\definecolor{currentstroke}{rgb}{0.000000,0.000000,0.000000}%
\pgfsetstrokecolor{currentstroke}%
\pgfsetdash{}{0pt}%
\pgfsys@defobject{currentmarker}{\pgfqpoint{-0.048611in}{0.000000in}}{\pgfqpoint{0.000000in}{0.000000in}}{%
\pgfpathmoveto{\pgfqpoint{0.000000in}{0.000000in}}%
\pgfpathlineto{\pgfqpoint{-0.048611in}{0.000000in}}%
\pgfusepath{stroke,fill}%
}%
\begin{pgfscope}%
\pgfsys@transformshift{2.069525in}{2.262707in}%
\pgfsys@useobject{currentmarker}{}%
\end{pgfscope}%
\end{pgfscope}%
\begin{pgfscope}%
\definecolor{textcolor}{rgb}{0.000000,0.000000,0.000000}%
\pgfsetstrokecolor{textcolor}%
\pgfsetfillcolor{textcolor}%
\pgftext[x=0.387808in, y=2.214482in, left, base]{\color{textcolor}\rmfamily\fontsize{10.000000}{12.000000}\selectfont Dependency of timestamp}%
\end{pgfscope}%
\begin{pgfscope}%
\pgfsetbuttcap%
\pgfsetroundjoin%
\definecolor{currentfill}{rgb}{0.000000,0.000000,0.000000}%
\pgfsetfillcolor{currentfill}%
\pgfsetlinewidth{0.803000pt}%
\definecolor{currentstroke}{rgb}{0.000000,0.000000,0.000000}%
\pgfsetstrokecolor{currentstroke}%
\pgfsetdash{}{0pt}%
\pgfsys@defobject{currentmarker}{\pgfqpoint{-0.048611in}{0.000000in}}{\pgfqpoint{0.000000in}{0.000000in}}{%
\pgfpathmoveto{\pgfqpoint{0.000000in}{0.000000in}}%
\pgfpathlineto{\pgfqpoint{-0.048611in}{0.000000in}}%
\pgfusepath{stroke,fill}%
}%
\begin{pgfscope}%
\pgfsys@transformshift{2.069525in}{2.384378in}%
\pgfsys@useobject{currentmarker}{}%
\end{pgfscope}%
\end{pgfscope}%
\begin{pgfscope}%
\definecolor{textcolor}{rgb}{0.000000,0.000000,0.000000}%
\pgfsetstrokecolor{textcolor}%
\pgfsetfillcolor{textcolor}%
\pgftext[x=0.521683in, y=2.336153in, left, base]{\color{textcolor}\rmfamily\fontsize{10.000000}{12.000000}\selectfont Exceed authority access}%
\end{pgfscope}%
\begin{pgfscope}%
\pgfsetbuttcap%
\pgfsetroundjoin%
\definecolor{currentfill}{rgb}{0.000000,0.000000,0.000000}%
\pgfsetfillcolor{currentfill}%
\pgfsetlinewidth{0.803000pt}%
\definecolor{currentstroke}{rgb}{0.000000,0.000000,0.000000}%
\pgfsetstrokecolor{currentstroke}%
\pgfsetdash{}{0pt}%
\pgfsys@defobject{currentmarker}{\pgfqpoint{-0.048611in}{0.000000in}}{\pgfqpoint{0.000000in}{0.000000in}}{%
\pgfpathmoveto{\pgfqpoint{0.000000in}{0.000000in}}%
\pgfpathlineto{\pgfqpoint{-0.048611in}{0.000000in}}%
\pgfusepath{stroke,fill}%
}%
\begin{pgfscope}%
\pgfsys@transformshift{2.069525in}{2.506049in}%
\pgfsys@useobject{currentmarker}{}%
\end{pgfscope}%
\end{pgfscope}%
\begin{pgfscope}%
\definecolor{textcolor}{rgb}{0.000000,0.000000,0.000000}%
\pgfsetstrokecolor{textcolor}%
\pgfsetfillcolor{textcolor}%
\pgftext[x=1.180634in, y=2.457824in, left, base]{\color{textcolor}\rmfamily\fontsize{10.000000}{12.000000}\selectfont Frozen Ether}%
\end{pgfscope}%
\begin{pgfscope}%
\pgfsetbuttcap%
\pgfsetroundjoin%
\definecolor{currentfill}{rgb}{0.000000,0.000000,0.000000}%
\pgfsetfillcolor{currentfill}%
\pgfsetlinewidth{0.803000pt}%
\definecolor{currentstroke}{rgb}{0.000000,0.000000,0.000000}%
\pgfsetstrokecolor{currentstroke}%
\pgfsetdash{}{0pt}%
\pgfsys@defobject{currentmarker}{\pgfqpoint{-0.048611in}{0.000000in}}{\pgfqpoint{0.000000in}{0.000000in}}{%
\pgfpathmoveto{\pgfqpoint{0.000000in}{0.000000in}}%
\pgfpathlineto{\pgfqpoint{-0.048611in}{0.000000in}}%
\pgfusepath{stroke,fill}%
}%
\begin{pgfscope}%
\pgfsys@transformshift{2.069525in}{2.627721in}%
\pgfsys@useobject{currentmarker}{}%
\end{pgfscope}%
\end{pgfscope}%
\begin{pgfscope}%
\definecolor{textcolor}{rgb}{0.000000,0.000000,0.000000}%
\pgfsetstrokecolor{textcolor}%
\pgfsetfillcolor{textcolor}%
\pgftext[x=1.412502in, y=2.579495in, left, base]{\color{textcolor}\rmfamily\fontsize{10.000000}{12.000000}\selectfont TxOrigin}%
\end{pgfscope}%
\begin{pgfscope}%
\pgfsetbuttcap%
\pgfsetroundjoin%
\definecolor{currentfill}{rgb}{0.000000,0.000000,0.000000}%
\pgfsetfillcolor{currentfill}%
\pgfsetlinewidth{0.803000pt}%
\definecolor{currentstroke}{rgb}{0.000000,0.000000,0.000000}%
\pgfsetstrokecolor{currentstroke}%
\pgfsetdash{}{0pt}%
\pgfsys@defobject{currentmarker}{\pgfqpoint{-0.048611in}{0.000000in}}{\pgfqpoint{0.000000in}{0.000000in}}{%
\pgfpathmoveto{\pgfqpoint{0.000000in}{0.000000in}}%
\pgfpathlineto{\pgfqpoint{-0.048611in}{0.000000in}}%
\pgfusepath{stroke,fill}%
}%
\begin{pgfscope}%
\pgfsys@transformshift{2.069525in}{2.871063in}%
\pgfsys@useobject{currentmarker}{}%
\end{pgfscope}%
\end{pgfscope}%
\begin{pgfscope}%
\definecolor{textcolor}{rgb}{0.000000,0.000000,0.000000}%
\pgfsetstrokecolor{textcolor}%
\pgfsetfillcolor{textcolor}%
\pgftext[x=0.662501in, y=2.822838in, left, base]{\color{textcolor}\rmfamily\fontsize{10.000000}{12.000000}\selectfont Unused state variable}%
\end{pgfscope}%
\begin{pgfscope}%
\pgfsetbuttcap%
\pgfsetroundjoin%
\definecolor{currentfill}{rgb}{0.000000,0.000000,0.000000}%
\pgfsetfillcolor{currentfill}%
\pgfsetlinewidth{0.803000pt}%
\definecolor{currentstroke}{rgb}{0.000000,0.000000,0.000000}%
\pgfsetstrokecolor{currentstroke}%
\pgfsetdash{}{0pt}%
\pgfsys@defobject{currentmarker}{\pgfqpoint{-0.048611in}{0.000000in}}{\pgfqpoint{0.000000in}{0.000000in}}{%
\pgfpathmoveto{\pgfqpoint{0.000000in}{0.000000in}}%
\pgfpathlineto{\pgfqpoint{-0.048611in}{0.000000in}}%
\pgfusepath{stroke,fill}%
}%
\begin{pgfscope}%
\pgfsys@transformshift{2.069525in}{2.992734in}%
\pgfsys@useobject{currentmarker}{}%
\end{pgfscope}%
\end{pgfscope}%
\begin{pgfscope}%
\definecolor{textcolor}{rgb}{0.000000,0.000000,0.000000}%
\pgfsetstrokecolor{textcolor}%
\pgfsetfillcolor{textcolor}%
\pgftext[x=0.441436in, y=2.944509in, left, base]{\color{textcolor}\rmfamily\fontsize{10.000000}{12.000000}\selectfont Missing return statement}%
\end{pgfscope}%
\begin{pgfscope}%
\pgfsetbuttcap%
\pgfsetroundjoin%
\definecolor{currentfill}{rgb}{0.000000,0.000000,0.000000}%
\pgfsetfillcolor{currentfill}%
\pgfsetlinewidth{0.803000pt}%
\definecolor{currentstroke}{rgb}{0.000000,0.000000,0.000000}%
\pgfsetstrokecolor{currentstroke}%
\pgfsetdash{}{0pt}%
\pgfsys@defobject{currentmarker}{\pgfqpoint{-0.048611in}{0.000000in}}{\pgfqpoint{0.000000in}{0.000000in}}{%
\pgfpathmoveto{\pgfqpoint{0.000000in}{0.000000in}}%
\pgfpathlineto{\pgfqpoint{-0.048611in}{0.000000in}}%
\pgfusepath{stroke,fill}%
}%
\begin{pgfscope}%
\pgfsys@transformshift{2.069525in}{3.114405in}%
\pgfsys@useobject{currentmarker}{}%
\end{pgfscope}%
\end{pgfscope}%
\begin{pgfscope}%
\definecolor{textcolor}{rgb}{0.000000,0.000000,0.000000}%
\pgfsetstrokecolor{textcolor}%
\pgfsetfillcolor{textcolor}%
\pgftext[x=0.932177in, y=3.066180in, left, base]{\color{textcolor}\rmfamily\fontsize{10.000000}{12.000000}\selectfont Redefine variable}%
\end{pgfscope}%
\begin{pgfscope}%
\pgfsetbuttcap%
\pgfsetroundjoin%
\definecolor{currentfill}{rgb}{0.000000,0.000000,0.000000}%
\pgfsetfillcolor{currentfill}%
\pgfsetlinewidth{0.803000pt}%
\definecolor{currentstroke}{rgb}{0.000000,0.000000,0.000000}%
\pgfsetstrokecolor{currentstroke}%
\pgfsetdash{}{0pt}%
\pgfsys@defobject{currentmarker}{\pgfqpoint{-0.048611in}{0.000000in}}{\pgfqpoint{0.000000in}{0.000000in}}{%
\pgfpathmoveto{\pgfqpoint{0.000000in}{0.000000in}}%
\pgfpathlineto{\pgfqpoint{-0.048611in}{0.000000in}}%
\pgfusepath{stroke,fill}%
}%
\begin{pgfscope}%
\pgfsys@transformshift{2.069525in}{3.236077in}%
\pgfsys@useobject{currentmarker}{}%
\end{pgfscope}%
\end{pgfscope}%
\begin{pgfscope}%
\definecolor{textcolor}{rgb}{0.000000,0.000000,0.000000}%
\pgfsetstrokecolor{textcolor}%
\pgfsetfillcolor{textcolor}%
\pgftext[x=0.826468in, y=3.187851in, left, base]{\color{textcolor}\rmfamily\fontsize{10.000000}{12.000000}\selectfont Useless assignment}%
\end{pgfscope}%
\begin{pgfscope}%
\pgfsetbuttcap%
\pgfsetroundjoin%
\definecolor{currentfill}{rgb}{0.000000,0.000000,0.000000}%
\pgfsetfillcolor{currentfill}%
\pgfsetlinewidth{0.803000pt}%
\definecolor{currentstroke}{rgb}{0.000000,0.000000,0.000000}%
\pgfsetstrokecolor{currentstroke}%
\pgfsetdash{}{0pt}%
\pgfsys@defobject{currentmarker}{\pgfqpoint{-0.138889in}{0.000000in}}{\pgfqpoint{0.000000in}{0.000000in}}{%
\pgfpathmoveto{\pgfqpoint{0.000000in}{0.000000in}}%
\pgfpathlineto{\pgfqpoint{-0.138889in}{0.000000in}}%
\pgfusepath{stroke,fill}%
}%
\begin{pgfscope}%
\pgfsys@transformshift{2.069525in}{0.499691in}%
\pgfsys@useobject{currentmarker}{}%
\end{pgfscope}%
\end{pgfscope}%
\begin{pgfscope}%
\pgfsetbuttcap%
\pgfsetroundjoin%
\definecolor{currentfill}{rgb}{0.000000,0.000000,0.000000}%
\pgfsetfillcolor{currentfill}%
\pgfsetlinewidth{0.803000pt}%
\definecolor{currentstroke}{rgb}{0.000000,0.000000,0.000000}%
\pgfsetstrokecolor{currentstroke}%
\pgfsetdash{}{0pt}%
\pgfsys@defobject{currentmarker}{\pgfqpoint{-0.138889in}{0.000000in}}{\pgfqpoint{0.000000in}{0.000000in}}{%
\pgfpathmoveto{\pgfqpoint{0.000000in}{0.000000in}}%
\pgfpathlineto{\pgfqpoint{-0.138889in}{0.000000in}}%
\pgfusepath{stroke,fill}%
}%
\begin{pgfscope}%
\pgfsys@transformshift{2.069525in}{1.411008in}%
\pgfsys@useobject{currentmarker}{}%
\end{pgfscope}%
\end{pgfscope}%
\begin{pgfscope}%
\pgfsetbuttcap%
\pgfsetroundjoin%
\definecolor{currentfill}{rgb}{0.000000,0.000000,0.000000}%
\pgfsetfillcolor{currentfill}%
\pgfsetlinewidth{0.803000pt}%
\definecolor{currentstroke}{rgb}{0.000000,0.000000,0.000000}%
\pgfsetstrokecolor{currentstroke}%
\pgfsetdash{}{0pt}%
\pgfsys@defobject{currentmarker}{\pgfqpoint{-0.138889in}{0.000000in}}{\pgfqpoint{0.000000in}{0.000000in}}{%
\pgfpathmoveto{\pgfqpoint{0.000000in}{0.000000in}}%
\pgfpathlineto{\pgfqpoint{-0.138889in}{0.000000in}}%
\pgfusepath{stroke,fill}%
}%
\begin{pgfscope}%
\pgfsys@transformshift{2.069525in}{3.417367in}%
\pgfsys@useobject{currentmarker}{}%
\end{pgfscope}%
\end{pgfscope}%
\begin{pgfscope}%
\pgfsetbuttcap%
\pgfsetroundjoin%
\definecolor{currentfill}{rgb}{0.000000,0.000000,0.000000}%
\pgfsetfillcolor{currentfill}%
\pgfsetlinewidth{0.803000pt}%
\definecolor{currentstroke}{rgb}{0.000000,0.000000,0.000000}%
\pgfsetstrokecolor{currentstroke}%
\pgfsetdash{}{0pt}%
\pgfsys@defobject{currentmarker}{\pgfqpoint{-0.138889in}{0.000000in}}{\pgfqpoint{0.000000in}{0.000000in}}{%
\pgfpathmoveto{\pgfqpoint{0.000000in}{0.000000in}}%
\pgfpathlineto{\pgfqpoint{-0.138889in}{0.000000in}}%
\pgfusepath{stroke,fill}%
}%
\begin{pgfscope}%
\pgfsys@transformshift{2.069525in}{2.749392in}%
\pgfsys@useobject{currentmarker}{}%
\end{pgfscope}%
\end{pgfscope}%
\begin{pgfscope}%
\pgfpathrectangle{\pgfqpoint{2.069525in}{0.499691in}}{\pgfqpoint{2.856580in}{2.917676in}}%
\pgfusepath{clip}%
\pgfsetbuttcap%
\pgfsetmiterjoin%
\definecolor{currentfill}{rgb}{0.121569,0.466667,0.705882}%
\pgfsetfillcolor{currentfill}%
\pgfsetlinewidth{0.000000pt}%
\definecolor{currentstroke}{rgb}{0.000000,0.000000,0.000000}%
\pgfsetstrokecolor{currentstroke}%
\pgfsetstrokeopacity{0.000000}%
\pgfsetdash{}{0pt}%
\pgfpathmoveto{\pgfqpoint{-0.704782in}{0.632313in}}%
\pgfpathlineto{\pgfqpoint{3.184833in}{0.632313in}}%
\pgfpathlineto{\pgfqpoint{3.184833in}{0.729650in}}%
\pgfpathlineto{\pgfqpoint{-0.704782in}{0.729650in}}%
\pgfpathclose%
\pgfusepath{fill}%
\end{pgfscope}%
\begin{pgfscope}%
\pgfpathrectangle{\pgfqpoint{2.069525in}{0.499691in}}{\pgfqpoint{2.856580in}{2.917676in}}%
\pgfusepath{clip}%
\pgfsetbuttcap%
\pgfsetmiterjoin%
\definecolor{currentfill}{rgb}{0.121569,0.466667,0.705882}%
\pgfsetfillcolor{currentfill}%
\pgfsetlinewidth{0.000000pt}%
\definecolor{currentstroke}{rgb}{0.000000,0.000000,0.000000}%
\pgfsetstrokecolor{currentstroke}%
\pgfsetstrokeopacity{0.000000}%
\pgfsetdash{}{0pt}%
\pgfpathmoveto{\pgfqpoint{-0.704782in}{0.753984in}}%
\pgfpathlineto{\pgfqpoint{3.078937in}{0.753984in}}%
\pgfpathlineto{\pgfqpoint{3.078937in}{0.851321in}}%
\pgfpathlineto{\pgfqpoint{-0.704782in}{0.851321in}}%
\pgfpathclose%
\pgfusepath{fill}%
\end{pgfscope}%
\begin{pgfscope}%
\pgfpathrectangle{\pgfqpoint{2.069525in}{0.499691in}}{\pgfqpoint{2.856580in}{2.917676in}}%
\pgfusepath{clip}%
\pgfsetbuttcap%
\pgfsetmiterjoin%
\definecolor{currentfill}{rgb}{0.121569,0.466667,0.705882}%
\pgfsetfillcolor{currentfill}%
\pgfsetlinewidth{0.000000pt}%
\definecolor{currentstroke}{rgb}{0.000000,0.000000,0.000000}%
\pgfsetstrokecolor{currentstroke}%
\pgfsetstrokeopacity{0.000000}%
\pgfsetdash{}{0pt}%
\pgfpathmoveto{\pgfqpoint{-0.704782in}{0.875655in}}%
\pgfpathlineto{\pgfqpoint{3.170275in}{0.875655in}}%
\pgfpathlineto{\pgfqpoint{3.170275in}{0.972992in}}%
\pgfpathlineto{\pgfqpoint{-0.704782in}{0.972992in}}%
\pgfpathclose%
\pgfusepath{fill}%
\end{pgfscope}%
\begin{pgfscope}%
\pgfpathrectangle{\pgfqpoint{2.069525in}{0.499691in}}{\pgfqpoint{2.856580in}{2.917676in}}%
\pgfusepath{clip}%
\pgfsetbuttcap%
\pgfsetmiterjoin%
\definecolor{currentfill}{rgb}{0.121569,0.466667,0.705882}%
\pgfsetfillcolor{currentfill}%
\pgfsetlinewidth{0.000000pt}%
\definecolor{currentstroke}{rgb}{0.000000,0.000000,0.000000}%
\pgfsetstrokecolor{currentstroke}%
\pgfsetstrokeopacity{0.000000}%
\pgfsetdash{}{0pt}%
\pgfpathmoveto{\pgfqpoint{-0.704782in}{0.997326in}}%
\pgfpathlineto{\pgfqpoint{4.905185in}{0.997326in}}%
\pgfpathlineto{\pgfqpoint{4.905185in}{1.094663in}}%
\pgfpathlineto{\pgfqpoint{-0.704782in}{1.094663in}}%
\pgfpathclose%
\pgfusepath{fill}%
\end{pgfscope}%
\begin{pgfscope}%
\pgfpathrectangle{\pgfqpoint{2.069525in}{0.499691in}}{\pgfqpoint{2.856580in}{2.917676in}}%
\pgfusepath{clip}%
\pgfsetbuttcap%
\pgfsetmiterjoin%
\definecolor{currentfill}{rgb}{0.121569,0.466667,0.705882}%
\pgfsetfillcolor{currentfill}%
\pgfsetlinewidth{0.000000pt}%
\definecolor{currentstroke}{rgb}{0.000000,0.000000,0.000000}%
\pgfsetstrokecolor{currentstroke}%
\pgfsetstrokeopacity{0.000000}%
\pgfsetdash{}{0pt}%
\pgfpathmoveto{\pgfqpoint{-0.704782in}{1.118998in}}%
\pgfpathlineto{\pgfqpoint{3.700450in}{1.118998in}}%
\pgfpathlineto{\pgfqpoint{3.700450in}{1.216335in}}%
\pgfpathlineto{\pgfqpoint{-0.704782in}{1.216335in}}%
\pgfpathclose%
\pgfusepath{fill}%
\end{pgfscope}%
\begin{pgfscope}%
\pgfpathrectangle{\pgfqpoint{2.069525in}{0.499691in}}{\pgfqpoint{2.856580in}{2.917676in}}%
\pgfusepath{clip}%
\pgfsetbuttcap%
\pgfsetmiterjoin%
\definecolor{currentfill}{rgb}{0.121569,0.466667,0.705882}%
\pgfsetfillcolor{currentfill}%
\pgfsetlinewidth{0.000000pt}%
\definecolor{currentstroke}{rgb}{0.000000,0.000000,0.000000}%
\pgfsetstrokecolor{currentstroke}%
\pgfsetstrokeopacity{0.000000}%
\pgfsetdash{}{0pt}%
\pgfpathmoveto{\pgfqpoint{-0.704782in}{1.240669in}}%
\pgfpathlineto{\pgfqpoint{4.554779in}{1.240669in}}%
\pgfpathlineto{\pgfqpoint{4.554779in}{1.338006in}}%
\pgfpathlineto{\pgfqpoint{-0.704782in}{1.338006in}}%
\pgfpathclose%
\pgfusepath{fill}%
\end{pgfscope}%
\begin{pgfscope}%
\pgfpathrectangle{\pgfqpoint{2.069525in}{0.499691in}}{\pgfqpoint{2.856580in}{2.917676in}}%
\pgfusepath{clip}%
\pgfsetbuttcap%
\pgfsetmiterjoin%
\definecolor{currentfill}{rgb}{1.000000,0.498039,0.054902}%
\pgfsetfillcolor{currentfill}%
\pgfsetlinewidth{0.000000pt}%
\definecolor{currentstroke}{rgb}{0.000000,0.000000,0.000000}%
\pgfsetstrokecolor{currentstroke}%
\pgfsetstrokeopacity{0.000000}%
\pgfsetdash{}{0pt}%
\pgfpathmoveto{\pgfqpoint{-0.704782in}{1.484011in}}%
\pgfpathlineto{\pgfqpoint{4.645781in}{1.484011in}}%
\pgfpathlineto{\pgfqpoint{4.645781in}{1.581348in}}%
\pgfpathlineto{\pgfqpoint{-0.704782in}{1.581348in}}%
\pgfpathclose%
\pgfusepath{fill}%
\end{pgfscope}%
\begin{pgfscope}%
\pgfpathrectangle{\pgfqpoint{2.069525in}{0.499691in}}{\pgfqpoint{2.856580in}{2.917676in}}%
\pgfusepath{clip}%
\pgfsetbuttcap%
\pgfsetmiterjoin%
\definecolor{currentfill}{rgb}{1.000000,0.498039,0.054902}%
\pgfsetfillcolor{currentfill}%
\pgfsetlinewidth{0.000000pt}%
\definecolor{currentstroke}{rgb}{0.000000,0.000000,0.000000}%
\pgfsetstrokecolor{currentstroke}%
\pgfsetstrokeopacity{0.000000}%
\pgfsetdash{}{0pt}%
\pgfpathmoveto{\pgfqpoint{-0.704782in}{1.605682in}}%
\pgfpathlineto{\pgfqpoint{4.545307in}{1.605682in}}%
\pgfpathlineto{\pgfqpoint{4.545307in}{1.703019in}}%
\pgfpathlineto{\pgfqpoint{-0.704782in}{1.703019in}}%
\pgfpathclose%
\pgfusepath{fill}%
\end{pgfscope}%
\begin{pgfscope}%
\pgfpathrectangle{\pgfqpoint{2.069525in}{0.499691in}}{\pgfqpoint{2.856580in}{2.917676in}}%
\pgfusepath{clip}%
\pgfsetbuttcap%
\pgfsetmiterjoin%
\definecolor{currentfill}{rgb}{1.000000,0.498039,0.054902}%
\pgfsetfillcolor{currentfill}%
\pgfsetlinewidth{0.000000pt}%
\definecolor{currentstroke}{rgb}{0.000000,0.000000,0.000000}%
\pgfsetstrokecolor{currentstroke}%
\pgfsetstrokeopacity{0.000000}%
\pgfsetdash{}{0pt}%
\pgfpathmoveto{\pgfqpoint{-0.704782in}{1.727354in}}%
\pgfpathlineto{\pgfqpoint{4.266708in}{1.727354in}}%
\pgfpathlineto{\pgfqpoint{4.266708in}{1.824691in}}%
\pgfpathlineto{\pgfqpoint{-0.704782in}{1.824691in}}%
\pgfpathclose%
\pgfusepath{fill}%
\end{pgfscope}%
\begin{pgfscope}%
\pgfpathrectangle{\pgfqpoint{2.069525in}{0.499691in}}{\pgfqpoint{2.856580in}{2.917676in}}%
\pgfusepath{clip}%
\pgfsetbuttcap%
\pgfsetmiterjoin%
\definecolor{currentfill}{rgb}{1.000000,0.498039,0.054902}%
\pgfsetfillcolor{currentfill}%
\pgfsetlinewidth{0.000000pt}%
\definecolor{currentstroke}{rgb}{0.000000,0.000000,0.000000}%
\pgfsetstrokecolor{currentstroke}%
\pgfsetstrokeopacity{0.000000}%
\pgfsetdash{}{0pt}%
\pgfpathmoveto{\pgfqpoint{-0.704782in}{1.849025in}}%
\pgfpathlineto{\pgfqpoint{3.994771in}{1.849025in}}%
\pgfpathlineto{\pgfqpoint{3.994771in}{1.946362in}}%
\pgfpathlineto{\pgfqpoint{-0.704782in}{1.946362in}}%
\pgfpathclose%
\pgfusepath{fill}%
\end{pgfscope}%
\begin{pgfscope}%
\pgfpathrectangle{\pgfqpoint{2.069525in}{0.499691in}}{\pgfqpoint{2.856580in}{2.917676in}}%
\pgfusepath{clip}%
\pgfsetbuttcap%
\pgfsetmiterjoin%
\definecolor{currentfill}{rgb}{1.000000,0.498039,0.054902}%
\pgfsetfillcolor{currentfill}%
\pgfsetlinewidth{0.000000pt}%
\definecolor{currentstroke}{rgb}{0.000000,0.000000,0.000000}%
\pgfsetstrokecolor{currentstroke}%
\pgfsetstrokeopacity{0.000000}%
\pgfsetdash{}{0pt}%
\pgfpathmoveto{\pgfqpoint{-0.704782in}{1.970696in}}%
\pgfpathlineto{\pgfqpoint{4.191352in}{1.970696in}}%
\pgfpathlineto{\pgfqpoint{4.191352in}{2.068033in}}%
\pgfpathlineto{\pgfqpoint{-0.704782in}{2.068033in}}%
\pgfpathclose%
\pgfusepath{fill}%
\end{pgfscope}%
\begin{pgfscope}%
\pgfpathrectangle{\pgfqpoint{2.069525in}{0.499691in}}{\pgfqpoint{2.856580in}{2.917676in}}%
\pgfusepath{clip}%
\pgfsetbuttcap%
\pgfsetmiterjoin%
\definecolor{currentfill}{rgb}{1.000000,0.498039,0.054902}%
\pgfsetfillcolor{currentfill}%
\pgfsetlinewidth{0.000000pt}%
\definecolor{currentstroke}{rgb}{0.000000,0.000000,0.000000}%
\pgfsetstrokecolor{currentstroke}%
\pgfsetstrokeopacity{0.000000}%
\pgfsetdash{}{0pt}%
\pgfpathmoveto{\pgfqpoint{-0.704782in}{2.092367in}}%
\pgfpathlineto{\pgfqpoint{3.907537in}{2.092367in}}%
\pgfpathlineto{\pgfqpoint{3.907537in}{2.189704in}}%
\pgfpathlineto{\pgfqpoint{-0.704782in}{2.189704in}}%
\pgfpathclose%
\pgfusepath{fill}%
\end{pgfscope}%
\begin{pgfscope}%
\pgfpathrectangle{\pgfqpoint{2.069525in}{0.499691in}}{\pgfqpoint{2.856580in}{2.917676in}}%
\pgfusepath{clip}%
\pgfsetbuttcap%
\pgfsetmiterjoin%
\definecolor{currentfill}{rgb}{1.000000,0.498039,0.054902}%
\pgfsetfillcolor{currentfill}%
\pgfsetlinewidth{0.000000pt}%
\definecolor{currentstroke}{rgb}{0.000000,0.000000,0.000000}%
\pgfsetstrokecolor{currentstroke}%
\pgfsetstrokeopacity{0.000000}%
\pgfsetdash{}{0pt}%
\pgfpathmoveto{\pgfqpoint{-0.704782in}{2.214038in}}%
\pgfpathlineto{\pgfqpoint{4.491822in}{2.214038in}}%
\pgfpathlineto{\pgfqpoint{4.491822in}{2.311375in}}%
\pgfpathlineto{\pgfqpoint{-0.704782in}{2.311375in}}%
\pgfpathclose%
\pgfusepath{fill}%
\end{pgfscope}%
\begin{pgfscope}%
\pgfpathrectangle{\pgfqpoint{2.069525in}{0.499691in}}{\pgfqpoint{2.856580in}{2.917676in}}%
\pgfusepath{clip}%
\pgfsetbuttcap%
\pgfsetmiterjoin%
\definecolor{currentfill}{rgb}{1.000000,0.498039,0.054902}%
\pgfsetfillcolor{currentfill}%
\pgfsetlinewidth{0.000000pt}%
\definecolor{currentstroke}{rgb}{0.000000,0.000000,0.000000}%
\pgfsetstrokecolor{currentstroke}%
\pgfsetstrokeopacity{0.000000}%
\pgfsetdash{}{0pt}%
\pgfpathmoveto{\pgfqpoint{-0.704782in}{2.335710in}}%
\pgfpathlineto{\pgfqpoint{4.883850in}{2.335710in}}%
\pgfpathlineto{\pgfqpoint{4.883850in}{2.433047in}}%
\pgfpathlineto{\pgfqpoint{-0.704782in}{2.433047in}}%
\pgfpathclose%
\pgfusepath{fill}%
\end{pgfscope}%
\begin{pgfscope}%
\pgfpathrectangle{\pgfqpoint{2.069525in}{0.499691in}}{\pgfqpoint{2.856580in}{2.917676in}}%
\pgfusepath{clip}%
\pgfsetbuttcap%
\pgfsetmiterjoin%
\definecolor{currentfill}{rgb}{1.000000,0.498039,0.054902}%
\pgfsetfillcolor{currentfill}%
\pgfsetlinewidth{0.000000pt}%
\definecolor{currentstroke}{rgb}{0.000000,0.000000,0.000000}%
\pgfsetstrokecolor{currentstroke}%
\pgfsetstrokeopacity{0.000000}%
\pgfsetdash{}{0pt}%
\pgfpathmoveto{\pgfqpoint{-0.704782in}{2.457381in}}%
\pgfpathlineto{\pgfqpoint{4.288734in}{2.457381in}}%
\pgfpathlineto{\pgfqpoint{4.288734in}{2.554718in}}%
\pgfpathlineto{\pgfqpoint{-0.704782in}{2.554718in}}%
\pgfpathclose%
\pgfusepath{fill}%
\end{pgfscope}%
\begin{pgfscope}%
\pgfpathrectangle{\pgfqpoint{2.069525in}{0.499691in}}{\pgfqpoint{2.856580in}{2.917676in}}%
\pgfusepath{clip}%
\pgfsetbuttcap%
\pgfsetmiterjoin%
\definecolor{currentfill}{rgb}{1.000000,0.498039,0.054902}%
\pgfsetfillcolor{currentfill}%
\pgfsetlinewidth{0.000000pt}%
\definecolor{currentstroke}{rgb}{0.000000,0.000000,0.000000}%
\pgfsetstrokecolor{currentstroke}%
\pgfsetstrokeopacity{0.000000}%
\pgfsetdash{}{0pt}%
\pgfpathmoveto{\pgfqpoint{-0.704782in}{2.579052in}}%
\pgfpathlineto{\pgfqpoint{2.188564in}{2.579052in}}%
\pgfpathlineto{\pgfqpoint{2.188564in}{2.676389in}}%
\pgfpathlineto{\pgfqpoint{-0.704782in}{2.676389in}}%
\pgfpathclose%
\pgfusepath{fill}%
\end{pgfscope}%
\begin{pgfscope}%
\pgfpathrectangle{\pgfqpoint{2.069525in}{0.499691in}}{\pgfqpoint{2.856580in}{2.917676in}}%
\pgfusepath{clip}%
\pgfsetbuttcap%
\pgfsetmiterjoin%
\definecolor{currentfill}{rgb}{0.172549,0.627451,0.172549}%
\pgfsetfillcolor{currentfill}%
\pgfsetlinewidth{0.000000pt}%
\definecolor{currentstroke}{rgb}{0.000000,0.000000,0.000000}%
\pgfsetstrokecolor{currentstroke}%
\pgfsetstrokeopacity{0.000000}%
\pgfsetdash{}{0pt}%
\pgfpathmoveto{\pgfqpoint{-0.704782in}{2.822395in}}%
\pgfpathlineto{\pgfqpoint{4.887037in}{2.822395in}}%
\pgfpathlineto{\pgfqpoint{4.887037in}{2.919732in}}%
\pgfpathlineto{\pgfqpoint{-0.704782in}{2.919732in}}%
\pgfpathclose%
\pgfusepath{fill}%
\end{pgfscope}%
\begin{pgfscope}%
\pgfpathrectangle{\pgfqpoint{2.069525in}{0.499691in}}{\pgfqpoint{2.856580in}{2.917676in}}%
\pgfusepath{clip}%
\pgfsetbuttcap%
\pgfsetmiterjoin%
\definecolor{currentfill}{rgb}{0.172549,0.627451,0.172549}%
\pgfsetfillcolor{currentfill}%
\pgfsetlinewidth{0.000000pt}%
\definecolor{currentstroke}{rgb}{0.000000,0.000000,0.000000}%
\pgfsetstrokecolor{currentstroke}%
\pgfsetstrokeopacity{0.000000}%
\pgfsetdash{}{0pt}%
\pgfpathmoveto{\pgfqpoint{-0.704782in}{2.944066in}}%
\pgfpathlineto{\pgfqpoint{3.610045in}{2.944066in}}%
\pgfpathlineto{\pgfqpoint{3.610045in}{3.041403in}}%
\pgfpathlineto{\pgfqpoint{-0.704782in}{3.041403in}}%
\pgfpathclose%
\pgfusepath{fill}%
\end{pgfscope}%
\begin{pgfscope}%
\pgfpathrectangle{\pgfqpoint{2.069525in}{0.499691in}}{\pgfqpoint{2.856580in}{2.917676in}}%
\pgfusepath{clip}%
\pgfsetbuttcap%
\pgfsetmiterjoin%
\definecolor{currentfill}{rgb}{0.172549,0.627451,0.172549}%
\pgfsetfillcolor{currentfill}%
\pgfsetlinewidth{0.000000pt}%
\definecolor{currentstroke}{rgb}{0.000000,0.000000,0.000000}%
\pgfsetstrokecolor{currentstroke}%
\pgfsetstrokeopacity{0.000000}%
\pgfsetdash{}{0pt}%
\pgfpathmoveto{\pgfqpoint{-0.704782in}{3.065737in}}%
\pgfpathlineto{\pgfqpoint{3.351842in}{3.065737in}}%
\pgfpathlineto{\pgfqpoint{3.351842in}{3.163074in}}%
\pgfpathlineto{\pgfqpoint{-0.704782in}{3.163074in}}%
\pgfpathclose%
\pgfusepath{fill}%
\end{pgfscope}%
\begin{pgfscope}%
\pgfpathrectangle{\pgfqpoint{2.069525in}{0.499691in}}{\pgfqpoint{2.856580in}{2.917676in}}%
\pgfusepath{clip}%
\pgfsetbuttcap%
\pgfsetmiterjoin%
\definecolor{currentfill}{rgb}{0.172549,0.627451,0.172549}%
\pgfsetfillcolor{currentfill}%
\pgfsetlinewidth{0.000000pt}%
\definecolor{currentstroke}{rgb}{0.000000,0.000000,0.000000}%
\pgfsetstrokecolor{currentstroke}%
\pgfsetstrokeopacity{0.000000}%
\pgfsetdash{}{0pt}%
\pgfpathmoveto{\pgfqpoint{-0.704782in}{3.187408in}}%
\pgfpathlineto{\pgfqpoint{2.523069in}{3.187408in}}%
\pgfpathlineto{\pgfqpoint{2.523069in}{3.284745in}}%
\pgfpathlineto{\pgfqpoint{-0.704782in}{3.284745in}}%
\pgfpathclose%
\pgfusepath{fill}%
\end{pgfscope}%
\begin{pgfscope}%
\pgfsetrectcap%
\pgfsetmiterjoin%
\pgfsetlinewidth{0.803000pt}%
\definecolor{currentstroke}{rgb}{0.000000,0.000000,0.000000}%
\pgfsetstrokecolor{currentstroke}%
\pgfsetdash{}{0pt}%
\pgfpathmoveto{\pgfqpoint{2.069525in}{0.499691in}}%
\pgfpathlineto{\pgfqpoint{2.069525in}{3.417367in}}%
\pgfusepath{stroke}%
\end{pgfscope}%
\begin{pgfscope}%
\pgfsetrectcap%
\pgfsetmiterjoin%
\pgfsetlinewidth{0.803000pt}%
\definecolor{currentstroke}{rgb}{0.000000,0.000000,0.000000}%
\pgfsetstrokecolor{currentstroke}%
\pgfsetdash{}{0pt}%
\pgfpathmoveto{\pgfqpoint{2.069525in}{0.499691in}}%
\pgfpathlineto{\pgfqpoint{4.926105in}{0.499691in}}%
\pgfusepath{stroke}%
\end{pgfscope}%
\begin{pgfscope}%
\pgfsetbuttcap%
\pgfsetmiterjoin%
\definecolor{currentfill}{rgb}{0.121569,0.466667,0.705882}%
\pgfsetfillcolor{currentfill}%
\pgfsetlinewidth{1.003750pt}%
\definecolor{currentstroke}{rgb}{0.121569,0.466667,0.705882}%
\pgfsetstrokecolor{currentstroke}%
\pgfsetdash{}{0pt}%
\pgfpathmoveto{\pgfqpoint{2.116834in}{3.632796in}}%
\pgfpathlineto{\pgfqpoint{2.394611in}{3.632796in}}%
\pgfpathlineto{\pgfqpoint{2.394611in}{3.730018in}}%
\pgfpathlineto{\pgfqpoint{2.116834in}{3.730018in}}%
\pgfpathclose%
\pgfusepath{stroke,fill}%
\end{pgfscope}%
\begin{pgfscope}%
\definecolor{textcolor}{rgb}{0.000000,0.000000,0.000000}%
\pgfsetstrokecolor{textcolor}%
\pgfsetfillcolor{textcolor}%
\pgftext[x=2.505723in,y=3.632796in,left,base]{\color{textcolor}\rmfamily\fontsize{10.000000}{12.000000}\selectfont High}%
\end{pgfscope}%
\begin{pgfscope}%
\pgfsetbuttcap%
\pgfsetmiterjoin%
\definecolor{currentfill}{rgb}{1.000000,0.498039,0.054902}%
\pgfsetfillcolor{currentfill}%
\pgfsetlinewidth{1.003750pt}%
\definecolor{currentstroke}{rgb}{1.000000,0.498039,0.054902}%
\pgfsetstrokecolor{currentstroke}%
\pgfsetdash{}{0pt}%
\pgfpathmoveto{\pgfqpoint{3.072853in}{3.632796in}}%
\pgfpathlineto{\pgfqpoint{3.350631in}{3.632796in}}%
\pgfpathlineto{\pgfqpoint{3.350631in}{3.730018in}}%
\pgfpathlineto{\pgfqpoint{3.072853in}{3.730018in}}%
\pgfpathclose%
\pgfusepath{stroke,fill}%
\end{pgfscope}%
\begin{pgfscope}%
\definecolor{textcolor}{rgb}{0.000000,0.000000,0.000000}%
\pgfsetstrokecolor{textcolor}%
\pgfsetfillcolor{textcolor}%
\pgftext[x=3.461742in,y=3.632796in,left,base]{\color{textcolor}\rmfamily\fontsize{10.000000}{12.000000}\selectfont Medium}%
\end{pgfscope}%
\begin{pgfscope}%
\pgfsetbuttcap%
\pgfsetmiterjoin%
\definecolor{currentfill}{rgb}{0.172549,0.627451,0.172549}%
\pgfsetfillcolor{currentfill}%
\pgfsetlinewidth{1.003750pt}%
\definecolor{currentstroke}{rgb}{0.172549,0.627451,0.172549}%
\pgfsetstrokecolor{currentstroke}%
\pgfsetdash{}{0pt}%
\pgfpathmoveto{\pgfqpoint{4.237206in}{3.632796in}}%
\pgfpathlineto{\pgfqpoint{4.514984in}{3.632796in}}%
\pgfpathlineto{\pgfqpoint{4.514984in}{3.730018in}}%
\pgfpathlineto{\pgfqpoint{4.237206in}{3.730018in}}%
\pgfpathclose%
\pgfusepath{stroke,fill}%
\end{pgfscope}%
\begin{pgfscope}%
\definecolor{textcolor}{rgb}{0.000000,0.000000,0.000000}%
\pgfsetstrokecolor{textcolor}%
\pgfsetfillcolor{textcolor}%
\pgftext[x=4.626095in,y=3.632796in,left,base]{\color{textcolor}\rmfamily\fontsize{10.000000}{12.000000}\selectfont Low}%
\end{pgfscope}%
\end{pgfpicture}%
\makeatother%
\endgroup%

    \caption{Histogram \small x of verified \acrshortpl{sc} on Ethereum}
\end{figure}

\subsubsection{Plain text}
\label{sec:verified-smart-contracts-plain-text}
For easy use of the dataset for casual language modeling training, a "plain\_text" version of both the raw, the flattened, and the inflated dataset is made available. This is done through a custom builder script for the dataset, a feature of the Dataset library by Hugging Face.

\subsubsection{Parsed}
\label{sec:verified-smart-contracts-parsed}

\subsection{Verified Smart Contracts Audit}
\label{sec:verified-smart-contracts-audit}
\url{https://github.com/andstor/verified-smart-contracts-audit}
\url{https://huggingface.co/datasets/andstor/smart_contracts_audit}

Subsets:
\subsubsection{SoliDetector}
\label{sec:verified-smart-contracts-audit-solidector}



\subsection{Smart Contract Comments}
\label{sec:verified-smart-contracts-comments}

\url{https://huggingface.co/datasets/andstor/smart_contract_comments}
See \cref{sec:code-comment-clustering} for more information.


\begin{figure}[ht]
    \centering
    %% Creator: Matplotlib, PGF backend
%%
%% To include the figure in your LaTeX document, write
%%   \input{<filename>.pgf}
%%
%% Make sure the required packages are loaded in your preamble
%%   \usepackage{pgf}
%%
%% and, on pdftex
%%   \usepackage[utf8]{inputenc}\DeclareUnicodeCharacter{2212}{-}
%%
%% or, on luatex and xetex
%%   \usepackage{unicode-math}
%%
%% Figures using additional raster images can only be included by \input if
%% they are in the same directory as the main LaTeX file. For loading figures
%% from other directories you can use the `import` package
%%   \usepackage{import}
%%
%% and then include the figures with
%%   \import{<path to file>}{<filename>.pgf}
%%
%% Matplotlib used the following preamble
%%
\begingroup%
\makeatletter%
\begin{pgfpicture}%
\pgfpathrectangle{\pgfpointorigin}{\pgfqpoint{5.159114in}{5.020022in}}%
\pgfusepath{use as bounding box, clip}%
\begin{pgfscope}%
\pgfsetbuttcap%
\pgfsetmiterjoin%
\definecolor{currentfill}{rgb}{1.000000,1.000000,1.000000}%
\pgfsetfillcolor{currentfill}%
\pgfsetlinewidth{0.000000pt}%
\definecolor{currentstroke}{rgb}{1.000000,1.000000,1.000000}%
\pgfsetstrokecolor{currentstroke}%
\pgfsetdash{}{0pt}%
\pgfpathmoveto{\pgfqpoint{0.000000in}{0.000000in}}%
\pgfpathlineto{\pgfqpoint{5.159114in}{0.000000in}}%
\pgfpathlineto{\pgfqpoint{5.159114in}{5.020022in}}%
\pgfpathlineto{\pgfqpoint{0.000000in}{5.020022in}}%
\pgfpathclose%
\pgfusepath{fill}%
\end{pgfscope}%
\begin{pgfscope}%
\pgfsetbuttcap%
\pgfsetmiterjoin%
\definecolor{currentfill}{rgb}{1.000000,1.000000,1.000000}%
\pgfsetfillcolor{currentfill}%
\pgfsetlinewidth{0.000000pt}%
\definecolor{currentstroke}{rgb}{0.000000,0.000000,0.000000}%
\pgfsetstrokecolor{currentstroke}%
\pgfsetstrokeopacity{0.000000}%
\pgfsetdash{}{0pt}%
\pgfpathmoveto{\pgfqpoint{0.692593in}{0.499691in}}%
\pgfpathlineto{\pgfqpoint{4.310567in}{0.499691in}}%
\pgfpathlineto{\pgfqpoint{4.310567in}{4.162251in}}%
\pgfpathlineto{\pgfqpoint{0.692593in}{4.162251in}}%
\pgfpathclose%
\pgfusepath{fill}%
\end{pgfscope}%
\begin{pgfscope}%
\pgfsys@transformshift{0.889167in}{0.500022in}%
\pgftext[left,bottom]{\includegraphics[interpolate=true,width=3.170000in,height=3.662500in]{figures/2d_cluster_marginals-img0.png}}%
\end{pgfscope}%
\begin{pgfscope}%
\pgfpathrectangle{\pgfqpoint{0.692593in}{0.499691in}}{\pgfqpoint{3.617974in}{3.662560in}}%
\pgfusepath{clip}%
\pgfsetbuttcap%
\pgfsetroundjoin%
\definecolor{currentfill}{rgb}{0.121569,0.466667,0.705882}%
\pgfsetfillcolor{currentfill}%
\pgfsetlinewidth{1.003750pt}%
\definecolor{currentstroke}{rgb}{0.121569,0.466667,0.705882}%
\pgfsetstrokecolor{currentstroke}%
\pgfsetdash{}{0pt}%
\pgfsys@defobject{currentmarker}{\pgfqpoint{-0.041667in}{-0.041667in}}{\pgfqpoint{0.041667in}{0.041667in}}{%
\pgfpathmoveto{\pgfqpoint{0.000000in}{-0.041667in}}%
\pgfpathcurveto{\pgfqpoint{0.011050in}{-0.041667in}}{\pgfqpoint{0.021649in}{-0.037276in}}{\pgfqpoint{0.029463in}{-0.029463in}}%
\pgfpathcurveto{\pgfqpoint{0.037276in}{-0.021649in}}{\pgfqpoint{0.041667in}{-0.011050in}}{\pgfqpoint{0.041667in}{0.000000in}}%
\pgfpathcurveto{\pgfqpoint{0.041667in}{0.011050in}}{\pgfqpoint{0.037276in}{0.021649in}}{\pgfqpoint{0.029463in}{0.029463in}}%
\pgfpathcurveto{\pgfqpoint{0.021649in}{0.037276in}}{\pgfqpoint{0.011050in}{0.041667in}}{\pgfqpoint{0.000000in}{0.041667in}}%
\pgfpathcurveto{\pgfqpoint{-0.011050in}{0.041667in}}{\pgfqpoint{-0.021649in}{0.037276in}}{\pgfqpoint{-0.029463in}{0.029463in}}%
\pgfpathcurveto{\pgfqpoint{-0.037276in}{0.021649in}}{\pgfqpoint{-0.041667in}{0.011050in}}{\pgfqpoint{-0.041667in}{0.000000in}}%
\pgfpathcurveto{\pgfqpoint{-0.041667in}{-0.011050in}}{\pgfqpoint{-0.037276in}{-0.021649in}}{\pgfqpoint{-0.029463in}{-0.029463in}}%
\pgfpathcurveto{\pgfqpoint{-0.021649in}{-0.037276in}}{\pgfqpoint{-0.011050in}{-0.041667in}}{\pgfqpoint{0.000000in}{-0.041667in}}%
\pgfpathclose%
\pgfusepath{stroke,fill}%
}%
\end{pgfscope}%
\begin{pgfscope}%
\pgfpathrectangle{\pgfqpoint{0.692593in}{0.499691in}}{\pgfqpoint{3.617974in}{3.662560in}}%
\pgfusepath{clip}%
\pgfsetbuttcap%
\pgfsetroundjoin%
\definecolor{currentfill}{rgb}{1.000000,0.498039,0.054902}%
\pgfsetfillcolor{currentfill}%
\pgfsetlinewidth{1.003750pt}%
\definecolor{currentstroke}{rgb}{1.000000,0.498039,0.054902}%
\pgfsetstrokecolor{currentstroke}%
\pgfsetdash{}{0pt}%
\pgfsys@defobject{currentmarker}{\pgfqpoint{-0.041667in}{-0.041667in}}{\pgfqpoint{0.041667in}{0.041667in}}{%
\pgfpathmoveto{\pgfqpoint{0.000000in}{-0.041667in}}%
\pgfpathcurveto{\pgfqpoint{0.011050in}{-0.041667in}}{\pgfqpoint{0.021649in}{-0.037276in}}{\pgfqpoint{0.029463in}{-0.029463in}}%
\pgfpathcurveto{\pgfqpoint{0.037276in}{-0.021649in}}{\pgfqpoint{0.041667in}{-0.011050in}}{\pgfqpoint{0.041667in}{0.000000in}}%
\pgfpathcurveto{\pgfqpoint{0.041667in}{0.011050in}}{\pgfqpoint{0.037276in}{0.021649in}}{\pgfqpoint{0.029463in}{0.029463in}}%
\pgfpathcurveto{\pgfqpoint{0.021649in}{0.037276in}}{\pgfqpoint{0.011050in}{0.041667in}}{\pgfqpoint{0.000000in}{0.041667in}}%
\pgfpathcurveto{\pgfqpoint{-0.011050in}{0.041667in}}{\pgfqpoint{-0.021649in}{0.037276in}}{\pgfqpoint{-0.029463in}{0.029463in}}%
\pgfpathcurveto{\pgfqpoint{-0.037276in}{0.021649in}}{\pgfqpoint{-0.041667in}{0.011050in}}{\pgfqpoint{-0.041667in}{0.000000in}}%
\pgfpathcurveto{\pgfqpoint{-0.041667in}{-0.011050in}}{\pgfqpoint{-0.037276in}{-0.021649in}}{\pgfqpoint{-0.029463in}{-0.029463in}}%
\pgfpathcurveto{\pgfqpoint{-0.021649in}{-0.037276in}}{\pgfqpoint{-0.011050in}{-0.041667in}}{\pgfqpoint{0.000000in}{-0.041667in}}%
\pgfpathclose%
\pgfusepath{stroke,fill}%
}%
\end{pgfscope}%
\begin{pgfscope}%
\pgfpathrectangle{\pgfqpoint{0.692593in}{0.499691in}}{\pgfqpoint{3.617974in}{3.662560in}}%
\pgfusepath{clip}%
\pgfsetbuttcap%
\pgfsetroundjoin%
\definecolor{currentfill}{rgb}{0.172549,0.627451,0.172549}%
\pgfsetfillcolor{currentfill}%
\pgfsetlinewidth{1.003750pt}%
\definecolor{currentstroke}{rgb}{0.172549,0.627451,0.172549}%
\pgfsetstrokecolor{currentstroke}%
\pgfsetdash{}{0pt}%
\pgfsys@defobject{currentmarker}{\pgfqpoint{-0.041667in}{-0.041667in}}{\pgfqpoint{0.041667in}{0.041667in}}{%
\pgfpathmoveto{\pgfqpoint{0.000000in}{-0.041667in}}%
\pgfpathcurveto{\pgfqpoint{0.011050in}{-0.041667in}}{\pgfqpoint{0.021649in}{-0.037276in}}{\pgfqpoint{0.029463in}{-0.029463in}}%
\pgfpathcurveto{\pgfqpoint{0.037276in}{-0.021649in}}{\pgfqpoint{0.041667in}{-0.011050in}}{\pgfqpoint{0.041667in}{0.000000in}}%
\pgfpathcurveto{\pgfqpoint{0.041667in}{0.011050in}}{\pgfqpoint{0.037276in}{0.021649in}}{\pgfqpoint{0.029463in}{0.029463in}}%
\pgfpathcurveto{\pgfqpoint{0.021649in}{0.037276in}}{\pgfqpoint{0.011050in}{0.041667in}}{\pgfqpoint{0.000000in}{0.041667in}}%
\pgfpathcurveto{\pgfqpoint{-0.011050in}{0.041667in}}{\pgfqpoint{-0.021649in}{0.037276in}}{\pgfqpoint{-0.029463in}{0.029463in}}%
\pgfpathcurveto{\pgfqpoint{-0.037276in}{0.021649in}}{\pgfqpoint{-0.041667in}{0.011050in}}{\pgfqpoint{-0.041667in}{0.000000in}}%
\pgfpathcurveto{\pgfqpoint{-0.041667in}{-0.011050in}}{\pgfqpoint{-0.037276in}{-0.021649in}}{\pgfqpoint{-0.029463in}{-0.029463in}}%
\pgfpathcurveto{\pgfqpoint{-0.021649in}{-0.037276in}}{\pgfqpoint{-0.011050in}{-0.041667in}}{\pgfqpoint{0.000000in}{-0.041667in}}%
\pgfpathclose%
\pgfusepath{stroke,fill}%
}%
\end{pgfscope}%
\begin{pgfscope}%
\pgfpathrectangle{\pgfqpoint{0.692593in}{0.499691in}}{\pgfqpoint{3.617974in}{3.662560in}}%
\pgfusepath{clip}%
\pgfsetbuttcap%
\pgfsetroundjoin%
\definecolor{currentfill}{rgb}{0.839216,0.152941,0.156863}%
\pgfsetfillcolor{currentfill}%
\pgfsetlinewidth{1.003750pt}%
\definecolor{currentstroke}{rgb}{0.839216,0.152941,0.156863}%
\pgfsetstrokecolor{currentstroke}%
\pgfsetdash{}{0pt}%
\pgfsys@defobject{currentmarker}{\pgfqpoint{-0.041667in}{-0.041667in}}{\pgfqpoint{0.041667in}{0.041667in}}{%
\pgfpathmoveto{\pgfqpoint{0.000000in}{-0.041667in}}%
\pgfpathcurveto{\pgfqpoint{0.011050in}{-0.041667in}}{\pgfqpoint{0.021649in}{-0.037276in}}{\pgfqpoint{0.029463in}{-0.029463in}}%
\pgfpathcurveto{\pgfqpoint{0.037276in}{-0.021649in}}{\pgfqpoint{0.041667in}{-0.011050in}}{\pgfqpoint{0.041667in}{0.000000in}}%
\pgfpathcurveto{\pgfqpoint{0.041667in}{0.011050in}}{\pgfqpoint{0.037276in}{0.021649in}}{\pgfqpoint{0.029463in}{0.029463in}}%
\pgfpathcurveto{\pgfqpoint{0.021649in}{0.037276in}}{\pgfqpoint{0.011050in}{0.041667in}}{\pgfqpoint{0.000000in}{0.041667in}}%
\pgfpathcurveto{\pgfqpoint{-0.011050in}{0.041667in}}{\pgfqpoint{-0.021649in}{0.037276in}}{\pgfqpoint{-0.029463in}{0.029463in}}%
\pgfpathcurveto{\pgfqpoint{-0.037276in}{0.021649in}}{\pgfqpoint{-0.041667in}{0.011050in}}{\pgfqpoint{-0.041667in}{0.000000in}}%
\pgfpathcurveto{\pgfqpoint{-0.041667in}{-0.011050in}}{\pgfqpoint{-0.037276in}{-0.021649in}}{\pgfqpoint{-0.029463in}{-0.029463in}}%
\pgfpathcurveto{\pgfqpoint{-0.021649in}{-0.037276in}}{\pgfqpoint{-0.011050in}{-0.041667in}}{\pgfqpoint{0.000000in}{-0.041667in}}%
\pgfpathclose%
\pgfusepath{stroke,fill}%
}%
\end{pgfscope}%
\begin{pgfscope}%
\pgfpathrectangle{\pgfqpoint{0.692593in}{0.499691in}}{\pgfqpoint{3.617974in}{3.662560in}}%
\pgfusepath{clip}%
\pgfsetbuttcap%
\pgfsetroundjoin%
\definecolor{currentfill}{rgb}{0.000000,0.000000,0.000000}%
\pgfsetfillcolor{currentfill}%
\pgfsetlinewidth{0.000000pt}%
\definecolor{currentstroke}{rgb}{1.000000,1.000000,1.000000}%
\pgfsetstrokecolor{currentstroke}%
\pgfsetdash{}{0pt}%
\pgfsys@defobject{currentmarker}{\pgfqpoint{-0.049105in}{-0.049105in}}{\pgfqpoint{0.049105in}{0.049105in}}{%
\pgfpathmoveto{\pgfqpoint{-0.024552in}{-0.049105in}}%
\pgfpathlineto{\pgfqpoint{0.000000in}{-0.024552in}}%
\pgfpathlineto{\pgfqpoint{0.024552in}{-0.049105in}}%
\pgfpathlineto{\pgfqpoint{0.049105in}{-0.024552in}}%
\pgfpathlineto{\pgfqpoint{0.024552in}{0.000000in}}%
\pgfpathlineto{\pgfqpoint{0.049105in}{0.024552in}}%
\pgfpathlineto{\pgfqpoint{0.024552in}{0.049105in}}%
\pgfpathlineto{\pgfqpoint{0.000000in}{0.024552in}}%
\pgfpathlineto{\pgfqpoint{-0.024552in}{0.049105in}}%
\pgfpathlineto{\pgfqpoint{-0.049105in}{0.024552in}}%
\pgfpathlineto{\pgfqpoint{-0.024552in}{0.000000in}}%
\pgfpathlineto{\pgfqpoint{-0.049105in}{-0.024552in}}%
\pgfpathclose%
\pgfusepath{fill}%
}%
\begin{pgfscope}%
\pgfsys@transformshift{1.378286in}{2.642801in}%
\pgfsys@useobject{currentmarker}{}%
\end{pgfscope}%
\begin{pgfscope}%
\pgfsys@transformshift{3.675327in}{1.871835in}%
\pgfsys@useobject{currentmarker}{}%
\end{pgfscope}%
\begin{pgfscope}%
\pgfsys@transformshift{1.710220in}{2.428542in}%
\pgfsys@useobject{currentmarker}{}%
\end{pgfscope}%
\begin{pgfscope}%
\pgfsys@transformshift{1.211857in}{1.872588in}%
\pgfsys@useobject{currentmarker}{}%
\end{pgfscope}%
\end{pgfscope}%
\begin{pgfscope}%
\pgfpathrectangle{\pgfqpoint{0.692593in}{0.499691in}}{\pgfqpoint{3.617974in}{3.662560in}}%
\pgfusepath{clip}%
\pgfsetrectcap%
\pgfsetroundjoin%
\pgfsetlinewidth{0.803000pt}%
\definecolor{currentstroke}{rgb}{0.690196,0.690196,0.690196}%
\pgfsetstrokecolor{currentstroke}%
\pgfsetdash{}{0pt}%
\pgfpathmoveto{\pgfqpoint{0.757626in}{0.499691in}}%
\pgfpathlineto{\pgfqpoint{0.757626in}{4.162251in}}%
\pgfusepath{stroke}%
\end{pgfscope}%
\begin{pgfscope}%
\pgfsetbuttcap%
\pgfsetroundjoin%
\definecolor{currentfill}{rgb}{0.000000,0.000000,0.000000}%
\pgfsetfillcolor{currentfill}%
\pgfsetlinewidth{0.803000pt}%
\definecolor{currentstroke}{rgb}{0.000000,0.000000,0.000000}%
\pgfsetstrokecolor{currentstroke}%
\pgfsetdash{}{0pt}%
\pgfsys@defobject{currentmarker}{\pgfqpoint{0.000000in}{-0.048611in}}{\pgfqpoint{0.000000in}{0.000000in}}{%
\pgfpathmoveto{\pgfqpoint{0.000000in}{0.000000in}}%
\pgfpathlineto{\pgfqpoint{0.000000in}{-0.048611in}}%
\pgfusepath{stroke,fill}%
}%
\begin{pgfscope}%
\pgfsys@transformshift{0.757626in}{0.499691in}%
\pgfsys@useobject{currentmarker}{}%
\end{pgfscope}%
\end{pgfscope}%
\begin{pgfscope}%
\definecolor{textcolor}{rgb}{0.000000,0.000000,0.000000}%
\pgfsetstrokecolor{textcolor}%
\pgfsetfillcolor{textcolor}%
\pgftext[x=0.757626in,y=0.402469in,,top]{\color{textcolor}\rmfamily\fontsize{10.000000}{12.000000}\selectfont \(\displaystyle {-100}\)}%
\end{pgfscope}%
\begin{pgfscope}%
\pgfpathrectangle{\pgfqpoint{0.692593in}{0.499691in}}{\pgfqpoint{3.617974in}{3.662560in}}%
\pgfusepath{clip}%
\pgfsetrectcap%
\pgfsetroundjoin%
\pgfsetlinewidth{0.803000pt}%
\definecolor{currentstroke}{rgb}{0.690196,0.690196,0.690196}%
\pgfsetstrokecolor{currentstroke}%
\pgfsetdash{}{0pt}%
\pgfpathmoveto{\pgfqpoint{1.445847in}{0.499691in}}%
\pgfpathlineto{\pgfqpoint{1.445847in}{4.162251in}}%
\pgfusepath{stroke}%
\end{pgfscope}%
\begin{pgfscope}%
\pgfsetbuttcap%
\pgfsetroundjoin%
\definecolor{currentfill}{rgb}{0.000000,0.000000,0.000000}%
\pgfsetfillcolor{currentfill}%
\pgfsetlinewidth{0.803000pt}%
\definecolor{currentstroke}{rgb}{0.000000,0.000000,0.000000}%
\pgfsetstrokecolor{currentstroke}%
\pgfsetdash{}{0pt}%
\pgfsys@defobject{currentmarker}{\pgfqpoint{0.000000in}{-0.048611in}}{\pgfqpoint{0.000000in}{0.000000in}}{%
\pgfpathmoveto{\pgfqpoint{0.000000in}{0.000000in}}%
\pgfpathlineto{\pgfqpoint{0.000000in}{-0.048611in}}%
\pgfusepath{stroke,fill}%
}%
\begin{pgfscope}%
\pgfsys@transformshift{1.445847in}{0.499691in}%
\pgfsys@useobject{currentmarker}{}%
\end{pgfscope}%
\end{pgfscope}%
\begin{pgfscope}%
\definecolor{textcolor}{rgb}{0.000000,0.000000,0.000000}%
\pgfsetstrokecolor{textcolor}%
\pgfsetfillcolor{textcolor}%
\pgftext[x=1.445847in,y=0.402469in,,top]{\color{textcolor}\rmfamily\fontsize{10.000000}{12.000000}\selectfont \(\displaystyle {0}\)}%
\end{pgfscope}%
\begin{pgfscope}%
\pgfpathrectangle{\pgfqpoint{0.692593in}{0.499691in}}{\pgfqpoint{3.617974in}{3.662560in}}%
\pgfusepath{clip}%
\pgfsetrectcap%
\pgfsetroundjoin%
\pgfsetlinewidth{0.803000pt}%
\definecolor{currentstroke}{rgb}{0.690196,0.690196,0.690196}%
\pgfsetstrokecolor{currentstroke}%
\pgfsetdash{}{0pt}%
\pgfpathmoveto{\pgfqpoint{2.134067in}{0.499691in}}%
\pgfpathlineto{\pgfqpoint{2.134067in}{4.162251in}}%
\pgfusepath{stroke}%
\end{pgfscope}%
\begin{pgfscope}%
\pgfsetbuttcap%
\pgfsetroundjoin%
\definecolor{currentfill}{rgb}{0.000000,0.000000,0.000000}%
\pgfsetfillcolor{currentfill}%
\pgfsetlinewidth{0.803000pt}%
\definecolor{currentstroke}{rgb}{0.000000,0.000000,0.000000}%
\pgfsetstrokecolor{currentstroke}%
\pgfsetdash{}{0pt}%
\pgfsys@defobject{currentmarker}{\pgfqpoint{0.000000in}{-0.048611in}}{\pgfqpoint{0.000000in}{0.000000in}}{%
\pgfpathmoveto{\pgfqpoint{0.000000in}{0.000000in}}%
\pgfpathlineto{\pgfqpoint{0.000000in}{-0.048611in}}%
\pgfusepath{stroke,fill}%
}%
\begin{pgfscope}%
\pgfsys@transformshift{2.134067in}{0.499691in}%
\pgfsys@useobject{currentmarker}{}%
\end{pgfscope}%
\end{pgfscope}%
\begin{pgfscope}%
\definecolor{textcolor}{rgb}{0.000000,0.000000,0.000000}%
\pgfsetstrokecolor{textcolor}%
\pgfsetfillcolor{textcolor}%
\pgftext[x=2.134067in,y=0.402469in,,top]{\color{textcolor}\rmfamily\fontsize{10.000000}{12.000000}\selectfont \(\displaystyle {100}\)}%
\end{pgfscope}%
\begin{pgfscope}%
\pgfpathrectangle{\pgfqpoint{0.692593in}{0.499691in}}{\pgfqpoint{3.617974in}{3.662560in}}%
\pgfusepath{clip}%
\pgfsetrectcap%
\pgfsetroundjoin%
\pgfsetlinewidth{0.803000pt}%
\definecolor{currentstroke}{rgb}{0.690196,0.690196,0.690196}%
\pgfsetstrokecolor{currentstroke}%
\pgfsetdash{}{0pt}%
\pgfpathmoveto{\pgfqpoint{2.822288in}{0.499691in}}%
\pgfpathlineto{\pgfqpoint{2.822288in}{4.162251in}}%
\pgfusepath{stroke}%
\end{pgfscope}%
\begin{pgfscope}%
\pgfsetbuttcap%
\pgfsetroundjoin%
\definecolor{currentfill}{rgb}{0.000000,0.000000,0.000000}%
\pgfsetfillcolor{currentfill}%
\pgfsetlinewidth{0.803000pt}%
\definecolor{currentstroke}{rgb}{0.000000,0.000000,0.000000}%
\pgfsetstrokecolor{currentstroke}%
\pgfsetdash{}{0pt}%
\pgfsys@defobject{currentmarker}{\pgfqpoint{0.000000in}{-0.048611in}}{\pgfqpoint{0.000000in}{0.000000in}}{%
\pgfpathmoveto{\pgfqpoint{0.000000in}{0.000000in}}%
\pgfpathlineto{\pgfqpoint{0.000000in}{-0.048611in}}%
\pgfusepath{stroke,fill}%
}%
\begin{pgfscope}%
\pgfsys@transformshift{2.822288in}{0.499691in}%
\pgfsys@useobject{currentmarker}{}%
\end{pgfscope}%
\end{pgfscope}%
\begin{pgfscope}%
\definecolor{textcolor}{rgb}{0.000000,0.000000,0.000000}%
\pgfsetstrokecolor{textcolor}%
\pgfsetfillcolor{textcolor}%
\pgftext[x=2.822288in,y=0.402469in,,top]{\color{textcolor}\rmfamily\fontsize{10.000000}{12.000000}\selectfont \(\displaystyle {200}\)}%
\end{pgfscope}%
\begin{pgfscope}%
\pgfpathrectangle{\pgfqpoint{0.692593in}{0.499691in}}{\pgfqpoint{3.617974in}{3.662560in}}%
\pgfusepath{clip}%
\pgfsetrectcap%
\pgfsetroundjoin%
\pgfsetlinewidth{0.803000pt}%
\definecolor{currentstroke}{rgb}{0.690196,0.690196,0.690196}%
\pgfsetstrokecolor{currentstroke}%
\pgfsetdash{}{0pt}%
\pgfpathmoveto{\pgfqpoint{3.510509in}{0.499691in}}%
\pgfpathlineto{\pgfqpoint{3.510509in}{4.162251in}}%
\pgfusepath{stroke}%
\end{pgfscope}%
\begin{pgfscope}%
\pgfsetbuttcap%
\pgfsetroundjoin%
\definecolor{currentfill}{rgb}{0.000000,0.000000,0.000000}%
\pgfsetfillcolor{currentfill}%
\pgfsetlinewidth{0.803000pt}%
\definecolor{currentstroke}{rgb}{0.000000,0.000000,0.000000}%
\pgfsetstrokecolor{currentstroke}%
\pgfsetdash{}{0pt}%
\pgfsys@defobject{currentmarker}{\pgfqpoint{0.000000in}{-0.048611in}}{\pgfqpoint{0.000000in}{0.000000in}}{%
\pgfpathmoveto{\pgfqpoint{0.000000in}{0.000000in}}%
\pgfpathlineto{\pgfqpoint{0.000000in}{-0.048611in}}%
\pgfusepath{stroke,fill}%
}%
\begin{pgfscope}%
\pgfsys@transformshift{3.510509in}{0.499691in}%
\pgfsys@useobject{currentmarker}{}%
\end{pgfscope}%
\end{pgfscope}%
\begin{pgfscope}%
\definecolor{textcolor}{rgb}{0.000000,0.000000,0.000000}%
\pgfsetstrokecolor{textcolor}%
\pgfsetfillcolor{textcolor}%
\pgftext[x=3.510509in,y=0.402469in,,top]{\color{textcolor}\rmfamily\fontsize{10.000000}{12.000000}\selectfont \(\displaystyle {300}\)}%
\end{pgfscope}%
\begin{pgfscope}%
\pgfpathrectangle{\pgfqpoint{0.692593in}{0.499691in}}{\pgfqpoint{3.617974in}{3.662560in}}%
\pgfusepath{clip}%
\pgfsetrectcap%
\pgfsetroundjoin%
\pgfsetlinewidth{0.803000pt}%
\definecolor{currentstroke}{rgb}{0.690196,0.690196,0.690196}%
\pgfsetstrokecolor{currentstroke}%
\pgfsetdash{}{0pt}%
\pgfpathmoveto{\pgfqpoint{4.198729in}{0.499691in}}%
\pgfpathlineto{\pgfqpoint{4.198729in}{4.162251in}}%
\pgfusepath{stroke}%
\end{pgfscope}%
\begin{pgfscope}%
\pgfsetbuttcap%
\pgfsetroundjoin%
\definecolor{currentfill}{rgb}{0.000000,0.000000,0.000000}%
\pgfsetfillcolor{currentfill}%
\pgfsetlinewidth{0.803000pt}%
\definecolor{currentstroke}{rgb}{0.000000,0.000000,0.000000}%
\pgfsetstrokecolor{currentstroke}%
\pgfsetdash{}{0pt}%
\pgfsys@defobject{currentmarker}{\pgfqpoint{0.000000in}{-0.048611in}}{\pgfqpoint{0.000000in}{0.000000in}}{%
\pgfpathmoveto{\pgfqpoint{0.000000in}{0.000000in}}%
\pgfpathlineto{\pgfqpoint{0.000000in}{-0.048611in}}%
\pgfusepath{stroke,fill}%
}%
\begin{pgfscope}%
\pgfsys@transformshift{4.198729in}{0.499691in}%
\pgfsys@useobject{currentmarker}{}%
\end{pgfscope}%
\end{pgfscope}%
\begin{pgfscope}%
\definecolor{textcolor}{rgb}{0.000000,0.000000,0.000000}%
\pgfsetstrokecolor{textcolor}%
\pgfsetfillcolor{textcolor}%
\pgftext[x=4.198729in,y=0.402469in,,top]{\color{textcolor}\rmfamily\fontsize{10.000000}{12.000000}\selectfont \(\displaystyle {400}\)}%
\end{pgfscope}%
\begin{pgfscope}%
\definecolor{textcolor}{rgb}{0.000000,0.000000,0.000000}%
\pgfsetstrokecolor{textcolor}%
\pgfsetfillcolor{textcolor}%
\pgftext[x=2.501580in,y=0.223457in,,top]{\color{textcolor}\rmfamily\fontsize{10.000000}{12.000000}\selectfont PC1}%
\end{pgfscope}%
\begin{pgfscope}%
\pgfpathrectangle{\pgfqpoint{0.692593in}{0.499691in}}{\pgfqpoint{3.617974in}{3.662560in}}%
\pgfusepath{clip}%
\pgfsetrectcap%
\pgfsetroundjoin%
\pgfsetlinewidth{0.803000pt}%
\definecolor{currentstroke}{rgb}{0.690196,0.690196,0.690196}%
\pgfsetstrokecolor{currentstroke}%
\pgfsetdash{}{0pt}%
\pgfpathmoveto{\pgfqpoint{0.692593in}{0.499691in}}%
\pgfpathlineto{\pgfqpoint{4.310567in}{0.499691in}}%
\pgfusepath{stroke}%
\end{pgfscope}%
\begin{pgfscope}%
\pgfsetbuttcap%
\pgfsetroundjoin%
\definecolor{currentfill}{rgb}{0.000000,0.000000,0.000000}%
\pgfsetfillcolor{currentfill}%
\pgfsetlinewidth{0.803000pt}%
\definecolor{currentstroke}{rgb}{0.000000,0.000000,0.000000}%
\pgfsetstrokecolor{currentstroke}%
\pgfsetdash{}{0pt}%
\pgfsys@defobject{currentmarker}{\pgfqpoint{-0.048611in}{0.000000in}}{\pgfqpoint{0.000000in}{0.000000in}}{%
\pgfpathmoveto{\pgfqpoint{0.000000in}{0.000000in}}%
\pgfpathlineto{\pgfqpoint{-0.048611in}{0.000000in}}%
\pgfusepath{stroke,fill}%
}%
\begin{pgfscope}%
\pgfsys@transformshift{0.692593in}{0.499691in}%
\pgfsys@useobject{currentmarker}{}%
\end{pgfscope}%
\end{pgfscope}%
\begin{pgfscope}%
\definecolor{textcolor}{rgb}{0.000000,0.000000,0.000000}%
\pgfsetstrokecolor{textcolor}%
\pgfsetfillcolor{textcolor}%
\pgftext[x=0.279012in, y=0.451466in, left, base]{\color{textcolor}\rmfamily\fontsize{10.000000}{12.000000}\selectfont \(\displaystyle {-200}\)}%
\end{pgfscope}%
\begin{pgfscope}%
\pgfpathrectangle{\pgfqpoint{0.692593in}{0.499691in}}{\pgfqpoint{3.617974in}{3.662560in}}%
\pgfusepath{clip}%
\pgfsetrectcap%
\pgfsetroundjoin%
\pgfsetlinewidth{0.803000pt}%
\definecolor{currentstroke}{rgb}{0.690196,0.690196,0.690196}%
\pgfsetstrokecolor{currentstroke}%
\pgfsetdash{}{0pt}%
\pgfpathmoveto{\pgfqpoint{0.692593in}{0.957511in}}%
\pgfpathlineto{\pgfqpoint{4.310567in}{0.957511in}}%
\pgfusepath{stroke}%
\end{pgfscope}%
\begin{pgfscope}%
\pgfsetbuttcap%
\pgfsetroundjoin%
\definecolor{currentfill}{rgb}{0.000000,0.000000,0.000000}%
\pgfsetfillcolor{currentfill}%
\pgfsetlinewidth{0.803000pt}%
\definecolor{currentstroke}{rgb}{0.000000,0.000000,0.000000}%
\pgfsetstrokecolor{currentstroke}%
\pgfsetdash{}{0pt}%
\pgfsys@defobject{currentmarker}{\pgfqpoint{-0.048611in}{0.000000in}}{\pgfqpoint{0.000000in}{0.000000in}}{%
\pgfpathmoveto{\pgfqpoint{0.000000in}{0.000000in}}%
\pgfpathlineto{\pgfqpoint{-0.048611in}{0.000000in}}%
\pgfusepath{stroke,fill}%
}%
\begin{pgfscope}%
\pgfsys@transformshift{0.692593in}{0.957511in}%
\pgfsys@useobject{currentmarker}{}%
\end{pgfscope}%
\end{pgfscope}%
\begin{pgfscope}%
\definecolor{textcolor}{rgb}{0.000000,0.000000,0.000000}%
\pgfsetstrokecolor{textcolor}%
\pgfsetfillcolor{textcolor}%
\pgftext[x=0.279012in, y=0.909286in, left, base]{\color{textcolor}\rmfamily\fontsize{10.000000}{12.000000}\selectfont \(\displaystyle {-150}\)}%
\end{pgfscope}%
\begin{pgfscope}%
\pgfpathrectangle{\pgfqpoint{0.692593in}{0.499691in}}{\pgfqpoint{3.617974in}{3.662560in}}%
\pgfusepath{clip}%
\pgfsetrectcap%
\pgfsetroundjoin%
\pgfsetlinewidth{0.803000pt}%
\definecolor{currentstroke}{rgb}{0.690196,0.690196,0.690196}%
\pgfsetstrokecolor{currentstroke}%
\pgfsetdash{}{0pt}%
\pgfpathmoveto{\pgfqpoint{0.692593in}{1.415331in}}%
\pgfpathlineto{\pgfqpoint{4.310567in}{1.415331in}}%
\pgfusepath{stroke}%
\end{pgfscope}%
\begin{pgfscope}%
\pgfsetbuttcap%
\pgfsetroundjoin%
\definecolor{currentfill}{rgb}{0.000000,0.000000,0.000000}%
\pgfsetfillcolor{currentfill}%
\pgfsetlinewidth{0.803000pt}%
\definecolor{currentstroke}{rgb}{0.000000,0.000000,0.000000}%
\pgfsetstrokecolor{currentstroke}%
\pgfsetdash{}{0pt}%
\pgfsys@defobject{currentmarker}{\pgfqpoint{-0.048611in}{0.000000in}}{\pgfqpoint{0.000000in}{0.000000in}}{%
\pgfpathmoveto{\pgfqpoint{0.000000in}{0.000000in}}%
\pgfpathlineto{\pgfqpoint{-0.048611in}{0.000000in}}%
\pgfusepath{stroke,fill}%
}%
\begin{pgfscope}%
\pgfsys@transformshift{0.692593in}{1.415331in}%
\pgfsys@useobject{currentmarker}{}%
\end{pgfscope}%
\end{pgfscope}%
\begin{pgfscope}%
\definecolor{textcolor}{rgb}{0.000000,0.000000,0.000000}%
\pgfsetstrokecolor{textcolor}%
\pgfsetfillcolor{textcolor}%
\pgftext[x=0.279012in, y=1.367106in, left, base]{\color{textcolor}\rmfamily\fontsize{10.000000}{12.000000}\selectfont \(\displaystyle {-100}\)}%
\end{pgfscope}%
\begin{pgfscope}%
\pgfpathrectangle{\pgfqpoint{0.692593in}{0.499691in}}{\pgfqpoint{3.617974in}{3.662560in}}%
\pgfusepath{clip}%
\pgfsetrectcap%
\pgfsetroundjoin%
\pgfsetlinewidth{0.803000pt}%
\definecolor{currentstroke}{rgb}{0.690196,0.690196,0.690196}%
\pgfsetstrokecolor{currentstroke}%
\pgfsetdash{}{0pt}%
\pgfpathmoveto{\pgfqpoint{0.692593in}{1.873151in}}%
\pgfpathlineto{\pgfqpoint{4.310567in}{1.873151in}}%
\pgfusepath{stroke}%
\end{pgfscope}%
\begin{pgfscope}%
\pgfsetbuttcap%
\pgfsetroundjoin%
\definecolor{currentfill}{rgb}{0.000000,0.000000,0.000000}%
\pgfsetfillcolor{currentfill}%
\pgfsetlinewidth{0.803000pt}%
\definecolor{currentstroke}{rgb}{0.000000,0.000000,0.000000}%
\pgfsetstrokecolor{currentstroke}%
\pgfsetdash{}{0pt}%
\pgfsys@defobject{currentmarker}{\pgfqpoint{-0.048611in}{0.000000in}}{\pgfqpoint{0.000000in}{0.000000in}}{%
\pgfpathmoveto{\pgfqpoint{0.000000in}{0.000000in}}%
\pgfpathlineto{\pgfqpoint{-0.048611in}{0.000000in}}%
\pgfusepath{stroke,fill}%
}%
\begin{pgfscope}%
\pgfsys@transformshift{0.692593in}{1.873151in}%
\pgfsys@useobject{currentmarker}{}%
\end{pgfscope}%
\end{pgfscope}%
\begin{pgfscope}%
\definecolor{textcolor}{rgb}{0.000000,0.000000,0.000000}%
\pgfsetstrokecolor{textcolor}%
\pgfsetfillcolor{textcolor}%
\pgftext[x=0.348457in, y=1.824926in, left, base]{\color{textcolor}\rmfamily\fontsize{10.000000}{12.000000}\selectfont \(\displaystyle {-50}\)}%
\end{pgfscope}%
\begin{pgfscope}%
\pgfpathrectangle{\pgfqpoint{0.692593in}{0.499691in}}{\pgfqpoint{3.617974in}{3.662560in}}%
\pgfusepath{clip}%
\pgfsetrectcap%
\pgfsetroundjoin%
\pgfsetlinewidth{0.803000pt}%
\definecolor{currentstroke}{rgb}{0.690196,0.690196,0.690196}%
\pgfsetstrokecolor{currentstroke}%
\pgfsetdash{}{0pt}%
\pgfpathmoveto{\pgfqpoint{0.692593in}{2.330971in}}%
\pgfpathlineto{\pgfqpoint{4.310567in}{2.330971in}}%
\pgfusepath{stroke}%
\end{pgfscope}%
\begin{pgfscope}%
\pgfsetbuttcap%
\pgfsetroundjoin%
\definecolor{currentfill}{rgb}{0.000000,0.000000,0.000000}%
\pgfsetfillcolor{currentfill}%
\pgfsetlinewidth{0.803000pt}%
\definecolor{currentstroke}{rgb}{0.000000,0.000000,0.000000}%
\pgfsetstrokecolor{currentstroke}%
\pgfsetdash{}{0pt}%
\pgfsys@defobject{currentmarker}{\pgfqpoint{-0.048611in}{0.000000in}}{\pgfqpoint{0.000000in}{0.000000in}}{%
\pgfpathmoveto{\pgfqpoint{0.000000in}{0.000000in}}%
\pgfpathlineto{\pgfqpoint{-0.048611in}{0.000000in}}%
\pgfusepath{stroke,fill}%
}%
\begin{pgfscope}%
\pgfsys@transformshift{0.692593in}{2.330971in}%
\pgfsys@useobject{currentmarker}{}%
\end{pgfscope}%
\end{pgfscope}%
\begin{pgfscope}%
\definecolor{textcolor}{rgb}{0.000000,0.000000,0.000000}%
\pgfsetstrokecolor{textcolor}%
\pgfsetfillcolor{textcolor}%
\pgftext[x=0.525927in, y=2.282746in, left, base]{\color{textcolor}\rmfamily\fontsize{10.000000}{12.000000}\selectfont \(\displaystyle {0}\)}%
\end{pgfscope}%
\begin{pgfscope}%
\pgfpathrectangle{\pgfqpoint{0.692593in}{0.499691in}}{\pgfqpoint{3.617974in}{3.662560in}}%
\pgfusepath{clip}%
\pgfsetrectcap%
\pgfsetroundjoin%
\pgfsetlinewidth{0.803000pt}%
\definecolor{currentstroke}{rgb}{0.690196,0.690196,0.690196}%
\pgfsetstrokecolor{currentstroke}%
\pgfsetdash{}{0pt}%
\pgfpathmoveto{\pgfqpoint{0.692593in}{2.788791in}}%
\pgfpathlineto{\pgfqpoint{4.310567in}{2.788791in}}%
\pgfusepath{stroke}%
\end{pgfscope}%
\begin{pgfscope}%
\pgfsetbuttcap%
\pgfsetroundjoin%
\definecolor{currentfill}{rgb}{0.000000,0.000000,0.000000}%
\pgfsetfillcolor{currentfill}%
\pgfsetlinewidth{0.803000pt}%
\definecolor{currentstroke}{rgb}{0.000000,0.000000,0.000000}%
\pgfsetstrokecolor{currentstroke}%
\pgfsetdash{}{0pt}%
\pgfsys@defobject{currentmarker}{\pgfqpoint{-0.048611in}{0.000000in}}{\pgfqpoint{0.000000in}{0.000000in}}{%
\pgfpathmoveto{\pgfqpoint{0.000000in}{0.000000in}}%
\pgfpathlineto{\pgfqpoint{-0.048611in}{0.000000in}}%
\pgfusepath{stroke,fill}%
}%
\begin{pgfscope}%
\pgfsys@transformshift{0.692593in}{2.788791in}%
\pgfsys@useobject{currentmarker}{}%
\end{pgfscope}%
\end{pgfscope}%
\begin{pgfscope}%
\definecolor{textcolor}{rgb}{0.000000,0.000000,0.000000}%
\pgfsetstrokecolor{textcolor}%
\pgfsetfillcolor{textcolor}%
\pgftext[x=0.456482in, y=2.740566in, left, base]{\color{textcolor}\rmfamily\fontsize{10.000000}{12.000000}\selectfont \(\displaystyle {50}\)}%
\end{pgfscope}%
\begin{pgfscope}%
\pgfpathrectangle{\pgfqpoint{0.692593in}{0.499691in}}{\pgfqpoint{3.617974in}{3.662560in}}%
\pgfusepath{clip}%
\pgfsetrectcap%
\pgfsetroundjoin%
\pgfsetlinewidth{0.803000pt}%
\definecolor{currentstroke}{rgb}{0.690196,0.690196,0.690196}%
\pgfsetstrokecolor{currentstroke}%
\pgfsetdash{}{0pt}%
\pgfpathmoveto{\pgfqpoint{0.692593in}{3.246611in}}%
\pgfpathlineto{\pgfqpoint{4.310567in}{3.246611in}}%
\pgfusepath{stroke}%
\end{pgfscope}%
\begin{pgfscope}%
\pgfsetbuttcap%
\pgfsetroundjoin%
\definecolor{currentfill}{rgb}{0.000000,0.000000,0.000000}%
\pgfsetfillcolor{currentfill}%
\pgfsetlinewidth{0.803000pt}%
\definecolor{currentstroke}{rgb}{0.000000,0.000000,0.000000}%
\pgfsetstrokecolor{currentstroke}%
\pgfsetdash{}{0pt}%
\pgfsys@defobject{currentmarker}{\pgfqpoint{-0.048611in}{0.000000in}}{\pgfqpoint{0.000000in}{0.000000in}}{%
\pgfpathmoveto{\pgfqpoint{0.000000in}{0.000000in}}%
\pgfpathlineto{\pgfqpoint{-0.048611in}{0.000000in}}%
\pgfusepath{stroke,fill}%
}%
\begin{pgfscope}%
\pgfsys@transformshift{0.692593in}{3.246611in}%
\pgfsys@useobject{currentmarker}{}%
\end{pgfscope}%
\end{pgfscope}%
\begin{pgfscope}%
\definecolor{textcolor}{rgb}{0.000000,0.000000,0.000000}%
\pgfsetstrokecolor{textcolor}%
\pgfsetfillcolor{textcolor}%
\pgftext[x=0.387037in, y=3.198386in, left, base]{\color{textcolor}\rmfamily\fontsize{10.000000}{12.000000}\selectfont \(\displaystyle {100}\)}%
\end{pgfscope}%
\begin{pgfscope}%
\pgfpathrectangle{\pgfqpoint{0.692593in}{0.499691in}}{\pgfqpoint{3.617974in}{3.662560in}}%
\pgfusepath{clip}%
\pgfsetrectcap%
\pgfsetroundjoin%
\pgfsetlinewidth{0.803000pt}%
\definecolor{currentstroke}{rgb}{0.690196,0.690196,0.690196}%
\pgfsetstrokecolor{currentstroke}%
\pgfsetdash{}{0pt}%
\pgfpathmoveto{\pgfqpoint{0.692593in}{3.704431in}}%
\pgfpathlineto{\pgfqpoint{4.310567in}{3.704431in}}%
\pgfusepath{stroke}%
\end{pgfscope}%
\begin{pgfscope}%
\pgfsetbuttcap%
\pgfsetroundjoin%
\definecolor{currentfill}{rgb}{0.000000,0.000000,0.000000}%
\pgfsetfillcolor{currentfill}%
\pgfsetlinewidth{0.803000pt}%
\definecolor{currentstroke}{rgb}{0.000000,0.000000,0.000000}%
\pgfsetstrokecolor{currentstroke}%
\pgfsetdash{}{0pt}%
\pgfsys@defobject{currentmarker}{\pgfqpoint{-0.048611in}{0.000000in}}{\pgfqpoint{0.000000in}{0.000000in}}{%
\pgfpathmoveto{\pgfqpoint{0.000000in}{0.000000in}}%
\pgfpathlineto{\pgfqpoint{-0.048611in}{0.000000in}}%
\pgfusepath{stroke,fill}%
}%
\begin{pgfscope}%
\pgfsys@transformshift{0.692593in}{3.704431in}%
\pgfsys@useobject{currentmarker}{}%
\end{pgfscope}%
\end{pgfscope}%
\begin{pgfscope}%
\definecolor{textcolor}{rgb}{0.000000,0.000000,0.000000}%
\pgfsetstrokecolor{textcolor}%
\pgfsetfillcolor{textcolor}%
\pgftext[x=0.387037in, y=3.656206in, left, base]{\color{textcolor}\rmfamily\fontsize{10.000000}{12.000000}\selectfont \(\displaystyle {150}\)}%
\end{pgfscope}%
\begin{pgfscope}%
\pgfpathrectangle{\pgfqpoint{0.692593in}{0.499691in}}{\pgfqpoint{3.617974in}{3.662560in}}%
\pgfusepath{clip}%
\pgfsetrectcap%
\pgfsetroundjoin%
\pgfsetlinewidth{0.803000pt}%
\definecolor{currentstroke}{rgb}{0.690196,0.690196,0.690196}%
\pgfsetstrokecolor{currentstroke}%
\pgfsetdash{}{0pt}%
\pgfpathmoveto{\pgfqpoint{0.692593in}{4.162251in}}%
\pgfpathlineto{\pgfqpoint{4.310567in}{4.162251in}}%
\pgfusepath{stroke}%
\end{pgfscope}%
\begin{pgfscope}%
\pgfsetbuttcap%
\pgfsetroundjoin%
\definecolor{currentfill}{rgb}{0.000000,0.000000,0.000000}%
\pgfsetfillcolor{currentfill}%
\pgfsetlinewidth{0.803000pt}%
\definecolor{currentstroke}{rgb}{0.000000,0.000000,0.000000}%
\pgfsetstrokecolor{currentstroke}%
\pgfsetdash{}{0pt}%
\pgfsys@defobject{currentmarker}{\pgfqpoint{-0.048611in}{0.000000in}}{\pgfqpoint{0.000000in}{0.000000in}}{%
\pgfpathmoveto{\pgfqpoint{0.000000in}{0.000000in}}%
\pgfpathlineto{\pgfqpoint{-0.048611in}{0.000000in}}%
\pgfusepath{stroke,fill}%
}%
\begin{pgfscope}%
\pgfsys@transformshift{0.692593in}{4.162251in}%
\pgfsys@useobject{currentmarker}{}%
\end{pgfscope}%
\end{pgfscope}%
\begin{pgfscope}%
\definecolor{textcolor}{rgb}{0.000000,0.000000,0.000000}%
\pgfsetstrokecolor{textcolor}%
\pgfsetfillcolor{textcolor}%
\pgftext[x=0.387037in, y=4.114026in, left, base]{\color{textcolor}\rmfamily\fontsize{10.000000}{12.000000}\selectfont \(\displaystyle {200}\)}%
\end{pgfscope}%
\begin{pgfscope}%
\definecolor{textcolor}{rgb}{0.000000,0.000000,0.000000}%
\pgfsetstrokecolor{textcolor}%
\pgfsetfillcolor{textcolor}%
\pgftext[x=0.223457in,y=2.330971in,,bottom,rotate=90.000000]{\color{textcolor}\rmfamily\fontsize{10.000000}{12.000000}\selectfont PC2}%
\end{pgfscope}%
\begin{pgfscope}%
\pgfsetrectcap%
\pgfsetmiterjoin%
\pgfsetlinewidth{0.803000pt}%
\definecolor{currentstroke}{rgb}{0.000000,0.000000,0.000000}%
\pgfsetstrokecolor{currentstroke}%
\pgfsetdash{}{0pt}%
\pgfpathmoveto{\pgfqpoint{0.692593in}{0.499691in}}%
\pgfpathlineto{\pgfqpoint{0.692593in}{4.162251in}}%
\pgfusepath{stroke}%
\end{pgfscope}%
\begin{pgfscope}%
\pgfsetrectcap%
\pgfsetmiterjoin%
\pgfsetlinewidth{0.803000pt}%
\definecolor{currentstroke}{rgb}{0.000000,0.000000,0.000000}%
\pgfsetstrokecolor{currentstroke}%
\pgfsetdash{}{0pt}%
\pgfpathmoveto{\pgfqpoint{0.692593in}{0.499691in}}%
\pgfpathlineto{\pgfqpoint{4.310567in}{0.499691in}}%
\pgfusepath{stroke}%
\end{pgfscope}%
\begin{pgfscope}%
\pgfsetbuttcap%
\pgfsetmiterjoin%
\definecolor{currentfill}{rgb}{1.000000,1.000000,1.000000}%
\pgfsetfillcolor{currentfill}%
\pgfsetfillopacity{0.800000}%
\pgfsetlinewidth{1.003750pt}%
\definecolor{currentstroke}{rgb}{0.800000,0.800000,0.800000}%
\pgfsetstrokecolor{currentstroke}%
\pgfsetstrokeopacity{0.800000}%
\pgfsetdash{}{0pt}%
\pgfpathmoveto{\pgfqpoint{3.212187in}{3.082776in}}%
\pgfpathlineto{\pgfqpoint{4.213345in}{3.082776in}}%
\pgfpathquadraticcurveto{\pgfqpoint{4.241123in}{3.082776in}}{\pgfqpoint{4.241123in}{3.110554in}}%
\pgfpathlineto{\pgfqpoint{4.241123in}{4.065029in}}%
\pgfpathquadraticcurveto{\pgfqpoint{4.241123in}{4.092807in}}{\pgfqpoint{4.213345in}{4.092807in}}%
\pgfpathlineto{\pgfqpoint{3.212187in}{4.092807in}}%
\pgfpathquadraticcurveto{\pgfqpoint{3.184409in}{4.092807in}}{\pgfqpoint{3.184409in}{4.065029in}}%
\pgfpathlineto{\pgfqpoint{3.184409in}{3.110554in}}%
\pgfpathquadraticcurveto{\pgfqpoint{3.184409in}{3.082776in}}{\pgfqpoint{3.212187in}{3.082776in}}%
\pgfpathclose%
\pgfusepath{stroke,fill}%
\end{pgfscope}%
\begin{pgfscope}%
\pgfsetbuttcap%
\pgfsetroundjoin%
\definecolor{currentfill}{rgb}{0.121569,0.466667,0.705882}%
\pgfsetfillcolor{currentfill}%
\pgfsetlinewidth{1.003750pt}%
\definecolor{currentstroke}{rgb}{0.121569,0.466667,0.705882}%
\pgfsetstrokecolor{currentstroke}%
\pgfsetdash{}{0pt}%
\pgfsys@defobject{currentmarker}{\pgfqpoint{-0.041667in}{-0.041667in}}{\pgfqpoint{0.041667in}{0.041667in}}{%
\pgfpathmoveto{\pgfqpoint{0.000000in}{-0.041667in}}%
\pgfpathcurveto{\pgfqpoint{0.011050in}{-0.041667in}}{\pgfqpoint{0.021649in}{-0.037276in}}{\pgfqpoint{0.029463in}{-0.029463in}}%
\pgfpathcurveto{\pgfqpoint{0.037276in}{-0.021649in}}{\pgfqpoint{0.041667in}{-0.011050in}}{\pgfqpoint{0.041667in}{0.000000in}}%
\pgfpathcurveto{\pgfqpoint{0.041667in}{0.011050in}}{\pgfqpoint{0.037276in}{0.021649in}}{\pgfqpoint{0.029463in}{0.029463in}}%
\pgfpathcurveto{\pgfqpoint{0.021649in}{0.037276in}}{\pgfqpoint{0.011050in}{0.041667in}}{\pgfqpoint{0.000000in}{0.041667in}}%
\pgfpathcurveto{\pgfqpoint{-0.011050in}{0.041667in}}{\pgfqpoint{-0.021649in}{0.037276in}}{\pgfqpoint{-0.029463in}{0.029463in}}%
\pgfpathcurveto{\pgfqpoint{-0.037276in}{0.021649in}}{\pgfqpoint{-0.041667in}{0.011050in}}{\pgfqpoint{-0.041667in}{0.000000in}}%
\pgfpathcurveto{\pgfqpoint{-0.041667in}{-0.011050in}}{\pgfqpoint{-0.037276in}{-0.021649in}}{\pgfqpoint{-0.029463in}{-0.029463in}}%
\pgfpathcurveto{\pgfqpoint{-0.021649in}{-0.037276in}}{\pgfqpoint{-0.011050in}{-0.041667in}}{\pgfqpoint{0.000000in}{-0.041667in}}%
\pgfpathclose%
\pgfusepath{stroke,fill}%
}%
\begin{pgfscope}%
\pgfsys@transformshift{3.378853in}{3.976487in}%
\pgfsys@useobject{currentmarker}{}%
\end{pgfscope}%
\end{pgfscope}%
\begin{pgfscope}%
\definecolor{textcolor}{rgb}{0.000000,0.000000,0.000000}%
\pgfsetstrokecolor{textcolor}%
\pgfsetfillcolor{textcolor}%
\pgftext[x=3.628853in,y=3.940029in,left,base]{\color{textcolor}\rmfamily\fontsize{10.000000}{12.000000}\selectfont Cluster 0}%
\end{pgfscope}%
\begin{pgfscope}%
\pgfsetbuttcap%
\pgfsetroundjoin%
\definecolor{currentfill}{rgb}{0.172549,0.627451,0.172549}%
\pgfsetfillcolor{currentfill}%
\pgfsetlinewidth{1.003750pt}%
\definecolor{currentstroke}{rgb}{0.172549,0.627451,0.172549}%
\pgfsetstrokecolor{currentstroke}%
\pgfsetdash{}{0pt}%
\pgfsys@defobject{currentmarker}{\pgfqpoint{-0.041667in}{-0.041667in}}{\pgfqpoint{0.041667in}{0.041667in}}{%
\pgfpathmoveto{\pgfqpoint{0.000000in}{-0.041667in}}%
\pgfpathcurveto{\pgfqpoint{0.011050in}{-0.041667in}}{\pgfqpoint{0.021649in}{-0.037276in}}{\pgfqpoint{0.029463in}{-0.029463in}}%
\pgfpathcurveto{\pgfqpoint{0.037276in}{-0.021649in}}{\pgfqpoint{0.041667in}{-0.011050in}}{\pgfqpoint{0.041667in}{0.000000in}}%
\pgfpathcurveto{\pgfqpoint{0.041667in}{0.011050in}}{\pgfqpoint{0.037276in}{0.021649in}}{\pgfqpoint{0.029463in}{0.029463in}}%
\pgfpathcurveto{\pgfqpoint{0.021649in}{0.037276in}}{\pgfqpoint{0.011050in}{0.041667in}}{\pgfqpoint{0.000000in}{0.041667in}}%
\pgfpathcurveto{\pgfqpoint{-0.011050in}{0.041667in}}{\pgfqpoint{-0.021649in}{0.037276in}}{\pgfqpoint{-0.029463in}{0.029463in}}%
\pgfpathcurveto{\pgfqpoint{-0.037276in}{0.021649in}}{\pgfqpoint{-0.041667in}{0.011050in}}{\pgfqpoint{-0.041667in}{0.000000in}}%
\pgfpathcurveto{\pgfqpoint{-0.041667in}{-0.011050in}}{\pgfqpoint{-0.037276in}{-0.021649in}}{\pgfqpoint{-0.029463in}{-0.029463in}}%
\pgfpathcurveto{\pgfqpoint{-0.021649in}{-0.037276in}}{\pgfqpoint{-0.011050in}{-0.041667in}}{\pgfqpoint{0.000000in}{-0.041667in}}%
\pgfpathclose%
\pgfusepath{stroke,fill}%
}%
\begin{pgfscope}%
\pgfsys@transformshift{3.378853in}{3.782815in}%
\pgfsys@useobject{currentmarker}{}%
\end{pgfscope}%
\end{pgfscope}%
\begin{pgfscope}%
\definecolor{textcolor}{rgb}{0.000000,0.000000,0.000000}%
\pgfsetstrokecolor{textcolor}%
\pgfsetfillcolor{textcolor}%
\pgftext[x=3.628853in,y=3.746356in,left,base]{\color{textcolor}\rmfamily\fontsize{10.000000}{12.000000}\selectfont Cluster 1}%
\end{pgfscope}%
\begin{pgfscope}%
\pgfsetbuttcap%
\pgfsetroundjoin%
\definecolor{currentfill}{rgb}{0.839216,0.152941,0.156863}%
\pgfsetfillcolor{currentfill}%
\pgfsetlinewidth{1.003750pt}%
\definecolor{currentstroke}{rgb}{0.839216,0.152941,0.156863}%
\pgfsetstrokecolor{currentstroke}%
\pgfsetdash{}{0pt}%
\pgfsys@defobject{currentmarker}{\pgfqpoint{-0.041667in}{-0.041667in}}{\pgfqpoint{0.041667in}{0.041667in}}{%
\pgfpathmoveto{\pgfqpoint{0.000000in}{-0.041667in}}%
\pgfpathcurveto{\pgfqpoint{0.011050in}{-0.041667in}}{\pgfqpoint{0.021649in}{-0.037276in}}{\pgfqpoint{0.029463in}{-0.029463in}}%
\pgfpathcurveto{\pgfqpoint{0.037276in}{-0.021649in}}{\pgfqpoint{0.041667in}{-0.011050in}}{\pgfqpoint{0.041667in}{0.000000in}}%
\pgfpathcurveto{\pgfqpoint{0.041667in}{0.011050in}}{\pgfqpoint{0.037276in}{0.021649in}}{\pgfqpoint{0.029463in}{0.029463in}}%
\pgfpathcurveto{\pgfqpoint{0.021649in}{0.037276in}}{\pgfqpoint{0.011050in}{0.041667in}}{\pgfqpoint{0.000000in}{0.041667in}}%
\pgfpathcurveto{\pgfqpoint{-0.011050in}{0.041667in}}{\pgfqpoint{-0.021649in}{0.037276in}}{\pgfqpoint{-0.029463in}{0.029463in}}%
\pgfpathcurveto{\pgfqpoint{-0.037276in}{0.021649in}}{\pgfqpoint{-0.041667in}{0.011050in}}{\pgfqpoint{-0.041667in}{0.000000in}}%
\pgfpathcurveto{\pgfqpoint{-0.041667in}{-0.011050in}}{\pgfqpoint{-0.037276in}{-0.021649in}}{\pgfqpoint{-0.029463in}{-0.029463in}}%
\pgfpathcurveto{\pgfqpoint{-0.021649in}{-0.037276in}}{\pgfqpoint{-0.011050in}{-0.041667in}}{\pgfqpoint{0.000000in}{-0.041667in}}%
\pgfpathclose%
\pgfusepath{stroke,fill}%
}%
\begin{pgfscope}%
\pgfsys@transformshift{3.378853in}{3.589142in}%
\pgfsys@useobject{currentmarker}{}%
\end{pgfscope}%
\end{pgfscope}%
\begin{pgfscope}%
\definecolor{textcolor}{rgb}{0.000000,0.000000,0.000000}%
\pgfsetstrokecolor{textcolor}%
\pgfsetfillcolor{textcolor}%
\pgftext[x=3.628853in,y=3.552683in,left,base]{\color{textcolor}\rmfamily\fontsize{10.000000}{12.000000}\selectfont Cluster 2}%
\end{pgfscope}%
\begin{pgfscope}%
\pgfsetbuttcap%
\pgfsetroundjoin%
\definecolor{currentfill}{rgb}{1.000000,0.498039,0.054902}%
\pgfsetfillcolor{currentfill}%
\pgfsetlinewidth{1.003750pt}%
\definecolor{currentstroke}{rgb}{1.000000,0.498039,0.054902}%
\pgfsetstrokecolor{currentstroke}%
\pgfsetdash{}{0pt}%
\pgfsys@defobject{currentmarker}{\pgfqpoint{-0.041667in}{-0.041667in}}{\pgfqpoint{0.041667in}{0.041667in}}{%
\pgfpathmoveto{\pgfqpoint{0.000000in}{-0.041667in}}%
\pgfpathcurveto{\pgfqpoint{0.011050in}{-0.041667in}}{\pgfqpoint{0.021649in}{-0.037276in}}{\pgfqpoint{0.029463in}{-0.029463in}}%
\pgfpathcurveto{\pgfqpoint{0.037276in}{-0.021649in}}{\pgfqpoint{0.041667in}{-0.011050in}}{\pgfqpoint{0.041667in}{0.000000in}}%
\pgfpathcurveto{\pgfqpoint{0.041667in}{0.011050in}}{\pgfqpoint{0.037276in}{0.021649in}}{\pgfqpoint{0.029463in}{0.029463in}}%
\pgfpathcurveto{\pgfqpoint{0.021649in}{0.037276in}}{\pgfqpoint{0.011050in}{0.041667in}}{\pgfqpoint{0.000000in}{0.041667in}}%
\pgfpathcurveto{\pgfqpoint{-0.011050in}{0.041667in}}{\pgfqpoint{-0.021649in}{0.037276in}}{\pgfqpoint{-0.029463in}{0.029463in}}%
\pgfpathcurveto{\pgfqpoint{-0.037276in}{0.021649in}}{\pgfqpoint{-0.041667in}{0.011050in}}{\pgfqpoint{-0.041667in}{0.000000in}}%
\pgfpathcurveto{\pgfqpoint{-0.041667in}{-0.011050in}}{\pgfqpoint{-0.037276in}{-0.021649in}}{\pgfqpoint{-0.029463in}{-0.029463in}}%
\pgfpathcurveto{\pgfqpoint{-0.021649in}{-0.037276in}}{\pgfqpoint{-0.011050in}{-0.041667in}}{\pgfqpoint{0.000000in}{-0.041667in}}%
\pgfpathclose%
\pgfusepath{stroke,fill}%
}%
\begin{pgfscope}%
\pgfsys@transformshift{3.378853in}{3.395469in}%
\pgfsys@useobject{currentmarker}{}%
\end{pgfscope}%
\end{pgfscope}%
\begin{pgfscope}%
\definecolor{textcolor}{rgb}{0.000000,0.000000,0.000000}%
\pgfsetstrokecolor{textcolor}%
\pgfsetfillcolor{textcolor}%
\pgftext[x=3.628853in,y=3.359011in,left,base]{\color{textcolor}\rmfamily\fontsize{10.000000}{12.000000}\selectfont Cluster 3}%
\end{pgfscope}%
\begin{pgfscope}%
\pgfsetbuttcap%
\pgfsetroundjoin%
\definecolor{currentfill}{rgb}{0.000000,0.000000,0.000000}%
\pgfsetfillcolor{currentfill}%
\pgfsetlinewidth{0.000000pt}%
\definecolor{currentstroke}{rgb}{1.000000,1.000000,1.000000}%
\pgfsetstrokecolor{currentstroke}%
\pgfsetdash{}{0pt}%
\pgfsys@defobject{currentmarker}{\pgfqpoint{-0.049105in}{-0.049105in}}{\pgfqpoint{0.049105in}{0.049105in}}{%
\pgfpathmoveto{\pgfqpoint{-0.024552in}{-0.049105in}}%
\pgfpathlineto{\pgfqpoint{0.000000in}{-0.024552in}}%
\pgfpathlineto{\pgfqpoint{0.024552in}{-0.049105in}}%
\pgfpathlineto{\pgfqpoint{0.049105in}{-0.024552in}}%
\pgfpathlineto{\pgfqpoint{0.024552in}{0.000000in}}%
\pgfpathlineto{\pgfqpoint{0.049105in}{0.024552in}}%
\pgfpathlineto{\pgfqpoint{0.024552in}{0.049105in}}%
\pgfpathlineto{\pgfqpoint{0.000000in}{0.024552in}}%
\pgfpathlineto{\pgfqpoint{-0.024552in}{0.049105in}}%
\pgfpathlineto{\pgfqpoint{-0.049105in}{0.024552in}}%
\pgfpathlineto{\pgfqpoint{-0.024552in}{0.000000in}}%
\pgfpathlineto{\pgfqpoint{-0.049105in}{-0.024552in}}%
\pgfpathclose%
\pgfusepath{fill}%
}%
\begin{pgfscope}%
\pgfsys@transformshift{3.378853in}{3.201796in}%
\pgfsys@useobject{currentmarker}{}%
\end{pgfscope}%
\end{pgfscope}%
\begin{pgfscope}%
\definecolor{textcolor}{rgb}{0.000000,0.000000,0.000000}%
\pgfsetstrokecolor{textcolor}%
\pgfsetfillcolor{textcolor}%
\pgftext[x=3.628853in,y=3.165338in,left,base]{\color{textcolor}\rmfamily\fontsize{10.000000}{12.000000}\selectfont Centroid}%
\end{pgfscope}%
\begin{pgfscope}%
\pgfsetbuttcap%
\pgfsetmiterjoin%
\definecolor{currentfill}{rgb}{1.000000,1.000000,1.000000}%
\pgfsetfillcolor{currentfill}%
\pgfsetlinewidth{0.000000pt}%
\definecolor{currentstroke}{rgb}{0.000000,0.000000,0.000000}%
\pgfsetstrokecolor{currentstroke}%
\pgfsetstrokeopacity{0.000000}%
\pgfsetdash{}{0pt}%
\pgfpathmoveto{\pgfqpoint{0.692593in}{4.288546in}}%
\pgfpathlineto{\pgfqpoint{4.310567in}{4.288546in}}%
\pgfpathlineto{\pgfqpoint{4.310567in}{4.920022in}}%
\pgfpathlineto{\pgfqpoint{0.692593in}{4.920022in}}%
\pgfpathclose%
\pgfusepath{fill}%
\end{pgfscope}%
\begin{pgfscope}%
\pgfpathrectangle{\pgfqpoint{0.692593in}{4.288546in}}{\pgfqpoint{3.617974in}{0.631476in}}%
\pgfusepath{clip}%
\pgfsetbuttcap%
\pgfsetroundjoin%
\definecolor{currentfill}{rgb}{0.839216,0.152941,0.156863}%
\pgfsetfillcolor{currentfill}%
\pgfsetfillopacity{0.200000}%
\pgfsetlinewidth{1.003750pt}%
\definecolor{currentstroke}{rgb}{0.839216,0.152941,0.156863}%
\pgfsetstrokecolor{currentstroke}%
\pgfsetdash{}{0pt}%
\pgfsys@defobject{currentmarker}{\pgfqpoint{0.876204in}{4.288546in}}{\pgfqpoint{2.013912in}{4.643080in}}{%
\pgfpathmoveto{\pgfqpoint{0.876204in}{4.288547in}}%
\pgfpathlineto{\pgfqpoint{0.876204in}{4.288546in}}%
\pgfpathlineto{\pgfqpoint{0.881921in}{4.288546in}}%
\pgfpathlineto{\pgfqpoint{0.887638in}{4.288546in}}%
\pgfpathlineto{\pgfqpoint{0.893355in}{4.288546in}}%
\pgfpathlineto{\pgfqpoint{0.899073in}{4.288546in}}%
\pgfpathlineto{\pgfqpoint{0.904790in}{4.288546in}}%
\pgfpathlineto{\pgfqpoint{0.910507in}{4.288546in}}%
\pgfpathlineto{\pgfqpoint{0.916224in}{4.288546in}}%
\pgfpathlineto{\pgfqpoint{0.921941in}{4.288546in}}%
\pgfpathlineto{\pgfqpoint{0.927658in}{4.288546in}}%
\pgfpathlineto{\pgfqpoint{0.933375in}{4.288546in}}%
\pgfpathlineto{\pgfqpoint{0.939092in}{4.288546in}}%
\pgfpathlineto{\pgfqpoint{0.944810in}{4.288546in}}%
\pgfpathlineto{\pgfqpoint{0.950527in}{4.288546in}}%
\pgfpathlineto{\pgfqpoint{0.956244in}{4.288546in}}%
\pgfpathlineto{\pgfqpoint{0.961961in}{4.288546in}}%
\pgfpathlineto{\pgfqpoint{0.967678in}{4.288546in}}%
\pgfpathlineto{\pgfqpoint{0.973395in}{4.288546in}}%
\pgfpathlineto{\pgfqpoint{0.979112in}{4.288546in}}%
\pgfpathlineto{\pgfqpoint{0.984829in}{4.288546in}}%
\pgfpathlineto{\pgfqpoint{0.990547in}{4.288546in}}%
\pgfpathlineto{\pgfqpoint{0.996264in}{4.288546in}}%
\pgfpathlineto{\pgfqpoint{1.001981in}{4.288546in}}%
\pgfpathlineto{\pgfqpoint{1.007698in}{4.288546in}}%
\pgfpathlineto{\pgfqpoint{1.013415in}{4.288546in}}%
\pgfpathlineto{\pgfqpoint{1.019132in}{4.288546in}}%
\pgfpathlineto{\pgfqpoint{1.024849in}{4.288546in}}%
\pgfpathlineto{\pgfqpoint{1.030567in}{4.288546in}}%
\pgfpathlineto{\pgfqpoint{1.036284in}{4.288546in}}%
\pgfpathlineto{\pgfqpoint{1.042001in}{4.288546in}}%
\pgfpathlineto{\pgfqpoint{1.047718in}{4.288546in}}%
\pgfpathlineto{\pgfqpoint{1.053435in}{4.288546in}}%
\pgfpathlineto{\pgfqpoint{1.059152in}{4.288546in}}%
\pgfpathlineto{\pgfqpoint{1.064869in}{4.288546in}}%
\pgfpathlineto{\pgfqpoint{1.070586in}{4.288546in}}%
\pgfpathlineto{\pgfqpoint{1.076304in}{4.288546in}}%
\pgfpathlineto{\pgfqpoint{1.082021in}{4.288546in}}%
\pgfpathlineto{\pgfqpoint{1.087738in}{4.288546in}}%
\pgfpathlineto{\pgfqpoint{1.093455in}{4.288546in}}%
\pgfpathlineto{\pgfqpoint{1.099172in}{4.288546in}}%
\pgfpathlineto{\pgfqpoint{1.104889in}{4.288546in}}%
\pgfpathlineto{\pgfqpoint{1.110606in}{4.288546in}}%
\pgfpathlineto{\pgfqpoint{1.116323in}{4.288546in}}%
\pgfpathlineto{\pgfqpoint{1.122041in}{4.288546in}}%
\pgfpathlineto{\pgfqpoint{1.127758in}{4.288546in}}%
\pgfpathlineto{\pgfqpoint{1.133475in}{4.288546in}}%
\pgfpathlineto{\pgfqpoint{1.139192in}{4.288546in}}%
\pgfpathlineto{\pgfqpoint{1.144909in}{4.288546in}}%
\pgfpathlineto{\pgfqpoint{1.150626in}{4.288546in}}%
\pgfpathlineto{\pgfqpoint{1.156343in}{4.288546in}}%
\pgfpathlineto{\pgfqpoint{1.162060in}{4.288546in}}%
\pgfpathlineto{\pgfqpoint{1.167778in}{4.288546in}}%
\pgfpathlineto{\pgfqpoint{1.173495in}{4.288546in}}%
\pgfpathlineto{\pgfqpoint{1.179212in}{4.288546in}}%
\pgfpathlineto{\pgfqpoint{1.184929in}{4.288546in}}%
\pgfpathlineto{\pgfqpoint{1.190646in}{4.288546in}}%
\pgfpathlineto{\pgfqpoint{1.196363in}{4.288546in}}%
\pgfpathlineto{\pgfqpoint{1.202080in}{4.288546in}}%
\pgfpathlineto{\pgfqpoint{1.207797in}{4.288546in}}%
\pgfpathlineto{\pgfqpoint{1.213515in}{4.288546in}}%
\pgfpathlineto{\pgfqpoint{1.219232in}{4.288546in}}%
\pgfpathlineto{\pgfqpoint{1.224949in}{4.288546in}}%
\pgfpathlineto{\pgfqpoint{1.230666in}{4.288546in}}%
\pgfpathlineto{\pgfqpoint{1.236383in}{4.288546in}}%
\pgfpathlineto{\pgfqpoint{1.242100in}{4.288546in}}%
\pgfpathlineto{\pgfqpoint{1.247817in}{4.288546in}}%
\pgfpathlineto{\pgfqpoint{1.253534in}{4.288546in}}%
\pgfpathlineto{\pgfqpoint{1.259252in}{4.288546in}}%
\pgfpathlineto{\pgfqpoint{1.264969in}{4.288546in}}%
\pgfpathlineto{\pgfqpoint{1.270686in}{4.288546in}}%
\pgfpathlineto{\pgfqpoint{1.276403in}{4.288546in}}%
\pgfpathlineto{\pgfqpoint{1.282120in}{4.288546in}}%
\pgfpathlineto{\pgfqpoint{1.287837in}{4.288546in}}%
\pgfpathlineto{\pgfqpoint{1.293554in}{4.288546in}}%
\pgfpathlineto{\pgfqpoint{1.299271in}{4.288546in}}%
\pgfpathlineto{\pgfqpoint{1.304989in}{4.288546in}}%
\pgfpathlineto{\pgfqpoint{1.310706in}{4.288546in}}%
\pgfpathlineto{\pgfqpoint{1.316423in}{4.288546in}}%
\pgfpathlineto{\pgfqpoint{1.322140in}{4.288546in}}%
\pgfpathlineto{\pgfqpoint{1.327857in}{4.288546in}}%
\pgfpathlineto{\pgfqpoint{1.333574in}{4.288546in}}%
\pgfpathlineto{\pgfqpoint{1.339291in}{4.288546in}}%
\pgfpathlineto{\pgfqpoint{1.345008in}{4.288546in}}%
\pgfpathlineto{\pgfqpoint{1.350726in}{4.288546in}}%
\pgfpathlineto{\pgfqpoint{1.356443in}{4.288546in}}%
\pgfpathlineto{\pgfqpoint{1.362160in}{4.288546in}}%
\pgfpathlineto{\pgfqpoint{1.367877in}{4.288546in}}%
\pgfpathlineto{\pgfqpoint{1.373594in}{4.288546in}}%
\pgfpathlineto{\pgfqpoint{1.379311in}{4.288546in}}%
\pgfpathlineto{\pgfqpoint{1.385028in}{4.288546in}}%
\pgfpathlineto{\pgfqpoint{1.390745in}{4.288546in}}%
\pgfpathlineto{\pgfqpoint{1.396463in}{4.288546in}}%
\pgfpathlineto{\pgfqpoint{1.402180in}{4.288546in}}%
\pgfpathlineto{\pgfqpoint{1.407897in}{4.288546in}}%
\pgfpathlineto{\pgfqpoint{1.413614in}{4.288546in}}%
\pgfpathlineto{\pgfqpoint{1.419331in}{4.288546in}}%
\pgfpathlineto{\pgfqpoint{1.425048in}{4.288546in}}%
\pgfpathlineto{\pgfqpoint{1.430765in}{4.288546in}}%
\pgfpathlineto{\pgfqpoint{1.436483in}{4.288546in}}%
\pgfpathlineto{\pgfqpoint{1.442200in}{4.288546in}}%
\pgfpathlineto{\pgfqpoint{1.447917in}{4.288546in}}%
\pgfpathlineto{\pgfqpoint{1.453634in}{4.288546in}}%
\pgfpathlineto{\pgfqpoint{1.459351in}{4.288546in}}%
\pgfpathlineto{\pgfqpoint{1.465068in}{4.288546in}}%
\pgfpathlineto{\pgfqpoint{1.470785in}{4.288546in}}%
\pgfpathlineto{\pgfqpoint{1.476502in}{4.288546in}}%
\pgfpathlineto{\pgfqpoint{1.482220in}{4.288546in}}%
\pgfpathlineto{\pgfqpoint{1.487937in}{4.288546in}}%
\pgfpathlineto{\pgfqpoint{1.493654in}{4.288546in}}%
\pgfpathlineto{\pgfqpoint{1.499371in}{4.288546in}}%
\pgfpathlineto{\pgfqpoint{1.505088in}{4.288546in}}%
\pgfpathlineto{\pgfqpoint{1.510805in}{4.288546in}}%
\pgfpathlineto{\pgfqpoint{1.516522in}{4.288546in}}%
\pgfpathlineto{\pgfqpoint{1.522239in}{4.288546in}}%
\pgfpathlineto{\pgfqpoint{1.527957in}{4.288546in}}%
\pgfpathlineto{\pgfqpoint{1.533674in}{4.288546in}}%
\pgfpathlineto{\pgfqpoint{1.539391in}{4.288546in}}%
\pgfpathlineto{\pgfqpoint{1.545108in}{4.288546in}}%
\pgfpathlineto{\pgfqpoint{1.550825in}{4.288546in}}%
\pgfpathlineto{\pgfqpoint{1.556542in}{4.288546in}}%
\pgfpathlineto{\pgfqpoint{1.562259in}{4.288546in}}%
\pgfpathlineto{\pgfqpoint{1.567976in}{4.288546in}}%
\pgfpathlineto{\pgfqpoint{1.573694in}{4.288546in}}%
\pgfpathlineto{\pgfqpoint{1.579411in}{4.288546in}}%
\pgfpathlineto{\pgfqpoint{1.585128in}{4.288546in}}%
\pgfpathlineto{\pgfqpoint{1.590845in}{4.288546in}}%
\pgfpathlineto{\pgfqpoint{1.596562in}{4.288546in}}%
\pgfpathlineto{\pgfqpoint{1.602279in}{4.288546in}}%
\pgfpathlineto{\pgfqpoint{1.607996in}{4.288546in}}%
\pgfpathlineto{\pgfqpoint{1.613713in}{4.288546in}}%
\pgfpathlineto{\pgfqpoint{1.619431in}{4.288546in}}%
\pgfpathlineto{\pgfqpoint{1.625148in}{4.288546in}}%
\pgfpathlineto{\pgfqpoint{1.630865in}{4.288546in}}%
\pgfpathlineto{\pgfqpoint{1.636582in}{4.288546in}}%
\pgfpathlineto{\pgfqpoint{1.642299in}{4.288546in}}%
\pgfpathlineto{\pgfqpoint{1.648016in}{4.288546in}}%
\pgfpathlineto{\pgfqpoint{1.653733in}{4.288546in}}%
\pgfpathlineto{\pgfqpoint{1.659450in}{4.288546in}}%
\pgfpathlineto{\pgfqpoint{1.665168in}{4.288546in}}%
\pgfpathlineto{\pgfqpoint{1.670885in}{4.288546in}}%
\pgfpathlineto{\pgfqpoint{1.676602in}{4.288546in}}%
\pgfpathlineto{\pgfqpoint{1.682319in}{4.288546in}}%
\pgfpathlineto{\pgfqpoint{1.688036in}{4.288546in}}%
\pgfpathlineto{\pgfqpoint{1.693753in}{4.288546in}}%
\pgfpathlineto{\pgfqpoint{1.699470in}{4.288546in}}%
\pgfpathlineto{\pgfqpoint{1.705187in}{4.288546in}}%
\pgfpathlineto{\pgfqpoint{1.710905in}{4.288546in}}%
\pgfpathlineto{\pgfqpoint{1.716622in}{4.288546in}}%
\pgfpathlineto{\pgfqpoint{1.722339in}{4.288546in}}%
\pgfpathlineto{\pgfqpoint{1.728056in}{4.288546in}}%
\pgfpathlineto{\pgfqpoint{1.733773in}{4.288546in}}%
\pgfpathlineto{\pgfqpoint{1.739490in}{4.288546in}}%
\pgfpathlineto{\pgfqpoint{1.745207in}{4.288546in}}%
\pgfpathlineto{\pgfqpoint{1.750924in}{4.288546in}}%
\pgfpathlineto{\pgfqpoint{1.756642in}{4.288546in}}%
\pgfpathlineto{\pgfqpoint{1.762359in}{4.288546in}}%
\pgfpathlineto{\pgfqpoint{1.768076in}{4.288546in}}%
\pgfpathlineto{\pgfqpoint{1.773793in}{4.288546in}}%
\pgfpathlineto{\pgfqpoint{1.779510in}{4.288546in}}%
\pgfpathlineto{\pgfqpoint{1.785227in}{4.288546in}}%
\pgfpathlineto{\pgfqpoint{1.790944in}{4.288546in}}%
\pgfpathlineto{\pgfqpoint{1.796661in}{4.288546in}}%
\pgfpathlineto{\pgfqpoint{1.802379in}{4.288546in}}%
\pgfpathlineto{\pgfqpoint{1.808096in}{4.288546in}}%
\pgfpathlineto{\pgfqpoint{1.813813in}{4.288546in}}%
\pgfpathlineto{\pgfqpoint{1.819530in}{4.288546in}}%
\pgfpathlineto{\pgfqpoint{1.825247in}{4.288546in}}%
\pgfpathlineto{\pgfqpoint{1.830964in}{4.288546in}}%
\pgfpathlineto{\pgfqpoint{1.836681in}{4.288546in}}%
\pgfpathlineto{\pgfqpoint{1.842398in}{4.288546in}}%
\pgfpathlineto{\pgfqpoint{1.848116in}{4.288546in}}%
\pgfpathlineto{\pgfqpoint{1.853833in}{4.288546in}}%
\pgfpathlineto{\pgfqpoint{1.859550in}{4.288546in}}%
\pgfpathlineto{\pgfqpoint{1.865267in}{4.288546in}}%
\pgfpathlineto{\pgfqpoint{1.870984in}{4.288546in}}%
\pgfpathlineto{\pgfqpoint{1.876701in}{4.288546in}}%
\pgfpathlineto{\pgfqpoint{1.882418in}{4.288546in}}%
\pgfpathlineto{\pgfqpoint{1.888136in}{4.288546in}}%
\pgfpathlineto{\pgfqpoint{1.893853in}{4.288546in}}%
\pgfpathlineto{\pgfqpoint{1.899570in}{4.288546in}}%
\pgfpathlineto{\pgfqpoint{1.905287in}{4.288546in}}%
\pgfpathlineto{\pgfqpoint{1.911004in}{4.288546in}}%
\pgfpathlineto{\pgfqpoint{1.916721in}{4.288546in}}%
\pgfpathlineto{\pgfqpoint{1.922438in}{4.288546in}}%
\pgfpathlineto{\pgfqpoint{1.928155in}{4.288546in}}%
\pgfpathlineto{\pgfqpoint{1.933873in}{4.288546in}}%
\pgfpathlineto{\pgfqpoint{1.939590in}{4.288546in}}%
\pgfpathlineto{\pgfqpoint{1.945307in}{4.288546in}}%
\pgfpathlineto{\pgfqpoint{1.951024in}{4.288546in}}%
\pgfpathlineto{\pgfqpoint{1.956741in}{4.288546in}}%
\pgfpathlineto{\pgfqpoint{1.962458in}{4.288546in}}%
\pgfpathlineto{\pgfqpoint{1.968175in}{4.288546in}}%
\pgfpathlineto{\pgfqpoint{1.973892in}{4.288546in}}%
\pgfpathlineto{\pgfqpoint{1.979610in}{4.288546in}}%
\pgfpathlineto{\pgfqpoint{1.985327in}{4.288546in}}%
\pgfpathlineto{\pgfqpoint{1.991044in}{4.288546in}}%
\pgfpathlineto{\pgfqpoint{1.996761in}{4.288546in}}%
\pgfpathlineto{\pgfqpoint{2.002478in}{4.288546in}}%
\pgfpathlineto{\pgfqpoint{2.008195in}{4.288546in}}%
\pgfpathlineto{\pgfqpoint{2.013912in}{4.288546in}}%
\pgfpathlineto{\pgfqpoint{2.013912in}{4.288546in}}%
\pgfpathlineto{\pgfqpoint{2.013912in}{4.288546in}}%
\pgfpathlineto{\pgfqpoint{2.008195in}{4.288547in}}%
\pgfpathlineto{\pgfqpoint{2.002478in}{4.288551in}}%
\pgfpathlineto{\pgfqpoint{1.996761in}{4.288552in}}%
\pgfpathlineto{\pgfqpoint{1.991044in}{4.288549in}}%
\pgfpathlineto{\pgfqpoint{1.985327in}{4.288547in}}%
\pgfpathlineto{\pgfqpoint{1.979610in}{4.288546in}}%
\pgfpathlineto{\pgfqpoint{1.973892in}{4.288546in}}%
\pgfpathlineto{\pgfqpoint{1.968175in}{4.288546in}}%
\pgfpathlineto{\pgfqpoint{1.962458in}{4.288546in}}%
\pgfpathlineto{\pgfqpoint{1.956741in}{4.288546in}}%
\pgfpathlineto{\pgfqpoint{1.951024in}{4.288546in}}%
\pgfpathlineto{\pgfqpoint{1.945307in}{4.288546in}}%
\pgfpathlineto{\pgfqpoint{1.939590in}{4.288546in}}%
\pgfpathlineto{\pgfqpoint{1.933873in}{4.288546in}}%
\pgfpathlineto{\pgfqpoint{1.928155in}{4.288546in}}%
\pgfpathlineto{\pgfqpoint{1.922438in}{4.288546in}}%
\pgfpathlineto{\pgfqpoint{1.916721in}{4.288547in}}%
\pgfpathlineto{\pgfqpoint{1.911004in}{4.288548in}}%
\pgfpathlineto{\pgfqpoint{1.905287in}{4.288556in}}%
\pgfpathlineto{\pgfqpoint{1.899570in}{4.288564in}}%
\pgfpathlineto{\pgfqpoint{1.893853in}{4.288557in}}%
\pgfpathlineto{\pgfqpoint{1.888136in}{4.288548in}}%
\pgfpathlineto{\pgfqpoint{1.882418in}{4.288547in}}%
\pgfpathlineto{\pgfqpoint{1.876701in}{4.288546in}}%
\pgfpathlineto{\pgfqpoint{1.870984in}{4.288546in}}%
\pgfpathlineto{\pgfqpoint{1.865267in}{4.288546in}}%
\pgfpathlineto{\pgfqpoint{1.859550in}{4.288546in}}%
\pgfpathlineto{\pgfqpoint{1.853833in}{4.288546in}}%
\pgfpathlineto{\pgfqpoint{1.848116in}{4.288546in}}%
\pgfpathlineto{\pgfqpoint{1.842398in}{4.288546in}}%
\pgfpathlineto{\pgfqpoint{1.836681in}{4.288547in}}%
\pgfpathlineto{\pgfqpoint{1.830964in}{4.288549in}}%
\pgfpathlineto{\pgfqpoint{1.825247in}{4.288552in}}%
\pgfpathlineto{\pgfqpoint{1.819530in}{4.288551in}}%
\pgfpathlineto{\pgfqpoint{1.813813in}{4.288547in}}%
\pgfpathlineto{\pgfqpoint{1.808096in}{4.288548in}}%
\pgfpathlineto{\pgfqpoint{1.802379in}{4.288551in}}%
\pgfpathlineto{\pgfqpoint{1.796661in}{4.288552in}}%
\pgfpathlineto{\pgfqpoint{1.790944in}{4.288548in}}%
\pgfpathlineto{\pgfqpoint{1.785227in}{4.288547in}}%
\pgfpathlineto{\pgfqpoint{1.779510in}{4.288546in}}%
\pgfpathlineto{\pgfqpoint{1.773793in}{4.288546in}}%
\pgfpathlineto{\pgfqpoint{1.768076in}{4.288547in}}%
\pgfpathlineto{\pgfqpoint{1.762359in}{4.288549in}}%
\pgfpathlineto{\pgfqpoint{1.756642in}{4.288552in}}%
\pgfpathlineto{\pgfqpoint{1.750924in}{4.288550in}}%
\pgfpathlineto{\pgfqpoint{1.745207in}{4.288547in}}%
\pgfpathlineto{\pgfqpoint{1.739490in}{4.288549in}}%
\pgfpathlineto{\pgfqpoint{1.733773in}{4.288565in}}%
\pgfpathlineto{\pgfqpoint{1.728056in}{4.288599in}}%
\pgfpathlineto{\pgfqpoint{1.722339in}{4.288601in}}%
\pgfpathlineto{\pgfqpoint{1.716622in}{4.288581in}}%
\pgfpathlineto{\pgfqpoint{1.710905in}{4.288569in}}%
\pgfpathlineto{\pgfqpoint{1.705187in}{4.288559in}}%
\pgfpathlineto{\pgfqpoint{1.699470in}{4.288560in}}%
\pgfpathlineto{\pgfqpoint{1.693753in}{4.288563in}}%
\pgfpathlineto{\pgfqpoint{1.688036in}{4.288580in}}%
\pgfpathlineto{\pgfqpoint{1.682319in}{4.288606in}}%
\pgfpathlineto{\pgfqpoint{1.676602in}{4.288593in}}%
\pgfpathlineto{\pgfqpoint{1.670885in}{4.288567in}}%
\pgfpathlineto{\pgfqpoint{1.665168in}{4.288553in}}%
\pgfpathlineto{\pgfqpoint{1.659450in}{4.288554in}}%
\pgfpathlineto{\pgfqpoint{1.653733in}{4.288567in}}%
\pgfpathlineto{\pgfqpoint{1.648016in}{4.288582in}}%
\pgfpathlineto{\pgfqpoint{1.642299in}{4.288598in}}%
\pgfpathlineto{\pgfqpoint{1.636582in}{4.288610in}}%
\pgfpathlineto{\pgfqpoint{1.630865in}{4.288597in}}%
\pgfpathlineto{\pgfqpoint{1.625148in}{4.288583in}}%
\pgfpathlineto{\pgfqpoint{1.619431in}{4.288623in}}%
\pgfpathlineto{\pgfqpoint{1.613713in}{4.288725in}}%
\pgfpathlineto{\pgfqpoint{1.607996in}{4.288807in}}%
\pgfpathlineto{\pgfqpoint{1.602279in}{4.288850in}}%
\pgfpathlineto{\pgfqpoint{1.596562in}{4.288920in}}%
\pgfpathlineto{\pgfqpoint{1.590845in}{4.288968in}}%
\pgfpathlineto{\pgfqpoint{1.585128in}{4.288823in}}%
\pgfpathlineto{\pgfqpoint{1.579411in}{4.288692in}}%
\pgfpathlineto{\pgfqpoint{1.573694in}{4.288749in}}%
\pgfpathlineto{\pgfqpoint{1.567976in}{4.288811in}}%
\pgfpathlineto{\pgfqpoint{1.562259in}{4.288858in}}%
\pgfpathlineto{\pgfqpoint{1.556542in}{4.288771in}}%
\pgfpathlineto{\pgfqpoint{1.550825in}{4.288619in}}%
\pgfpathlineto{\pgfqpoint{1.545108in}{4.288588in}}%
\pgfpathlineto{\pgfqpoint{1.539391in}{4.288649in}}%
\pgfpathlineto{\pgfqpoint{1.533674in}{4.288774in}}%
\pgfpathlineto{\pgfqpoint{1.527957in}{4.288803in}}%
\pgfpathlineto{\pgfqpoint{1.522239in}{4.288757in}}%
\pgfpathlineto{\pgfqpoint{1.516522in}{4.288761in}}%
\pgfpathlineto{\pgfqpoint{1.510805in}{4.288723in}}%
\pgfpathlineto{\pgfqpoint{1.505088in}{4.288684in}}%
\pgfpathlineto{\pgfqpoint{1.499371in}{4.288708in}}%
\pgfpathlineto{\pgfqpoint{1.493654in}{4.288741in}}%
\pgfpathlineto{\pgfqpoint{1.487937in}{4.288739in}}%
\pgfpathlineto{\pgfqpoint{1.482220in}{4.288698in}}%
\pgfpathlineto{\pgfqpoint{1.476502in}{4.288678in}}%
\pgfpathlineto{\pgfqpoint{1.470785in}{4.288679in}}%
\pgfpathlineto{\pgfqpoint{1.465068in}{4.288684in}}%
\pgfpathlineto{\pgfqpoint{1.459351in}{4.288714in}}%
\pgfpathlineto{\pgfqpoint{1.453634in}{4.288817in}}%
\pgfpathlineto{\pgfqpoint{1.447917in}{4.289030in}}%
\pgfpathlineto{\pgfqpoint{1.442200in}{4.289284in}}%
\pgfpathlineto{\pgfqpoint{1.436483in}{4.289339in}}%
\pgfpathlineto{\pgfqpoint{1.430765in}{4.289200in}}%
\pgfpathlineto{\pgfqpoint{1.425048in}{4.289036in}}%
\pgfpathlineto{\pgfqpoint{1.419331in}{4.289062in}}%
\pgfpathlineto{\pgfqpoint{1.413614in}{4.289354in}}%
\pgfpathlineto{\pgfqpoint{1.407897in}{4.289461in}}%
\pgfpathlineto{\pgfqpoint{1.402180in}{4.289290in}}%
\pgfpathlineto{\pgfqpoint{1.396463in}{4.289293in}}%
\pgfpathlineto{\pgfqpoint{1.390745in}{4.289537in}}%
\pgfpathlineto{\pgfqpoint{1.385028in}{4.290030in}}%
\pgfpathlineto{\pgfqpoint{1.379311in}{4.291550in}}%
\pgfpathlineto{\pgfqpoint{1.373594in}{4.297093in}}%
\pgfpathlineto{\pgfqpoint{1.367877in}{4.303887in}}%
\pgfpathlineto{\pgfqpoint{1.362160in}{4.301319in}}%
\pgfpathlineto{\pgfqpoint{1.356443in}{4.312049in}}%
\pgfpathlineto{\pgfqpoint{1.350726in}{4.360161in}}%
\pgfpathlineto{\pgfqpoint{1.345008in}{4.379176in}}%
\pgfpathlineto{\pgfqpoint{1.339291in}{4.335277in}}%
\pgfpathlineto{\pgfqpoint{1.333574in}{4.317337in}}%
\pgfpathlineto{\pgfqpoint{1.327857in}{4.332625in}}%
\pgfpathlineto{\pgfqpoint{1.322140in}{4.329787in}}%
\pgfpathlineto{\pgfqpoint{1.316423in}{4.322217in}}%
\pgfpathlineto{\pgfqpoint{1.310706in}{4.370027in}}%
\pgfpathlineto{\pgfqpoint{1.304989in}{4.468102in}}%
\pgfpathlineto{\pgfqpoint{1.299271in}{4.564038in}}%
\pgfpathlineto{\pgfqpoint{1.293554in}{4.525605in}}%
\pgfpathlineto{\pgfqpoint{1.287837in}{4.395309in}}%
\pgfpathlineto{\pgfqpoint{1.282120in}{4.336996in}}%
\pgfpathlineto{\pgfqpoint{1.276403in}{4.330995in}}%
\pgfpathlineto{\pgfqpoint{1.270686in}{4.336174in}}%
\pgfpathlineto{\pgfqpoint{1.264969in}{4.373707in}}%
\pgfpathlineto{\pgfqpoint{1.259252in}{4.518051in}}%
\pgfpathlineto{\pgfqpoint{1.253534in}{4.643080in}}%
\pgfpathlineto{\pgfqpoint{1.247817in}{4.640610in}}%
\pgfpathlineto{\pgfqpoint{1.242100in}{4.555330in}}%
\pgfpathlineto{\pgfqpoint{1.236383in}{4.460934in}}%
\pgfpathlineto{\pgfqpoint{1.230666in}{4.407492in}}%
\pgfpathlineto{\pgfqpoint{1.224949in}{4.443933in}}%
\pgfpathlineto{\pgfqpoint{1.219232in}{4.557634in}}%
\pgfpathlineto{\pgfqpoint{1.213515in}{4.639202in}}%
\pgfpathlineto{\pgfqpoint{1.207797in}{4.642923in}}%
\pgfpathlineto{\pgfqpoint{1.202080in}{4.527886in}}%
\pgfpathlineto{\pgfqpoint{1.196363in}{4.465477in}}%
\pgfpathlineto{\pgfqpoint{1.190646in}{4.478620in}}%
\pgfpathlineto{\pgfqpoint{1.184929in}{4.437250in}}%
\pgfpathlineto{\pgfqpoint{1.179212in}{4.359081in}}%
\pgfpathlineto{\pgfqpoint{1.173495in}{4.318468in}}%
\pgfpathlineto{\pgfqpoint{1.167778in}{4.315191in}}%
\pgfpathlineto{\pgfqpoint{1.162060in}{4.312740in}}%
\pgfpathlineto{\pgfqpoint{1.156343in}{4.311675in}}%
\pgfpathlineto{\pgfqpoint{1.150626in}{4.328812in}}%
\pgfpathlineto{\pgfqpoint{1.144909in}{4.368574in}}%
\pgfpathlineto{\pgfqpoint{1.139192in}{4.399083in}}%
\pgfpathlineto{\pgfqpoint{1.133475in}{4.425814in}}%
\pgfpathlineto{\pgfqpoint{1.127758in}{4.488716in}}%
\pgfpathlineto{\pgfqpoint{1.122041in}{4.537870in}}%
\pgfpathlineto{\pgfqpoint{1.116323in}{4.510918in}}%
\pgfpathlineto{\pgfqpoint{1.110606in}{4.491995in}}%
\pgfpathlineto{\pgfqpoint{1.104889in}{4.489193in}}%
\pgfpathlineto{\pgfqpoint{1.099172in}{4.426061in}}%
\pgfpathlineto{\pgfqpoint{1.093455in}{4.348728in}}%
\pgfpathlineto{\pgfqpoint{1.087738in}{4.305072in}}%
\pgfpathlineto{\pgfqpoint{1.082021in}{4.303892in}}%
\pgfpathlineto{\pgfqpoint{1.076304in}{4.328738in}}%
\pgfpathlineto{\pgfqpoint{1.070586in}{4.334410in}}%
\pgfpathlineto{\pgfqpoint{1.064869in}{4.331233in}}%
\pgfpathlineto{\pgfqpoint{1.059152in}{4.329105in}}%
\pgfpathlineto{\pgfqpoint{1.053435in}{4.306208in}}%
\pgfpathlineto{\pgfqpoint{1.047718in}{4.291182in}}%
\pgfpathlineto{\pgfqpoint{1.042001in}{4.288767in}}%
\pgfpathlineto{\pgfqpoint{1.036284in}{4.288733in}}%
\pgfpathlineto{\pgfqpoint{1.030567in}{4.288839in}}%
\pgfpathlineto{\pgfqpoint{1.024849in}{4.288781in}}%
\pgfpathlineto{\pgfqpoint{1.019132in}{4.288709in}}%
\pgfpathlineto{\pgfqpoint{1.013415in}{4.288815in}}%
\pgfpathlineto{\pgfqpoint{1.007698in}{4.288872in}}%
\pgfpathlineto{\pgfqpoint{1.001981in}{4.288763in}}%
\pgfpathlineto{\pgfqpoint{0.996264in}{4.288631in}}%
\pgfpathlineto{\pgfqpoint{0.990547in}{4.288586in}}%
\pgfpathlineto{\pgfqpoint{0.984829in}{4.288602in}}%
\pgfpathlineto{\pgfqpoint{0.979112in}{4.288600in}}%
\pgfpathlineto{\pgfqpoint{0.973395in}{4.288593in}}%
\pgfpathlineto{\pgfqpoint{0.967678in}{4.288623in}}%
\pgfpathlineto{\pgfqpoint{0.961961in}{4.288612in}}%
\pgfpathlineto{\pgfqpoint{0.956244in}{4.288655in}}%
\pgfpathlineto{\pgfqpoint{0.950527in}{4.288776in}}%
\pgfpathlineto{\pgfqpoint{0.944810in}{4.288717in}}%
\pgfpathlineto{\pgfqpoint{0.939092in}{4.288602in}}%
\pgfpathlineto{\pgfqpoint{0.933375in}{4.288570in}}%
\pgfpathlineto{\pgfqpoint{0.927658in}{4.288554in}}%
\pgfpathlineto{\pgfqpoint{0.921941in}{4.288547in}}%
\pgfpathlineto{\pgfqpoint{0.916224in}{4.288546in}}%
\pgfpathlineto{\pgfqpoint{0.910507in}{4.288546in}}%
\pgfpathlineto{\pgfqpoint{0.904790in}{4.288547in}}%
\pgfpathlineto{\pgfqpoint{0.899073in}{4.288554in}}%
\pgfpathlineto{\pgfqpoint{0.893355in}{4.288563in}}%
\pgfpathlineto{\pgfqpoint{0.887638in}{4.288559in}}%
\pgfpathlineto{\pgfqpoint{0.881921in}{4.288549in}}%
\pgfpathlineto{\pgfqpoint{0.876204in}{4.288547in}}%
\pgfpathclose%
\pgfusepath{stroke,fill}%
}%
\begin{pgfscope}%
\pgfsys@transformshift{0.000000in}{0.000000in}%
\pgfsys@useobject{currentmarker}{}%
\end{pgfscope}%
\end{pgfscope}%
\begin{pgfscope}%
\pgfpathrectangle{\pgfqpoint{0.692593in}{4.288546in}}{\pgfqpoint{3.617974in}{0.631476in}}%
\pgfusepath{clip}%
\pgfsetbuttcap%
\pgfsetroundjoin%
\definecolor{currentfill}{rgb}{0.172549,0.627451,0.172549}%
\pgfsetfillcolor{currentfill}%
\pgfsetfillopacity{0.200000}%
\pgfsetlinewidth{1.003750pt}%
\definecolor{currentstroke}{rgb}{0.172549,0.627451,0.172549}%
\pgfsetstrokecolor{currentstroke}%
\pgfsetdash{}{0pt}%
\pgfsys@defobject{currentmarker}{\pgfqpoint{0.857047in}{4.288546in}}{\pgfqpoint{3.007590in}{4.320270in}}{%
\pgfpathmoveto{\pgfqpoint{0.857047in}{4.288546in}}%
\pgfpathlineto{\pgfqpoint{0.857047in}{4.288546in}}%
\pgfpathlineto{\pgfqpoint{0.867854in}{4.288546in}}%
\pgfpathlineto{\pgfqpoint{0.878660in}{4.288546in}}%
\pgfpathlineto{\pgfqpoint{0.889467in}{4.288546in}}%
\pgfpathlineto{\pgfqpoint{0.900274in}{4.288546in}}%
\pgfpathlineto{\pgfqpoint{0.911081in}{4.288546in}}%
\pgfpathlineto{\pgfqpoint{0.921887in}{4.288546in}}%
\pgfpathlineto{\pgfqpoint{0.932694in}{4.288546in}}%
\pgfpathlineto{\pgfqpoint{0.943501in}{4.288546in}}%
\pgfpathlineto{\pgfqpoint{0.954308in}{4.288546in}}%
\pgfpathlineto{\pgfqpoint{0.965114in}{4.288546in}}%
\pgfpathlineto{\pgfqpoint{0.975921in}{4.288546in}}%
\pgfpathlineto{\pgfqpoint{0.986728in}{4.288546in}}%
\pgfpathlineto{\pgfqpoint{0.997535in}{4.288546in}}%
\pgfpathlineto{\pgfqpoint{1.008341in}{4.288546in}}%
\pgfpathlineto{\pgfqpoint{1.019148in}{4.288546in}}%
\pgfpathlineto{\pgfqpoint{1.029955in}{4.288546in}}%
\pgfpathlineto{\pgfqpoint{1.040762in}{4.288546in}}%
\pgfpathlineto{\pgfqpoint{1.051568in}{4.288546in}}%
\pgfpathlineto{\pgfqpoint{1.062375in}{4.288546in}}%
\pgfpathlineto{\pgfqpoint{1.073182in}{4.288546in}}%
\pgfpathlineto{\pgfqpoint{1.083989in}{4.288546in}}%
\pgfpathlineto{\pgfqpoint{1.094795in}{4.288546in}}%
\pgfpathlineto{\pgfqpoint{1.105602in}{4.288546in}}%
\pgfpathlineto{\pgfqpoint{1.116409in}{4.288546in}}%
\pgfpathlineto{\pgfqpoint{1.127216in}{4.288546in}}%
\pgfpathlineto{\pgfqpoint{1.138022in}{4.288546in}}%
\pgfpathlineto{\pgfqpoint{1.148829in}{4.288546in}}%
\pgfpathlineto{\pgfqpoint{1.159636in}{4.288546in}}%
\pgfpathlineto{\pgfqpoint{1.170443in}{4.288546in}}%
\pgfpathlineto{\pgfqpoint{1.181249in}{4.288546in}}%
\pgfpathlineto{\pgfqpoint{1.192056in}{4.288546in}}%
\pgfpathlineto{\pgfqpoint{1.202863in}{4.288546in}}%
\pgfpathlineto{\pgfqpoint{1.213670in}{4.288546in}}%
\pgfpathlineto{\pgfqpoint{1.224476in}{4.288546in}}%
\pgfpathlineto{\pgfqpoint{1.235283in}{4.288546in}}%
\pgfpathlineto{\pgfqpoint{1.246090in}{4.288546in}}%
\pgfpathlineto{\pgfqpoint{1.256897in}{4.288546in}}%
\pgfpathlineto{\pgfqpoint{1.267703in}{4.288546in}}%
\pgfpathlineto{\pgfqpoint{1.278510in}{4.288546in}}%
\pgfpathlineto{\pgfqpoint{1.289317in}{4.288546in}}%
\pgfpathlineto{\pgfqpoint{1.300124in}{4.288546in}}%
\pgfpathlineto{\pgfqpoint{1.310930in}{4.288546in}}%
\pgfpathlineto{\pgfqpoint{1.321737in}{4.288546in}}%
\pgfpathlineto{\pgfqpoint{1.332544in}{4.288546in}}%
\pgfpathlineto{\pgfqpoint{1.343351in}{4.288546in}}%
\pgfpathlineto{\pgfqpoint{1.354157in}{4.288546in}}%
\pgfpathlineto{\pgfqpoint{1.364964in}{4.288546in}}%
\pgfpathlineto{\pgfqpoint{1.375771in}{4.288546in}}%
\pgfpathlineto{\pgfqpoint{1.386578in}{4.288546in}}%
\pgfpathlineto{\pgfqpoint{1.397384in}{4.288546in}}%
\pgfpathlineto{\pgfqpoint{1.408191in}{4.288546in}}%
\pgfpathlineto{\pgfqpoint{1.418998in}{4.288546in}}%
\pgfpathlineto{\pgfqpoint{1.429805in}{4.288546in}}%
\pgfpathlineto{\pgfqpoint{1.440611in}{4.288546in}}%
\pgfpathlineto{\pgfqpoint{1.451418in}{4.288546in}}%
\pgfpathlineto{\pgfqpoint{1.462225in}{4.288546in}}%
\pgfpathlineto{\pgfqpoint{1.473032in}{4.288546in}}%
\pgfpathlineto{\pgfqpoint{1.483838in}{4.288546in}}%
\pgfpathlineto{\pgfqpoint{1.494645in}{4.288546in}}%
\pgfpathlineto{\pgfqpoint{1.505452in}{4.288546in}}%
\pgfpathlineto{\pgfqpoint{1.516259in}{4.288546in}}%
\pgfpathlineto{\pgfqpoint{1.527065in}{4.288546in}}%
\pgfpathlineto{\pgfqpoint{1.537872in}{4.288546in}}%
\pgfpathlineto{\pgfqpoint{1.548679in}{4.288546in}}%
\pgfpathlineto{\pgfqpoint{1.559486in}{4.288546in}}%
\pgfpathlineto{\pgfqpoint{1.570292in}{4.288546in}}%
\pgfpathlineto{\pgfqpoint{1.581099in}{4.288546in}}%
\pgfpathlineto{\pgfqpoint{1.591906in}{4.288546in}}%
\pgfpathlineto{\pgfqpoint{1.602713in}{4.288546in}}%
\pgfpathlineto{\pgfqpoint{1.613519in}{4.288546in}}%
\pgfpathlineto{\pgfqpoint{1.624326in}{4.288546in}}%
\pgfpathlineto{\pgfqpoint{1.635133in}{4.288546in}}%
\pgfpathlineto{\pgfqpoint{1.645940in}{4.288546in}}%
\pgfpathlineto{\pgfqpoint{1.656746in}{4.288546in}}%
\pgfpathlineto{\pgfqpoint{1.667553in}{4.288546in}}%
\pgfpathlineto{\pgfqpoint{1.678360in}{4.288546in}}%
\pgfpathlineto{\pgfqpoint{1.689167in}{4.288546in}}%
\pgfpathlineto{\pgfqpoint{1.699973in}{4.288546in}}%
\pgfpathlineto{\pgfqpoint{1.710780in}{4.288546in}}%
\pgfpathlineto{\pgfqpoint{1.721587in}{4.288546in}}%
\pgfpathlineto{\pgfqpoint{1.732394in}{4.288546in}}%
\pgfpathlineto{\pgfqpoint{1.743200in}{4.288546in}}%
\pgfpathlineto{\pgfqpoint{1.754007in}{4.288546in}}%
\pgfpathlineto{\pgfqpoint{1.764814in}{4.288546in}}%
\pgfpathlineto{\pgfqpoint{1.775621in}{4.288546in}}%
\pgfpathlineto{\pgfqpoint{1.786427in}{4.288546in}}%
\pgfpathlineto{\pgfqpoint{1.797234in}{4.288546in}}%
\pgfpathlineto{\pgfqpoint{1.808041in}{4.288546in}}%
\pgfpathlineto{\pgfqpoint{1.818848in}{4.288546in}}%
\pgfpathlineto{\pgfqpoint{1.829654in}{4.288546in}}%
\pgfpathlineto{\pgfqpoint{1.840461in}{4.288546in}}%
\pgfpathlineto{\pgfqpoint{1.851268in}{4.288546in}}%
\pgfpathlineto{\pgfqpoint{1.862075in}{4.288546in}}%
\pgfpathlineto{\pgfqpoint{1.872881in}{4.288546in}}%
\pgfpathlineto{\pgfqpoint{1.883688in}{4.288546in}}%
\pgfpathlineto{\pgfqpoint{1.894495in}{4.288546in}}%
\pgfpathlineto{\pgfqpoint{1.905302in}{4.288546in}}%
\pgfpathlineto{\pgfqpoint{1.916108in}{4.288546in}}%
\pgfpathlineto{\pgfqpoint{1.926915in}{4.288546in}}%
\pgfpathlineto{\pgfqpoint{1.937722in}{4.288546in}}%
\pgfpathlineto{\pgfqpoint{1.948529in}{4.288546in}}%
\pgfpathlineto{\pgfqpoint{1.959335in}{4.288546in}}%
\pgfpathlineto{\pgfqpoint{1.970142in}{4.288546in}}%
\pgfpathlineto{\pgfqpoint{1.980949in}{4.288546in}}%
\pgfpathlineto{\pgfqpoint{1.991756in}{4.288546in}}%
\pgfpathlineto{\pgfqpoint{2.002562in}{4.288546in}}%
\pgfpathlineto{\pgfqpoint{2.013369in}{4.288546in}}%
\pgfpathlineto{\pgfqpoint{2.024176in}{4.288546in}}%
\pgfpathlineto{\pgfqpoint{2.034983in}{4.288546in}}%
\pgfpathlineto{\pgfqpoint{2.045789in}{4.288546in}}%
\pgfpathlineto{\pgfqpoint{2.056596in}{4.288546in}}%
\pgfpathlineto{\pgfqpoint{2.067403in}{4.288546in}}%
\pgfpathlineto{\pgfqpoint{2.078210in}{4.288546in}}%
\pgfpathlineto{\pgfqpoint{2.089016in}{4.288546in}}%
\pgfpathlineto{\pgfqpoint{2.099823in}{4.288546in}}%
\pgfpathlineto{\pgfqpoint{2.110630in}{4.288546in}}%
\pgfpathlineto{\pgfqpoint{2.121437in}{4.288546in}}%
\pgfpathlineto{\pgfqpoint{2.132243in}{4.288546in}}%
\pgfpathlineto{\pgfqpoint{2.143050in}{4.288546in}}%
\pgfpathlineto{\pgfqpoint{2.153857in}{4.288546in}}%
\pgfpathlineto{\pgfqpoint{2.164664in}{4.288546in}}%
\pgfpathlineto{\pgfqpoint{2.175470in}{4.288546in}}%
\pgfpathlineto{\pgfqpoint{2.186277in}{4.288546in}}%
\pgfpathlineto{\pgfqpoint{2.197084in}{4.288546in}}%
\pgfpathlineto{\pgfqpoint{2.207891in}{4.288546in}}%
\pgfpathlineto{\pgfqpoint{2.218697in}{4.288546in}}%
\pgfpathlineto{\pgfqpoint{2.229504in}{4.288546in}}%
\pgfpathlineto{\pgfqpoint{2.240311in}{4.288546in}}%
\pgfpathlineto{\pgfqpoint{2.251118in}{4.288546in}}%
\pgfpathlineto{\pgfqpoint{2.261924in}{4.288546in}}%
\pgfpathlineto{\pgfqpoint{2.272731in}{4.288546in}}%
\pgfpathlineto{\pgfqpoint{2.283538in}{4.288546in}}%
\pgfpathlineto{\pgfqpoint{2.294345in}{4.288546in}}%
\pgfpathlineto{\pgfqpoint{2.305151in}{4.288546in}}%
\pgfpathlineto{\pgfqpoint{2.315958in}{4.288546in}}%
\pgfpathlineto{\pgfqpoint{2.326765in}{4.288546in}}%
\pgfpathlineto{\pgfqpoint{2.337572in}{4.288546in}}%
\pgfpathlineto{\pgfqpoint{2.348378in}{4.288546in}}%
\pgfpathlineto{\pgfqpoint{2.359185in}{4.288546in}}%
\pgfpathlineto{\pgfqpoint{2.369992in}{4.288546in}}%
\pgfpathlineto{\pgfqpoint{2.380799in}{4.288546in}}%
\pgfpathlineto{\pgfqpoint{2.391605in}{4.288546in}}%
\pgfpathlineto{\pgfqpoint{2.402412in}{4.288546in}}%
\pgfpathlineto{\pgfqpoint{2.413219in}{4.288546in}}%
\pgfpathlineto{\pgfqpoint{2.424026in}{4.288546in}}%
\pgfpathlineto{\pgfqpoint{2.434832in}{4.288546in}}%
\pgfpathlineto{\pgfqpoint{2.445639in}{4.288546in}}%
\pgfpathlineto{\pgfqpoint{2.456446in}{4.288546in}}%
\pgfpathlineto{\pgfqpoint{2.467253in}{4.288546in}}%
\pgfpathlineto{\pgfqpoint{2.478059in}{4.288546in}}%
\pgfpathlineto{\pgfqpoint{2.488866in}{4.288546in}}%
\pgfpathlineto{\pgfqpoint{2.499673in}{4.288546in}}%
\pgfpathlineto{\pgfqpoint{2.510480in}{4.288546in}}%
\pgfpathlineto{\pgfqpoint{2.521286in}{4.288546in}}%
\pgfpathlineto{\pgfqpoint{2.532093in}{4.288546in}}%
\pgfpathlineto{\pgfqpoint{2.542900in}{4.288546in}}%
\pgfpathlineto{\pgfqpoint{2.553707in}{4.288546in}}%
\pgfpathlineto{\pgfqpoint{2.564513in}{4.288546in}}%
\pgfpathlineto{\pgfqpoint{2.575320in}{4.288546in}}%
\pgfpathlineto{\pgfqpoint{2.586127in}{4.288546in}}%
\pgfpathlineto{\pgfqpoint{2.596934in}{4.288546in}}%
\pgfpathlineto{\pgfqpoint{2.607740in}{4.288546in}}%
\pgfpathlineto{\pgfqpoint{2.618547in}{4.288546in}}%
\pgfpathlineto{\pgfqpoint{2.629354in}{4.288546in}}%
\pgfpathlineto{\pgfqpoint{2.640161in}{4.288546in}}%
\pgfpathlineto{\pgfqpoint{2.650967in}{4.288546in}}%
\pgfpathlineto{\pgfqpoint{2.661774in}{4.288546in}}%
\pgfpathlineto{\pgfqpoint{2.672581in}{4.288546in}}%
\pgfpathlineto{\pgfqpoint{2.683388in}{4.288546in}}%
\pgfpathlineto{\pgfqpoint{2.694194in}{4.288546in}}%
\pgfpathlineto{\pgfqpoint{2.705001in}{4.288546in}}%
\pgfpathlineto{\pgfqpoint{2.715808in}{4.288546in}}%
\pgfpathlineto{\pgfqpoint{2.726615in}{4.288546in}}%
\pgfpathlineto{\pgfqpoint{2.737421in}{4.288546in}}%
\pgfpathlineto{\pgfqpoint{2.748228in}{4.288546in}}%
\pgfpathlineto{\pgfqpoint{2.759035in}{4.288546in}}%
\pgfpathlineto{\pgfqpoint{2.769841in}{4.288546in}}%
\pgfpathlineto{\pgfqpoint{2.780648in}{4.288546in}}%
\pgfpathlineto{\pgfqpoint{2.791455in}{4.288546in}}%
\pgfpathlineto{\pgfqpoint{2.802262in}{4.288546in}}%
\pgfpathlineto{\pgfqpoint{2.813068in}{4.288546in}}%
\pgfpathlineto{\pgfqpoint{2.823875in}{4.288546in}}%
\pgfpathlineto{\pgfqpoint{2.834682in}{4.288546in}}%
\pgfpathlineto{\pgfqpoint{2.845489in}{4.288546in}}%
\pgfpathlineto{\pgfqpoint{2.856295in}{4.288546in}}%
\pgfpathlineto{\pgfqpoint{2.867102in}{4.288546in}}%
\pgfpathlineto{\pgfqpoint{2.877909in}{4.288546in}}%
\pgfpathlineto{\pgfqpoint{2.888716in}{4.288546in}}%
\pgfpathlineto{\pgfqpoint{2.899522in}{4.288546in}}%
\pgfpathlineto{\pgfqpoint{2.910329in}{4.288546in}}%
\pgfpathlineto{\pgfqpoint{2.921136in}{4.288546in}}%
\pgfpathlineto{\pgfqpoint{2.931943in}{4.288546in}}%
\pgfpathlineto{\pgfqpoint{2.942749in}{4.288546in}}%
\pgfpathlineto{\pgfqpoint{2.953556in}{4.288546in}}%
\pgfpathlineto{\pgfqpoint{2.964363in}{4.288546in}}%
\pgfpathlineto{\pgfqpoint{2.975170in}{4.288546in}}%
\pgfpathlineto{\pgfqpoint{2.985976in}{4.288546in}}%
\pgfpathlineto{\pgfqpoint{2.996783in}{4.288546in}}%
\pgfpathlineto{\pgfqpoint{3.007590in}{4.288546in}}%
\pgfpathlineto{\pgfqpoint{3.007590in}{4.288546in}}%
\pgfpathlineto{\pgfqpoint{3.007590in}{4.288546in}}%
\pgfpathlineto{\pgfqpoint{2.996783in}{4.288547in}}%
\pgfpathlineto{\pgfqpoint{2.985976in}{4.288547in}}%
\pgfpathlineto{\pgfqpoint{2.975170in}{4.288549in}}%
\pgfpathlineto{\pgfqpoint{2.964363in}{4.288552in}}%
\pgfpathlineto{\pgfqpoint{2.953556in}{4.288556in}}%
\pgfpathlineto{\pgfqpoint{2.942749in}{4.288563in}}%
\pgfpathlineto{\pgfqpoint{2.931943in}{4.288572in}}%
\pgfpathlineto{\pgfqpoint{2.921136in}{4.288581in}}%
\pgfpathlineto{\pgfqpoint{2.910329in}{4.288590in}}%
\pgfpathlineto{\pgfqpoint{2.899522in}{4.288596in}}%
\pgfpathlineto{\pgfqpoint{2.888716in}{4.288597in}}%
\pgfpathlineto{\pgfqpoint{2.877909in}{4.288593in}}%
\pgfpathlineto{\pgfqpoint{2.867102in}{4.288586in}}%
\pgfpathlineto{\pgfqpoint{2.856295in}{4.288579in}}%
\pgfpathlineto{\pgfqpoint{2.845489in}{4.288577in}}%
\pgfpathlineto{\pgfqpoint{2.834682in}{4.288582in}}%
\pgfpathlineto{\pgfqpoint{2.823875in}{4.288592in}}%
\pgfpathlineto{\pgfqpoint{2.813068in}{4.288607in}}%
\pgfpathlineto{\pgfqpoint{2.802262in}{4.288625in}}%
\pgfpathlineto{\pgfqpoint{2.791455in}{4.288640in}}%
\pgfpathlineto{\pgfqpoint{2.780648in}{4.288649in}}%
\pgfpathlineto{\pgfqpoint{2.769841in}{4.288652in}}%
\pgfpathlineto{\pgfqpoint{2.759035in}{4.288649in}}%
\pgfpathlineto{\pgfqpoint{2.748228in}{4.288644in}}%
\pgfpathlineto{\pgfqpoint{2.737421in}{4.288641in}}%
\pgfpathlineto{\pgfqpoint{2.726615in}{4.288641in}}%
\pgfpathlineto{\pgfqpoint{2.715808in}{4.288648in}}%
\pgfpathlineto{\pgfqpoint{2.705001in}{4.288662in}}%
\pgfpathlineto{\pgfqpoint{2.694194in}{4.288689in}}%
\pgfpathlineto{\pgfqpoint{2.683388in}{4.288730in}}%
\pgfpathlineto{\pgfqpoint{2.672581in}{4.288781in}}%
\pgfpathlineto{\pgfqpoint{2.661774in}{4.288834in}}%
\pgfpathlineto{\pgfqpoint{2.650967in}{4.288880in}}%
\pgfpathlineto{\pgfqpoint{2.640161in}{4.288918in}}%
\pgfpathlineto{\pgfqpoint{2.629354in}{4.288955in}}%
\pgfpathlineto{\pgfqpoint{2.618547in}{4.289004in}}%
\pgfpathlineto{\pgfqpoint{2.607740in}{4.289063in}}%
\pgfpathlineto{\pgfqpoint{2.596934in}{4.289125in}}%
\pgfpathlineto{\pgfqpoint{2.586127in}{4.289191in}}%
\pgfpathlineto{\pgfqpoint{2.575320in}{4.289285in}}%
\pgfpathlineto{\pgfqpoint{2.564513in}{4.289435in}}%
\pgfpathlineto{\pgfqpoint{2.553707in}{4.289633in}}%
\pgfpathlineto{\pgfqpoint{2.542900in}{4.289833in}}%
\pgfpathlineto{\pgfqpoint{2.532093in}{4.289986in}}%
\pgfpathlineto{\pgfqpoint{2.521286in}{4.290111in}}%
\pgfpathlineto{\pgfqpoint{2.510480in}{4.290343in}}%
\pgfpathlineto{\pgfqpoint{2.499673in}{4.290906in}}%
\pgfpathlineto{\pgfqpoint{2.488866in}{4.292008in}}%
\pgfpathlineto{\pgfqpoint{2.478059in}{4.293666in}}%
\pgfpathlineto{\pgfqpoint{2.467253in}{4.295574in}}%
\pgfpathlineto{\pgfqpoint{2.456446in}{4.297182in}}%
\pgfpathlineto{\pgfqpoint{2.445639in}{4.297985in}}%
\pgfpathlineto{\pgfqpoint{2.434832in}{4.297815in}}%
\pgfpathlineto{\pgfqpoint{2.424026in}{4.296868in}}%
\pgfpathlineto{\pgfqpoint{2.413219in}{4.295522in}}%
\pgfpathlineto{\pgfqpoint{2.402412in}{4.294134in}}%
\pgfpathlineto{\pgfqpoint{2.391605in}{4.292939in}}%
\pgfpathlineto{\pgfqpoint{2.380799in}{4.292025in}}%
\pgfpathlineto{\pgfqpoint{2.369992in}{4.291352in}}%
\pgfpathlineto{\pgfqpoint{2.359185in}{4.290818in}}%
\pgfpathlineto{\pgfqpoint{2.348378in}{4.290348in}}%
\pgfpathlineto{\pgfqpoint{2.337572in}{4.289941in}}%
\pgfpathlineto{\pgfqpoint{2.326765in}{4.289647in}}%
\pgfpathlineto{\pgfqpoint{2.315958in}{4.289518in}}%
\pgfpathlineto{\pgfqpoint{2.305151in}{4.289584in}}%
\pgfpathlineto{\pgfqpoint{2.294345in}{4.289846in}}%
\pgfpathlineto{\pgfqpoint{2.283538in}{4.290242in}}%
\pgfpathlineto{\pgfqpoint{2.272731in}{4.290636in}}%
\pgfpathlineto{\pgfqpoint{2.261924in}{4.290858in}}%
\pgfpathlineto{\pgfqpoint{2.251118in}{4.290820in}}%
\pgfpathlineto{\pgfqpoint{2.240311in}{4.290572in}}%
\pgfpathlineto{\pgfqpoint{2.229504in}{4.290260in}}%
\pgfpathlineto{\pgfqpoint{2.218697in}{4.290018in}}%
\pgfpathlineto{\pgfqpoint{2.207891in}{4.289911in}}%
\pgfpathlineto{\pgfqpoint{2.197084in}{4.289943in}}%
\pgfpathlineto{\pgfqpoint{2.186277in}{4.290099in}}%
\pgfpathlineto{\pgfqpoint{2.175470in}{4.290372in}}%
\pgfpathlineto{\pgfqpoint{2.164664in}{4.290774in}}%
\pgfpathlineto{\pgfqpoint{2.153857in}{4.291329in}}%
\pgfpathlineto{\pgfqpoint{2.143050in}{4.292049in}}%
\pgfpathlineto{\pgfqpoint{2.132243in}{4.292900in}}%
\pgfpathlineto{\pgfqpoint{2.121437in}{4.293794in}}%
\pgfpathlineto{\pgfqpoint{2.110630in}{4.294606in}}%
\pgfpathlineto{\pgfqpoint{2.099823in}{4.295232in}}%
\pgfpathlineto{\pgfqpoint{2.089016in}{4.295632in}}%
\pgfpathlineto{\pgfqpoint{2.078210in}{4.295846in}}%
\pgfpathlineto{\pgfqpoint{2.067403in}{4.295946in}}%
\pgfpathlineto{\pgfqpoint{2.056596in}{4.295963in}}%
\pgfpathlineto{\pgfqpoint{2.045789in}{4.295862in}}%
\pgfpathlineto{\pgfqpoint{2.034983in}{4.295592in}}%
\pgfpathlineto{\pgfqpoint{2.024176in}{4.295185in}}%
\pgfpathlineto{\pgfqpoint{2.013369in}{4.294801in}}%
\pgfpathlineto{\pgfqpoint{2.002562in}{4.294646in}}%
\pgfpathlineto{\pgfqpoint{1.991756in}{4.294870in}}%
\pgfpathlineto{\pgfqpoint{1.980949in}{4.295529in}}%
\pgfpathlineto{\pgfqpoint{1.970142in}{4.296624in}}%
\pgfpathlineto{\pgfqpoint{1.959335in}{4.298097in}}%
\pgfpathlineto{\pgfqpoint{1.948529in}{4.299752in}}%
\pgfpathlineto{\pgfqpoint{1.937722in}{4.301220in}}%
\pgfpathlineto{\pgfqpoint{1.926915in}{4.302109in}}%
\pgfpathlineto{\pgfqpoint{1.916108in}{4.302266in}}%
\pgfpathlineto{\pgfqpoint{1.905302in}{4.301887in}}%
\pgfpathlineto{\pgfqpoint{1.894495in}{4.301358in}}%
\pgfpathlineto{\pgfqpoint{1.883688in}{4.301012in}}%
\pgfpathlineto{\pgfqpoint{1.872881in}{4.301020in}}%
\pgfpathlineto{\pgfqpoint{1.862075in}{4.301418in}}%
\pgfpathlineto{\pgfqpoint{1.851268in}{4.302169in}}%
\pgfpathlineto{\pgfqpoint{1.840461in}{4.303178in}}%
\pgfpathlineto{\pgfqpoint{1.829654in}{4.304320in}}%
\pgfpathlineto{\pgfqpoint{1.818848in}{4.305497in}}%
\pgfpathlineto{\pgfqpoint{1.808041in}{4.306680in}}%
\pgfpathlineto{\pgfqpoint{1.797234in}{4.307883in}}%
\pgfpathlineto{\pgfqpoint{1.786427in}{4.309105in}}%
\pgfpathlineto{\pgfqpoint{1.775621in}{4.310314in}}%
\pgfpathlineto{\pgfqpoint{1.764814in}{4.311486in}}%
\pgfpathlineto{\pgfqpoint{1.754007in}{4.312598in}}%
\pgfpathlineto{\pgfqpoint{1.743200in}{4.313592in}}%
\pgfpathlineto{\pgfqpoint{1.732394in}{4.314378in}}%
\pgfpathlineto{\pgfqpoint{1.721587in}{4.314931in}}%
\pgfpathlineto{\pgfqpoint{1.710780in}{4.315397in}}%
\pgfpathlineto{\pgfqpoint{1.699973in}{4.316065in}}%
\pgfpathlineto{\pgfqpoint{1.689167in}{4.317136in}}%
\pgfpathlineto{\pgfqpoint{1.678360in}{4.318487in}}%
\pgfpathlineto{\pgfqpoint{1.667553in}{4.319681in}}%
\pgfpathlineto{\pgfqpoint{1.656746in}{4.320270in}}%
\pgfpathlineto{\pgfqpoint{1.645940in}{4.320112in}}%
\pgfpathlineto{\pgfqpoint{1.635133in}{4.319424in}}%
\pgfpathlineto{\pgfqpoint{1.624326in}{4.318561in}}%
\pgfpathlineto{\pgfqpoint{1.613519in}{4.317811in}}%
\pgfpathlineto{\pgfqpoint{1.602713in}{4.317351in}}%
\pgfpathlineto{\pgfqpoint{1.591906in}{4.317299in}}%
\pgfpathlineto{\pgfqpoint{1.581099in}{4.317622in}}%
\pgfpathlineto{\pgfqpoint{1.570292in}{4.317957in}}%
\pgfpathlineto{\pgfqpoint{1.559486in}{4.317686in}}%
\pgfpathlineto{\pgfqpoint{1.548679in}{4.316409in}}%
\pgfpathlineto{\pgfqpoint{1.537872in}{4.314388in}}%
\pgfpathlineto{\pgfqpoint{1.527065in}{4.312456in}}%
\pgfpathlineto{\pgfqpoint{1.516259in}{4.311454in}}%
\pgfpathlineto{\pgfqpoint{1.505452in}{4.311674in}}%
\pgfpathlineto{\pgfqpoint{1.494645in}{4.312724in}}%
\pgfpathlineto{\pgfqpoint{1.483838in}{4.313823in}}%
\pgfpathlineto{\pgfqpoint{1.473032in}{4.314317in}}%
\pgfpathlineto{\pgfqpoint{1.462225in}{4.314014in}}%
\pgfpathlineto{\pgfqpoint{1.451418in}{4.313132in}}%
\pgfpathlineto{\pgfqpoint{1.440611in}{4.311984in}}%
\pgfpathlineto{\pgfqpoint{1.429805in}{4.310763in}}%
\pgfpathlineto{\pgfqpoint{1.418998in}{4.309547in}}%
\pgfpathlineto{\pgfqpoint{1.408191in}{4.308407in}}%
\pgfpathlineto{\pgfqpoint{1.397384in}{4.307457in}}%
\pgfpathlineto{\pgfqpoint{1.386578in}{4.306821in}}%
\pgfpathlineto{\pgfqpoint{1.375771in}{4.306515in}}%
\pgfpathlineto{\pgfqpoint{1.364964in}{4.306323in}}%
\pgfpathlineto{\pgfqpoint{1.354157in}{4.305792in}}%
\pgfpathlineto{\pgfqpoint{1.343351in}{4.304462in}}%
\pgfpathlineto{\pgfqpoint{1.332544in}{4.302194in}}%
\pgfpathlineto{\pgfqpoint{1.321737in}{4.299329in}}%
\pgfpathlineto{\pgfqpoint{1.310930in}{4.296475in}}%
\pgfpathlineto{\pgfqpoint{1.300124in}{4.294125in}}%
\pgfpathlineto{\pgfqpoint{1.289317in}{4.292427in}}%
\pgfpathlineto{\pgfqpoint{1.278510in}{4.291265in}}%
\pgfpathlineto{\pgfqpoint{1.267703in}{4.290463in}}%
\pgfpathlineto{\pgfqpoint{1.256897in}{4.289903in}}%
\pgfpathlineto{\pgfqpoint{1.246090in}{4.289527in}}%
\pgfpathlineto{\pgfqpoint{1.235283in}{4.289291in}}%
\pgfpathlineto{\pgfqpoint{1.224476in}{4.289145in}}%
\pgfpathlineto{\pgfqpoint{1.213670in}{4.289042in}}%
\pgfpathlineto{\pgfqpoint{1.202863in}{4.288950in}}%
\pgfpathlineto{\pgfqpoint{1.192056in}{4.288855in}}%
\pgfpathlineto{\pgfqpoint{1.181249in}{4.288761in}}%
\pgfpathlineto{\pgfqpoint{1.170443in}{4.288682in}}%
\pgfpathlineto{\pgfqpoint{1.159636in}{4.288625in}}%
\pgfpathlineto{\pgfqpoint{1.148829in}{4.288590in}}%
\pgfpathlineto{\pgfqpoint{1.138022in}{4.288572in}}%
\pgfpathlineto{\pgfqpoint{1.127216in}{4.288563in}}%
\pgfpathlineto{\pgfqpoint{1.116409in}{4.288559in}}%
\pgfpathlineto{\pgfqpoint{1.105602in}{4.288558in}}%
\pgfpathlineto{\pgfqpoint{1.094795in}{4.288557in}}%
\pgfpathlineto{\pgfqpoint{1.083989in}{4.288556in}}%
\pgfpathlineto{\pgfqpoint{1.073182in}{4.288555in}}%
\pgfpathlineto{\pgfqpoint{1.062375in}{4.288553in}}%
\pgfpathlineto{\pgfqpoint{1.051568in}{4.288552in}}%
\pgfpathlineto{\pgfqpoint{1.040762in}{4.288550in}}%
\pgfpathlineto{\pgfqpoint{1.029955in}{4.288549in}}%
\pgfpathlineto{\pgfqpoint{1.019148in}{4.288548in}}%
\pgfpathlineto{\pgfqpoint{1.008341in}{4.288547in}}%
\pgfpathlineto{\pgfqpoint{0.997535in}{4.288547in}}%
\pgfpathlineto{\pgfqpoint{0.986728in}{4.288547in}}%
\pgfpathlineto{\pgfqpoint{0.975921in}{4.288547in}}%
\pgfpathlineto{\pgfqpoint{0.965114in}{4.288548in}}%
\pgfpathlineto{\pgfqpoint{0.954308in}{4.288549in}}%
\pgfpathlineto{\pgfqpoint{0.943501in}{4.288550in}}%
\pgfpathlineto{\pgfqpoint{0.932694in}{4.288550in}}%
\pgfpathlineto{\pgfqpoint{0.921887in}{4.288550in}}%
\pgfpathlineto{\pgfqpoint{0.911081in}{4.288549in}}%
\pgfpathlineto{\pgfqpoint{0.900274in}{4.288548in}}%
\pgfpathlineto{\pgfqpoint{0.889467in}{4.288547in}}%
\pgfpathlineto{\pgfqpoint{0.878660in}{4.288547in}}%
\pgfpathlineto{\pgfqpoint{0.867854in}{4.288547in}}%
\pgfpathlineto{\pgfqpoint{0.857047in}{4.288546in}}%
\pgfpathclose%
\pgfusepath{stroke,fill}%
}%
\begin{pgfscope}%
\pgfsys@transformshift{0.000000in}{0.000000in}%
\pgfsys@useobject{currentmarker}{}%
\end{pgfscope}%
\end{pgfscope}%
\begin{pgfscope}%
\pgfpathrectangle{\pgfqpoint{0.692593in}{4.288546in}}{\pgfqpoint{3.617974in}{0.631476in}}%
\pgfusepath{clip}%
\pgfsetbuttcap%
\pgfsetroundjoin%
\definecolor{currentfill}{rgb}{1.000000,0.498039,0.054902}%
\pgfsetfillcolor{currentfill}%
\pgfsetfillopacity{0.200000}%
\pgfsetlinewidth{1.003750pt}%
\definecolor{currentstroke}{rgb}{1.000000,0.498039,0.054902}%
\pgfsetstrokecolor{currentstroke}%
\pgfsetdash{}{0pt}%
\pgfsys@defobject{currentmarker}{\pgfqpoint{2.442983in}{4.288546in}}{\pgfqpoint{4.146114in}{4.299618in}}{%
\pgfpathmoveto{\pgfqpoint{2.442983in}{4.288546in}}%
\pgfpathlineto{\pgfqpoint{2.442983in}{4.288546in}}%
\pgfpathlineto{\pgfqpoint{2.451542in}{4.288546in}}%
\pgfpathlineto{\pgfqpoint{2.460100in}{4.288546in}}%
\pgfpathlineto{\pgfqpoint{2.468659in}{4.288546in}}%
\pgfpathlineto{\pgfqpoint{2.477217in}{4.288546in}}%
\pgfpathlineto{\pgfqpoint{2.485776in}{4.288546in}}%
\pgfpathlineto{\pgfqpoint{2.494334in}{4.288546in}}%
\pgfpathlineto{\pgfqpoint{2.502892in}{4.288546in}}%
\pgfpathlineto{\pgfqpoint{2.511451in}{4.288546in}}%
\pgfpathlineto{\pgfqpoint{2.520009in}{4.288546in}}%
\pgfpathlineto{\pgfqpoint{2.528568in}{4.288546in}}%
\pgfpathlineto{\pgfqpoint{2.537126in}{4.288546in}}%
\pgfpathlineto{\pgfqpoint{2.545685in}{4.288546in}}%
\pgfpathlineto{\pgfqpoint{2.554243in}{4.288546in}}%
\pgfpathlineto{\pgfqpoint{2.562802in}{4.288546in}}%
\pgfpathlineto{\pgfqpoint{2.571360in}{4.288546in}}%
\pgfpathlineto{\pgfqpoint{2.579918in}{4.288546in}}%
\pgfpathlineto{\pgfqpoint{2.588477in}{4.288546in}}%
\pgfpathlineto{\pgfqpoint{2.597035in}{4.288546in}}%
\pgfpathlineto{\pgfqpoint{2.605594in}{4.288546in}}%
\pgfpathlineto{\pgfqpoint{2.614152in}{4.288546in}}%
\pgfpathlineto{\pgfqpoint{2.622711in}{4.288546in}}%
\pgfpathlineto{\pgfqpoint{2.631269in}{4.288546in}}%
\pgfpathlineto{\pgfqpoint{2.639828in}{4.288546in}}%
\pgfpathlineto{\pgfqpoint{2.648386in}{4.288546in}}%
\pgfpathlineto{\pgfqpoint{2.656944in}{4.288546in}}%
\pgfpathlineto{\pgfqpoint{2.665503in}{4.288546in}}%
\pgfpathlineto{\pgfqpoint{2.674061in}{4.288546in}}%
\pgfpathlineto{\pgfqpoint{2.682620in}{4.288546in}}%
\pgfpathlineto{\pgfqpoint{2.691178in}{4.288546in}}%
\pgfpathlineto{\pgfqpoint{2.699737in}{4.288546in}}%
\pgfpathlineto{\pgfqpoint{2.708295in}{4.288546in}}%
\pgfpathlineto{\pgfqpoint{2.716854in}{4.288546in}}%
\pgfpathlineto{\pgfqpoint{2.725412in}{4.288546in}}%
\pgfpathlineto{\pgfqpoint{2.733970in}{4.288546in}}%
\pgfpathlineto{\pgfqpoint{2.742529in}{4.288546in}}%
\pgfpathlineto{\pgfqpoint{2.751087in}{4.288546in}}%
\pgfpathlineto{\pgfqpoint{2.759646in}{4.288546in}}%
\pgfpathlineto{\pgfqpoint{2.768204in}{4.288546in}}%
\pgfpathlineto{\pgfqpoint{2.776763in}{4.288546in}}%
\pgfpathlineto{\pgfqpoint{2.785321in}{4.288546in}}%
\pgfpathlineto{\pgfqpoint{2.793880in}{4.288546in}}%
\pgfpathlineto{\pgfqpoint{2.802438in}{4.288546in}}%
\pgfpathlineto{\pgfqpoint{2.810996in}{4.288546in}}%
\pgfpathlineto{\pgfqpoint{2.819555in}{4.288546in}}%
\pgfpathlineto{\pgfqpoint{2.828113in}{4.288546in}}%
\pgfpathlineto{\pgfqpoint{2.836672in}{4.288546in}}%
\pgfpathlineto{\pgfqpoint{2.845230in}{4.288546in}}%
\pgfpathlineto{\pgfqpoint{2.853789in}{4.288546in}}%
\pgfpathlineto{\pgfqpoint{2.862347in}{4.288546in}}%
\pgfpathlineto{\pgfqpoint{2.870906in}{4.288546in}}%
\pgfpathlineto{\pgfqpoint{2.879464in}{4.288546in}}%
\pgfpathlineto{\pgfqpoint{2.888022in}{4.288546in}}%
\pgfpathlineto{\pgfqpoint{2.896581in}{4.288546in}}%
\pgfpathlineto{\pgfqpoint{2.905139in}{4.288546in}}%
\pgfpathlineto{\pgfqpoint{2.913698in}{4.288546in}}%
\pgfpathlineto{\pgfqpoint{2.922256in}{4.288546in}}%
\pgfpathlineto{\pgfqpoint{2.930815in}{4.288546in}}%
\pgfpathlineto{\pgfqpoint{2.939373in}{4.288546in}}%
\pgfpathlineto{\pgfqpoint{2.947932in}{4.288546in}}%
\pgfpathlineto{\pgfqpoint{2.956490in}{4.288546in}}%
\pgfpathlineto{\pgfqpoint{2.965048in}{4.288546in}}%
\pgfpathlineto{\pgfqpoint{2.973607in}{4.288546in}}%
\pgfpathlineto{\pgfqpoint{2.982165in}{4.288546in}}%
\pgfpathlineto{\pgfqpoint{2.990724in}{4.288546in}}%
\pgfpathlineto{\pgfqpoint{2.999282in}{4.288546in}}%
\pgfpathlineto{\pgfqpoint{3.007841in}{4.288546in}}%
\pgfpathlineto{\pgfqpoint{3.016399in}{4.288546in}}%
\pgfpathlineto{\pgfqpoint{3.024958in}{4.288546in}}%
\pgfpathlineto{\pgfqpoint{3.033516in}{4.288546in}}%
\pgfpathlineto{\pgfqpoint{3.042074in}{4.288546in}}%
\pgfpathlineto{\pgfqpoint{3.050633in}{4.288546in}}%
\pgfpathlineto{\pgfqpoint{3.059191in}{4.288546in}}%
\pgfpathlineto{\pgfqpoint{3.067750in}{4.288546in}}%
\pgfpathlineto{\pgfqpoint{3.076308in}{4.288546in}}%
\pgfpathlineto{\pgfqpoint{3.084867in}{4.288546in}}%
\pgfpathlineto{\pgfqpoint{3.093425in}{4.288546in}}%
\pgfpathlineto{\pgfqpoint{3.101984in}{4.288546in}}%
\pgfpathlineto{\pgfqpoint{3.110542in}{4.288546in}}%
\pgfpathlineto{\pgfqpoint{3.119100in}{4.288546in}}%
\pgfpathlineto{\pgfqpoint{3.127659in}{4.288546in}}%
\pgfpathlineto{\pgfqpoint{3.136217in}{4.288546in}}%
\pgfpathlineto{\pgfqpoint{3.144776in}{4.288546in}}%
\pgfpathlineto{\pgfqpoint{3.153334in}{4.288546in}}%
\pgfpathlineto{\pgfqpoint{3.161893in}{4.288546in}}%
\pgfpathlineto{\pgfqpoint{3.170451in}{4.288546in}}%
\pgfpathlineto{\pgfqpoint{3.179010in}{4.288546in}}%
\pgfpathlineto{\pgfqpoint{3.187568in}{4.288546in}}%
\pgfpathlineto{\pgfqpoint{3.196126in}{4.288546in}}%
\pgfpathlineto{\pgfqpoint{3.204685in}{4.288546in}}%
\pgfpathlineto{\pgfqpoint{3.213243in}{4.288546in}}%
\pgfpathlineto{\pgfqpoint{3.221802in}{4.288546in}}%
\pgfpathlineto{\pgfqpoint{3.230360in}{4.288546in}}%
\pgfpathlineto{\pgfqpoint{3.238919in}{4.288546in}}%
\pgfpathlineto{\pgfqpoint{3.247477in}{4.288546in}}%
\pgfpathlineto{\pgfqpoint{3.256036in}{4.288546in}}%
\pgfpathlineto{\pgfqpoint{3.264594in}{4.288546in}}%
\pgfpathlineto{\pgfqpoint{3.273152in}{4.288546in}}%
\pgfpathlineto{\pgfqpoint{3.281711in}{4.288546in}}%
\pgfpathlineto{\pgfqpoint{3.290269in}{4.288546in}}%
\pgfpathlineto{\pgfqpoint{3.298828in}{4.288546in}}%
\pgfpathlineto{\pgfqpoint{3.307386in}{4.288546in}}%
\pgfpathlineto{\pgfqpoint{3.315945in}{4.288546in}}%
\pgfpathlineto{\pgfqpoint{3.324503in}{4.288546in}}%
\pgfpathlineto{\pgfqpoint{3.333062in}{4.288546in}}%
\pgfpathlineto{\pgfqpoint{3.341620in}{4.288546in}}%
\pgfpathlineto{\pgfqpoint{3.350178in}{4.288546in}}%
\pgfpathlineto{\pgfqpoint{3.358737in}{4.288546in}}%
\pgfpathlineto{\pgfqpoint{3.367295in}{4.288546in}}%
\pgfpathlineto{\pgfqpoint{3.375854in}{4.288546in}}%
\pgfpathlineto{\pgfqpoint{3.384412in}{4.288546in}}%
\pgfpathlineto{\pgfqpoint{3.392971in}{4.288546in}}%
\pgfpathlineto{\pgfqpoint{3.401529in}{4.288546in}}%
\pgfpathlineto{\pgfqpoint{3.410088in}{4.288546in}}%
\pgfpathlineto{\pgfqpoint{3.418646in}{4.288546in}}%
\pgfpathlineto{\pgfqpoint{3.427205in}{4.288546in}}%
\pgfpathlineto{\pgfqpoint{3.435763in}{4.288546in}}%
\pgfpathlineto{\pgfqpoint{3.444321in}{4.288546in}}%
\pgfpathlineto{\pgfqpoint{3.452880in}{4.288546in}}%
\pgfpathlineto{\pgfqpoint{3.461438in}{4.288546in}}%
\pgfpathlineto{\pgfqpoint{3.469997in}{4.288546in}}%
\pgfpathlineto{\pgfqpoint{3.478555in}{4.288546in}}%
\pgfpathlineto{\pgfqpoint{3.487114in}{4.288546in}}%
\pgfpathlineto{\pgfqpoint{3.495672in}{4.288546in}}%
\pgfpathlineto{\pgfqpoint{3.504231in}{4.288546in}}%
\pgfpathlineto{\pgfqpoint{3.512789in}{4.288546in}}%
\pgfpathlineto{\pgfqpoint{3.521347in}{4.288546in}}%
\pgfpathlineto{\pgfqpoint{3.529906in}{4.288546in}}%
\pgfpathlineto{\pgfqpoint{3.538464in}{4.288546in}}%
\pgfpathlineto{\pgfqpoint{3.547023in}{4.288546in}}%
\pgfpathlineto{\pgfqpoint{3.555581in}{4.288546in}}%
\pgfpathlineto{\pgfqpoint{3.564140in}{4.288546in}}%
\pgfpathlineto{\pgfqpoint{3.572698in}{4.288546in}}%
\pgfpathlineto{\pgfqpoint{3.581257in}{4.288546in}}%
\pgfpathlineto{\pgfqpoint{3.589815in}{4.288546in}}%
\pgfpathlineto{\pgfqpoint{3.598373in}{4.288546in}}%
\pgfpathlineto{\pgfqpoint{3.606932in}{4.288546in}}%
\pgfpathlineto{\pgfqpoint{3.615490in}{4.288546in}}%
\pgfpathlineto{\pgfqpoint{3.624049in}{4.288546in}}%
\pgfpathlineto{\pgfqpoint{3.632607in}{4.288546in}}%
\pgfpathlineto{\pgfqpoint{3.641166in}{4.288546in}}%
\pgfpathlineto{\pgfqpoint{3.649724in}{4.288546in}}%
\pgfpathlineto{\pgfqpoint{3.658283in}{4.288546in}}%
\pgfpathlineto{\pgfqpoint{3.666841in}{4.288546in}}%
\pgfpathlineto{\pgfqpoint{3.675399in}{4.288546in}}%
\pgfpathlineto{\pgfqpoint{3.683958in}{4.288546in}}%
\pgfpathlineto{\pgfqpoint{3.692516in}{4.288546in}}%
\pgfpathlineto{\pgfqpoint{3.701075in}{4.288546in}}%
\pgfpathlineto{\pgfqpoint{3.709633in}{4.288546in}}%
\pgfpathlineto{\pgfqpoint{3.718192in}{4.288546in}}%
\pgfpathlineto{\pgfqpoint{3.726750in}{4.288546in}}%
\pgfpathlineto{\pgfqpoint{3.735309in}{4.288546in}}%
\pgfpathlineto{\pgfqpoint{3.743867in}{4.288546in}}%
\pgfpathlineto{\pgfqpoint{3.752425in}{4.288546in}}%
\pgfpathlineto{\pgfqpoint{3.760984in}{4.288546in}}%
\pgfpathlineto{\pgfqpoint{3.769542in}{4.288546in}}%
\pgfpathlineto{\pgfqpoint{3.778101in}{4.288546in}}%
\pgfpathlineto{\pgfqpoint{3.786659in}{4.288546in}}%
\pgfpathlineto{\pgfqpoint{3.795218in}{4.288546in}}%
\pgfpathlineto{\pgfqpoint{3.803776in}{4.288546in}}%
\pgfpathlineto{\pgfqpoint{3.812335in}{4.288546in}}%
\pgfpathlineto{\pgfqpoint{3.820893in}{4.288546in}}%
\pgfpathlineto{\pgfqpoint{3.829451in}{4.288546in}}%
\pgfpathlineto{\pgfqpoint{3.838010in}{4.288546in}}%
\pgfpathlineto{\pgfqpoint{3.846568in}{4.288546in}}%
\pgfpathlineto{\pgfqpoint{3.855127in}{4.288546in}}%
\pgfpathlineto{\pgfqpoint{3.863685in}{4.288546in}}%
\pgfpathlineto{\pgfqpoint{3.872244in}{4.288546in}}%
\pgfpathlineto{\pgfqpoint{3.880802in}{4.288546in}}%
\pgfpathlineto{\pgfqpoint{3.889361in}{4.288546in}}%
\pgfpathlineto{\pgfqpoint{3.897919in}{4.288546in}}%
\pgfpathlineto{\pgfqpoint{3.906477in}{4.288546in}}%
\pgfpathlineto{\pgfqpoint{3.915036in}{4.288546in}}%
\pgfpathlineto{\pgfqpoint{3.923594in}{4.288546in}}%
\pgfpathlineto{\pgfqpoint{3.932153in}{4.288546in}}%
\pgfpathlineto{\pgfqpoint{3.940711in}{4.288546in}}%
\pgfpathlineto{\pgfqpoint{3.949270in}{4.288546in}}%
\pgfpathlineto{\pgfqpoint{3.957828in}{4.288546in}}%
\pgfpathlineto{\pgfqpoint{3.966387in}{4.288546in}}%
\pgfpathlineto{\pgfqpoint{3.974945in}{4.288546in}}%
\pgfpathlineto{\pgfqpoint{3.983503in}{4.288546in}}%
\pgfpathlineto{\pgfqpoint{3.992062in}{4.288546in}}%
\pgfpathlineto{\pgfqpoint{4.000620in}{4.288546in}}%
\pgfpathlineto{\pgfqpoint{4.009179in}{4.288546in}}%
\pgfpathlineto{\pgfqpoint{4.017737in}{4.288546in}}%
\pgfpathlineto{\pgfqpoint{4.026296in}{4.288546in}}%
\pgfpathlineto{\pgfqpoint{4.034854in}{4.288546in}}%
\pgfpathlineto{\pgfqpoint{4.043413in}{4.288546in}}%
\pgfpathlineto{\pgfqpoint{4.051971in}{4.288546in}}%
\pgfpathlineto{\pgfqpoint{4.060529in}{4.288546in}}%
\pgfpathlineto{\pgfqpoint{4.069088in}{4.288546in}}%
\pgfpathlineto{\pgfqpoint{4.077646in}{4.288546in}}%
\pgfpathlineto{\pgfqpoint{4.086205in}{4.288546in}}%
\pgfpathlineto{\pgfqpoint{4.094763in}{4.288546in}}%
\pgfpathlineto{\pgfqpoint{4.103322in}{4.288546in}}%
\pgfpathlineto{\pgfqpoint{4.111880in}{4.288546in}}%
\pgfpathlineto{\pgfqpoint{4.120439in}{4.288546in}}%
\pgfpathlineto{\pgfqpoint{4.128997in}{4.288546in}}%
\pgfpathlineto{\pgfqpoint{4.137555in}{4.288546in}}%
\pgfpathlineto{\pgfqpoint{4.146114in}{4.288546in}}%
\pgfpathlineto{\pgfqpoint{4.146114in}{4.288548in}}%
\pgfpathlineto{\pgfqpoint{4.146114in}{4.288548in}}%
\pgfpathlineto{\pgfqpoint{4.137555in}{4.288550in}}%
\pgfpathlineto{\pgfqpoint{4.128997in}{4.288555in}}%
\pgfpathlineto{\pgfqpoint{4.120439in}{4.288565in}}%
\pgfpathlineto{\pgfqpoint{4.111880in}{4.288586in}}%
\pgfpathlineto{\pgfqpoint{4.103322in}{4.288622in}}%
\pgfpathlineto{\pgfqpoint{4.094763in}{4.288682in}}%
\pgfpathlineto{\pgfqpoint{4.086205in}{4.288774in}}%
\pgfpathlineto{\pgfqpoint{4.077646in}{4.288906in}}%
\pgfpathlineto{\pgfqpoint{4.069088in}{4.289082in}}%
\pgfpathlineto{\pgfqpoint{4.060529in}{4.289309in}}%
\pgfpathlineto{\pgfqpoint{4.051971in}{4.289598in}}%
\pgfpathlineto{\pgfqpoint{4.043413in}{4.289966in}}%
\pgfpathlineto{\pgfqpoint{4.034854in}{4.290438in}}%
\pgfpathlineto{\pgfqpoint{4.026296in}{4.291043in}}%
\pgfpathlineto{\pgfqpoint{4.017737in}{4.291807in}}%
\pgfpathlineto{\pgfqpoint{4.009179in}{4.292732in}}%
\pgfpathlineto{\pgfqpoint{4.000620in}{4.293794in}}%
\pgfpathlineto{\pgfqpoint{3.992062in}{4.294928in}}%
\pgfpathlineto{\pgfqpoint{3.983503in}{4.296040in}}%
\pgfpathlineto{\pgfqpoint{3.974945in}{4.297019in}}%
\pgfpathlineto{\pgfqpoint{3.966387in}{4.297764in}}%
\pgfpathlineto{\pgfqpoint{3.957828in}{4.298203in}}%
\pgfpathlineto{\pgfqpoint{3.949270in}{4.298323in}}%
\pgfpathlineto{\pgfqpoint{3.940711in}{4.298172in}}%
\pgfpathlineto{\pgfqpoint{3.932153in}{4.297855in}}%
\pgfpathlineto{\pgfqpoint{3.923594in}{4.297512in}}%
\pgfpathlineto{\pgfqpoint{3.915036in}{4.297277in}}%
\pgfpathlineto{\pgfqpoint{3.906477in}{4.297234in}}%
\pgfpathlineto{\pgfqpoint{3.897919in}{4.297389in}}%
\pgfpathlineto{\pgfqpoint{3.889361in}{4.297662in}}%
\pgfpathlineto{\pgfqpoint{3.880802in}{4.297914in}}%
\pgfpathlineto{\pgfqpoint{3.872244in}{4.297990in}}%
\pgfpathlineto{\pgfqpoint{3.863685in}{4.297774in}}%
\pgfpathlineto{\pgfqpoint{3.855127in}{4.297219in}}%
\pgfpathlineto{\pgfqpoint{3.846568in}{4.296364in}}%
\pgfpathlineto{\pgfqpoint{3.838010in}{4.295324in}}%
\pgfpathlineto{\pgfqpoint{3.829451in}{4.294260in}}%
\pgfpathlineto{\pgfqpoint{3.820893in}{4.293347in}}%
\pgfpathlineto{\pgfqpoint{3.812335in}{4.292730in}}%
\pgfpathlineto{\pgfqpoint{3.803776in}{4.292489in}}%
\pgfpathlineto{\pgfqpoint{3.795218in}{4.292616in}}%
\pgfpathlineto{\pgfqpoint{3.786659in}{4.293010in}}%
\pgfpathlineto{\pgfqpoint{3.778101in}{4.293503in}}%
\pgfpathlineto{\pgfqpoint{3.769542in}{4.293901in}}%
\pgfpathlineto{\pgfqpoint{3.760984in}{4.294047in}}%
\pgfpathlineto{\pgfqpoint{3.752425in}{4.293862in}}%
\pgfpathlineto{\pgfqpoint{3.743867in}{4.293369in}}%
\pgfpathlineto{\pgfqpoint{3.735309in}{4.292672in}}%
\pgfpathlineto{\pgfqpoint{3.726750in}{4.291912in}}%
\pgfpathlineto{\pgfqpoint{3.718192in}{4.291216in}}%
\pgfpathlineto{\pgfqpoint{3.709633in}{4.290663in}}%
\pgfpathlineto{\pgfqpoint{3.701075in}{4.290272in}}%
\pgfpathlineto{\pgfqpoint{3.692516in}{4.290020in}}%
\pgfpathlineto{\pgfqpoint{3.683958in}{4.289859in}}%
\pgfpathlineto{\pgfqpoint{3.675399in}{4.289744in}}%
\pgfpathlineto{\pgfqpoint{3.666841in}{4.289642in}}%
\pgfpathlineto{\pgfqpoint{3.658283in}{4.289537in}}%
\pgfpathlineto{\pgfqpoint{3.649724in}{4.289427in}}%
\pgfpathlineto{\pgfqpoint{3.641166in}{4.289319in}}%
\pgfpathlineto{\pgfqpoint{3.632607in}{4.289221in}}%
\pgfpathlineto{\pgfqpoint{3.624049in}{4.289141in}}%
\pgfpathlineto{\pgfqpoint{3.615490in}{4.289085in}}%
\pgfpathlineto{\pgfqpoint{3.606932in}{4.289058in}}%
\pgfpathlineto{\pgfqpoint{3.598373in}{4.289071in}}%
\pgfpathlineto{\pgfqpoint{3.589815in}{4.289147in}}%
\pgfpathlineto{\pgfqpoint{3.581257in}{4.289318in}}%
\pgfpathlineto{\pgfqpoint{3.572698in}{4.289634in}}%
\pgfpathlineto{\pgfqpoint{3.564140in}{4.290152in}}%
\pgfpathlineto{\pgfqpoint{3.555581in}{4.290923in}}%
\pgfpathlineto{\pgfqpoint{3.547023in}{4.291973in}}%
\pgfpathlineto{\pgfqpoint{3.538464in}{4.293281in}}%
\pgfpathlineto{\pgfqpoint{3.529906in}{4.294768in}}%
\pgfpathlineto{\pgfqpoint{3.521347in}{4.296294in}}%
\pgfpathlineto{\pgfqpoint{3.512789in}{4.297684in}}%
\pgfpathlineto{\pgfqpoint{3.504231in}{4.298769in}}%
\pgfpathlineto{\pgfqpoint{3.495672in}{4.299429in}}%
\pgfpathlineto{\pgfqpoint{3.487114in}{4.299618in}}%
\pgfpathlineto{\pgfqpoint{3.478555in}{4.299375in}}%
\pgfpathlineto{\pgfqpoint{3.469997in}{4.298791in}}%
\pgfpathlineto{\pgfqpoint{3.461438in}{4.297968in}}%
\pgfpathlineto{\pgfqpoint{3.452880in}{4.296988in}}%
\pgfpathlineto{\pgfqpoint{3.444321in}{4.295904in}}%
\pgfpathlineto{\pgfqpoint{3.435763in}{4.294758in}}%
\pgfpathlineto{\pgfqpoint{3.427205in}{4.293595in}}%
\pgfpathlineto{\pgfqpoint{3.418646in}{4.292476in}}%
\pgfpathlineto{\pgfqpoint{3.410088in}{4.291471in}}%
\pgfpathlineto{\pgfqpoint{3.401529in}{4.290639in}}%
\pgfpathlineto{\pgfqpoint{3.392971in}{4.290011in}}%
\pgfpathlineto{\pgfqpoint{3.384412in}{4.289589in}}%
\pgfpathlineto{\pgfqpoint{3.375854in}{4.289355in}}%
\pgfpathlineto{\pgfqpoint{3.367295in}{4.289284in}}%
\pgfpathlineto{\pgfqpoint{3.358737in}{4.289359in}}%
\pgfpathlineto{\pgfqpoint{3.350178in}{4.289570in}}%
\pgfpathlineto{\pgfqpoint{3.341620in}{4.289913in}}%
\pgfpathlineto{\pgfqpoint{3.333062in}{4.290375in}}%
\pgfpathlineto{\pgfqpoint{3.324503in}{4.290920in}}%
\pgfpathlineto{\pgfqpoint{3.315945in}{4.291487in}}%
\pgfpathlineto{\pgfqpoint{3.307386in}{4.291994in}}%
\pgfpathlineto{\pgfqpoint{3.298828in}{4.292361in}}%
\pgfpathlineto{\pgfqpoint{3.290269in}{4.292532in}}%
\pgfpathlineto{\pgfqpoint{3.281711in}{4.292493in}}%
\pgfpathlineto{\pgfqpoint{3.273152in}{4.292270in}}%
\pgfpathlineto{\pgfqpoint{3.264594in}{4.291917in}}%
\pgfpathlineto{\pgfqpoint{3.256036in}{4.291489in}}%
\pgfpathlineto{\pgfqpoint{3.247477in}{4.291036in}}%
\pgfpathlineto{\pgfqpoint{3.238919in}{4.290588in}}%
\pgfpathlineto{\pgfqpoint{3.230360in}{4.290168in}}%
\pgfpathlineto{\pgfqpoint{3.221802in}{4.289793in}}%
\pgfpathlineto{\pgfqpoint{3.213243in}{4.289476in}}%
\pgfpathlineto{\pgfqpoint{3.204685in}{4.289227in}}%
\pgfpathlineto{\pgfqpoint{3.196126in}{4.289048in}}%
\pgfpathlineto{\pgfqpoint{3.187568in}{4.288936in}}%
\pgfpathlineto{\pgfqpoint{3.179010in}{4.288880in}}%
\pgfpathlineto{\pgfqpoint{3.170451in}{4.288866in}}%
\pgfpathlineto{\pgfqpoint{3.161893in}{4.288879in}}%
\pgfpathlineto{\pgfqpoint{3.153334in}{4.288905in}}%
\pgfpathlineto{\pgfqpoint{3.144776in}{4.288928in}}%
\pgfpathlineto{\pgfqpoint{3.136217in}{4.288939in}}%
\pgfpathlineto{\pgfqpoint{3.127659in}{4.288931in}}%
\pgfpathlineto{\pgfqpoint{3.119100in}{4.288905in}}%
\pgfpathlineto{\pgfqpoint{3.110542in}{4.288866in}}%
\pgfpathlineto{\pgfqpoint{3.101984in}{4.288823in}}%
\pgfpathlineto{\pgfqpoint{3.093425in}{4.288783in}}%
\pgfpathlineto{\pgfqpoint{3.084867in}{4.288754in}}%
\pgfpathlineto{\pgfqpoint{3.076308in}{4.288739in}}%
\pgfpathlineto{\pgfqpoint{3.067750in}{4.288739in}}%
\pgfpathlineto{\pgfqpoint{3.059191in}{4.288752in}}%
\pgfpathlineto{\pgfqpoint{3.050633in}{4.288774in}}%
\pgfpathlineto{\pgfqpoint{3.042074in}{4.288801in}}%
\pgfpathlineto{\pgfqpoint{3.033516in}{4.288827in}}%
\pgfpathlineto{\pgfqpoint{3.024958in}{4.288849in}}%
\pgfpathlineto{\pgfqpoint{3.016399in}{4.288865in}}%
\pgfpathlineto{\pgfqpoint{3.007841in}{4.288873in}}%
\pgfpathlineto{\pgfqpoint{2.999282in}{4.288874in}}%
\pgfpathlineto{\pgfqpoint{2.990724in}{4.288871in}}%
\pgfpathlineto{\pgfqpoint{2.982165in}{4.288863in}}%
\pgfpathlineto{\pgfqpoint{2.973607in}{4.288851in}}%
\pgfpathlineto{\pgfqpoint{2.965048in}{4.288835in}}%
\pgfpathlineto{\pgfqpoint{2.956490in}{4.288816in}}%
\pgfpathlineto{\pgfqpoint{2.947932in}{4.288794in}}%
\pgfpathlineto{\pgfqpoint{2.939373in}{4.288771in}}%
\pgfpathlineto{\pgfqpoint{2.930815in}{4.288749in}}%
\pgfpathlineto{\pgfqpoint{2.922256in}{4.288730in}}%
\pgfpathlineto{\pgfqpoint{2.913698in}{4.288715in}}%
\pgfpathlineto{\pgfqpoint{2.905139in}{4.288703in}}%
\pgfpathlineto{\pgfqpoint{2.896581in}{4.288694in}}%
\pgfpathlineto{\pgfqpoint{2.888022in}{4.288688in}}%
\pgfpathlineto{\pgfqpoint{2.879464in}{4.288684in}}%
\pgfpathlineto{\pgfqpoint{2.870906in}{4.288680in}}%
\pgfpathlineto{\pgfqpoint{2.862347in}{4.288677in}}%
\pgfpathlineto{\pgfqpoint{2.853789in}{4.288674in}}%
\pgfpathlineto{\pgfqpoint{2.845230in}{4.288670in}}%
\pgfpathlineto{\pgfqpoint{2.836672in}{4.288666in}}%
\pgfpathlineto{\pgfqpoint{2.828113in}{4.288661in}}%
\pgfpathlineto{\pgfqpoint{2.819555in}{4.288655in}}%
\pgfpathlineto{\pgfqpoint{2.810996in}{4.288648in}}%
\pgfpathlineto{\pgfqpoint{2.802438in}{4.288641in}}%
\pgfpathlineto{\pgfqpoint{2.793880in}{4.288636in}}%
\pgfpathlineto{\pgfqpoint{2.785321in}{4.288633in}}%
\pgfpathlineto{\pgfqpoint{2.776763in}{4.288632in}}%
\pgfpathlineto{\pgfqpoint{2.768204in}{4.288633in}}%
\pgfpathlineto{\pgfqpoint{2.759646in}{4.288634in}}%
\pgfpathlineto{\pgfqpoint{2.751087in}{4.288635in}}%
\pgfpathlineto{\pgfqpoint{2.742529in}{4.288635in}}%
\pgfpathlineto{\pgfqpoint{2.733970in}{4.288633in}}%
\pgfpathlineto{\pgfqpoint{2.725412in}{4.288630in}}%
\pgfpathlineto{\pgfqpoint{2.716854in}{4.288625in}}%
\pgfpathlineto{\pgfqpoint{2.708295in}{4.288619in}}%
\pgfpathlineto{\pgfqpoint{2.699737in}{4.288613in}}%
\pgfpathlineto{\pgfqpoint{2.691178in}{4.288606in}}%
\pgfpathlineto{\pgfqpoint{2.682620in}{4.288598in}}%
\pgfpathlineto{\pgfqpoint{2.674061in}{4.288591in}}%
\pgfpathlineto{\pgfqpoint{2.665503in}{4.288584in}}%
\pgfpathlineto{\pgfqpoint{2.656944in}{4.288578in}}%
\pgfpathlineto{\pgfqpoint{2.648386in}{4.288573in}}%
\pgfpathlineto{\pgfqpoint{2.639828in}{4.288570in}}%
\pgfpathlineto{\pgfqpoint{2.631269in}{4.288568in}}%
\pgfpathlineto{\pgfqpoint{2.622711in}{4.288566in}}%
\pgfpathlineto{\pgfqpoint{2.614152in}{4.288565in}}%
\pgfpathlineto{\pgfqpoint{2.605594in}{4.288565in}}%
\pgfpathlineto{\pgfqpoint{2.597035in}{4.288565in}}%
\pgfpathlineto{\pgfqpoint{2.588477in}{4.288564in}}%
\pgfpathlineto{\pgfqpoint{2.579918in}{4.288564in}}%
\pgfpathlineto{\pgfqpoint{2.571360in}{4.288563in}}%
\pgfpathlineto{\pgfqpoint{2.562802in}{4.288561in}}%
\pgfpathlineto{\pgfqpoint{2.554243in}{4.288559in}}%
\pgfpathlineto{\pgfqpoint{2.545685in}{4.288557in}}%
\pgfpathlineto{\pgfqpoint{2.537126in}{4.288554in}}%
\pgfpathlineto{\pgfqpoint{2.528568in}{4.288552in}}%
\pgfpathlineto{\pgfqpoint{2.520009in}{4.288551in}}%
\pgfpathlineto{\pgfqpoint{2.511451in}{4.288549in}}%
\pgfpathlineto{\pgfqpoint{2.502892in}{4.288548in}}%
\pgfpathlineto{\pgfqpoint{2.494334in}{4.288548in}}%
\pgfpathlineto{\pgfqpoint{2.485776in}{4.288547in}}%
\pgfpathlineto{\pgfqpoint{2.477217in}{4.288547in}}%
\pgfpathlineto{\pgfqpoint{2.468659in}{4.288547in}}%
\pgfpathlineto{\pgfqpoint{2.460100in}{4.288547in}}%
\pgfpathlineto{\pgfqpoint{2.451542in}{4.288546in}}%
\pgfpathlineto{\pgfqpoint{2.442983in}{4.288546in}}%
\pgfpathclose%
\pgfusepath{stroke,fill}%
}%
\begin{pgfscope}%
\pgfsys@transformshift{0.000000in}{0.000000in}%
\pgfsys@useobject{currentmarker}{}%
\end{pgfscope}%
\end{pgfscope}%
\begin{pgfscope}%
\pgfpathrectangle{\pgfqpoint{0.692593in}{4.288546in}}{\pgfqpoint{3.617974in}{0.631476in}}%
\pgfusepath{clip}%
\pgfsetbuttcap%
\pgfsetroundjoin%
\definecolor{currentfill}{rgb}{0.121569,0.466667,0.705882}%
\pgfsetfillcolor{currentfill}%
\pgfsetfillopacity{0.200000}%
\pgfsetlinewidth{1.003750pt}%
\definecolor{currentstroke}{rgb}{0.121569,0.466667,0.705882}%
\pgfsetstrokecolor{currentstroke}%
\pgfsetdash{}{0pt}%
\pgfsys@defobject{currentmarker}{\pgfqpoint{1.015651in}{4.288546in}}{\pgfqpoint{2.552276in}{4.889952in}}{%
\pgfpathmoveto{\pgfqpoint{1.015651in}{4.288547in}}%
\pgfpathlineto{\pgfqpoint{1.015651in}{4.288546in}}%
\pgfpathlineto{\pgfqpoint{1.023373in}{4.288546in}}%
\pgfpathlineto{\pgfqpoint{1.031095in}{4.288546in}}%
\pgfpathlineto{\pgfqpoint{1.038816in}{4.288546in}}%
\pgfpathlineto{\pgfqpoint{1.046538in}{4.288546in}}%
\pgfpathlineto{\pgfqpoint{1.054260in}{4.288546in}}%
\pgfpathlineto{\pgfqpoint{1.061981in}{4.288546in}}%
\pgfpathlineto{\pgfqpoint{1.069703in}{4.288546in}}%
\pgfpathlineto{\pgfqpoint{1.077425in}{4.288546in}}%
\pgfpathlineto{\pgfqpoint{1.085147in}{4.288546in}}%
\pgfpathlineto{\pgfqpoint{1.092868in}{4.288546in}}%
\pgfpathlineto{\pgfqpoint{1.100590in}{4.288546in}}%
\pgfpathlineto{\pgfqpoint{1.108312in}{4.288546in}}%
\pgfpathlineto{\pgfqpoint{1.116034in}{4.288546in}}%
\pgfpathlineto{\pgfqpoint{1.123755in}{4.288546in}}%
\pgfpathlineto{\pgfqpoint{1.131477in}{4.288546in}}%
\pgfpathlineto{\pgfqpoint{1.139199in}{4.288546in}}%
\pgfpathlineto{\pgfqpoint{1.146921in}{4.288546in}}%
\pgfpathlineto{\pgfqpoint{1.154642in}{4.288546in}}%
\pgfpathlineto{\pgfqpoint{1.162364in}{4.288546in}}%
\pgfpathlineto{\pgfqpoint{1.170086in}{4.288546in}}%
\pgfpathlineto{\pgfqpoint{1.177808in}{4.288546in}}%
\pgfpathlineto{\pgfqpoint{1.185529in}{4.288546in}}%
\pgfpathlineto{\pgfqpoint{1.193251in}{4.288546in}}%
\pgfpathlineto{\pgfqpoint{1.200973in}{4.288546in}}%
\pgfpathlineto{\pgfqpoint{1.208694in}{4.288546in}}%
\pgfpathlineto{\pgfqpoint{1.216416in}{4.288546in}}%
\pgfpathlineto{\pgfqpoint{1.224138in}{4.288546in}}%
\pgfpathlineto{\pgfqpoint{1.231860in}{4.288546in}}%
\pgfpathlineto{\pgfqpoint{1.239581in}{4.288546in}}%
\pgfpathlineto{\pgfqpoint{1.247303in}{4.288546in}}%
\pgfpathlineto{\pgfqpoint{1.255025in}{4.288546in}}%
\pgfpathlineto{\pgfqpoint{1.262747in}{4.288546in}}%
\pgfpathlineto{\pgfqpoint{1.270468in}{4.288546in}}%
\pgfpathlineto{\pgfqpoint{1.278190in}{4.288546in}}%
\pgfpathlineto{\pgfqpoint{1.285912in}{4.288546in}}%
\pgfpathlineto{\pgfqpoint{1.293634in}{4.288546in}}%
\pgfpathlineto{\pgfqpoint{1.301355in}{4.288546in}}%
\pgfpathlineto{\pgfqpoint{1.309077in}{4.288546in}}%
\pgfpathlineto{\pgfqpoint{1.316799in}{4.288546in}}%
\pgfpathlineto{\pgfqpoint{1.324520in}{4.288546in}}%
\pgfpathlineto{\pgfqpoint{1.332242in}{4.288546in}}%
\pgfpathlineto{\pgfqpoint{1.339964in}{4.288546in}}%
\pgfpathlineto{\pgfqpoint{1.347686in}{4.288546in}}%
\pgfpathlineto{\pgfqpoint{1.355407in}{4.288546in}}%
\pgfpathlineto{\pgfqpoint{1.363129in}{4.288546in}}%
\pgfpathlineto{\pgfqpoint{1.370851in}{4.288546in}}%
\pgfpathlineto{\pgfqpoint{1.378573in}{4.288546in}}%
\pgfpathlineto{\pgfqpoint{1.386294in}{4.288546in}}%
\pgfpathlineto{\pgfqpoint{1.394016in}{4.288546in}}%
\pgfpathlineto{\pgfqpoint{1.401738in}{4.288546in}}%
\pgfpathlineto{\pgfqpoint{1.409460in}{4.288546in}}%
\pgfpathlineto{\pgfqpoint{1.417181in}{4.288546in}}%
\pgfpathlineto{\pgfqpoint{1.424903in}{4.288546in}}%
\pgfpathlineto{\pgfqpoint{1.432625in}{4.288546in}}%
\pgfpathlineto{\pgfqpoint{1.440347in}{4.288546in}}%
\pgfpathlineto{\pgfqpoint{1.448068in}{4.288546in}}%
\pgfpathlineto{\pgfqpoint{1.455790in}{4.288546in}}%
\pgfpathlineto{\pgfqpoint{1.463512in}{4.288546in}}%
\pgfpathlineto{\pgfqpoint{1.471233in}{4.288546in}}%
\pgfpathlineto{\pgfqpoint{1.478955in}{4.288546in}}%
\pgfpathlineto{\pgfqpoint{1.486677in}{4.288546in}}%
\pgfpathlineto{\pgfqpoint{1.494399in}{4.288546in}}%
\pgfpathlineto{\pgfqpoint{1.502120in}{4.288546in}}%
\pgfpathlineto{\pgfqpoint{1.509842in}{4.288546in}}%
\pgfpathlineto{\pgfqpoint{1.517564in}{4.288546in}}%
\pgfpathlineto{\pgfqpoint{1.525286in}{4.288546in}}%
\pgfpathlineto{\pgfqpoint{1.533007in}{4.288546in}}%
\pgfpathlineto{\pgfqpoint{1.540729in}{4.288546in}}%
\pgfpathlineto{\pgfqpoint{1.548451in}{4.288546in}}%
\pgfpathlineto{\pgfqpoint{1.556173in}{4.288546in}}%
\pgfpathlineto{\pgfqpoint{1.563894in}{4.288546in}}%
\pgfpathlineto{\pgfqpoint{1.571616in}{4.288546in}}%
\pgfpathlineto{\pgfqpoint{1.579338in}{4.288546in}}%
\pgfpathlineto{\pgfqpoint{1.587059in}{4.288546in}}%
\pgfpathlineto{\pgfqpoint{1.594781in}{4.288546in}}%
\pgfpathlineto{\pgfqpoint{1.602503in}{4.288546in}}%
\pgfpathlineto{\pgfqpoint{1.610225in}{4.288546in}}%
\pgfpathlineto{\pgfqpoint{1.617946in}{4.288546in}}%
\pgfpathlineto{\pgfqpoint{1.625668in}{4.288546in}}%
\pgfpathlineto{\pgfqpoint{1.633390in}{4.288546in}}%
\pgfpathlineto{\pgfqpoint{1.641112in}{4.288546in}}%
\pgfpathlineto{\pgfqpoint{1.648833in}{4.288546in}}%
\pgfpathlineto{\pgfqpoint{1.656555in}{4.288546in}}%
\pgfpathlineto{\pgfqpoint{1.664277in}{4.288546in}}%
\pgfpathlineto{\pgfqpoint{1.671999in}{4.288546in}}%
\pgfpathlineto{\pgfqpoint{1.679720in}{4.288546in}}%
\pgfpathlineto{\pgfqpoint{1.687442in}{4.288546in}}%
\pgfpathlineto{\pgfqpoint{1.695164in}{4.288546in}}%
\pgfpathlineto{\pgfqpoint{1.702886in}{4.288546in}}%
\pgfpathlineto{\pgfqpoint{1.710607in}{4.288546in}}%
\pgfpathlineto{\pgfqpoint{1.718329in}{4.288546in}}%
\pgfpathlineto{\pgfqpoint{1.726051in}{4.288546in}}%
\pgfpathlineto{\pgfqpoint{1.733772in}{4.288546in}}%
\pgfpathlineto{\pgfqpoint{1.741494in}{4.288546in}}%
\pgfpathlineto{\pgfqpoint{1.749216in}{4.288546in}}%
\pgfpathlineto{\pgfqpoint{1.756938in}{4.288546in}}%
\pgfpathlineto{\pgfqpoint{1.764659in}{4.288546in}}%
\pgfpathlineto{\pgfqpoint{1.772381in}{4.288546in}}%
\pgfpathlineto{\pgfqpoint{1.780103in}{4.288546in}}%
\pgfpathlineto{\pgfqpoint{1.787825in}{4.288546in}}%
\pgfpathlineto{\pgfqpoint{1.795546in}{4.288546in}}%
\pgfpathlineto{\pgfqpoint{1.803268in}{4.288546in}}%
\pgfpathlineto{\pgfqpoint{1.810990in}{4.288546in}}%
\pgfpathlineto{\pgfqpoint{1.818712in}{4.288546in}}%
\pgfpathlineto{\pgfqpoint{1.826433in}{4.288546in}}%
\pgfpathlineto{\pgfqpoint{1.834155in}{4.288546in}}%
\pgfpathlineto{\pgfqpoint{1.841877in}{4.288546in}}%
\pgfpathlineto{\pgfqpoint{1.849598in}{4.288546in}}%
\pgfpathlineto{\pgfqpoint{1.857320in}{4.288546in}}%
\pgfpathlineto{\pgfqpoint{1.865042in}{4.288546in}}%
\pgfpathlineto{\pgfqpoint{1.872764in}{4.288546in}}%
\pgfpathlineto{\pgfqpoint{1.880485in}{4.288546in}}%
\pgfpathlineto{\pgfqpoint{1.888207in}{4.288546in}}%
\pgfpathlineto{\pgfqpoint{1.895929in}{4.288546in}}%
\pgfpathlineto{\pgfqpoint{1.903651in}{4.288546in}}%
\pgfpathlineto{\pgfqpoint{1.911372in}{4.288546in}}%
\pgfpathlineto{\pgfqpoint{1.919094in}{4.288546in}}%
\pgfpathlineto{\pgfqpoint{1.926816in}{4.288546in}}%
\pgfpathlineto{\pgfqpoint{1.934538in}{4.288546in}}%
\pgfpathlineto{\pgfqpoint{1.942259in}{4.288546in}}%
\pgfpathlineto{\pgfqpoint{1.949981in}{4.288546in}}%
\pgfpathlineto{\pgfqpoint{1.957703in}{4.288546in}}%
\pgfpathlineto{\pgfqpoint{1.965425in}{4.288546in}}%
\pgfpathlineto{\pgfqpoint{1.973146in}{4.288546in}}%
\pgfpathlineto{\pgfqpoint{1.980868in}{4.288546in}}%
\pgfpathlineto{\pgfqpoint{1.988590in}{4.288546in}}%
\pgfpathlineto{\pgfqpoint{1.996311in}{4.288546in}}%
\pgfpathlineto{\pgfqpoint{2.004033in}{4.288546in}}%
\pgfpathlineto{\pgfqpoint{2.011755in}{4.288546in}}%
\pgfpathlineto{\pgfqpoint{2.019477in}{4.288546in}}%
\pgfpathlineto{\pgfqpoint{2.027198in}{4.288546in}}%
\pgfpathlineto{\pgfqpoint{2.034920in}{4.288546in}}%
\pgfpathlineto{\pgfqpoint{2.042642in}{4.288546in}}%
\pgfpathlineto{\pgfqpoint{2.050364in}{4.288546in}}%
\pgfpathlineto{\pgfqpoint{2.058085in}{4.288546in}}%
\pgfpathlineto{\pgfqpoint{2.065807in}{4.288546in}}%
\pgfpathlineto{\pgfqpoint{2.073529in}{4.288546in}}%
\pgfpathlineto{\pgfqpoint{2.081251in}{4.288546in}}%
\pgfpathlineto{\pgfqpoint{2.088972in}{4.288546in}}%
\pgfpathlineto{\pgfqpoint{2.096694in}{4.288546in}}%
\pgfpathlineto{\pgfqpoint{2.104416in}{4.288546in}}%
\pgfpathlineto{\pgfqpoint{2.112137in}{4.288546in}}%
\pgfpathlineto{\pgfqpoint{2.119859in}{4.288546in}}%
\pgfpathlineto{\pgfqpoint{2.127581in}{4.288546in}}%
\pgfpathlineto{\pgfqpoint{2.135303in}{4.288546in}}%
\pgfpathlineto{\pgfqpoint{2.143024in}{4.288546in}}%
\pgfpathlineto{\pgfqpoint{2.150746in}{4.288546in}}%
\pgfpathlineto{\pgfqpoint{2.158468in}{4.288546in}}%
\pgfpathlineto{\pgfqpoint{2.166190in}{4.288546in}}%
\pgfpathlineto{\pgfqpoint{2.173911in}{4.288546in}}%
\pgfpathlineto{\pgfqpoint{2.181633in}{4.288546in}}%
\pgfpathlineto{\pgfqpoint{2.189355in}{4.288546in}}%
\pgfpathlineto{\pgfqpoint{2.197077in}{4.288546in}}%
\pgfpathlineto{\pgfqpoint{2.204798in}{4.288546in}}%
\pgfpathlineto{\pgfqpoint{2.212520in}{4.288546in}}%
\pgfpathlineto{\pgfqpoint{2.220242in}{4.288546in}}%
\pgfpathlineto{\pgfqpoint{2.227963in}{4.288546in}}%
\pgfpathlineto{\pgfqpoint{2.235685in}{4.288546in}}%
\pgfpathlineto{\pgfqpoint{2.243407in}{4.288546in}}%
\pgfpathlineto{\pgfqpoint{2.251129in}{4.288546in}}%
\pgfpathlineto{\pgfqpoint{2.258850in}{4.288546in}}%
\pgfpathlineto{\pgfqpoint{2.266572in}{4.288546in}}%
\pgfpathlineto{\pgfqpoint{2.274294in}{4.288546in}}%
\pgfpathlineto{\pgfqpoint{2.282016in}{4.288546in}}%
\pgfpathlineto{\pgfqpoint{2.289737in}{4.288546in}}%
\pgfpathlineto{\pgfqpoint{2.297459in}{4.288546in}}%
\pgfpathlineto{\pgfqpoint{2.305181in}{4.288546in}}%
\pgfpathlineto{\pgfqpoint{2.312903in}{4.288546in}}%
\pgfpathlineto{\pgfqpoint{2.320624in}{4.288546in}}%
\pgfpathlineto{\pgfqpoint{2.328346in}{4.288546in}}%
\pgfpathlineto{\pgfqpoint{2.336068in}{4.288546in}}%
\pgfpathlineto{\pgfqpoint{2.343790in}{4.288546in}}%
\pgfpathlineto{\pgfqpoint{2.351511in}{4.288546in}}%
\pgfpathlineto{\pgfqpoint{2.359233in}{4.288546in}}%
\pgfpathlineto{\pgfqpoint{2.366955in}{4.288546in}}%
\pgfpathlineto{\pgfqpoint{2.374676in}{4.288546in}}%
\pgfpathlineto{\pgfqpoint{2.382398in}{4.288546in}}%
\pgfpathlineto{\pgfqpoint{2.390120in}{4.288546in}}%
\pgfpathlineto{\pgfqpoint{2.397842in}{4.288546in}}%
\pgfpathlineto{\pgfqpoint{2.405563in}{4.288546in}}%
\pgfpathlineto{\pgfqpoint{2.413285in}{4.288546in}}%
\pgfpathlineto{\pgfqpoint{2.421007in}{4.288546in}}%
\pgfpathlineto{\pgfqpoint{2.428729in}{4.288546in}}%
\pgfpathlineto{\pgfqpoint{2.436450in}{4.288546in}}%
\pgfpathlineto{\pgfqpoint{2.444172in}{4.288546in}}%
\pgfpathlineto{\pgfqpoint{2.451894in}{4.288546in}}%
\pgfpathlineto{\pgfqpoint{2.459616in}{4.288546in}}%
\pgfpathlineto{\pgfqpoint{2.467337in}{4.288546in}}%
\pgfpathlineto{\pgfqpoint{2.475059in}{4.288546in}}%
\pgfpathlineto{\pgfqpoint{2.482781in}{4.288546in}}%
\pgfpathlineto{\pgfqpoint{2.490502in}{4.288546in}}%
\pgfpathlineto{\pgfqpoint{2.498224in}{4.288546in}}%
\pgfpathlineto{\pgfqpoint{2.505946in}{4.288546in}}%
\pgfpathlineto{\pgfqpoint{2.513668in}{4.288546in}}%
\pgfpathlineto{\pgfqpoint{2.521389in}{4.288546in}}%
\pgfpathlineto{\pgfqpoint{2.529111in}{4.288546in}}%
\pgfpathlineto{\pgfqpoint{2.536833in}{4.288546in}}%
\pgfpathlineto{\pgfqpoint{2.544555in}{4.288546in}}%
\pgfpathlineto{\pgfqpoint{2.552276in}{4.288546in}}%
\pgfpathlineto{\pgfqpoint{2.552276in}{4.288547in}}%
\pgfpathlineto{\pgfqpoint{2.552276in}{4.288547in}}%
\pgfpathlineto{\pgfqpoint{2.544555in}{4.288552in}}%
\pgfpathlineto{\pgfqpoint{2.536833in}{4.288559in}}%
\pgfpathlineto{\pgfqpoint{2.529111in}{4.288548in}}%
\pgfpathlineto{\pgfqpoint{2.521389in}{4.288546in}}%
\pgfpathlineto{\pgfqpoint{2.513668in}{4.288546in}}%
\pgfpathlineto{\pgfqpoint{2.505946in}{4.288546in}}%
\pgfpathlineto{\pgfqpoint{2.498224in}{4.288546in}}%
\pgfpathlineto{\pgfqpoint{2.490502in}{4.288546in}}%
\pgfpathlineto{\pgfqpoint{2.482781in}{4.288546in}}%
\pgfpathlineto{\pgfqpoint{2.475059in}{4.288546in}}%
\pgfpathlineto{\pgfqpoint{2.467337in}{4.288546in}}%
\pgfpathlineto{\pgfqpoint{2.459616in}{4.288546in}}%
\pgfpathlineto{\pgfqpoint{2.451894in}{4.288546in}}%
\pgfpathlineto{\pgfqpoint{2.444172in}{4.288546in}}%
\pgfpathlineto{\pgfqpoint{2.436450in}{4.288546in}}%
\pgfpathlineto{\pgfqpoint{2.428729in}{4.288546in}}%
\pgfpathlineto{\pgfqpoint{2.421007in}{4.288546in}}%
\pgfpathlineto{\pgfqpoint{2.413285in}{4.288546in}}%
\pgfpathlineto{\pgfqpoint{2.405563in}{4.288546in}}%
\pgfpathlineto{\pgfqpoint{2.397842in}{4.288546in}}%
\pgfpathlineto{\pgfqpoint{2.390120in}{4.288546in}}%
\pgfpathlineto{\pgfqpoint{2.382398in}{4.288546in}}%
\pgfpathlineto{\pgfqpoint{2.374676in}{4.288546in}}%
\pgfpathlineto{\pgfqpoint{2.366955in}{4.288546in}}%
\pgfpathlineto{\pgfqpoint{2.359233in}{4.288546in}}%
\pgfpathlineto{\pgfqpoint{2.351511in}{4.288546in}}%
\pgfpathlineto{\pgfqpoint{2.343790in}{4.288546in}}%
\pgfpathlineto{\pgfqpoint{2.336068in}{4.288546in}}%
\pgfpathlineto{\pgfqpoint{2.328346in}{4.288546in}}%
\pgfpathlineto{\pgfqpoint{2.320624in}{4.288550in}}%
\pgfpathlineto{\pgfqpoint{2.312903in}{4.288560in}}%
\pgfpathlineto{\pgfqpoint{2.305181in}{4.288549in}}%
\pgfpathlineto{\pgfqpoint{2.297459in}{4.288546in}}%
\pgfpathlineto{\pgfqpoint{2.289737in}{4.288546in}}%
\pgfpathlineto{\pgfqpoint{2.282016in}{4.288546in}}%
\pgfpathlineto{\pgfqpoint{2.274294in}{4.288546in}}%
\pgfpathlineto{\pgfqpoint{2.266572in}{4.288546in}}%
\pgfpathlineto{\pgfqpoint{2.258850in}{4.288546in}}%
\pgfpathlineto{\pgfqpoint{2.251129in}{4.288546in}}%
\pgfpathlineto{\pgfqpoint{2.243407in}{4.288546in}}%
\pgfpathlineto{\pgfqpoint{2.235685in}{4.288546in}}%
\pgfpathlineto{\pgfqpoint{2.227963in}{4.288546in}}%
\pgfpathlineto{\pgfqpoint{2.220242in}{4.288546in}}%
\pgfpathlineto{\pgfqpoint{2.212520in}{4.288546in}}%
\pgfpathlineto{\pgfqpoint{2.204798in}{4.288546in}}%
\pgfpathlineto{\pgfqpoint{2.197077in}{4.288548in}}%
\pgfpathlineto{\pgfqpoint{2.189355in}{4.288579in}}%
\pgfpathlineto{\pgfqpoint{2.181633in}{4.288607in}}%
\pgfpathlineto{\pgfqpoint{2.173911in}{4.288575in}}%
\pgfpathlineto{\pgfqpoint{2.166190in}{4.288821in}}%
\pgfpathlineto{\pgfqpoint{2.158468in}{4.288958in}}%
\pgfpathlineto{\pgfqpoint{2.150746in}{4.288598in}}%
\pgfpathlineto{\pgfqpoint{2.143024in}{4.288568in}}%
\pgfpathlineto{\pgfqpoint{2.135303in}{4.288559in}}%
\pgfpathlineto{\pgfqpoint{2.127581in}{4.288553in}}%
\pgfpathlineto{\pgfqpoint{2.119859in}{4.288559in}}%
\pgfpathlineto{\pgfqpoint{2.112137in}{4.288548in}}%
\pgfpathlineto{\pgfqpoint{2.104416in}{4.288547in}}%
\pgfpathlineto{\pgfqpoint{2.096694in}{4.288555in}}%
\pgfpathlineto{\pgfqpoint{2.088972in}{4.288566in}}%
\pgfpathlineto{\pgfqpoint{2.081251in}{4.288556in}}%
\pgfpathlineto{\pgfqpoint{2.073529in}{4.288550in}}%
\pgfpathlineto{\pgfqpoint{2.065807in}{4.288558in}}%
\pgfpathlineto{\pgfqpoint{2.058085in}{4.288554in}}%
\pgfpathlineto{\pgfqpoint{2.050364in}{4.288551in}}%
\pgfpathlineto{\pgfqpoint{2.042642in}{4.288567in}}%
\pgfpathlineto{\pgfqpoint{2.034920in}{4.288563in}}%
\pgfpathlineto{\pgfqpoint{2.027198in}{4.288561in}}%
\pgfpathlineto{\pgfqpoint{2.019477in}{4.288552in}}%
\pgfpathlineto{\pgfqpoint{2.011755in}{4.288557in}}%
\pgfpathlineto{\pgfqpoint{2.004033in}{4.288558in}}%
\pgfpathlineto{\pgfqpoint{1.996311in}{4.288564in}}%
\pgfpathlineto{\pgfqpoint{1.988590in}{4.288564in}}%
\pgfpathlineto{\pgfqpoint{1.980868in}{4.288577in}}%
\pgfpathlineto{\pgfqpoint{1.973146in}{4.288565in}}%
\pgfpathlineto{\pgfqpoint{1.965425in}{4.288570in}}%
\pgfpathlineto{\pgfqpoint{1.957703in}{4.288611in}}%
\pgfpathlineto{\pgfqpoint{1.949981in}{4.288586in}}%
\pgfpathlineto{\pgfqpoint{1.942259in}{4.288576in}}%
\pgfpathlineto{\pgfqpoint{1.934538in}{4.288599in}}%
\pgfpathlineto{\pgfqpoint{1.926816in}{4.288578in}}%
\pgfpathlineto{\pgfqpoint{1.919094in}{4.288608in}}%
\pgfpathlineto{\pgfqpoint{1.911372in}{4.288588in}}%
\pgfpathlineto{\pgfqpoint{1.903651in}{4.288579in}}%
\pgfpathlineto{\pgfqpoint{1.895929in}{4.288598in}}%
\pgfpathlineto{\pgfqpoint{1.888207in}{4.288644in}}%
\pgfpathlineto{\pgfqpoint{1.880485in}{4.288634in}}%
\pgfpathlineto{\pgfqpoint{1.872764in}{4.288649in}}%
\pgfpathlineto{\pgfqpoint{1.865042in}{4.288616in}}%
\pgfpathlineto{\pgfqpoint{1.857320in}{4.288577in}}%
\pgfpathlineto{\pgfqpoint{1.849598in}{4.288644in}}%
\pgfpathlineto{\pgfqpoint{1.841877in}{4.288703in}}%
\pgfpathlineto{\pgfqpoint{1.834155in}{4.289028in}}%
\pgfpathlineto{\pgfqpoint{1.826433in}{4.288923in}}%
\pgfpathlineto{\pgfqpoint{1.818712in}{4.288733in}}%
\pgfpathlineto{\pgfqpoint{1.810990in}{4.288703in}}%
\pgfpathlineto{\pgfqpoint{1.803268in}{4.288693in}}%
\pgfpathlineto{\pgfqpoint{1.795546in}{4.288774in}}%
\pgfpathlineto{\pgfqpoint{1.787825in}{4.288769in}}%
\pgfpathlineto{\pgfqpoint{1.780103in}{4.288861in}}%
\pgfpathlineto{\pgfqpoint{1.772381in}{4.289699in}}%
\pgfpathlineto{\pgfqpoint{1.764659in}{4.289531in}}%
\pgfpathlineto{\pgfqpoint{1.756938in}{4.289273in}}%
\pgfpathlineto{\pgfqpoint{1.749216in}{4.289358in}}%
\pgfpathlineto{\pgfqpoint{1.741494in}{4.289279in}}%
\pgfpathlineto{\pgfqpoint{1.733772in}{4.289040in}}%
\pgfpathlineto{\pgfqpoint{1.726051in}{4.289016in}}%
\pgfpathlineto{\pgfqpoint{1.718329in}{4.289380in}}%
\pgfpathlineto{\pgfqpoint{1.710607in}{4.292260in}}%
\pgfpathlineto{\pgfqpoint{1.702886in}{4.292282in}}%
\pgfpathlineto{\pgfqpoint{1.695164in}{4.289406in}}%
\pgfpathlineto{\pgfqpoint{1.687442in}{4.289238in}}%
\pgfpathlineto{\pgfqpoint{1.679720in}{4.289333in}}%
\pgfpathlineto{\pgfqpoint{1.671999in}{4.289581in}}%
\pgfpathlineto{\pgfqpoint{1.664277in}{4.289692in}}%
\pgfpathlineto{\pgfqpoint{1.656555in}{4.290098in}}%
\pgfpathlineto{\pgfqpoint{1.648833in}{4.289805in}}%
\pgfpathlineto{\pgfqpoint{1.641112in}{4.289671in}}%
\pgfpathlineto{\pgfqpoint{1.633390in}{4.289812in}}%
\pgfpathlineto{\pgfqpoint{1.625668in}{4.289823in}}%
\pgfpathlineto{\pgfqpoint{1.617946in}{4.290497in}}%
\pgfpathlineto{\pgfqpoint{1.610225in}{4.298383in}}%
\pgfpathlineto{\pgfqpoint{1.602503in}{4.299684in}}%
\pgfpathlineto{\pgfqpoint{1.594781in}{4.291682in}}%
\pgfpathlineto{\pgfqpoint{1.587059in}{4.291834in}}%
\pgfpathlineto{\pgfqpoint{1.579338in}{4.292113in}}%
\pgfpathlineto{\pgfqpoint{1.571616in}{4.293235in}}%
\pgfpathlineto{\pgfqpoint{1.563894in}{4.295484in}}%
\pgfpathlineto{\pgfqpoint{1.556173in}{4.300338in}}%
\pgfpathlineto{\pgfqpoint{1.548451in}{4.301902in}}%
\pgfpathlineto{\pgfqpoint{1.540729in}{4.305752in}}%
\pgfpathlineto{\pgfqpoint{1.533007in}{4.306658in}}%
\pgfpathlineto{\pgfqpoint{1.525286in}{4.319195in}}%
\pgfpathlineto{\pgfqpoint{1.517564in}{4.360732in}}%
\pgfpathlineto{\pgfqpoint{1.509842in}{4.379626in}}%
\pgfpathlineto{\pgfqpoint{1.502120in}{4.400051in}}%
\pgfpathlineto{\pgfqpoint{1.494399in}{4.392791in}}%
\pgfpathlineto{\pgfqpoint{1.486677in}{4.380259in}}%
\pgfpathlineto{\pgfqpoint{1.478955in}{4.395630in}}%
\pgfpathlineto{\pgfqpoint{1.471233in}{4.412199in}}%
\pgfpathlineto{\pgfqpoint{1.463512in}{4.390867in}}%
\pgfpathlineto{\pgfqpoint{1.455790in}{4.419226in}}%
\pgfpathlineto{\pgfqpoint{1.448068in}{4.490331in}}%
\pgfpathlineto{\pgfqpoint{1.440347in}{4.418477in}}%
\pgfpathlineto{\pgfqpoint{1.432625in}{4.402202in}}%
\pgfpathlineto{\pgfqpoint{1.424903in}{4.421272in}}%
\pgfpathlineto{\pgfqpoint{1.417181in}{4.454966in}}%
\pgfpathlineto{\pgfqpoint{1.409460in}{4.482866in}}%
\pgfpathlineto{\pgfqpoint{1.401738in}{4.515615in}}%
\pgfpathlineto{\pgfqpoint{1.394016in}{4.583806in}}%
\pgfpathlineto{\pgfqpoint{1.386294in}{4.674788in}}%
\pgfpathlineto{\pgfqpoint{1.378573in}{4.765353in}}%
\pgfpathlineto{\pgfqpoint{1.370851in}{4.880839in}}%
\pgfpathlineto{\pgfqpoint{1.363129in}{4.889952in}}%
\pgfpathlineto{\pgfqpoint{1.355407in}{4.752451in}}%
\pgfpathlineto{\pgfqpoint{1.347686in}{4.794469in}}%
\pgfpathlineto{\pgfqpoint{1.339964in}{4.687862in}}%
\pgfpathlineto{\pgfqpoint{1.332242in}{4.543323in}}%
\pgfpathlineto{\pgfqpoint{1.324520in}{4.481664in}}%
\pgfpathlineto{\pgfqpoint{1.316799in}{4.485660in}}%
\pgfpathlineto{\pgfqpoint{1.309077in}{4.506631in}}%
\pgfpathlineto{\pgfqpoint{1.301355in}{4.416144in}}%
\pgfpathlineto{\pgfqpoint{1.293634in}{4.362531in}}%
\pgfpathlineto{\pgfqpoint{1.285912in}{4.360158in}}%
\pgfpathlineto{\pgfqpoint{1.278190in}{4.379246in}}%
\pgfpathlineto{\pgfqpoint{1.270468in}{4.429555in}}%
\pgfpathlineto{\pgfqpoint{1.262747in}{4.357953in}}%
\pgfpathlineto{\pgfqpoint{1.255025in}{4.344103in}}%
\pgfpathlineto{\pgfqpoint{1.247303in}{4.323204in}}%
\pgfpathlineto{\pgfqpoint{1.239581in}{4.305135in}}%
\pgfpathlineto{\pgfqpoint{1.231860in}{4.293859in}}%
\pgfpathlineto{\pgfqpoint{1.224138in}{4.292434in}}%
\pgfpathlineto{\pgfqpoint{1.216416in}{4.293874in}}%
\pgfpathlineto{\pgfqpoint{1.208694in}{4.318856in}}%
\pgfpathlineto{\pgfqpoint{1.200973in}{4.321146in}}%
\pgfpathlineto{\pgfqpoint{1.193251in}{4.308663in}}%
\pgfpathlineto{\pgfqpoint{1.185529in}{4.302766in}}%
\pgfpathlineto{\pgfqpoint{1.177808in}{4.291553in}}%
\pgfpathlineto{\pgfqpoint{1.170086in}{4.289322in}}%
\pgfpathlineto{\pgfqpoint{1.162364in}{4.288673in}}%
\pgfpathlineto{\pgfqpoint{1.154642in}{4.288588in}}%
\pgfpathlineto{\pgfqpoint{1.146921in}{4.288657in}}%
\pgfpathlineto{\pgfqpoint{1.139199in}{4.288597in}}%
\pgfpathlineto{\pgfqpoint{1.131477in}{4.288549in}}%
\pgfpathlineto{\pgfqpoint{1.123755in}{4.288558in}}%
\pgfpathlineto{\pgfqpoint{1.116034in}{4.288562in}}%
\pgfpathlineto{\pgfqpoint{1.108312in}{4.288565in}}%
\pgfpathlineto{\pgfqpoint{1.100590in}{4.288567in}}%
\pgfpathlineto{\pgfqpoint{1.092868in}{4.288552in}}%
\pgfpathlineto{\pgfqpoint{1.085147in}{4.288553in}}%
\pgfpathlineto{\pgfqpoint{1.077425in}{4.288548in}}%
\pgfpathlineto{\pgfqpoint{1.069703in}{4.288546in}}%
\pgfpathlineto{\pgfqpoint{1.061981in}{4.288546in}}%
\pgfpathlineto{\pgfqpoint{1.054260in}{4.288546in}}%
\pgfpathlineto{\pgfqpoint{1.046538in}{4.288546in}}%
\pgfpathlineto{\pgfqpoint{1.038816in}{4.288547in}}%
\pgfpathlineto{\pgfqpoint{1.031095in}{4.288553in}}%
\pgfpathlineto{\pgfqpoint{1.023373in}{4.288549in}}%
\pgfpathlineto{\pgfqpoint{1.015651in}{4.288547in}}%
\pgfpathclose%
\pgfusepath{stroke,fill}%
}%
\begin{pgfscope}%
\pgfsys@transformshift{0.000000in}{0.000000in}%
\pgfsys@useobject{currentmarker}{}%
\end{pgfscope}%
\end{pgfscope}%
\begin{pgfscope}%
\pgfsetbuttcap%
\pgfsetroundjoin%
\definecolor{currentfill}{rgb}{0.000000,0.000000,0.000000}%
\pgfsetfillcolor{currentfill}%
\pgfsetlinewidth{0.803000pt}%
\definecolor{currentstroke}{rgb}{0.000000,0.000000,0.000000}%
\pgfsetstrokecolor{currentstroke}%
\pgfsetdash{}{0pt}%
\pgfsys@defobject{currentmarker}{\pgfqpoint{0.000000in}{-0.048611in}}{\pgfqpoint{0.000000in}{0.000000in}}{%
\pgfpathmoveto{\pgfqpoint{0.000000in}{0.000000in}}%
\pgfpathlineto{\pgfqpoint{0.000000in}{-0.048611in}}%
\pgfusepath{stroke,fill}%
}%
\begin{pgfscope}%
\pgfsys@transformshift{0.757626in}{4.288546in}%
\pgfsys@useobject{currentmarker}{}%
\end{pgfscope}%
\end{pgfscope}%
\begin{pgfscope}%
\pgfsetbuttcap%
\pgfsetroundjoin%
\definecolor{currentfill}{rgb}{0.000000,0.000000,0.000000}%
\pgfsetfillcolor{currentfill}%
\pgfsetlinewidth{0.803000pt}%
\definecolor{currentstroke}{rgb}{0.000000,0.000000,0.000000}%
\pgfsetstrokecolor{currentstroke}%
\pgfsetdash{}{0pt}%
\pgfsys@defobject{currentmarker}{\pgfqpoint{0.000000in}{-0.048611in}}{\pgfqpoint{0.000000in}{0.000000in}}{%
\pgfpathmoveto{\pgfqpoint{0.000000in}{0.000000in}}%
\pgfpathlineto{\pgfqpoint{0.000000in}{-0.048611in}}%
\pgfusepath{stroke,fill}%
}%
\begin{pgfscope}%
\pgfsys@transformshift{1.445847in}{4.288546in}%
\pgfsys@useobject{currentmarker}{}%
\end{pgfscope}%
\end{pgfscope}%
\begin{pgfscope}%
\pgfsetbuttcap%
\pgfsetroundjoin%
\definecolor{currentfill}{rgb}{0.000000,0.000000,0.000000}%
\pgfsetfillcolor{currentfill}%
\pgfsetlinewidth{0.803000pt}%
\definecolor{currentstroke}{rgb}{0.000000,0.000000,0.000000}%
\pgfsetstrokecolor{currentstroke}%
\pgfsetdash{}{0pt}%
\pgfsys@defobject{currentmarker}{\pgfqpoint{0.000000in}{-0.048611in}}{\pgfqpoint{0.000000in}{0.000000in}}{%
\pgfpathmoveto{\pgfqpoint{0.000000in}{0.000000in}}%
\pgfpathlineto{\pgfqpoint{0.000000in}{-0.048611in}}%
\pgfusepath{stroke,fill}%
}%
\begin{pgfscope}%
\pgfsys@transformshift{2.134067in}{4.288546in}%
\pgfsys@useobject{currentmarker}{}%
\end{pgfscope}%
\end{pgfscope}%
\begin{pgfscope}%
\pgfsetbuttcap%
\pgfsetroundjoin%
\definecolor{currentfill}{rgb}{0.000000,0.000000,0.000000}%
\pgfsetfillcolor{currentfill}%
\pgfsetlinewidth{0.803000pt}%
\definecolor{currentstroke}{rgb}{0.000000,0.000000,0.000000}%
\pgfsetstrokecolor{currentstroke}%
\pgfsetdash{}{0pt}%
\pgfsys@defobject{currentmarker}{\pgfqpoint{0.000000in}{-0.048611in}}{\pgfqpoint{0.000000in}{0.000000in}}{%
\pgfpathmoveto{\pgfqpoint{0.000000in}{0.000000in}}%
\pgfpathlineto{\pgfqpoint{0.000000in}{-0.048611in}}%
\pgfusepath{stroke,fill}%
}%
\begin{pgfscope}%
\pgfsys@transformshift{2.822288in}{4.288546in}%
\pgfsys@useobject{currentmarker}{}%
\end{pgfscope}%
\end{pgfscope}%
\begin{pgfscope}%
\pgfsetbuttcap%
\pgfsetroundjoin%
\definecolor{currentfill}{rgb}{0.000000,0.000000,0.000000}%
\pgfsetfillcolor{currentfill}%
\pgfsetlinewidth{0.803000pt}%
\definecolor{currentstroke}{rgb}{0.000000,0.000000,0.000000}%
\pgfsetstrokecolor{currentstroke}%
\pgfsetdash{}{0pt}%
\pgfsys@defobject{currentmarker}{\pgfqpoint{0.000000in}{-0.048611in}}{\pgfqpoint{0.000000in}{0.000000in}}{%
\pgfpathmoveto{\pgfqpoint{0.000000in}{0.000000in}}%
\pgfpathlineto{\pgfqpoint{0.000000in}{-0.048611in}}%
\pgfusepath{stroke,fill}%
}%
\begin{pgfscope}%
\pgfsys@transformshift{3.510509in}{4.288546in}%
\pgfsys@useobject{currentmarker}{}%
\end{pgfscope}%
\end{pgfscope}%
\begin{pgfscope}%
\pgfsetbuttcap%
\pgfsetroundjoin%
\definecolor{currentfill}{rgb}{0.000000,0.000000,0.000000}%
\pgfsetfillcolor{currentfill}%
\pgfsetlinewidth{0.803000pt}%
\definecolor{currentstroke}{rgb}{0.000000,0.000000,0.000000}%
\pgfsetstrokecolor{currentstroke}%
\pgfsetdash{}{0pt}%
\pgfsys@defobject{currentmarker}{\pgfqpoint{0.000000in}{-0.048611in}}{\pgfqpoint{0.000000in}{0.000000in}}{%
\pgfpathmoveto{\pgfqpoint{0.000000in}{0.000000in}}%
\pgfpathlineto{\pgfqpoint{0.000000in}{-0.048611in}}%
\pgfusepath{stroke,fill}%
}%
\begin{pgfscope}%
\pgfsys@transformshift{4.198729in}{4.288546in}%
\pgfsys@useobject{currentmarker}{}%
\end{pgfscope}%
\end{pgfscope}%
\begin{pgfscope}%
\pgfsetrectcap%
\pgfsetmiterjoin%
\pgfsetlinewidth{0.803000pt}%
\definecolor{currentstroke}{rgb}{0.000000,0.000000,0.000000}%
\pgfsetstrokecolor{currentstroke}%
\pgfsetdash{}{0pt}%
\pgfpathmoveto{\pgfqpoint{0.692593in}{4.288546in}}%
\pgfpathlineto{\pgfqpoint{4.310567in}{4.288546in}}%
\pgfusepath{stroke}%
\end{pgfscope}%
\begin{pgfscope}%
\pgfsetbuttcap%
\pgfsetmiterjoin%
\definecolor{currentfill}{rgb}{1.000000,1.000000,1.000000}%
\pgfsetfillcolor{currentfill}%
\pgfsetlinewidth{0.000000pt}%
\definecolor{currentstroke}{rgb}{0.000000,0.000000,0.000000}%
\pgfsetstrokecolor{currentstroke}%
\pgfsetstrokeopacity{0.000000}%
\pgfsetdash{}{0pt}%
\pgfpathmoveto{\pgfqpoint{4.435325in}{0.499691in}}%
\pgfpathlineto{\pgfqpoint{5.059114in}{0.499691in}}%
\pgfpathlineto{\pgfqpoint{5.059114in}{4.162251in}}%
\pgfpathlineto{\pgfqpoint{4.435325in}{4.162251in}}%
\pgfpathclose%
\pgfusepath{fill}%
\end{pgfscope}%
\begin{pgfscope}%
\pgfpathrectangle{\pgfqpoint{4.435325in}{0.499691in}}{\pgfqpoint{0.623789in}{3.662560in}}%
\pgfusepath{clip}%
\pgfsetbuttcap%
\pgfsetroundjoin%
\definecolor{currentfill}{rgb}{0.839216,0.152941,0.156863}%
\pgfsetfillcolor{currentfill}%
\pgfsetfillopacity{0.200000}%
\pgfsetlinewidth{1.003750pt}%
\definecolor{currentstroke}{rgb}{0.839216,0.152941,0.156863}%
\pgfsetstrokecolor{currentstroke}%
\pgfsetdash{}{0pt}%
\pgfsys@defobject{currentmarker}{\pgfqpoint{4.435325in}{-0.175797in}}{\pgfqpoint{4.854434in}{2.378500in}}{%
\pgfpathmoveto{\pgfqpoint{4.435325in}{-0.175797in}}%
\pgfpathlineto{\pgfqpoint{4.435325in}{-0.175797in}}%
\pgfpathlineto{\pgfqpoint{4.435325in}{-0.162962in}}%
\pgfpathlineto{\pgfqpoint{4.435325in}{-0.150126in}}%
\pgfpathlineto{\pgfqpoint{4.435325in}{-0.137290in}}%
\pgfpathlineto{\pgfqpoint{4.435325in}{-0.124455in}}%
\pgfpathlineto{\pgfqpoint{4.435325in}{-0.111619in}}%
\pgfpathlineto{\pgfqpoint{4.435325in}{-0.098783in}}%
\pgfpathlineto{\pgfqpoint{4.435325in}{-0.085948in}}%
\pgfpathlineto{\pgfqpoint{4.435325in}{-0.073112in}}%
\pgfpathlineto{\pgfqpoint{4.435325in}{-0.060276in}}%
\pgfpathlineto{\pgfqpoint{4.435325in}{-0.047441in}}%
\pgfpathlineto{\pgfqpoint{4.435325in}{-0.034605in}}%
\pgfpathlineto{\pgfqpoint{4.435325in}{-0.021769in}}%
\pgfpathlineto{\pgfqpoint{4.435325in}{-0.008934in}}%
\pgfpathlineto{\pgfqpoint{4.435325in}{0.003902in}}%
\pgfpathlineto{\pgfqpoint{4.435325in}{0.016738in}}%
\pgfpathlineto{\pgfqpoint{4.435325in}{0.029573in}}%
\pgfpathlineto{\pgfqpoint{4.435325in}{0.042409in}}%
\pgfpathlineto{\pgfqpoint{4.435325in}{0.055245in}}%
\pgfpathlineto{\pgfqpoint{4.435325in}{0.068080in}}%
\pgfpathlineto{\pgfqpoint{4.435325in}{0.080916in}}%
\pgfpathlineto{\pgfqpoint{4.435325in}{0.093752in}}%
\pgfpathlineto{\pgfqpoint{4.435325in}{0.106587in}}%
\pgfpathlineto{\pgfqpoint{4.435325in}{0.119423in}}%
\pgfpathlineto{\pgfqpoint{4.435325in}{0.132259in}}%
\pgfpathlineto{\pgfqpoint{4.435325in}{0.145094in}}%
\pgfpathlineto{\pgfqpoint{4.435325in}{0.157930in}}%
\pgfpathlineto{\pgfqpoint{4.435325in}{0.170766in}}%
\pgfpathlineto{\pgfqpoint{4.435325in}{0.183601in}}%
\pgfpathlineto{\pgfqpoint{4.435325in}{0.196437in}}%
\pgfpathlineto{\pgfqpoint{4.435325in}{0.209273in}}%
\pgfpathlineto{\pgfqpoint{4.435325in}{0.222108in}}%
\pgfpathlineto{\pgfqpoint{4.435325in}{0.234944in}}%
\pgfpathlineto{\pgfqpoint{4.435325in}{0.247780in}}%
\pgfpathlineto{\pgfqpoint{4.435325in}{0.260615in}}%
\pgfpathlineto{\pgfqpoint{4.435325in}{0.273451in}}%
\pgfpathlineto{\pgfqpoint{4.435325in}{0.286287in}}%
\pgfpathlineto{\pgfqpoint{4.435325in}{0.299122in}}%
\pgfpathlineto{\pgfqpoint{4.435325in}{0.311958in}}%
\pgfpathlineto{\pgfqpoint{4.435325in}{0.324794in}}%
\pgfpathlineto{\pgfqpoint{4.435325in}{0.337629in}}%
\pgfpathlineto{\pgfqpoint{4.435325in}{0.350465in}}%
\pgfpathlineto{\pgfqpoint{4.435325in}{0.363301in}}%
\pgfpathlineto{\pgfqpoint{4.435325in}{0.376136in}}%
\pgfpathlineto{\pgfqpoint{4.435325in}{0.388972in}}%
\pgfpathlineto{\pgfqpoint{4.435325in}{0.401808in}}%
\pgfpathlineto{\pgfqpoint{4.435325in}{0.414643in}}%
\pgfpathlineto{\pgfqpoint{4.435325in}{0.427479in}}%
\pgfpathlineto{\pgfqpoint{4.435325in}{0.440315in}}%
\pgfpathlineto{\pgfqpoint{4.435325in}{0.453150in}}%
\pgfpathlineto{\pgfqpoint{4.435325in}{0.465986in}}%
\pgfpathlineto{\pgfqpoint{4.435325in}{0.478822in}}%
\pgfpathlineto{\pgfqpoint{4.435325in}{0.491657in}}%
\pgfpathlineto{\pgfqpoint{4.435325in}{0.504493in}}%
\pgfpathlineto{\pgfqpoint{4.435325in}{0.517329in}}%
\pgfpathlineto{\pgfqpoint{4.435325in}{0.530164in}}%
\pgfpathlineto{\pgfqpoint{4.435325in}{0.543000in}}%
\pgfpathlineto{\pgfqpoint{4.435325in}{0.555836in}}%
\pgfpathlineto{\pgfqpoint{4.435325in}{0.568671in}}%
\pgfpathlineto{\pgfqpoint{4.435325in}{0.581507in}}%
\pgfpathlineto{\pgfqpoint{4.435325in}{0.594343in}}%
\pgfpathlineto{\pgfqpoint{4.435325in}{0.607178in}}%
\pgfpathlineto{\pgfqpoint{4.435325in}{0.620014in}}%
\pgfpathlineto{\pgfqpoint{4.435325in}{0.632850in}}%
\pgfpathlineto{\pgfqpoint{4.435325in}{0.645685in}}%
\pgfpathlineto{\pgfqpoint{4.435325in}{0.658521in}}%
\pgfpathlineto{\pgfqpoint{4.435325in}{0.671357in}}%
\pgfpathlineto{\pgfqpoint{4.435325in}{0.684192in}}%
\pgfpathlineto{\pgfqpoint{4.435325in}{0.697028in}}%
\pgfpathlineto{\pgfqpoint{4.435325in}{0.709864in}}%
\pgfpathlineto{\pgfqpoint{4.435325in}{0.722699in}}%
\pgfpathlineto{\pgfqpoint{4.435325in}{0.735535in}}%
\pgfpathlineto{\pgfqpoint{4.435325in}{0.748371in}}%
\pgfpathlineto{\pgfqpoint{4.435325in}{0.761206in}}%
\pgfpathlineto{\pgfqpoint{4.435325in}{0.774042in}}%
\pgfpathlineto{\pgfqpoint{4.435325in}{0.786878in}}%
\pgfpathlineto{\pgfqpoint{4.435325in}{0.799713in}}%
\pgfpathlineto{\pgfqpoint{4.435325in}{0.812549in}}%
\pgfpathlineto{\pgfqpoint{4.435325in}{0.825385in}}%
\pgfpathlineto{\pgfqpoint{4.435325in}{0.838220in}}%
\pgfpathlineto{\pgfqpoint{4.435325in}{0.851056in}}%
\pgfpathlineto{\pgfqpoint{4.435325in}{0.863892in}}%
\pgfpathlineto{\pgfqpoint{4.435325in}{0.876727in}}%
\pgfpathlineto{\pgfqpoint{4.435325in}{0.889563in}}%
\pgfpathlineto{\pgfqpoint{4.435325in}{0.902399in}}%
\pgfpathlineto{\pgfqpoint{4.435325in}{0.915234in}}%
\pgfpathlineto{\pgfqpoint{4.435325in}{0.928070in}}%
\pgfpathlineto{\pgfqpoint{4.435325in}{0.940906in}}%
\pgfpathlineto{\pgfqpoint{4.435325in}{0.953741in}}%
\pgfpathlineto{\pgfqpoint{4.435325in}{0.966577in}}%
\pgfpathlineto{\pgfqpoint{4.435325in}{0.979413in}}%
\pgfpathlineto{\pgfqpoint{4.435325in}{0.992248in}}%
\pgfpathlineto{\pgfqpoint{4.435325in}{1.005084in}}%
\pgfpathlineto{\pgfqpoint{4.435325in}{1.017920in}}%
\pgfpathlineto{\pgfqpoint{4.435325in}{1.030755in}}%
\pgfpathlineto{\pgfqpoint{4.435325in}{1.043591in}}%
\pgfpathlineto{\pgfqpoint{4.435325in}{1.056427in}}%
\pgfpathlineto{\pgfqpoint{4.435325in}{1.069262in}}%
\pgfpathlineto{\pgfqpoint{4.435325in}{1.082098in}}%
\pgfpathlineto{\pgfqpoint{4.435325in}{1.094934in}}%
\pgfpathlineto{\pgfqpoint{4.435325in}{1.107769in}}%
\pgfpathlineto{\pgfqpoint{4.435325in}{1.120605in}}%
\pgfpathlineto{\pgfqpoint{4.435325in}{1.133441in}}%
\pgfpathlineto{\pgfqpoint{4.435325in}{1.146276in}}%
\pgfpathlineto{\pgfqpoint{4.435325in}{1.159112in}}%
\pgfpathlineto{\pgfqpoint{4.435325in}{1.171948in}}%
\pgfpathlineto{\pgfqpoint{4.435325in}{1.184783in}}%
\pgfpathlineto{\pgfqpoint{4.435325in}{1.197619in}}%
\pgfpathlineto{\pgfqpoint{4.435325in}{1.210455in}}%
\pgfpathlineto{\pgfqpoint{4.435325in}{1.223290in}}%
\pgfpathlineto{\pgfqpoint{4.435325in}{1.236126in}}%
\pgfpathlineto{\pgfqpoint{4.435325in}{1.248962in}}%
\pgfpathlineto{\pgfqpoint{4.435325in}{1.261797in}}%
\pgfpathlineto{\pgfqpoint{4.435325in}{1.274633in}}%
\pgfpathlineto{\pgfqpoint{4.435325in}{1.287469in}}%
\pgfpathlineto{\pgfqpoint{4.435325in}{1.300304in}}%
\pgfpathlineto{\pgfqpoint{4.435325in}{1.313140in}}%
\pgfpathlineto{\pgfqpoint{4.435325in}{1.325976in}}%
\pgfpathlineto{\pgfqpoint{4.435325in}{1.338811in}}%
\pgfpathlineto{\pgfqpoint{4.435325in}{1.351647in}}%
\pgfpathlineto{\pgfqpoint{4.435325in}{1.364483in}}%
\pgfpathlineto{\pgfqpoint{4.435325in}{1.377318in}}%
\pgfpathlineto{\pgfqpoint{4.435325in}{1.390154in}}%
\pgfpathlineto{\pgfqpoint{4.435325in}{1.402990in}}%
\pgfpathlineto{\pgfqpoint{4.435325in}{1.415825in}}%
\pgfpathlineto{\pgfqpoint{4.435325in}{1.428661in}}%
\pgfpathlineto{\pgfqpoint{4.435325in}{1.441497in}}%
\pgfpathlineto{\pgfqpoint{4.435325in}{1.454332in}}%
\pgfpathlineto{\pgfqpoint{4.435325in}{1.467168in}}%
\pgfpathlineto{\pgfqpoint{4.435325in}{1.480004in}}%
\pgfpathlineto{\pgfqpoint{4.435325in}{1.492839in}}%
\pgfpathlineto{\pgfqpoint{4.435325in}{1.505675in}}%
\pgfpathlineto{\pgfqpoint{4.435325in}{1.518511in}}%
\pgfpathlineto{\pgfqpoint{4.435325in}{1.531346in}}%
\pgfpathlineto{\pgfqpoint{4.435325in}{1.544182in}}%
\pgfpathlineto{\pgfqpoint{4.435325in}{1.557018in}}%
\pgfpathlineto{\pgfqpoint{4.435325in}{1.569853in}}%
\pgfpathlineto{\pgfqpoint{4.435325in}{1.582689in}}%
\pgfpathlineto{\pgfqpoint{4.435325in}{1.595525in}}%
\pgfpathlineto{\pgfqpoint{4.435325in}{1.608360in}}%
\pgfpathlineto{\pgfqpoint{4.435325in}{1.621196in}}%
\pgfpathlineto{\pgfqpoint{4.435325in}{1.634032in}}%
\pgfpathlineto{\pgfqpoint{4.435325in}{1.646867in}}%
\pgfpathlineto{\pgfqpoint{4.435325in}{1.659703in}}%
\pgfpathlineto{\pgfqpoint{4.435325in}{1.672539in}}%
\pgfpathlineto{\pgfqpoint{4.435325in}{1.685374in}}%
\pgfpathlineto{\pgfqpoint{4.435325in}{1.698210in}}%
\pgfpathlineto{\pgfqpoint{4.435325in}{1.711046in}}%
\pgfpathlineto{\pgfqpoint{4.435325in}{1.723881in}}%
\pgfpathlineto{\pgfqpoint{4.435325in}{1.736717in}}%
\pgfpathlineto{\pgfqpoint{4.435325in}{1.749553in}}%
\pgfpathlineto{\pgfqpoint{4.435325in}{1.762388in}}%
\pgfpathlineto{\pgfqpoint{4.435325in}{1.775224in}}%
\pgfpathlineto{\pgfqpoint{4.435325in}{1.788060in}}%
\pgfpathlineto{\pgfqpoint{4.435325in}{1.800895in}}%
\pgfpathlineto{\pgfqpoint{4.435325in}{1.813731in}}%
\pgfpathlineto{\pgfqpoint{4.435325in}{1.826567in}}%
\pgfpathlineto{\pgfqpoint{4.435325in}{1.839402in}}%
\pgfpathlineto{\pgfqpoint{4.435325in}{1.852238in}}%
\pgfpathlineto{\pgfqpoint{4.435325in}{1.865074in}}%
\pgfpathlineto{\pgfqpoint{4.435325in}{1.877909in}}%
\pgfpathlineto{\pgfqpoint{4.435325in}{1.890745in}}%
\pgfpathlineto{\pgfqpoint{4.435325in}{1.903581in}}%
\pgfpathlineto{\pgfqpoint{4.435325in}{1.916416in}}%
\pgfpathlineto{\pgfqpoint{4.435325in}{1.929252in}}%
\pgfpathlineto{\pgfqpoint{4.435325in}{1.942088in}}%
\pgfpathlineto{\pgfqpoint{4.435325in}{1.954923in}}%
\pgfpathlineto{\pgfqpoint{4.435325in}{1.967759in}}%
\pgfpathlineto{\pgfqpoint{4.435325in}{1.980595in}}%
\pgfpathlineto{\pgfqpoint{4.435325in}{1.993430in}}%
\pgfpathlineto{\pgfqpoint{4.435325in}{2.006266in}}%
\pgfpathlineto{\pgfqpoint{4.435325in}{2.019102in}}%
\pgfpathlineto{\pgfqpoint{4.435325in}{2.031937in}}%
\pgfpathlineto{\pgfqpoint{4.435325in}{2.044773in}}%
\pgfpathlineto{\pgfqpoint{4.435325in}{2.057609in}}%
\pgfpathlineto{\pgfqpoint{4.435325in}{2.070444in}}%
\pgfpathlineto{\pgfqpoint{4.435325in}{2.083280in}}%
\pgfpathlineto{\pgfqpoint{4.435325in}{2.096116in}}%
\pgfpathlineto{\pgfqpoint{4.435325in}{2.108951in}}%
\pgfpathlineto{\pgfqpoint{4.435325in}{2.121787in}}%
\pgfpathlineto{\pgfqpoint{4.435325in}{2.134623in}}%
\pgfpathlineto{\pgfqpoint{4.435325in}{2.147458in}}%
\pgfpathlineto{\pgfqpoint{4.435325in}{2.160294in}}%
\pgfpathlineto{\pgfqpoint{4.435325in}{2.173130in}}%
\pgfpathlineto{\pgfqpoint{4.435325in}{2.185965in}}%
\pgfpathlineto{\pgfqpoint{4.435325in}{2.198801in}}%
\pgfpathlineto{\pgfqpoint{4.435325in}{2.211637in}}%
\pgfpathlineto{\pgfqpoint{4.435325in}{2.224472in}}%
\pgfpathlineto{\pgfqpoint{4.435325in}{2.237308in}}%
\pgfpathlineto{\pgfqpoint{4.435325in}{2.250144in}}%
\pgfpathlineto{\pgfqpoint{4.435325in}{2.262979in}}%
\pgfpathlineto{\pgfqpoint{4.435325in}{2.275815in}}%
\pgfpathlineto{\pgfqpoint{4.435325in}{2.288651in}}%
\pgfpathlineto{\pgfqpoint{4.435325in}{2.301486in}}%
\pgfpathlineto{\pgfqpoint{4.435325in}{2.314322in}}%
\pgfpathlineto{\pgfqpoint{4.435325in}{2.327158in}}%
\pgfpathlineto{\pgfqpoint{4.435325in}{2.339993in}}%
\pgfpathlineto{\pgfqpoint{4.435325in}{2.352829in}}%
\pgfpathlineto{\pgfqpoint{4.435325in}{2.365665in}}%
\pgfpathlineto{\pgfqpoint{4.435325in}{2.378500in}}%
\pgfpathlineto{\pgfqpoint{4.435325in}{2.378500in}}%
\pgfpathlineto{\pgfqpoint{4.435325in}{2.378500in}}%
\pgfpathlineto{\pgfqpoint{4.435326in}{2.365665in}}%
\pgfpathlineto{\pgfqpoint{4.435333in}{2.352829in}}%
\pgfpathlineto{\pgfqpoint{4.435369in}{2.339993in}}%
\pgfpathlineto{\pgfqpoint{4.435516in}{2.327158in}}%
\pgfpathlineto{\pgfqpoint{4.436015in}{2.314322in}}%
\pgfpathlineto{\pgfqpoint{4.437017in}{2.301486in}}%
\pgfpathlineto{\pgfqpoint{4.438016in}{2.288651in}}%
\pgfpathlineto{\pgfqpoint{4.439764in}{2.275815in}}%
\pgfpathlineto{\pgfqpoint{4.449383in}{2.262979in}}%
\pgfpathlineto{\pgfqpoint{4.484884in}{2.250144in}}%
\pgfpathlineto{\pgfqpoint{4.528307in}{2.237308in}}%
\pgfpathlineto{\pgfqpoint{4.515321in}{2.224472in}}%
\pgfpathlineto{\pgfqpoint{4.472170in}{2.211637in}}%
\pgfpathlineto{\pgfqpoint{4.451420in}{2.198801in}}%
\pgfpathlineto{\pgfqpoint{4.447487in}{2.185965in}}%
\pgfpathlineto{\pgfqpoint{4.450768in}{2.173130in}}%
\pgfpathlineto{\pgfqpoint{4.457556in}{2.160294in}}%
\pgfpathlineto{\pgfqpoint{4.462273in}{2.147458in}}%
\pgfpathlineto{\pgfqpoint{4.478206in}{2.134623in}}%
\pgfpathlineto{\pgfqpoint{4.536103in}{2.121787in}}%
\pgfpathlineto{\pgfqpoint{4.592490in}{2.108951in}}%
\pgfpathlineto{\pgfqpoint{4.563986in}{2.096116in}}%
\pgfpathlineto{\pgfqpoint{4.503716in}{2.083280in}}%
\pgfpathlineto{\pgfqpoint{4.492420in}{2.070444in}}%
\pgfpathlineto{\pgfqpoint{4.535883in}{2.057609in}}%
\pgfpathlineto{\pgfqpoint{4.599421in}{2.044773in}}%
\pgfpathlineto{\pgfqpoint{4.616257in}{2.031937in}}%
\pgfpathlineto{\pgfqpoint{4.605689in}{2.019102in}}%
\pgfpathlineto{\pgfqpoint{4.632979in}{2.006266in}}%
\pgfpathlineto{\pgfqpoint{4.693256in}{1.993430in}}%
\pgfpathlineto{\pgfqpoint{4.750438in}{1.980595in}}%
\pgfpathlineto{\pgfqpoint{4.770768in}{1.967759in}}%
\pgfpathlineto{\pgfqpoint{4.711556in}{1.954923in}}%
\pgfpathlineto{\pgfqpoint{4.712452in}{1.942088in}}%
\pgfpathlineto{\pgfqpoint{4.854434in}{1.929252in}}%
\pgfpathlineto{\pgfqpoint{4.838790in}{1.916416in}}%
\pgfpathlineto{\pgfqpoint{4.644418in}{1.903581in}}%
\pgfpathlineto{\pgfqpoint{4.536557in}{1.890745in}}%
\pgfpathlineto{\pgfqpoint{4.552086in}{1.877909in}}%
\pgfpathlineto{\pgfqpoint{4.599689in}{1.865074in}}%
\pgfpathlineto{\pgfqpoint{4.587405in}{1.852238in}}%
\pgfpathlineto{\pgfqpoint{4.523228in}{1.839402in}}%
\pgfpathlineto{\pgfqpoint{4.482961in}{1.826567in}}%
\pgfpathlineto{\pgfqpoint{4.511434in}{1.813731in}}%
\pgfpathlineto{\pgfqpoint{4.589206in}{1.800895in}}%
\pgfpathlineto{\pgfqpoint{4.618638in}{1.788060in}}%
\pgfpathlineto{\pgfqpoint{4.577117in}{1.775224in}}%
\pgfpathlineto{\pgfqpoint{4.552057in}{1.762388in}}%
\pgfpathlineto{\pgfqpoint{4.555408in}{1.749553in}}%
\pgfpathlineto{\pgfqpoint{4.527023in}{1.736717in}}%
\pgfpathlineto{\pgfqpoint{4.516307in}{1.723881in}}%
\pgfpathlineto{\pgfqpoint{4.552731in}{1.711046in}}%
\pgfpathlineto{\pgfqpoint{4.572807in}{1.698210in}}%
\pgfpathlineto{\pgfqpoint{4.562060in}{1.685374in}}%
\pgfpathlineto{\pgfqpoint{4.529853in}{1.672539in}}%
\pgfpathlineto{\pgfqpoint{4.502441in}{1.659703in}}%
\pgfpathlineto{\pgfqpoint{4.505586in}{1.646867in}}%
\pgfpathlineto{\pgfqpoint{4.512649in}{1.634032in}}%
\pgfpathlineto{\pgfqpoint{4.499130in}{1.621196in}}%
\pgfpathlineto{\pgfqpoint{4.506116in}{1.608360in}}%
\pgfpathlineto{\pgfqpoint{4.562611in}{1.595525in}}%
\pgfpathlineto{\pgfqpoint{4.577259in}{1.582689in}}%
\pgfpathlineto{\pgfqpoint{4.526292in}{1.569853in}}%
\pgfpathlineto{\pgfqpoint{4.512827in}{1.557018in}}%
\pgfpathlineto{\pgfqpoint{4.537303in}{1.544182in}}%
\pgfpathlineto{\pgfqpoint{4.533821in}{1.531346in}}%
\pgfpathlineto{\pgfqpoint{4.499053in}{1.518511in}}%
\pgfpathlineto{\pgfqpoint{4.479819in}{1.505675in}}%
\pgfpathlineto{\pgfqpoint{4.465733in}{1.492839in}}%
\pgfpathlineto{\pgfqpoint{4.446674in}{1.480004in}}%
\pgfpathlineto{\pgfqpoint{4.437716in}{1.467168in}}%
\pgfpathlineto{\pgfqpoint{4.437377in}{1.454332in}}%
\pgfpathlineto{\pgfqpoint{4.443382in}{1.441497in}}%
\pgfpathlineto{\pgfqpoint{4.454073in}{1.428661in}}%
\pgfpathlineto{\pgfqpoint{4.457430in}{1.415825in}}%
\pgfpathlineto{\pgfqpoint{4.452831in}{1.402990in}}%
\pgfpathlineto{\pgfqpoint{4.448229in}{1.390154in}}%
\pgfpathlineto{\pgfqpoint{4.442692in}{1.377318in}}%
\pgfpathlineto{\pgfqpoint{4.438999in}{1.364483in}}%
\pgfpathlineto{\pgfqpoint{4.436937in}{1.351647in}}%
\pgfpathlineto{\pgfqpoint{4.435777in}{1.338811in}}%
\pgfpathlineto{\pgfqpoint{4.435524in}{1.325976in}}%
\pgfpathlineto{\pgfqpoint{4.435603in}{1.313140in}}%
\pgfpathlineto{\pgfqpoint{4.435659in}{1.300304in}}%
\pgfpathlineto{\pgfqpoint{4.435583in}{1.287469in}}%
\pgfpathlineto{\pgfqpoint{4.435505in}{1.274633in}}%
\pgfpathlineto{\pgfqpoint{4.435483in}{1.261797in}}%
\pgfpathlineto{\pgfqpoint{4.435463in}{1.248962in}}%
\pgfpathlineto{\pgfqpoint{4.435442in}{1.236126in}}%
\pgfpathlineto{\pgfqpoint{4.435471in}{1.223290in}}%
\pgfpathlineto{\pgfqpoint{4.435499in}{1.210455in}}%
\pgfpathlineto{\pgfqpoint{4.435485in}{1.197619in}}%
\pgfpathlineto{\pgfqpoint{4.435474in}{1.184783in}}%
\pgfpathlineto{\pgfqpoint{4.435494in}{1.171948in}}%
\pgfpathlineto{\pgfqpoint{4.435562in}{1.159112in}}%
\pgfpathlineto{\pgfqpoint{4.435612in}{1.146276in}}%
\pgfpathlineto{\pgfqpoint{4.435552in}{1.133441in}}%
\pgfpathlineto{\pgfqpoint{4.435469in}{1.120605in}}%
\pgfpathlineto{\pgfqpoint{4.435532in}{1.107769in}}%
\pgfpathlineto{\pgfqpoint{4.435751in}{1.094934in}}%
\pgfpathlineto{\pgfqpoint{4.435838in}{1.082098in}}%
\pgfpathlineto{\pgfqpoint{4.435655in}{1.069262in}}%
\pgfpathlineto{\pgfqpoint{4.435496in}{1.056427in}}%
\pgfpathlineto{\pgfqpoint{4.435436in}{1.043591in}}%
\pgfpathlineto{\pgfqpoint{4.435384in}{1.030755in}}%
\pgfpathlineto{\pgfqpoint{4.435349in}{1.017920in}}%
\pgfpathlineto{\pgfqpoint{4.435338in}{1.005084in}}%
\pgfpathlineto{\pgfqpoint{4.435335in}{0.992248in}}%
\pgfpathlineto{\pgfqpoint{4.435333in}{0.979413in}}%
\pgfpathlineto{\pgfqpoint{4.435329in}{0.966577in}}%
\pgfpathlineto{\pgfqpoint{4.435326in}{0.953741in}}%
\pgfpathlineto{\pgfqpoint{4.435325in}{0.940906in}}%
\pgfpathlineto{\pgfqpoint{4.435327in}{0.928070in}}%
\pgfpathlineto{\pgfqpoint{4.435332in}{0.915234in}}%
\pgfpathlineto{\pgfqpoint{4.435336in}{0.902399in}}%
\pgfpathlineto{\pgfqpoint{4.435332in}{0.889563in}}%
\pgfpathlineto{\pgfqpoint{4.435327in}{0.876727in}}%
\pgfpathlineto{\pgfqpoint{4.435327in}{0.863892in}}%
\pgfpathlineto{\pgfqpoint{4.435336in}{0.851056in}}%
\pgfpathlineto{\pgfqpoint{4.435347in}{0.838220in}}%
\pgfpathlineto{\pgfqpoint{4.435342in}{0.825385in}}%
\pgfpathlineto{\pgfqpoint{4.435330in}{0.812549in}}%
\pgfpathlineto{\pgfqpoint{4.435326in}{0.799713in}}%
\pgfpathlineto{\pgfqpoint{4.435325in}{0.786878in}}%
\pgfpathlineto{\pgfqpoint{4.435325in}{0.774042in}}%
\pgfpathlineto{\pgfqpoint{4.435325in}{0.761206in}}%
\pgfpathlineto{\pgfqpoint{4.435325in}{0.748371in}}%
\pgfpathlineto{\pgfqpoint{4.435325in}{0.735535in}}%
\pgfpathlineto{\pgfqpoint{4.435325in}{0.722699in}}%
\pgfpathlineto{\pgfqpoint{4.435325in}{0.709864in}}%
\pgfpathlineto{\pgfqpoint{4.435325in}{0.697028in}}%
\pgfpathlineto{\pgfqpoint{4.435325in}{0.684192in}}%
\pgfpathlineto{\pgfqpoint{4.435325in}{0.671357in}}%
\pgfpathlineto{\pgfqpoint{4.435325in}{0.658521in}}%
\pgfpathlineto{\pgfqpoint{4.435326in}{0.645685in}}%
\pgfpathlineto{\pgfqpoint{4.435328in}{0.632850in}}%
\pgfpathlineto{\pgfqpoint{4.435331in}{0.620014in}}%
\pgfpathlineto{\pgfqpoint{4.435329in}{0.607178in}}%
\pgfpathlineto{\pgfqpoint{4.435328in}{0.594343in}}%
\pgfpathlineto{\pgfqpoint{4.435330in}{0.581507in}}%
\pgfpathlineto{\pgfqpoint{4.435330in}{0.568671in}}%
\pgfpathlineto{\pgfqpoint{4.435327in}{0.555836in}}%
\pgfpathlineto{\pgfqpoint{4.435326in}{0.543000in}}%
\pgfpathlineto{\pgfqpoint{4.435331in}{0.530164in}}%
\pgfpathlineto{\pgfqpoint{4.435354in}{0.517329in}}%
\pgfpathlineto{\pgfqpoint{4.435381in}{0.504493in}}%
\pgfpathlineto{\pgfqpoint{4.435367in}{0.491657in}}%
\pgfpathlineto{\pgfqpoint{4.435338in}{0.478822in}}%
\pgfpathlineto{\pgfqpoint{4.435327in}{0.465986in}}%
\pgfpathlineto{\pgfqpoint{4.435329in}{0.453150in}}%
\pgfpathlineto{\pgfqpoint{4.435335in}{0.440315in}}%
\pgfpathlineto{\pgfqpoint{4.435335in}{0.427479in}}%
\pgfpathlineto{\pgfqpoint{4.435329in}{0.414643in}}%
\pgfpathlineto{\pgfqpoint{4.435326in}{0.401808in}}%
\pgfpathlineto{\pgfqpoint{4.435325in}{0.388972in}}%
\pgfpathlineto{\pgfqpoint{4.435325in}{0.376136in}}%
\pgfpathlineto{\pgfqpoint{4.435325in}{0.363301in}}%
\pgfpathlineto{\pgfqpoint{4.435325in}{0.350465in}}%
\pgfpathlineto{\pgfqpoint{4.435325in}{0.337629in}}%
\pgfpathlineto{\pgfqpoint{4.435325in}{0.324794in}}%
\pgfpathlineto{\pgfqpoint{4.435325in}{0.311958in}}%
\pgfpathlineto{\pgfqpoint{4.435325in}{0.299122in}}%
\pgfpathlineto{\pgfqpoint{4.435325in}{0.286287in}}%
\pgfpathlineto{\pgfqpoint{4.435325in}{0.273451in}}%
\pgfpathlineto{\pgfqpoint{4.435325in}{0.260615in}}%
\pgfpathlineto{\pgfqpoint{4.435329in}{0.247780in}}%
\pgfpathlineto{\pgfqpoint{4.435343in}{0.234944in}}%
\pgfpathlineto{\pgfqpoint{4.435361in}{0.222108in}}%
\pgfpathlineto{\pgfqpoint{4.435356in}{0.209273in}}%
\pgfpathlineto{\pgfqpoint{4.435337in}{0.196437in}}%
\pgfpathlineto{\pgfqpoint{4.435327in}{0.183601in}}%
\pgfpathlineto{\pgfqpoint{4.435327in}{0.170766in}}%
\pgfpathlineto{\pgfqpoint{4.435331in}{0.157930in}}%
\pgfpathlineto{\pgfqpoint{4.435336in}{0.145094in}}%
\pgfpathlineto{\pgfqpoint{4.435333in}{0.132259in}}%
\pgfpathlineto{\pgfqpoint{4.435327in}{0.119423in}}%
\pgfpathlineto{\pgfqpoint{4.435325in}{0.106587in}}%
\pgfpathlineto{\pgfqpoint{4.435325in}{0.093752in}}%
\pgfpathlineto{\pgfqpoint{4.435325in}{0.080916in}}%
\pgfpathlineto{\pgfqpoint{4.435325in}{0.068080in}}%
\pgfpathlineto{\pgfqpoint{4.435325in}{0.055245in}}%
\pgfpathlineto{\pgfqpoint{4.435325in}{0.042409in}}%
\pgfpathlineto{\pgfqpoint{4.435325in}{0.029573in}}%
\pgfpathlineto{\pgfqpoint{4.435325in}{0.016738in}}%
\pgfpathlineto{\pgfqpoint{4.435325in}{0.003902in}}%
\pgfpathlineto{\pgfqpoint{4.435325in}{-0.008934in}}%
\pgfpathlineto{\pgfqpoint{4.435325in}{-0.021769in}}%
\pgfpathlineto{\pgfqpoint{4.435325in}{-0.034605in}}%
\pgfpathlineto{\pgfqpoint{4.435325in}{-0.047441in}}%
\pgfpathlineto{\pgfqpoint{4.435326in}{-0.060276in}}%
\pgfpathlineto{\pgfqpoint{4.435331in}{-0.073112in}}%
\pgfpathlineto{\pgfqpoint{4.435336in}{-0.085948in}}%
\pgfpathlineto{\pgfqpoint{4.435333in}{-0.098783in}}%
\pgfpathlineto{\pgfqpoint{4.435330in}{-0.111619in}}%
\pgfpathlineto{\pgfqpoint{4.435333in}{-0.124455in}}%
\pgfpathlineto{\pgfqpoint{4.435336in}{-0.137290in}}%
\pgfpathlineto{\pgfqpoint{4.435331in}{-0.150126in}}%
\pgfpathlineto{\pgfqpoint{4.435326in}{-0.162962in}}%
\pgfpathlineto{\pgfqpoint{4.435325in}{-0.175797in}}%
\pgfpathclose%
\pgfusepath{stroke,fill}%
}%
\begin{pgfscope}%
\pgfsys@transformshift{0.000000in}{0.000000in}%
\pgfsys@useobject{currentmarker}{}%
\end{pgfscope}%
\end{pgfscope}%
\begin{pgfscope}%
\pgfpathrectangle{\pgfqpoint{4.435325in}{0.499691in}}{\pgfqpoint{0.623789in}{3.662560in}}%
\pgfusepath{clip}%
\pgfsetbuttcap%
\pgfsetroundjoin%
\definecolor{currentfill}{rgb}{0.172549,0.627451,0.172549}%
\pgfsetfillcolor{currentfill}%
\pgfsetfillopacity{0.200000}%
\pgfsetlinewidth{1.003750pt}%
\definecolor{currentstroke}{rgb}{0.172549,0.627451,0.172549}%
\pgfsetstrokecolor{currentstroke}%
\pgfsetdash{}{0pt}%
\pgfsys@defobject{currentmarker}{\pgfqpoint{4.435325in}{0.963940in}}{\pgfqpoint{4.576922in}{3.265252in}}{%
\pgfpathmoveto{\pgfqpoint{4.435325in}{0.963940in}}%
\pgfpathlineto{\pgfqpoint{4.435325in}{0.963940in}}%
\pgfpathlineto{\pgfqpoint{4.435325in}{0.975504in}}%
\pgfpathlineto{\pgfqpoint{4.435325in}{0.987069in}}%
\pgfpathlineto{\pgfqpoint{4.435325in}{0.998633in}}%
\pgfpathlineto{\pgfqpoint{4.435325in}{1.010198in}}%
\pgfpathlineto{\pgfqpoint{4.435325in}{1.021762in}}%
\pgfpathlineto{\pgfqpoint{4.435325in}{1.033326in}}%
\pgfpathlineto{\pgfqpoint{4.435325in}{1.044891in}}%
\pgfpathlineto{\pgfqpoint{4.435325in}{1.056455in}}%
\pgfpathlineto{\pgfqpoint{4.435325in}{1.068019in}}%
\pgfpathlineto{\pgfqpoint{4.435325in}{1.079584in}}%
\pgfpathlineto{\pgfqpoint{4.435325in}{1.091148in}}%
\pgfpathlineto{\pgfqpoint{4.435325in}{1.102713in}}%
\pgfpathlineto{\pgfqpoint{4.435325in}{1.114277in}}%
\pgfpathlineto{\pgfqpoint{4.435325in}{1.125841in}}%
\pgfpathlineto{\pgfqpoint{4.435325in}{1.137406in}}%
\pgfpathlineto{\pgfqpoint{4.435325in}{1.148970in}}%
\pgfpathlineto{\pgfqpoint{4.435325in}{1.160535in}}%
\pgfpathlineto{\pgfqpoint{4.435325in}{1.172099in}}%
\pgfpathlineto{\pgfqpoint{4.435325in}{1.183663in}}%
\pgfpathlineto{\pgfqpoint{4.435325in}{1.195228in}}%
\pgfpathlineto{\pgfqpoint{4.435325in}{1.206792in}}%
\pgfpathlineto{\pgfqpoint{4.435325in}{1.218356in}}%
\pgfpathlineto{\pgfqpoint{4.435325in}{1.229921in}}%
\pgfpathlineto{\pgfqpoint{4.435325in}{1.241485in}}%
\pgfpathlineto{\pgfqpoint{4.435325in}{1.253050in}}%
\pgfpathlineto{\pgfqpoint{4.435325in}{1.264614in}}%
\pgfpathlineto{\pgfqpoint{4.435325in}{1.276178in}}%
\pgfpathlineto{\pgfqpoint{4.435325in}{1.287743in}}%
\pgfpathlineto{\pgfqpoint{4.435325in}{1.299307in}}%
\pgfpathlineto{\pgfqpoint{4.435325in}{1.310871in}}%
\pgfpathlineto{\pgfqpoint{4.435325in}{1.322436in}}%
\pgfpathlineto{\pgfqpoint{4.435325in}{1.334000in}}%
\pgfpathlineto{\pgfqpoint{4.435325in}{1.345565in}}%
\pgfpathlineto{\pgfqpoint{4.435325in}{1.357129in}}%
\pgfpathlineto{\pgfqpoint{4.435325in}{1.368693in}}%
\pgfpathlineto{\pgfqpoint{4.435325in}{1.380258in}}%
\pgfpathlineto{\pgfqpoint{4.435325in}{1.391822in}}%
\pgfpathlineto{\pgfqpoint{4.435325in}{1.403387in}}%
\pgfpathlineto{\pgfqpoint{4.435325in}{1.414951in}}%
\pgfpathlineto{\pgfqpoint{4.435325in}{1.426515in}}%
\pgfpathlineto{\pgfqpoint{4.435325in}{1.438080in}}%
\pgfpathlineto{\pgfqpoint{4.435325in}{1.449644in}}%
\pgfpathlineto{\pgfqpoint{4.435325in}{1.461208in}}%
\pgfpathlineto{\pgfqpoint{4.435325in}{1.472773in}}%
\pgfpathlineto{\pgfqpoint{4.435325in}{1.484337in}}%
\pgfpathlineto{\pgfqpoint{4.435325in}{1.495902in}}%
\pgfpathlineto{\pgfqpoint{4.435325in}{1.507466in}}%
\pgfpathlineto{\pgfqpoint{4.435325in}{1.519030in}}%
\pgfpathlineto{\pgfqpoint{4.435325in}{1.530595in}}%
\pgfpathlineto{\pgfqpoint{4.435325in}{1.542159in}}%
\pgfpathlineto{\pgfqpoint{4.435325in}{1.553723in}}%
\pgfpathlineto{\pgfqpoint{4.435325in}{1.565288in}}%
\pgfpathlineto{\pgfqpoint{4.435325in}{1.576852in}}%
\pgfpathlineto{\pgfqpoint{4.435325in}{1.588417in}}%
\pgfpathlineto{\pgfqpoint{4.435325in}{1.599981in}}%
\pgfpathlineto{\pgfqpoint{4.435325in}{1.611545in}}%
\pgfpathlineto{\pgfqpoint{4.435325in}{1.623110in}}%
\pgfpathlineto{\pgfqpoint{4.435325in}{1.634674in}}%
\pgfpathlineto{\pgfqpoint{4.435325in}{1.646238in}}%
\pgfpathlineto{\pgfqpoint{4.435325in}{1.657803in}}%
\pgfpathlineto{\pgfqpoint{4.435325in}{1.669367in}}%
\pgfpathlineto{\pgfqpoint{4.435325in}{1.680932in}}%
\pgfpathlineto{\pgfqpoint{4.435325in}{1.692496in}}%
\pgfpathlineto{\pgfqpoint{4.435325in}{1.704060in}}%
\pgfpathlineto{\pgfqpoint{4.435325in}{1.715625in}}%
\pgfpathlineto{\pgfqpoint{4.435325in}{1.727189in}}%
\pgfpathlineto{\pgfqpoint{4.435325in}{1.738754in}}%
\pgfpathlineto{\pgfqpoint{4.435325in}{1.750318in}}%
\pgfpathlineto{\pgfqpoint{4.435325in}{1.761882in}}%
\pgfpathlineto{\pgfqpoint{4.435325in}{1.773447in}}%
\pgfpathlineto{\pgfqpoint{4.435325in}{1.785011in}}%
\pgfpathlineto{\pgfqpoint{4.435325in}{1.796575in}}%
\pgfpathlineto{\pgfqpoint{4.435325in}{1.808140in}}%
\pgfpathlineto{\pgfqpoint{4.435325in}{1.819704in}}%
\pgfpathlineto{\pgfqpoint{4.435325in}{1.831269in}}%
\pgfpathlineto{\pgfqpoint{4.435325in}{1.842833in}}%
\pgfpathlineto{\pgfqpoint{4.435325in}{1.854397in}}%
\pgfpathlineto{\pgfqpoint{4.435325in}{1.865962in}}%
\pgfpathlineto{\pgfqpoint{4.435325in}{1.877526in}}%
\pgfpathlineto{\pgfqpoint{4.435325in}{1.889090in}}%
\pgfpathlineto{\pgfqpoint{4.435325in}{1.900655in}}%
\pgfpathlineto{\pgfqpoint{4.435325in}{1.912219in}}%
\pgfpathlineto{\pgfqpoint{4.435325in}{1.923784in}}%
\pgfpathlineto{\pgfqpoint{4.435325in}{1.935348in}}%
\pgfpathlineto{\pgfqpoint{4.435325in}{1.946912in}}%
\pgfpathlineto{\pgfqpoint{4.435325in}{1.958477in}}%
\pgfpathlineto{\pgfqpoint{4.435325in}{1.970041in}}%
\pgfpathlineto{\pgfqpoint{4.435325in}{1.981606in}}%
\pgfpathlineto{\pgfqpoint{4.435325in}{1.993170in}}%
\pgfpathlineto{\pgfqpoint{4.435325in}{2.004734in}}%
\pgfpathlineto{\pgfqpoint{4.435325in}{2.016299in}}%
\pgfpathlineto{\pgfqpoint{4.435325in}{2.027863in}}%
\pgfpathlineto{\pgfqpoint{4.435325in}{2.039427in}}%
\pgfpathlineto{\pgfqpoint{4.435325in}{2.050992in}}%
\pgfpathlineto{\pgfqpoint{4.435325in}{2.062556in}}%
\pgfpathlineto{\pgfqpoint{4.435325in}{2.074121in}}%
\pgfpathlineto{\pgfqpoint{4.435325in}{2.085685in}}%
\pgfpathlineto{\pgfqpoint{4.435325in}{2.097249in}}%
\pgfpathlineto{\pgfqpoint{4.435325in}{2.108814in}}%
\pgfpathlineto{\pgfqpoint{4.435325in}{2.120378in}}%
\pgfpathlineto{\pgfqpoint{4.435325in}{2.131942in}}%
\pgfpathlineto{\pgfqpoint{4.435325in}{2.143507in}}%
\pgfpathlineto{\pgfqpoint{4.435325in}{2.155071in}}%
\pgfpathlineto{\pgfqpoint{4.435325in}{2.166636in}}%
\pgfpathlineto{\pgfqpoint{4.435325in}{2.178200in}}%
\pgfpathlineto{\pgfqpoint{4.435325in}{2.189764in}}%
\pgfpathlineto{\pgfqpoint{4.435325in}{2.201329in}}%
\pgfpathlineto{\pgfqpoint{4.435325in}{2.212893in}}%
\pgfpathlineto{\pgfqpoint{4.435325in}{2.224458in}}%
\pgfpathlineto{\pgfqpoint{4.435325in}{2.236022in}}%
\pgfpathlineto{\pgfqpoint{4.435325in}{2.247586in}}%
\pgfpathlineto{\pgfqpoint{4.435325in}{2.259151in}}%
\pgfpathlineto{\pgfqpoint{4.435325in}{2.270715in}}%
\pgfpathlineto{\pgfqpoint{4.435325in}{2.282279in}}%
\pgfpathlineto{\pgfqpoint{4.435325in}{2.293844in}}%
\pgfpathlineto{\pgfqpoint{4.435325in}{2.305408in}}%
\pgfpathlineto{\pgfqpoint{4.435325in}{2.316973in}}%
\pgfpathlineto{\pgfqpoint{4.435325in}{2.328537in}}%
\pgfpathlineto{\pgfqpoint{4.435325in}{2.340101in}}%
\pgfpathlineto{\pgfqpoint{4.435325in}{2.351666in}}%
\pgfpathlineto{\pgfqpoint{4.435325in}{2.363230in}}%
\pgfpathlineto{\pgfqpoint{4.435325in}{2.374794in}}%
\pgfpathlineto{\pgfqpoint{4.435325in}{2.386359in}}%
\pgfpathlineto{\pgfqpoint{4.435325in}{2.397923in}}%
\pgfpathlineto{\pgfqpoint{4.435325in}{2.409488in}}%
\pgfpathlineto{\pgfqpoint{4.435325in}{2.421052in}}%
\pgfpathlineto{\pgfqpoint{4.435325in}{2.432616in}}%
\pgfpathlineto{\pgfqpoint{4.435325in}{2.444181in}}%
\pgfpathlineto{\pgfqpoint{4.435325in}{2.455745in}}%
\pgfpathlineto{\pgfqpoint{4.435325in}{2.467310in}}%
\pgfpathlineto{\pgfqpoint{4.435325in}{2.478874in}}%
\pgfpathlineto{\pgfqpoint{4.435325in}{2.490438in}}%
\pgfpathlineto{\pgfqpoint{4.435325in}{2.502003in}}%
\pgfpathlineto{\pgfqpoint{4.435325in}{2.513567in}}%
\pgfpathlineto{\pgfqpoint{4.435325in}{2.525131in}}%
\pgfpathlineto{\pgfqpoint{4.435325in}{2.536696in}}%
\pgfpathlineto{\pgfqpoint{4.435325in}{2.548260in}}%
\pgfpathlineto{\pgfqpoint{4.435325in}{2.559825in}}%
\pgfpathlineto{\pgfqpoint{4.435325in}{2.571389in}}%
\pgfpathlineto{\pgfqpoint{4.435325in}{2.582953in}}%
\pgfpathlineto{\pgfqpoint{4.435325in}{2.594518in}}%
\pgfpathlineto{\pgfqpoint{4.435325in}{2.606082in}}%
\pgfpathlineto{\pgfqpoint{4.435325in}{2.617646in}}%
\pgfpathlineto{\pgfqpoint{4.435325in}{2.629211in}}%
\pgfpathlineto{\pgfqpoint{4.435325in}{2.640775in}}%
\pgfpathlineto{\pgfqpoint{4.435325in}{2.652340in}}%
\pgfpathlineto{\pgfqpoint{4.435325in}{2.663904in}}%
\pgfpathlineto{\pgfqpoint{4.435325in}{2.675468in}}%
\pgfpathlineto{\pgfqpoint{4.435325in}{2.687033in}}%
\pgfpathlineto{\pgfqpoint{4.435325in}{2.698597in}}%
\pgfpathlineto{\pgfqpoint{4.435325in}{2.710162in}}%
\pgfpathlineto{\pgfqpoint{4.435325in}{2.721726in}}%
\pgfpathlineto{\pgfqpoint{4.435325in}{2.733290in}}%
\pgfpathlineto{\pgfqpoint{4.435325in}{2.744855in}}%
\pgfpathlineto{\pgfqpoint{4.435325in}{2.756419in}}%
\pgfpathlineto{\pgfqpoint{4.435325in}{2.767983in}}%
\pgfpathlineto{\pgfqpoint{4.435325in}{2.779548in}}%
\pgfpathlineto{\pgfqpoint{4.435325in}{2.791112in}}%
\pgfpathlineto{\pgfqpoint{4.435325in}{2.802677in}}%
\pgfpathlineto{\pgfqpoint{4.435325in}{2.814241in}}%
\pgfpathlineto{\pgfqpoint{4.435325in}{2.825805in}}%
\pgfpathlineto{\pgfqpoint{4.435325in}{2.837370in}}%
\pgfpathlineto{\pgfqpoint{4.435325in}{2.848934in}}%
\pgfpathlineto{\pgfqpoint{4.435325in}{2.860498in}}%
\pgfpathlineto{\pgfqpoint{4.435325in}{2.872063in}}%
\pgfpathlineto{\pgfqpoint{4.435325in}{2.883627in}}%
\pgfpathlineto{\pgfqpoint{4.435325in}{2.895192in}}%
\pgfpathlineto{\pgfqpoint{4.435325in}{2.906756in}}%
\pgfpathlineto{\pgfqpoint{4.435325in}{2.918320in}}%
\pgfpathlineto{\pgfqpoint{4.435325in}{2.929885in}}%
\pgfpathlineto{\pgfqpoint{4.435325in}{2.941449in}}%
\pgfpathlineto{\pgfqpoint{4.435325in}{2.953014in}}%
\pgfpathlineto{\pgfqpoint{4.435325in}{2.964578in}}%
\pgfpathlineto{\pgfqpoint{4.435325in}{2.976142in}}%
\pgfpathlineto{\pgfqpoint{4.435325in}{2.987707in}}%
\pgfpathlineto{\pgfqpoint{4.435325in}{2.999271in}}%
\pgfpathlineto{\pgfqpoint{4.435325in}{3.010835in}}%
\pgfpathlineto{\pgfqpoint{4.435325in}{3.022400in}}%
\pgfpathlineto{\pgfqpoint{4.435325in}{3.033964in}}%
\pgfpathlineto{\pgfqpoint{4.435325in}{3.045529in}}%
\pgfpathlineto{\pgfqpoint{4.435325in}{3.057093in}}%
\pgfpathlineto{\pgfqpoint{4.435325in}{3.068657in}}%
\pgfpathlineto{\pgfqpoint{4.435325in}{3.080222in}}%
\pgfpathlineto{\pgfqpoint{4.435325in}{3.091786in}}%
\pgfpathlineto{\pgfqpoint{4.435325in}{3.103350in}}%
\pgfpathlineto{\pgfqpoint{4.435325in}{3.114915in}}%
\pgfpathlineto{\pgfqpoint{4.435325in}{3.126479in}}%
\pgfpathlineto{\pgfqpoint{4.435325in}{3.138044in}}%
\pgfpathlineto{\pgfqpoint{4.435325in}{3.149608in}}%
\pgfpathlineto{\pgfqpoint{4.435325in}{3.161172in}}%
\pgfpathlineto{\pgfqpoint{4.435325in}{3.172737in}}%
\pgfpathlineto{\pgfqpoint{4.435325in}{3.184301in}}%
\pgfpathlineto{\pgfqpoint{4.435325in}{3.195866in}}%
\pgfpathlineto{\pgfqpoint{4.435325in}{3.207430in}}%
\pgfpathlineto{\pgfqpoint{4.435325in}{3.218994in}}%
\pgfpathlineto{\pgfqpoint{4.435325in}{3.230559in}}%
\pgfpathlineto{\pgfqpoint{4.435325in}{3.242123in}}%
\pgfpathlineto{\pgfqpoint{4.435325in}{3.253687in}}%
\pgfpathlineto{\pgfqpoint{4.435325in}{3.265252in}}%
\pgfpathlineto{\pgfqpoint{4.435325in}{3.265252in}}%
\pgfpathlineto{\pgfqpoint{4.435325in}{3.265252in}}%
\pgfpathlineto{\pgfqpoint{4.435325in}{3.253687in}}%
\pgfpathlineto{\pgfqpoint{4.435327in}{3.242123in}}%
\pgfpathlineto{\pgfqpoint{4.435330in}{3.230559in}}%
\pgfpathlineto{\pgfqpoint{4.435334in}{3.218994in}}%
\pgfpathlineto{\pgfqpoint{4.435341in}{3.207430in}}%
\pgfpathlineto{\pgfqpoint{4.435349in}{3.195866in}}%
\pgfpathlineto{\pgfqpoint{4.435357in}{3.184301in}}%
\pgfpathlineto{\pgfqpoint{4.435359in}{3.172737in}}%
\pgfpathlineto{\pgfqpoint{4.435353in}{3.161172in}}%
\pgfpathlineto{\pgfqpoint{4.435344in}{3.149608in}}%
\pgfpathlineto{\pgfqpoint{4.435336in}{3.138044in}}%
\pgfpathlineto{\pgfqpoint{4.435335in}{3.126479in}}%
\pgfpathlineto{\pgfqpoint{4.435339in}{3.114915in}}%
\pgfpathlineto{\pgfqpoint{4.435346in}{3.103350in}}%
\pgfpathlineto{\pgfqpoint{4.435359in}{3.091786in}}%
\pgfpathlineto{\pgfqpoint{4.435382in}{3.080222in}}%
\pgfpathlineto{\pgfqpoint{4.435410in}{3.068657in}}%
\pgfpathlineto{\pgfqpoint{4.435427in}{3.057093in}}%
\pgfpathlineto{\pgfqpoint{4.435426in}{3.045529in}}%
\pgfpathlineto{\pgfqpoint{4.435420in}{3.033964in}}%
\pgfpathlineto{\pgfqpoint{4.435438in}{3.022400in}}%
\pgfpathlineto{\pgfqpoint{4.435497in}{3.010835in}}%
\pgfpathlineto{\pgfqpoint{4.435576in}{2.999271in}}%
\pgfpathlineto{\pgfqpoint{4.435638in}{2.987707in}}%
\pgfpathlineto{\pgfqpoint{4.435693in}{2.976142in}}%
\pgfpathlineto{\pgfqpoint{4.435780in}{2.964578in}}%
\pgfpathlineto{\pgfqpoint{4.435875in}{2.953014in}}%
\pgfpathlineto{\pgfqpoint{4.435939in}{2.941449in}}%
\pgfpathlineto{\pgfqpoint{4.436022in}{2.929885in}}%
\pgfpathlineto{\pgfqpoint{4.436243in}{2.918320in}}%
\pgfpathlineto{\pgfqpoint{4.436763in}{2.906756in}}%
\pgfpathlineto{\pgfqpoint{4.437650in}{2.895192in}}%
\pgfpathlineto{\pgfqpoint{4.438519in}{2.883627in}}%
\pgfpathlineto{\pgfqpoint{4.438764in}{2.872063in}}%
\pgfpathlineto{\pgfqpoint{4.438514in}{2.860498in}}%
\pgfpathlineto{\pgfqpoint{4.438404in}{2.848934in}}%
\pgfpathlineto{\pgfqpoint{4.438627in}{2.837370in}}%
\pgfpathlineto{\pgfqpoint{4.439087in}{2.825805in}}%
\pgfpathlineto{\pgfqpoint{4.439719in}{2.814241in}}%
\pgfpathlineto{\pgfqpoint{4.440353in}{2.802677in}}%
\pgfpathlineto{\pgfqpoint{4.440880in}{2.791112in}}%
\pgfpathlineto{\pgfqpoint{4.441483in}{2.779548in}}%
\pgfpathlineto{\pgfqpoint{4.442353in}{2.767983in}}%
\pgfpathlineto{\pgfqpoint{4.443636in}{2.756419in}}%
\pgfpathlineto{\pgfqpoint{4.445667in}{2.744855in}}%
\pgfpathlineto{\pgfqpoint{4.448465in}{2.733290in}}%
\pgfpathlineto{\pgfqpoint{4.451312in}{2.721726in}}%
\pgfpathlineto{\pgfqpoint{4.453555in}{2.710162in}}%
\pgfpathlineto{\pgfqpoint{4.455404in}{2.698597in}}%
\pgfpathlineto{\pgfqpoint{4.457915in}{2.687033in}}%
\pgfpathlineto{\pgfqpoint{4.462143in}{2.675468in}}%
\pgfpathlineto{\pgfqpoint{4.468079in}{2.663904in}}%
\pgfpathlineto{\pgfqpoint{4.474259in}{2.652340in}}%
\pgfpathlineto{\pgfqpoint{4.479435in}{2.640775in}}%
\pgfpathlineto{\pgfqpoint{4.484883in}{2.629211in}}%
\pgfpathlineto{\pgfqpoint{4.492355in}{2.617646in}}%
\pgfpathlineto{\pgfqpoint{4.501104in}{2.606082in}}%
\pgfpathlineto{\pgfqpoint{4.508591in}{2.594518in}}%
\pgfpathlineto{\pgfqpoint{4.513225in}{2.582953in}}%
\pgfpathlineto{\pgfqpoint{4.516161in}{2.571389in}}%
\pgfpathlineto{\pgfqpoint{4.519917in}{2.559825in}}%
\pgfpathlineto{\pgfqpoint{4.526075in}{2.548260in}}%
\pgfpathlineto{\pgfqpoint{4.533916in}{2.536696in}}%
\pgfpathlineto{\pgfqpoint{4.539945in}{2.525131in}}%
\pgfpathlineto{\pgfqpoint{4.540503in}{2.513567in}}%
\pgfpathlineto{\pgfqpoint{4.536721in}{2.502003in}}%
\pgfpathlineto{\pgfqpoint{4.534474in}{2.490438in}}%
\pgfpathlineto{\pgfqpoint{4.537320in}{2.478874in}}%
\pgfpathlineto{\pgfqpoint{4.541122in}{2.467310in}}%
\pgfpathlineto{\pgfqpoint{4.540505in}{2.455745in}}%
\pgfpathlineto{\pgfqpoint{4.537880in}{2.444181in}}%
\pgfpathlineto{\pgfqpoint{4.540789in}{2.432616in}}%
\pgfpathlineto{\pgfqpoint{4.553579in}{2.421052in}}%
\pgfpathlineto{\pgfqpoint{4.570230in}{2.409488in}}%
\pgfpathlineto{\pgfqpoint{4.576922in}{2.397923in}}%
\pgfpathlineto{\pgfqpoint{4.566119in}{2.386359in}}%
\pgfpathlineto{\pgfqpoint{4.544592in}{2.374794in}}%
\pgfpathlineto{\pgfqpoint{4.525119in}{2.363230in}}%
\pgfpathlineto{\pgfqpoint{4.512841in}{2.351666in}}%
\pgfpathlineto{\pgfqpoint{4.506840in}{2.340101in}}%
\pgfpathlineto{\pgfqpoint{4.508075in}{2.328537in}}%
\pgfpathlineto{\pgfqpoint{4.512753in}{2.316973in}}%
\pgfpathlineto{\pgfqpoint{4.510718in}{2.305408in}}%
\pgfpathlineto{\pgfqpoint{4.500788in}{2.293844in}}%
\pgfpathlineto{\pgfqpoint{4.489742in}{2.282279in}}%
\pgfpathlineto{\pgfqpoint{4.479686in}{2.270715in}}%
\pgfpathlineto{\pgfqpoint{4.470490in}{2.259151in}}%
\pgfpathlineto{\pgfqpoint{4.463829in}{2.247586in}}%
\pgfpathlineto{\pgfqpoint{4.460238in}{2.236022in}}%
\pgfpathlineto{\pgfqpoint{4.457810in}{2.224458in}}%
\pgfpathlineto{\pgfqpoint{4.454659in}{2.212893in}}%
\pgfpathlineto{\pgfqpoint{4.450783in}{2.201329in}}%
\pgfpathlineto{\pgfqpoint{4.447492in}{2.189764in}}%
\pgfpathlineto{\pgfqpoint{4.445847in}{2.178200in}}%
\pgfpathlineto{\pgfqpoint{4.445496in}{2.166636in}}%
\pgfpathlineto{\pgfqpoint{4.445135in}{2.155071in}}%
\pgfpathlineto{\pgfqpoint{4.444348in}{2.143507in}}%
\pgfpathlineto{\pgfqpoint{4.443828in}{2.131942in}}%
\pgfpathlineto{\pgfqpoint{4.443533in}{2.120378in}}%
\pgfpathlineto{\pgfqpoint{4.442687in}{2.108814in}}%
\pgfpathlineto{\pgfqpoint{4.441380in}{2.097249in}}%
\pgfpathlineto{\pgfqpoint{4.440348in}{2.085685in}}%
\pgfpathlineto{\pgfqpoint{4.439852in}{2.074121in}}%
\pgfpathlineto{\pgfqpoint{4.439729in}{2.062556in}}%
\pgfpathlineto{\pgfqpoint{4.440005in}{2.050992in}}%
\pgfpathlineto{\pgfqpoint{4.440599in}{2.039427in}}%
\pgfpathlineto{\pgfqpoint{4.440878in}{2.027863in}}%
\pgfpathlineto{\pgfqpoint{4.440688in}{2.016299in}}%
\pgfpathlineto{\pgfqpoint{4.440906in}{2.004734in}}%
\pgfpathlineto{\pgfqpoint{4.441585in}{1.993170in}}%
\pgfpathlineto{\pgfqpoint{4.441393in}{1.981606in}}%
\pgfpathlineto{\pgfqpoint{4.440023in}{1.970041in}}%
\pgfpathlineto{\pgfqpoint{4.438558in}{1.958477in}}%
\pgfpathlineto{\pgfqpoint{4.437582in}{1.946912in}}%
\pgfpathlineto{\pgfqpoint{4.437116in}{1.935348in}}%
\pgfpathlineto{\pgfqpoint{4.437225in}{1.923784in}}%
\pgfpathlineto{\pgfqpoint{4.437884in}{1.912219in}}%
\pgfpathlineto{\pgfqpoint{4.438732in}{1.900655in}}%
\pgfpathlineto{\pgfqpoint{4.439035in}{1.889090in}}%
\pgfpathlineto{\pgfqpoint{4.438380in}{1.877526in}}%
\pgfpathlineto{\pgfqpoint{4.437300in}{1.865962in}}%
\pgfpathlineto{\pgfqpoint{4.436525in}{1.854397in}}%
\pgfpathlineto{\pgfqpoint{4.436212in}{1.842833in}}%
\pgfpathlineto{\pgfqpoint{4.436204in}{1.831269in}}%
\pgfpathlineto{\pgfqpoint{4.436355in}{1.819704in}}%
\pgfpathlineto{\pgfqpoint{4.436569in}{1.808140in}}%
\pgfpathlineto{\pgfqpoint{4.437280in}{1.796575in}}%
\pgfpathlineto{\pgfqpoint{4.439955in}{1.785011in}}%
\pgfpathlineto{\pgfqpoint{4.445111in}{1.773447in}}%
\pgfpathlineto{\pgfqpoint{4.448984in}{1.761882in}}%
\pgfpathlineto{\pgfqpoint{4.447257in}{1.750318in}}%
\pgfpathlineto{\pgfqpoint{4.442120in}{1.738754in}}%
\pgfpathlineto{\pgfqpoint{4.438334in}{1.727189in}}%
\pgfpathlineto{\pgfqpoint{4.436835in}{1.715625in}}%
\pgfpathlineto{\pgfqpoint{4.436465in}{1.704060in}}%
\pgfpathlineto{\pgfqpoint{4.436457in}{1.692496in}}%
\pgfpathlineto{\pgfqpoint{4.436373in}{1.680932in}}%
\pgfpathlineto{\pgfqpoint{4.436089in}{1.669367in}}%
\pgfpathlineto{\pgfqpoint{4.435846in}{1.657803in}}%
\pgfpathlineto{\pgfqpoint{4.435777in}{1.646238in}}%
\pgfpathlineto{\pgfqpoint{4.435759in}{1.634674in}}%
\pgfpathlineto{\pgfqpoint{4.435698in}{1.623110in}}%
\pgfpathlineto{\pgfqpoint{4.435623in}{1.611545in}}%
\pgfpathlineto{\pgfqpoint{4.435608in}{1.599981in}}%
\pgfpathlineto{\pgfqpoint{4.435707in}{1.588417in}}%
\pgfpathlineto{\pgfqpoint{4.435914in}{1.576852in}}%
\pgfpathlineto{\pgfqpoint{4.436120in}{1.565288in}}%
\pgfpathlineto{\pgfqpoint{4.436151in}{1.553723in}}%
\pgfpathlineto{\pgfqpoint{4.435974in}{1.542159in}}%
\pgfpathlineto{\pgfqpoint{4.435747in}{1.530595in}}%
\pgfpathlineto{\pgfqpoint{4.435615in}{1.519030in}}%
\pgfpathlineto{\pgfqpoint{4.435600in}{1.507466in}}%
\pgfpathlineto{\pgfqpoint{4.435658in}{1.495902in}}%
\pgfpathlineto{\pgfqpoint{4.435750in}{1.484337in}}%
\pgfpathlineto{\pgfqpoint{4.435843in}{1.472773in}}%
\pgfpathlineto{\pgfqpoint{4.435848in}{1.461208in}}%
\pgfpathlineto{\pgfqpoint{4.435729in}{1.449644in}}%
\pgfpathlineto{\pgfqpoint{4.435593in}{1.438080in}}%
\pgfpathlineto{\pgfqpoint{4.435535in}{1.426515in}}%
\pgfpathlineto{\pgfqpoint{4.435542in}{1.414951in}}%
\pgfpathlineto{\pgfqpoint{4.435584in}{1.403387in}}%
\pgfpathlineto{\pgfqpoint{4.435664in}{1.391822in}}%
\pgfpathlineto{\pgfqpoint{4.435761in}{1.380258in}}%
\pgfpathlineto{\pgfqpoint{4.435811in}{1.368693in}}%
\pgfpathlineto{\pgfqpoint{4.435819in}{1.357129in}}%
\pgfpathlineto{\pgfqpoint{4.435899in}{1.345565in}}%
\pgfpathlineto{\pgfqpoint{4.436191in}{1.334000in}}%
\pgfpathlineto{\pgfqpoint{4.436733in}{1.322436in}}%
\pgfpathlineto{\pgfqpoint{4.437179in}{1.310871in}}%
\pgfpathlineto{\pgfqpoint{4.436999in}{1.299307in}}%
\pgfpathlineto{\pgfqpoint{4.436304in}{1.287743in}}%
\pgfpathlineto{\pgfqpoint{4.435701in}{1.276178in}}%
\pgfpathlineto{\pgfqpoint{4.435428in}{1.264614in}}%
\pgfpathlineto{\pgfqpoint{4.435349in}{1.253050in}}%
\pgfpathlineto{\pgfqpoint{4.435331in}{1.241485in}}%
\pgfpathlineto{\pgfqpoint{4.435330in}{1.229921in}}%
\pgfpathlineto{\pgfqpoint{4.435337in}{1.218356in}}%
\pgfpathlineto{\pgfqpoint{4.435343in}{1.206792in}}%
\pgfpathlineto{\pgfqpoint{4.435342in}{1.195228in}}%
\pgfpathlineto{\pgfqpoint{4.435334in}{1.183663in}}%
\pgfpathlineto{\pgfqpoint{4.435328in}{1.172099in}}%
\pgfpathlineto{\pgfqpoint{4.435326in}{1.160535in}}%
\pgfpathlineto{\pgfqpoint{4.435325in}{1.148970in}}%
\pgfpathlineto{\pgfqpoint{4.435325in}{1.137406in}}%
\pgfpathlineto{\pgfqpoint{4.435325in}{1.125841in}}%
\pgfpathlineto{\pgfqpoint{4.435325in}{1.114277in}}%
\pgfpathlineto{\pgfqpoint{4.435325in}{1.102713in}}%
\pgfpathlineto{\pgfqpoint{4.435325in}{1.091148in}}%
\pgfpathlineto{\pgfqpoint{4.435325in}{1.079584in}}%
\pgfpathlineto{\pgfqpoint{4.435325in}{1.068019in}}%
\pgfpathlineto{\pgfqpoint{4.435326in}{1.056455in}}%
\pgfpathlineto{\pgfqpoint{4.435331in}{1.044891in}}%
\pgfpathlineto{\pgfqpoint{4.435343in}{1.033326in}}%
\pgfpathlineto{\pgfqpoint{4.435361in}{1.021762in}}%
\pgfpathlineto{\pgfqpoint{4.435367in}{1.010198in}}%
\pgfpathlineto{\pgfqpoint{4.435354in}{0.998633in}}%
\pgfpathlineto{\pgfqpoint{4.435337in}{0.987069in}}%
\pgfpathlineto{\pgfqpoint{4.435328in}{0.975504in}}%
\pgfpathlineto{\pgfqpoint{4.435325in}{0.963940in}}%
\pgfpathclose%
\pgfusepath{stroke,fill}%
}%
\begin{pgfscope}%
\pgfsys@transformshift{0.000000in}{0.000000in}%
\pgfsys@useobject{currentmarker}{}%
\end{pgfscope}%
\end{pgfscope}%
\begin{pgfscope}%
\pgfpathrectangle{\pgfqpoint{4.435325in}{0.499691in}}{\pgfqpoint{0.623789in}{3.662560in}}%
\pgfusepath{clip}%
\pgfsetbuttcap%
\pgfsetroundjoin%
\definecolor{currentfill}{rgb}{1.000000,0.498039,0.054902}%
\pgfsetfillcolor{currentfill}%
\pgfsetfillopacity{0.200000}%
\pgfsetlinewidth{1.003750pt}%
\definecolor{currentstroke}{rgb}{1.000000,0.498039,0.054902}%
\pgfsetstrokecolor{currentstroke}%
\pgfsetdash{}{0pt}%
\pgfsys@defobject{currentmarker}{\pgfqpoint{4.435325in}{1.221751in}}{\pgfqpoint{4.542202in}{2.446125in}}{%
\pgfpathmoveto{\pgfqpoint{4.435325in}{1.221751in}}%
\pgfpathlineto{\pgfqpoint{4.435325in}{1.221751in}}%
\pgfpathlineto{\pgfqpoint{4.435325in}{1.227904in}}%
\pgfpathlineto{\pgfqpoint{4.435325in}{1.234057in}}%
\pgfpathlineto{\pgfqpoint{4.435325in}{1.240209in}}%
\pgfpathlineto{\pgfqpoint{4.435325in}{1.246362in}}%
\pgfpathlineto{\pgfqpoint{4.435325in}{1.252515in}}%
\pgfpathlineto{\pgfqpoint{4.435325in}{1.258667in}}%
\pgfpathlineto{\pgfqpoint{4.435325in}{1.264820in}}%
\pgfpathlineto{\pgfqpoint{4.435325in}{1.270972in}}%
\pgfpathlineto{\pgfqpoint{4.435325in}{1.277125in}}%
\pgfpathlineto{\pgfqpoint{4.435325in}{1.283278in}}%
\pgfpathlineto{\pgfqpoint{4.435325in}{1.289430in}}%
\pgfpathlineto{\pgfqpoint{4.435325in}{1.295583in}}%
\pgfpathlineto{\pgfqpoint{4.435325in}{1.301736in}}%
\pgfpathlineto{\pgfqpoint{4.435325in}{1.307888in}}%
\pgfpathlineto{\pgfqpoint{4.435325in}{1.314041in}}%
\pgfpathlineto{\pgfqpoint{4.435325in}{1.320194in}}%
\pgfpathlineto{\pgfqpoint{4.435325in}{1.326346in}}%
\pgfpathlineto{\pgfqpoint{4.435325in}{1.332499in}}%
\pgfpathlineto{\pgfqpoint{4.435325in}{1.338651in}}%
\pgfpathlineto{\pgfqpoint{4.435325in}{1.344804in}}%
\pgfpathlineto{\pgfqpoint{4.435325in}{1.350957in}}%
\pgfpathlineto{\pgfqpoint{4.435325in}{1.357109in}}%
\pgfpathlineto{\pgfqpoint{4.435325in}{1.363262in}}%
\pgfpathlineto{\pgfqpoint{4.435325in}{1.369415in}}%
\pgfpathlineto{\pgfqpoint{4.435325in}{1.375567in}}%
\pgfpathlineto{\pgfqpoint{4.435325in}{1.381720in}}%
\pgfpathlineto{\pgfqpoint{4.435325in}{1.387872in}}%
\pgfpathlineto{\pgfqpoint{4.435325in}{1.394025in}}%
\pgfpathlineto{\pgfqpoint{4.435325in}{1.400178in}}%
\pgfpathlineto{\pgfqpoint{4.435325in}{1.406330in}}%
\pgfpathlineto{\pgfqpoint{4.435325in}{1.412483in}}%
\pgfpathlineto{\pgfqpoint{4.435325in}{1.418636in}}%
\pgfpathlineto{\pgfqpoint{4.435325in}{1.424788in}}%
\pgfpathlineto{\pgfqpoint{4.435325in}{1.430941in}}%
\pgfpathlineto{\pgfqpoint{4.435325in}{1.437094in}}%
\pgfpathlineto{\pgfqpoint{4.435325in}{1.443246in}}%
\pgfpathlineto{\pgfqpoint{4.435325in}{1.449399in}}%
\pgfpathlineto{\pgfqpoint{4.435325in}{1.455551in}}%
\pgfpathlineto{\pgfqpoint{4.435325in}{1.461704in}}%
\pgfpathlineto{\pgfqpoint{4.435325in}{1.467857in}}%
\pgfpathlineto{\pgfqpoint{4.435325in}{1.474009in}}%
\pgfpathlineto{\pgfqpoint{4.435325in}{1.480162in}}%
\pgfpathlineto{\pgfqpoint{4.435325in}{1.486315in}}%
\pgfpathlineto{\pgfqpoint{4.435325in}{1.492467in}}%
\pgfpathlineto{\pgfqpoint{4.435325in}{1.498620in}}%
\pgfpathlineto{\pgfqpoint{4.435325in}{1.504773in}}%
\pgfpathlineto{\pgfqpoint{4.435325in}{1.510925in}}%
\pgfpathlineto{\pgfqpoint{4.435325in}{1.517078in}}%
\pgfpathlineto{\pgfqpoint{4.435325in}{1.523230in}}%
\pgfpathlineto{\pgfqpoint{4.435325in}{1.529383in}}%
\pgfpathlineto{\pgfqpoint{4.435325in}{1.535536in}}%
\pgfpathlineto{\pgfqpoint{4.435325in}{1.541688in}}%
\pgfpathlineto{\pgfqpoint{4.435325in}{1.547841in}}%
\pgfpathlineto{\pgfqpoint{4.435325in}{1.553994in}}%
\pgfpathlineto{\pgfqpoint{4.435325in}{1.560146in}}%
\pgfpathlineto{\pgfqpoint{4.435325in}{1.566299in}}%
\pgfpathlineto{\pgfqpoint{4.435325in}{1.572451in}}%
\pgfpathlineto{\pgfqpoint{4.435325in}{1.578604in}}%
\pgfpathlineto{\pgfqpoint{4.435325in}{1.584757in}}%
\pgfpathlineto{\pgfqpoint{4.435325in}{1.590909in}}%
\pgfpathlineto{\pgfqpoint{4.435325in}{1.597062in}}%
\pgfpathlineto{\pgfqpoint{4.435325in}{1.603215in}}%
\pgfpathlineto{\pgfqpoint{4.435325in}{1.609367in}}%
\pgfpathlineto{\pgfqpoint{4.435325in}{1.615520in}}%
\pgfpathlineto{\pgfqpoint{4.435325in}{1.621673in}}%
\pgfpathlineto{\pgfqpoint{4.435325in}{1.627825in}}%
\pgfpathlineto{\pgfqpoint{4.435325in}{1.633978in}}%
\pgfpathlineto{\pgfqpoint{4.435325in}{1.640130in}}%
\pgfpathlineto{\pgfqpoint{4.435325in}{1.646283in}}%
\pgfpathlineto{\pgfqpoint{4.435325in}{1.652436in}}%
\pgfpathlineto{\pgfqpoint{4.435325in}{1.658588in}}%
\pgfpathlineto{\pgfqpoint{4.435325in}{1.664741in}}%
\pgfpathlineto{\pgfqpoint{4.435325in}{1.670894in}}%
\pgfpathlineto{\pgfqpoint{4.435325in}{1.677046in}}%
\pgfpathlineto{\pgfqpoint{4.435325in}{1.683199in}}%
\pgfpathlineto{\pgfqpoint{4.435325in}{1.689351in}}%
\pgfpathlineto{\pgfqpoint{4.435325in}{1.695504in}}%
\pgfpathlineto{\pgfqpoint{4.435325in}{1.701657in}}%
\pgfpathlineto{\pgfqpoint{4.435325in}{1.707809in}}%
\pgfpathlineto{\pgfqpoint{4.435325in}{1.713962in}}%
\pgfpathlineto{\pgfqpoint{4.435325in}{1.720115in}}%
\pgfpathlineto{\pgfqpoint{4.435325in}{1.726267in}}%
\pgfpathlineto{\pgfqpoint{4.435325in}{1.732420in}}%
\pgfpathlineto{\pgfqpoint{4.435325in}{1.738573in}}%
\pgfpathlineto{\pgfqpoint{4.435325in}{1.744725in}}%
\pgfpathlineto{\pgfqpoint{4.435325in}{1.750878in}}%
\pgfpathlineto{\pgfqpoint{4.435325in}{1.757030in}}%
\pgfpathlineto{\pgfqpoint{4.435325in}{1.763183in}}%
\pgfpathlineto{\pgfqpoint{4.435325in}{1.769336in}}%
\pgfpathlineto{\pgfqpoint{4.435325in}{1.775488in}}%
\pgfpathlineto{\pgfqpoint{4.435325in}{1.781641in}}%
\pgfpathlineto{\pgfqpoint{4.435325in}{1.787794in}}%
\pgfpathlineto{\pgfqpoint{4.435325in}{1.793946in}}%
\pgfpathlineto{\pgfqpoint{4.435325in}{1.800099in}}%
\pgfpathlineto{\pgfqpoint{4.435325in}{1.806252in}}%
\pgfpathlineto{\pgfqpoint{4.435325in}{1.812404in}}%
\pgfpathlineto{\pgfqpoint{4.435325in}{1.818557in}}%
\pgfpathlineto{\pgfqpoint{4.435325in}{1.824709in}}%
\pgfpathlineto{\pgfqpoint{4.435325in}{1.830862in}}%
\pgfpathlineto{\pgfqpoint{4.435325in}{1.837015in}}%
\pgfpathlineto{\pgfqpoint{4.435325in}{1.843167in}}%
\pgfpathlineto{\pgfqpoint{4.435325in}{1.849320in}}%
\pgfpathlineto{\pgfqpoint{4.435325in}{1.855473in}}%
\pgfpathlineto{\pgfqpoint{4.435325in}{1.861625in}}%
\pgfpathlineto{\pgfqpoint{4.435325in}{1.867778in}}%
\pgfpathlineto{\pgfqpoint{4.435325in}{1.873930in}}%
\pgfpathlineto{\pgfqpoint{4.435325in}{1.880083in}}%
\pgfpathlineto{\pgfqpoint{4.435325in}{1.886236in}}%
\pgfpathlineto{\pgfqpoint{4.435325in}{1.892388in}}%
\pgfpathlineto{\pgfqpoint{4.435325in}{1.898541in}}%
\pgfpathlineto{\pgfqpoint{4.435325in}{1.904694in}}%
\pgfpathlineto{\pgfqpoint{4.435325in}{1.910846in}}%
\pgfpathlineto{\pgfqpoint{4.435325in}{1.916999in}}%
\pgfpathlineto{\pgfqpoint{4.435325in}{1.923152in}}%
\pgfpathlineto{\pgfqpoint{4.435325in}{1.929304in}}%
\pgfpathlineto{\pgfqpoint{4.435325in}{1.935457in}}%
\pgfpathlineto{\pgfqpoint{4.435325in}{1.941609in}}%
\pgfpathlineto{\pgfqpoint{4.435325in}{1.947762in}}%
\pgfpathlineto{\pgfqpoint{4.435325in}{1.953915in}}%
\pgfpathlineto{\pgfqpoint{4.435325in}{1.960067in}}%
\pgfpathlineto{\pgfqpoint{4.435325in}{1.966220in}}%
\pgfpathlineto{\pgfqpoint{4.435325in}{1.972373in}}%
\pgfpathlineto{\pgfqpoint{4.435325in}{1.978525in}}%
\pgfpathlineto{\pgfqpoint{4.435325in}{1.984678in}}%
\pgfpathlineto{\pgfqpoint{4.435325in}{1.990831in}}%
\pgfpathlineto{\pgfqpoint{4.435325in}{1.996983in}}%
\pgfpathlineto{\pgfqpoint{4.435325in}{2.003136in}}%
\pgfpathlineto{\pgfqpoint{4.435325in}{2.009288in}}%
\pgfpathlineto{\pgfqpoint{4.435325in}{2.015441in}}%
\pgfpathlineto{\pgfqpoint{4.435325in}{2.021594in}}%
\pgfpathlineto{\pgfqpoint{4.435325in}{2.027746in}}%
\pgfpathlineto{\pgfqpoint{4.435325in}{2.033899in}}%
\pgfpathlineto{\pgfqpoint{4.435325in}{2.040052in}}%
\pgfpathlineto{\pgfqpoint{4.435325in}{2.046204in}}%
\pgfpathlineto{\pgfqpoint{4.435325in}{2.052357in}}%
\pgfpathlineto{\pgfqpoint{4.435325in}{2.058509in}}%
\pgfpathlineto{\pgfqpoint{4.435325in}{2.064662in}}%
\pgfpathlineto{\pgfqpoint{4.435325in}{2.070815in}}%
\pgfpathlineto{\pgfqpoint{4.435325in}{2.076967in}}%
\pgfpathlineto{\pgfqpoint{4.435325in}{2.083120in}}%
\pgfpathlineto{\pgfqpoint{4.435325in}{2.089273in}}%
\pgfpathlineto{\pgfqpoint{4.435325in}{2.095425in}}%
\pgfpathlineto{\pgfqpoint{4.435325in}{2.101578in}}%
\pgfpathlineto{\pgfqpoint{4.435325in}{2.107731in}}%
\pgfpathlineto{\pgfqpoint{4.435325in}{2.113883in}}%
\pgfpathlineto{\pgfqpoint{4.435325in}{2.120036in}}%
\pgfpathlineto{\pgfqpoint{4.435325in}{2.126188in}}%
\pgfpathlineto{\pgfqpoint{4.435325in}{2.132341in}}%
\pgfpathlineto{\pgfqpoint{4.435325in}{2.138494in}}%
\pgfpathlineto{\pgfqpoint{4.435325in}{2.144646in}}%
\pgfpathlineto{\pgfqpoint{4.435325in}{2.150799in}}%
\pgfpathlineto{\pgfqpoint{4.435325in}{2.156952in}}%
\pgfpathlineto{\pgfqpoint{4.435325in}{2.163104in}}%
\pgfpathlineto{\pgfqpoint{4.435325in}{2.169257in}}%
\pgfpathlineto{\pgfqpoint{4.435325in}{2.175409in}}%
\pgfpathlineto{\pgfqpoint{4.435325in}{2.181562in}}%
\pgfpathlineto{\pgfqpoint{4.435325in}{2.187715in}}%
\pgfpathlineto{\pgfqpoint{4.435325in}{2.193867in}}%
\pgfpathlineto{\pgfqpoint{4.435325in}{2.200020in}}%
\pgfpathlineto{\pgfqpoint{4.435325in}{2.206173in}}%
\pgfpathlineto{\pgfqpoint{4.435325in}{2.212325in}}%
\pgfpathlineto{\pgfqpoint{4.435325in}{2.218478in}}%
\pgfpathlineto{\pgfqpoint{4.435325in}{2.224631in}}%
\pgfpathlineto{\pgfqpoint{4.435325in}{2.230783in}}%
\pgfpathlineto{\pgfqpoint{4.435325in}{2.236936in}}%
\pgfpathlineto{\pgfqpoint{4.435325in}{2.243088in}}%
\pgfpathlineto{\pgfqpoint{4.435325in}{2.249241in}}%
\pgfpathlineto{\pgfqpoint{4.435325in}{2.255394in}}%
\pgfpathlineto{\pgfqpoint{4.435325in}{2.261546in}}%
\pgfpathlineto{\pgfqpoint{4.435325in}{2.267699in}}%
\pgfpathlineto{\pgfqpoint{4.435325in}{2.273852in}}%
\pgfpathlineto{\pgfqpoint{4.435325in}{2.280004in}}%
\pgfpathlineto{\pgfqpoint{4.435325in}{2.286157in}}%
\pgfpathlineto{\pgfqpoint{4.435325in}{2.292310in}}%
\pgfpathlineto{\pgfqpoint{4.435325in}{2.298462in}}%
\pgfpathlineto{\pgfqpoint{4.435325in}{2.304615in}}%
\pgfpathlineto{\pgfqpoint{4.435325in}{2.310767in}}%
\pgfpathlineto{\pgfqpoint{4.435325in}{2.316920in}}%
\pgfpathlineto{\pgfqpoint{4.435325in}{2.323073in}}%
\pgfpathlineto{\pgfqpoint{4.435325in}{2.329225in}}%
\pgfpathlineto{\pgfqpoint{4.435325in}{2.335378in}}%
\pgfpathlineto{\pgfqpoint{4.435325in}{2.341531in}}%
\pgfpathlineto{\pgfqpoint{4.435325in}{2.347683in}}%
\pgfpathlineto{\pgfqpoint{4.435325in}{2.353836in}}%
\pgfpathlineto{\pgfqpoint{4.435325in}{2.359988in}}%
\pgfpathlineto{\pgfqpoint{4.435325in}{2.366141in}}%
\pgfpathlineto{\pgfqpoint{4.435325in}{2.372294in}}%
\pgfpathlineto{\pgfqpoint{4.435325in}{2.378446in}}%
\pgfpathlineto{\pgfqpoint{4.435325in}{2.384599in}}%
\pgfpathlineto{\pgfqpoint{4.435325in}{2.390752in}}%
\pgfpathlineto{\pgfqpoint{4.435325in}{2.396904in}}%
\pgfpathlineto{\pgfqpoint{4.435325in}{2.403057in}}%
\pgfpathlineto{\pgfqpoint{4.435325in}{2.409210in}}%
\pgfpathlineto{\pgfqpoint{4.435325in}{2.415362in}}%
\pgfpathlineto{\pgfqpoint{4.435325in}{2.421515in}}%
\pgfpathlineto{\pgfqpoint{4.435325in}{2.427667in}}%
\pgfpathlineto{\pgfqpoint{4.435325in}{2.433820in}}%
\pgfpathlineto{\pgfqpoint{4.435325in}{2.439973in}}%
\pgfpathlineto{\pgfqpoint{4.435325in}{2.446125in}}%
\pgfpathlineto{\pgfqpoint{4.435325in}{2.446125in}}%
\pgfpathlineto{\pgfqpoint{4.435325in}{2.446125in}}%
\pgfpathlineto{\pgfqpoint{4.435326in}{2.439973in}}%
\pgfpathlineto{\pgfqpoint{4.435329in}{2.433820in}}%
\pgfpathlineto{\pgfqpoint{4.435339in}{2.427667in}}%
\pgfpathlineto{\pgfqpoint{4.435359in}{2.421515in}}%
\pgfpathlineto{\pgfqpoint{4.435384in}{2.415362in}}%
\pgfpathlineto{\pgfqpoint{4.435395in}{2.409210in}}%
\pgfpathlineto{\pgfqpoint{4.435383in}{2.403057in}}%
\pgfpathlineto{\pgfqpoint{4.435366in}{2.396904in}}%
\pgfpathlineto{\pgfqpoint{4.435367in}{2.390752in}}%
\pgfpathlineto{\pgfqpoint{4.435385in}{2.384599in}}%
\pgfpathlineto{\pgfqpoint{4.435395in}{2.378446in}}%
\pgfpathlineto{\pgfqpoint{4.435382in}{2.372294in}}%
\pgfpathlineto{\pgfqpoint{4.435357in}{2.366141in}}%
\pgfpathlineto{\pgfqpoint{4.435339in}{2.359988in}}%
\pgfpathlineto{\pgfqpoint{4.435335in}{2.353836in}}%
\pgfpathlineto{\pgfqpoint{4.435344in}{2.347683in}}%
\pgfpathlineto{\pgfqpoint{4.435363in}{2.341531in}}%
\pgfpathlineto{\pgfqpoint{4.435388in}{2.335378in}}%
\pgfpathlineto{\pgfqpoint{4.435406in}{2.329225in}}%
\pgfpathlineto{\pgfqpoint{4.435413in}{2.323073in}}%
\pgfpathlineto{\pgfqpoint{4.435422in}{2.316920in}}%
\pgfpathlineto{\pgfqpoint{4.435464in}{2.310767in}}%
\pgfpathlineto{\pgfqpoint{4.435546in}{2.304615in}}%
\pgfpathlineto{\pgfqpoint{4.435635in}{2.298462in}}%
\pgfpathlineto{\pgfqpoint{4.435680in}{2.292310in}}%
\pgfpathlineto{\pgfqpoint{4.435673in}{2.286157in}}%
\pgfpathlineto{\pgfqpoint{4.435633in}{2.280004in}}%
\pgfpathlineto{\pgfqpoint{4.435579in}{2.273852in}}%
\pgfpathlineto{\pgfqpoint{4.435531in}{2.267699in}}%
\pgfpathlineto{\pgfqpoint{4.435490in}{2.261546in}}%
\pgfpathlineto{\pgfqpoint{4.435464in}{2.255394in}}%
\pgfpathlineto{\pgfqpoint{4.435476in}{2.249241in}}%
\pgfpathlineto{\pgfqpoint{4.435531in}{2.243088in}}%
\pgfpathlineto{\pgfqpoint{4.435596in}{2.236936in}}%
\pgfpathlineto{\pgfqpoint{4.435651in}{2.230783in}}%
\pgfpathlineto{\pgfqpoint{4.435712in}{2.224631in}}%
\pgfpathlineto{\pgfqpoint{4.435770in}{2.218478in}}%
\pgfpathlineto{\pgfqpoint{4.435772in}{2.212325in}}%
\pgfpathlineto{\pgfqpoint{4.435705in}{2.206173in}}%
\pgfpathlineto{\pgfqpoint{4.435621in}{2.200020in}}%
\pgfpathlineto{\pgfqpoint{4.435588in}{2.193867in}}%
\pgfpathlineto{\pgfqpoint{4.435774in}{2.187715in}}%
\pgfpathlineto{\pgfqpoint{4.436839in}{2.181562in}}%
\pgfpathlineto{\pgfqpoint{4.440060in}{2.175409in}}%
\pgfpathlineto{\pgfqpoint{4.445653in}{2.169257in}}%
\pgfpathlineto{\pgfqpoint{4.450253in}{2.163104in}}%
\pgfpathlineto{\pgfqpoint{4.449500in}{2.156952in}}%
\pgfpathlineto{\pgfqpoint{4.444186in}{2.150799in}}%
\pgfpathlineto{\pgfqpoint{4.439048in}{2.144646in}}%
\pgfpathlineto{\pgfqpoint{4.436504in}{2.138494in}}%
\pgfpathlineto{\pgfqpoint{4.435772in}{2.132341in}}%
\pgfpathlineto{\pgfqpoint{4.435660in}{2.126188in}}%
\pgfpathlineto{\pgfqpoint{4.435689in}{2.120036in}}%
\pgfpathlineto{\pgfqpoint{4.435767in}{2.113883in}}%
\pgfpathlineto{\pgfqpoint{4.435868in}{2.107731in}}%
\pgfpathlineto{\pgfqpoint{4.435940in}{2.101578in}}%
\pgfpathlineto{\pgfqpoint{4.435955in}{2.095425in}}%
\pgfpathlineto{\pgfqpoint{4.435946in}{2.089273in}}%
\pgfpathlineto{\pgfqpoint{4.435954in}{2.083120in}}%
\pgfpathlineto{\pgfqpoint{4.435977in}{2.076967in}}%
\pgfpathlineto{\pgfqpoint{4.435996in}{2.070815in}}%
\pgfpathlineto{\pgfqpoint{4.436023in}{2.064662in}}%
\pgfpathlineto{\pgfqpoint{4.436089in}{2.058509in}}%
\pgfpathlineto{\pgfqpoint{4.436203in}{2.052357in}}%
\pgfpathlineto{\pgfqpoint{4.436353in}{2.046204in}}%
\pgfpathlineto{\pgfqpoint{4.436517in}{2.040052in}}%
\pgfpathlineto{\pgfqpoint{4.436710in}{2.033899in}}%
\pgfpathlineto{\pgfqpoint{4.437057in}{2.027746in}}%
\pgfpathlineto{\pgfqpoint{4.437697in}{2.021594in}}%
\pgfpathlineto{\pgfqpoint{4.438493in}{2.015441in}}%
\pgfpathlineto{\pgfqpoint{4.439013in}{2.009288in}}%
\pgfpathlineto{\pgfqpoint{4.439041in}{2.003136in}}%
\pgfpathlineto{\pgfqpoint{4.438774in}{1.996983in}}%
\pgfpathlineto{\pgfqpoint{4.438371in}{1.990831in}}%
\pgfpathlineto{\pgfqpoint{4.437884in}{1.984678in}}%
\pgfpathlineto{\pgfqpoint{4.437520in}{1.978525in}}%
\pgfpathlineto{\pgfqpoint{4.437532in}{1.972373in}}%
\pgfpathlineto{\pgfqpoint{4.438340in}{1.966220in}}%
\pgfpathlineto{\pgfqpoint{4.441231in}{1.960067in}}%
\pgfpathlineto{\pgfqpoint{4.448106in}{1.953915in}}%
\pgfpathlineto{\pgfqpoint{4.457897in}{1.947762in}}%
\pgfpathlineto{\pgfqpoint{4.463788in}{1.941609in}}%
\pgfpathlineto{\pgfqpoint{4.460199in}{1.935457in}}%
\pgfpathlineto{\pgfqpoint{4.451110in}{1.929304in}}%
\pgfpathlineto{\pgfqpoint{4.444536in}{1.923152in}}%
\pgfpathlineto{\pgfqpoint{4.444248in}{1.916999in}}%
\pgfpathlineto{\pgfqpoint{4.451207in}{1.910846in}}%
\pgfpathlineto{\pgfqpoint{4.465357in}{1.904694in}}%
\pgfpathlineto{\pgfqpoint{4.482313in}{1.898541in}}%
\pgfpathlineto{\pgfqpoint{4.494237in}{1.892388in}}%
\pgfpathlineto{\pgfqpoint{4.498403in}{1.886236in}}%
\pgfpathlineto{\pgfqpoint{4.500346in}{1.880083in}}%
\pgfpathlineto{\pgfqpoint{4.504698in}{1.873930in}}%
\pgfpathlineto{\pgfqpoint{4.510686in}{1.867778in}}%
\pgfpathlineto{\pgfqpoint{4.518630in}{1.861625in}}%
\pgfpathlineto{\pgfqpoint{4.530301in}{1.855473in}}%
\pgfpathlineto{\pgfqpoint{4.541609in}{1.849320in}}%
\pgfpathlineto{\pgfqpoint{4.542202in}{1.843167in}}%
\pgfpathlineto{\pgfqpoint{4.527009in}{1.837015in}}%
\pgfpathlineto{\pgfqpoint{4.506630in}{1.830862in}}%
\pgfpathlineto{\pgfqpoint{4.498487in}{1.824709in}}%
\pgfpathlineto{\pgfqpoint{4.503408in}{1.818557in}}%
\pgfpathlineto{\pgfqpoint{4.503302in}{1.812404in}}%
\pgfpathlineto{\pgfqpoint{4.488006in}{1.806252in}}%
\pgfpathlineto{\pgfqpoint{4.467170in}{1.800099in}}%
\pgfpathlineto{\pgfqpoint{4.452285in}{1.793946in}}%
\pgfpathlineto{\pgfqpoint{4.444650in}{1.787794in}}%
\pgfpathlineto{\pgfqpoint{4.441007in}{1.781641in}}%
\pgfpathlineto{\pgfqpoint{4.439192in}{1.775488in}}%
\pgfpathlineto{\pgfqpoint{4.438411in}{1.769336in}}%
\pgfpathlineto{\pgfqpoint{4.438427in}{1.763183in}}%
\pgfpathlineto{\pgfqpoint{4.439092in}{1.757030in}}%
\pgfpathlineto{\pgfqpoint{4.439882in}{1.750878in}}%
\pgfpathlineto{\pgfqpoint{4.439902in}{1.744725in}}%
\pgfpathlineto{\pgfqpoint{4.438851in}{1.738573in}}%
\pgfpathlineto{\pgfqpoint{4.437454in}{1.732420in}}%
\pgfpathlineto{\pgfqpoint{4.436466in}{1.726267in}}%
\pgfpathlineto{\pgfqpoint{4.435998in}{1.720115in}}%
\pgfpathlineto{\pgfqpoint{4.435849in}{1.713962in}}%
\pgfpathlineto{\pgfqpoint{4.435860in}{1.707809in}}%
\pgfpathlineto{\pgfqpoint{4.435957in}{1.701657in}}%
\pgfpathlineto{\pgfqpoint{4.436093in}{1.695504in}}%
\pgfpathlineto{\pgfqpoint{4.436185in}{1.689351in}}%
\pgfpathlineto{\pgfqpoint{4.436128in}{1.683199in}}%
\pgfpathlineto{\pgfqpoint{4.435931in}{1.677046in}}%
\pgfpathlineto{\pgfqpoint{4.435728in}{1.670894in}}%
\pgfpathlineto{\pgfqpoint{4.435608in}{1.664741in}}%
\pgfpathlineto{\pgfqpoint{4.435560in}{1.658588in}}%
\pgfpathlineto{\pgfqpoint{4.435548in}{1.652436in}}%
\pgfpathlineto{\pgfqpoint{4.435559in}{1.646283in}}%
\pgfpathlineto{\pgfqpoint{4.435582in}{1.640130in}}%
\pgfpathlineto{\pgfqpoint{4.435595in}{1.633978in}}%
\pgfpathlineto{\pgfqpoint{4.435591in}{1.627825in}}%
\pgfpathlineto{\pgfqpoint{4.435584in}{1.621673in}}%
\pgfpathlineto{\pgfqpoint{4.435609in}{1.615520in}}%
\pgfpathlineto{\pgfqpoint{4.435730in}{1.609367in}}%
\pgfpathlineto{\pgfqpoint{4.435979in}{1.603215in}}%
\pgfpathlineto{\pgfqpoint{4.436218in}{1.597062in}}%
\pgfpathlineto{\pgfqpoint{4.436203in}{1.590909in}}%
\pgfpathlineto{\pgfqpoint{4.435929in}{1.584757in}}%
\pgfpathlineto{\pgfqpoint{4.435642in}{1.578604in}}%
\pgfpathlineto{\pgfqpoint{4.435510in}{1.572451in}}%
\pgfpathlineto{\pgfqpoint{4.435500in}{1.566299in}}%
\pgfpathlineto{\pgfqpoint{4.435522in}{1.560146in}}%
\pgfpathlineto{\pgfqpoint{4.435526in}{1.553994in}}%
\pgfpathlineto{\pgfqpoint{4.435516in}{1.547841in}}%
\pgfpathlineto{\pgfqpoint{4.435513in}{1.541688in}}%
\pgfpathlineto{\pgfqpoint{4.435513in}{1.535536in}}%
\pgfpathlineto{\pgfqpoint{4.435490in}{1.529383in}}%
\pgfpathlineto{\pgfqpoint{4.435441in}{1.523230in}}%
\pgfpathlineto{\pgfqpoint{4.435390in}{1.517078in}}%
\pgfpathlineto{\pgfqpoint{4.435353in}{1.510925in}}%
\pgfpathlineto{\pgfqpoint{4.435334in}{1.504773in}}%
\pgfpathlineto{\pgfqpoint{4.435327in}{1.498620in}}%
\pgfpathlineto{\pgfqpoint{4.435325in}{1.492467in}}%
\pgfpathlineto{\pgfqpoint{4.435326in}{1.486315in}}%
\pgfpathlineto{\pgfqpoint{4.435329in}{1.480162in}}%
\pgfpathlineto{\pgfqpoint{4.435335in}{1.474009in}}%
\pgfpathlineto{\pgfqpoint{4.435340in}{1.467857in}}%
\pgfpathlineto{\pgfqpoint{4.435340in}{1.461704in}}%
\pgfpathlineto{\pgfqpoint{4.435335in}{1.455551in}}%
\pgfpathlineto{\pgfqpoint{4.435330in}{1.449399in}}%
\pgfpathlineto{\pgfqpoint{4.435328in}{1.443246in}}%
\pgfpathlineto{\pgfqpoint{4.435330in}{1.437094in}}%
\pgfpathlineto{\pgfqpoint{4.435332in}{1.430941in}}%
\pgfpathlineto{\pgfqpoint{4.435333in}{1.424788in}}%
\pgfpathlineto{\pgfqpoint{4.435332in}{1.418636in}}%
\pgfpathlineto{\pgfqpoint{4.435332in}{1.412483in}}%
\pgfpathlineto{\pgfqpoint{4.435335in}{1.406330in}}%
\pgfpathlineto{\pgfqpoint{4.435340in}{1.400178in}}%
\pgfpathlineto{\pgfqpoint{4.435341in}{1.394025in}}%
\pgfpathlineto{\pgfqpoint{4.435338in}{1.387872in}}%
\pgfpathlineto{\pgfqpoint{4.435340in}{1.381720in}}%
\pgfpathlineto{\pgfqpoint{4.435349in}{1.375567in}}%
\pgfpathlineto{\pgfqpoint{4.435358in}{1.369415in}}%
\pgfpathlineto{\pgfqpoint{4.435359in}{1.363262in}}%
\pgfpathlineto{\pgfqpoint{4.435355in}{1.357109in}}%
\pgfpathlineto{\pgfqpoint{4.435353in}{1.350957in}}%
\pgfpathlineto{\pgfqpoint{4.435350in}{1.344804in}}%
\pgfpathlineto{\pgfqpoint{4.435342in}{1.338651in}}%
\pgfpathlineto{\pgfqpoint{4.435333in}{1.332499in}}%
\pgfpathlineto{\pgfqpoint{4.435327in}{1.326346in}}%
\pgfpathlineto{\pgfqpoint{4.435325in}{1.320194in}}%
\pgfpathlineto{\pgfqpoint{4.435325in}{1.314041in}}%
\pgfpathlineto{\pgfqpoint{4.435325in}{1.307888in}}%
\pgfpathlineto{\pgfqpoint{4.435325in}{1.301736in}}%
\pgfpathlineto{\pgfqpoint{4.435329in}{1.295583in}}%
\pgfpathlineto{\pgfqpoint{4.435343in}{1.289430in}}%
\pgfpathlineto{\pgfqpoint{4.435379in}{1.283278in}}%
\pgfpathlineto{\pgfqpoint{4.435431in}{1.277125in}}%
\pgfpathlineto{\pgfqpoint{4.435462in}{1.270972in}}%
\pgfpathlineto{\pgfqpoint{4.435442in}{1.264820in}}%
\pgfpathlineto{\pgfqpoint{4.435393in}{1.258667in}}%
\pgfpathlineto{\pgfqpoint{4.435355in}{1.252515in}}%
\pgfpathlineto{\pgfqpoint{4.435338in}{1.246362in}}%
\pgfpathlineto{\pgfqpoint{4.435330in}{1.240209in}}%
\pgfpathlineto{\pgfqpoint{4.435327in}{1.234057in}}%
\pgfpathlineto{\pgfqpoint{4.435325in}{1.227904in}}%
\pgfpathlineto{\pgfqpoint{4.435325in}{1.221751in}}%
\pgfpathclose%
\pgfusepath{stroke,fill}%
}%
\begin{pgfscope}%
\pgfsys@transformshift{0.000000in}{0.000000in}%
\pgfsys@useobject{currentmarker}{}%
\end{pgfscope}%
\end{pgfscope}%
\begin{pgfscope}%
\pgfpathrectangle{\pgfqpoint{4.435325in}{0.499691in}}{\pgfqpoint{0.623789in}{3.662560in}}%
\pgfusepath{clip}%
\pgfsetbuttcap%
\pgfsetroundjoin%
\definecolor{currentfill}{rgb}{0.121569,0.466667,0.705882}%
\pgfsetfillcolor{currentfill}%
\pgfsetfillopacity{0.200000}%
\pgfsetlinewidth{1.003750pt}%
\definecolor{currentstroke}{rgb}{0.121569,0.466667,0.705882}%
\pgfsetstrokecolor{currentstroke}%
\pgfsetdash{}{0pt}%
\pgfsys@defobject{currentmarker}{\pgfqpoint{4.435325in}{1.988803in}}{\pgfqpoint{5.029409in}{4.973604in}}{%
\pgfpathmoveto{\pgfqpoint{4.435325in}{1.988803in}}%
\pgfpathlineto{\pgfqpoint{4.435325in}{1.988803in}}%
\pgfpathlineto{\pgfqpoint{4.435325in}{2.003802in}}%
\pgfpathlineto{\pgfqpoint{4.435325in}{2.018801in}}%
\pgfpathlineto{\pgfqpoint{4.435325in}{2.033800in}}%
\pgfpathlineto{\pgfqpoint{4.435325in}{2.048799in}}%
\pgfpathlineto{\pgfqpoint{4.435325in}{2.063798in}}%
\pgfpathlineto{\pgfqpoint{4.435325in}{2.078797in}}%
\pgfpathlineto{\pgfqpoint{4.435325in}{2.093796in}}%
\pgfpathlineto{\pgfqpoint{4.435325in}{2.108795in}}%
\pgfpathlineto{\pgfqpoint{4.435325in}{2.123794in}}%
\pgfpathlineto{\pgfqpoint{4.435325in}{2.138793in}}%
\pgfpathlineto{\pgfqpoint{4.435325in}{2.153792in}}%
\pgfpathlineto{\pgfqpoint{4.435325in}{2.168791in}}%
\pgfpathlineto{\pgfqpoint{4.435325in}{2.183790in}}%
\pgfpathlineto{\pgfqpoint{4.435325in}{2.198789in}}%
\pgfpathlineto{\pgfqpoint{4.435325in}{2.213788in}}%
\pgfpathlineto{\pgfqpoint{4.435325in}{2.228787in}}%
\pgfpathlineto{\pgfqpoint{4.435325in}{2.243786in}}%
\pgfpathlineto{\pgfqpoint{4.435325in}{2.258785in}}%
\pgfpathlineto{\pgfqpoint{4.435325in}{2.273784in}}%
\pgfpathlineto{\pgfqpoint{4.435325in}{2.288783in}}%
\pgfpathlineto{\pgfqpoint{4.435325in}{2.303782in}}%
\pgfpathlineto{\pgfqpoint{4.435325in}{2.318781in}}%
\pgfpathlineto{\pgfqpoint{4.435325in}{2.333780in}}%
\pgfpathlineto{\pgfqpoint{4.435325in}{2.348779in}}%
\pgfpathlineto{\pgfqpoint{4.435325in}{2.363778in}}%
\pgfpathlineto{\pgfqpoint{4.435325in}{2.378777in}}%
\pgfpathlineto{\pgfqpoint{4.435325in}{2.393776in}}%
\pgfpathlineto{\pgfqpoint{4.435325in}{2.408775in}}%
\pgfpathlineto{\pgfqpoint{4.435325in}{2.423774in}}%
\pgfpathlineto{\pgfqpoint{4.435325in}{2.438773in}}%
\pgfpathlineto{\pgfqpoint{4.435325in}{2.453772in}}%
\pgfpathlineto{\pgfqpoint{4.435325in}{2.468771in}}%
\pgfpathlineto{\pgfqpoint{4.435325in}{2.483770in}}%
\pgfpathlineto{\pgfqpoint{4.435325in}{2.498769in}}%
\pgfpathlineto{\pgfqpoint{4.435325in}{2.513768in}}%
\pgfpathlineto{\pgfqpoint{4.435325in}{2.528767in}}%
\pgfpathlineto{\pgfqpoint{4.435325in}{2.543766in}}%
\pgfpathlineto{\pgfqpoint{4.435325in}{2.558765in}}%
\pgfpathlineto{\pgfqpoint{4.435325in}{2.573764in}}%
\pgfpathlineto{\pgfqpoint{4.435325in}{2.588763in}}%
\pgfpathlineto{\pgfqpoint{4.435325in}{2.603762in}}%
\pgfpathlineto{\pgfqpoint{4.435325in}{2.618761in}}%
\pgfpathlineto{\pgfqpoint{4.435325in}{2.633760in}}%
\pgfpathlineto{\pgfqpoint{4.435325in}{2.648759in}}%
\pgfpathlineto{\pgfqpoint{4.435325in}{2.663758in}}%
\pgfpathlineto{\pgfqpoint{4.435325in}{2.678757in}}%
\pgfpathlineto{\pgfqpoint{4.435325in}{2.693756in}}%
\pgfpathlineto{\pgfqpoint{4.435325in}{2.708755in}}%
\pgfpathlineto{\pgfqpoint{4.435325in}{2.723754in}}%
\pgfpathlineto{\pgfqpoint{4.435325in}{2.738753in}}%
\pgfpathlineto{\pgfqpoint{4.435325in}{2.753752in}}%
\pgfpathlineto{\pgfqpoint{4.435325in}{2.768751in}}%
\pgfpathlineto{\pgfqpoint{4.435325in}{2.783750in}}%
\pgfpathlineto{\pgfqpoint{4.435325in}{2.798749in}}%
\pgfpathlineto{\pgfqpoint{4.435325in}{2.813748in}}%
\pgfpathlineto{\pgfqpoint{4.435325in}{2.828747in}}%
\pgfpathlineto{\pgfqpoint{4.435325in}{2.843746in}}%
\pgfpathlineto{\pgfqpoint{4.435325in}{2.858745in}}%
\pgfpathlineto{\pgfqpoint{4.435325in}{2.873744in}}%
\pgfpathlineto{\pgfqpoint{4.435325in}{2.888743in}}%
\pgfpathlineto{\pgfqpoint{4.435325in}{2.903742in}}%
\pgfpathlineto{\pgfqpoint{4.435325in}{2.918741in}}%
\pgfpathlineto{\pgfqpoint{4.435325in}{2.933740in}}%
\pgfpathlineto{\pgfqpoint{4.435325in}{2.948739in}}%
\pgfpathlineto{\pgfqpoint{4.435325in}{2.963738in}}%
\pgfpathlineto{\pgfqpoint{4.435325in}{2.978737in}}%
\pgfpathlineto{\pgfqpoint{4.435325in}{2.993736in}}%
\pgfpathlineto{\pgfqpoint{4.435325in}{3.008735in}}%
\pgfpathlineto{\pgfqpoint{4.435325in}{3.023734in}}%
\pgfpathlineto{\pgfqpoint{4.435325in}{3.038733in}}%
\pgfpathlineto{\pgfqpoint{4.435325in}{3.053732in}}%
\pgfpathlineto{\pgfqpoint{4.435325in}{3.068731in}}%
\pgfpathlineto{\pgfqpoint{4.435325in}{3.083730in}}%
\pgfpathlineto{\pgfqpoint{4.435325in}{3.098729in}}%
\pgfpathlineto{\pgfqpoint{4.435325in}{3.113728in}}%
\pgfpathlineto{\pgfqpoint{4.435325in}{3.128727in}}%
\pgfpathlineto{\pgfqpoint{4.435325in}{3.143726in}}%
\pgfpathlineto{\pgfqpoint{4.435325in}{3.158725in}}%
\pgfpathlineto{\pgfqpoint{4.435325in}{3.173724in}}%
\pgfpathlineto{\pgfqpoint{4.435325in}{3.188723in}}%
\pgfpathlineto{\pgfqpoint{4.435325in}{3.203722in}}%
\pgfpathlineto{\pgfqpoint{4.435325in}{3.218721in}}%
\pgfpathlineto{\pgfqpoint{4.435325in}{3.233720in}}%
\pgfpathlineto{\pgfqpoint{4.435325in}{3.248719in}}%
\pgfpathlineto{\pgfqpoint{4.435325in}{3.263718in}}%
\pgfpathlineto{\pgfqpoint{4.435325in}{3.278717in}}%
\pgfpathlineto{\pgfqpoint{4.435325in}{3.293716in}}%
\pgfpathlineto{\pgfqpoint{4.435325in}{3.308715in}}%
\pgfpathlineto{\pgfqpoint{4.435325in}{3.323714in}}%
\pgfpathlineto{\pgfqpoint{4.435325in}{3.338713in}}%
\pgfpathlineto{\pgfqpoint{4.435325in}{3.353712in}}%
\pgfpathlineto{\pgfqpoint{4.435325in}{3.368711in}}%
\pgfpathlineto{\pgfqpoint{4.435325in}{3.383710in}}%
\pgfpathlineto{\pgfqpoint{4.435325in}{3.398709in}}%
\pgfpathlineto{\pgfqpoint{4.435325in}{3.413708in}}%
\pgfpathlineto{\pgfqpoint{4.435325in}{3.428707in}}%
\pgfpathlineto{\pgfqpoint{4.435325in}{3.443706in}}%
\pgfpathlineto{\pgfqpoint{4.435325in}{3.458705in}}%
\pgfpathlineto{\pgfqpoint{4.435325in}{3.473704in}}%
\pgfpathlineto{\pgfqpoint{4.435325in}{3.488703in}}%
\pgfpathlineto{\pgfqpoint{4.435325in}{3.503702in}}%
\pgfpathlineto{\pgfqpoint{4.435325in}{3.518701in}}%
\pgfpathlineto{\pgfqpoint{4.435325in}{3.533700in}}%
\pgfpathlineto{\pgfqpoint{4.435325in}{3.548699in}}%
\pgfpathlineto{\pgfqpoint{4.435325in}{3.563698in}}%
\pgfpathlineto{\pgfqpoint{4.435325in}{3.578697in}}%
\pgfpathlineto{\pgfqpoint{4.435325in}{3.593696in}}%
\pgfpathlineto{\pgfqpoint{4.435325in}{3.608695in}}%
\pgfpathlineto{\pgfqpoint{4.435325in}{3.623694in}}%
\pgfpathlineto{\pgfqpoint{4.435325in}{3.638693in}}%
\pgfpathlineto{\pgfqpoint{4.435325in}{3.653692in}}%
\pgfpathlineto{\pgfqpoint{4.435325in}{3.668691in}}%
\pgfpathlineto{\pgfqpoint{4.435325in}{3.683690in}}%
\pgfpathlineto{\pgfqpoint{4.435325in}{3.698689in}}%
\pgfpathlineto{\pgfqpoint{4.435325in}{3.713688in}}%
\pgfpathlineto{\pgfqpoint{4.435325in}{3.728687in}}%
\pgfpathlineto{\pgfqpoint{4.435325in}{3.743686in}}%
\pgfpathlineto{\pgfqpoint{4.435325in}{3.758685in}}%
\pgfpathlineto{\pgfqpoint{4.435325in}{3.773684in}}%
\pgfpathlineto{\pgfqpoint{4.435325in}{3.788683in}}%
\pgfpathlineto{\pgfqpoint{4.435325in}{3.803682in}}%
\pgfpathlineto{\pgfqpoint{4.435325in}{3.818681in}}%
\pgfpathlineto{\pgfqpoint{4.435325in}{3.833680in}}%
\pgfpathlineto{\pgfqpoint{4.435325in}{3.848679in}}%
\pgfpathlineto{\pgfqpoint{4.435325in}{3.863678in}}%
\pgfpathlineto{\pgfqpoint{4.435325in}{3.878677in}}%
\pgfpathlineto{\pgfqpoint{4.435325in}{3.893676in}}%
\pgfpathlineto{\pgfqpoint{4.435325in}{3.908675in}}%
\pgfpathlineto{\pgfqpoint{4.435325in}{3.923674in}}%
\pgfpathlineto{\pgfqpoint{4.435325in}{3.938673in}}%
\pgfpathlineto{\pgfqpoint{4.435325in}{3.953672in}}%
\pgfpathlineto{\pgfqpoint{4.435325in}{3.968671in}}%
\pgfpathlineto{\pgfqpoint{4.435325in}{3.983670in}}%
\pgfpathlineto{\pgfqpoint{4.435325in}{3.998669in}}%
\pgfpathlineto{\pgfqpoint{4.435325in}{4.013668in}}%
\pgfpathlineto{\pgfqpoint{4.435325in}{4.028667in}}%
\pgfpathlineto{\pgfqpoint{4.435325in}{4.043666in}}%
\pgfpathlineto{\pgfqpoint{4.435325in}{4.058665in}}%
\pgfpathlineto{\pgfqpoint{4.435325in}{4.073664in}}%
\pgfpathlineto{\pgfqpoint{4.435325in}{4.088663in}}%
\pgfpathlineto{\pgfqpoint{4.435325in}{4.103662in}}%
\pgfpathlineto{\pgfqpoint{4.435325in}{4.118661in}}%
\pgfpathlineto{\pgfqpoint{4.435325in}{4.133660in}}%
\pgfpathlineto{\pgfqpoint{4.435325in}{4.148659in}}%
\pgfpathlineto{\pgfqpoint{4.435325in}{4.163658in}}%
\pgfpathlineto{\pgfqpoint{4.435325in}{4.178657in}}%
\pgfpathlineto{\pgfqpoint{4.435325in}{4.193656in}}%
\pgfpathlineto{\pgfqpoint{4.435325in}{4.208655in}}%
\pgfpathlineto{\pgfqpoint{4.435325in}{4.223654in}}%
\pgfpathlineto{\pgfqpoint{4.435325in}{4.238653in}}%
\pgfpathlineto{\pgfqpoint{4.435325in}{4.253652in}}%
\pgfpathlineto{\pgfqpoint{4.435325in}{4.268651in}}%
\pgfpathlineto{\pgfqpoint{4.435325in}{4.283650in}}%
\pgfpathlineto{\pgfqpoint{4.435325in}{4.298649in}}%
\pgfpathlineto{\pgfqpoint{4.435325in}{4.313648in}}%
\pgfpathlineto{\pgfqpoint{4.435325in}{4.328647in}}%
\pgfpathlineto{\pgfqpoint{4.435325in}{4.343646in}}%
\pgfpathlineto{\pgfqpoint{4.435325in}{4.358645in}}%
\pgfpathlineto{\pgfqpoint{4.435325in}{4.373644in}}%
\pgfpathlineto{\pgfqpoint{4.435325in}{4.388643in}}%
\pgfpathlineto{\pgfqpoint{4.435325in}{4.403642in}}%
\pgfpathlineto{\pgfqpoint{4.435325in}{4.418641in}}%
\pgfpathlineto{\pgfqpoint{4.435325in}{4.433640in}}%
\pgfpathlineto{\pgfqpoint{4.435325in}{4.448639in}}%
\pgfpathlineto{\pgfqpoint{4.435325in}{4.463638in}}%
\pgfpathlineto{\pgfqpoint{4.435325in}{4.478637in}}%
\pgfpathlineto{\pgfqpoint{4.435325in}{4.493636in}}%
\pgfpathlineto{\pgfqpoint{4.435325in}{4.508635in}}%
\pgfpathlineto{\pgfqpoint{4.435325in}{4.523634in}}%
\pgfpathlineto{\pgfqpoint{4.435325in}{4.538633in}}%
\pgfpathlineto{\pgfqpoint{4.435325in}{4.553632in}}%
\pgfpathlineto{\pgfqpoint{4.435325in}{4.568631in}}%
\pgfpathlineto{\pgfqpoint{4.435325in}{4.583630in}}%
\pgfpathlineto{\pgfqpoint{4.435325in}{4.598629in}}%
\pgfpathlineto{\pgfqpoint{4.435325in}{4.613628in}}%
\pgfpathlineto{\pgfqpoint{4.435325in}{4.628627in}}%
\pgfpathlineto{\pgfqpoint{4.435325in}{4.643626in}}%
\pgfpathlineto{\pgfqpoint{4.435325in}{4.658625in}}%
\pgfpathlineto{\pgfqpoint{4.435325in}{4.673624in}}%
\pgfpathlineto{\pgfqpoint{4.435325in}{4.688623in}}%
\pgfpathlineto{\pgfqpoint{4.435325in}{4.703622in}}%
\pgfpathlineto{\pgfqpoint{4.435325in}{4.718621in}}%
\pgfpathlineto{\pgfqpoint{4.435325in}{4.733620in}}%
\pgfpathlineto{\pgfqpoint{4.435325in}{4.748619in}}%
\pgfpathlineto{\pgfqpoint{4.435325in}{4.763618in}}%
\pgfpathlineto{\pgfqpoint{4.435325in}{4.778617in}}%
\pgfpathlineto{\pgfqpoint{4.435325in}{4.793616in}}%
\pgfpathlineto{\pgfqpoint{4.435325in}{4.808615in}}%
\pgfpathlineto{\pgfqpoint{4.435325in}{4.823614in}}%
\pgfpathlineto{\pgfqpoint{4.435325in}{4.838613in}}%
\pgfpathlineto{\pgfqpoint{4.435325in}{4.853612in}}%
\pgfpathlineto{\pgfqpoint{4.435325in}{4.868611in}}%
\pgfpathlineto{\pgfqpoint{4.435325in}{4.883610in}}%
\pgfpathlineto{\pgfqpoint{4.435325in}{4.898609in}}%
\pgfpathlineto{\pgfqpoint{4.435325in}{4.913608in}}%
\pgfpathlineto{\pgfqpoint{4.435325in}{4.928607in}}%
\pgfpathlineto{\pgfqpoint{4.435325in}{4.943606in}}%
\pgfpathlineto{\pgfqpoint{4.435325in}{4.958605in}}%
\pgfpathlineto{\pgfqpoint{4.435325in}{4.973604in}}%
\pgfpathlineto{\pgfqpoint{4.435329in}{4.973604in}}%
\pgfpathlineto{\pgfqpoint{4.435329in}{4.973604in}}%
\pgfpathlineto{\pgfqpoint{4.435369in}{4.958605in}}%
\pgfpathlineto{\pgfqpoint{4.435521in}{4.943606in}}%
\pgfpathlineto{\pgfqpoint{4.435703in}{4.928607in}}%
\pgfpathlineto{\pgfqpoint{4.435643in}{4.913608in}}%
\pgfpathlineto{\pgfqpoint{4.435442in}{4.898609in}}%
\pgfpathlineto{\pgfqpoint{4.435344in}{4.883610in}}%
\pgfpathlineto{\pgfqpoint{4.435326in}{4.868611in}}%
\pgfpathlineto{\pgfqpoint{4.435325in}{4.853612in}}%
\pgfpathlineto{\pgfqpoint{4.435325in}{4.838613in}}%
\pgfpathlineto{\pgfqpoint{4.435325in}{4.823614in}}%
\pgfpathlineto{\pgfqpoint{4.435325in}{4.808615in}}%
\pgfpathlineto{\pgfqpoint{4.435325in}{4.793616in}}%
\pgfpathlineto{\pgfqpoint{4.435325in}{4.778617in}}%
\pgfpathlineto{\pgfqpoint{4.435325in}{4.763618in}}%
\pgfpathlineto{\pgfqpoint{4.435325in}{4.748619in}}%
\pgfpathlineto{\pgfqpoint{4.435325in}{4.733620in}}%
\pgfpathlineto{\pgfqpoint{4.435325in}{4.718621in}}%
\pgfpathlineto{\pgfqpoint{4.435325in}{4.703622in}}%
\pgfpathlineto{\pgfqpoint{4.435325in}{4.688623in}}%
\pgfpathlineto{\pgfqpoint{4.435325in}{4.673624in}}%
\pgfpathlineto{\pgfqpoint{4.435325in}{4.658625in}}%
\pgfpathlineto{\pgfqpoint{4.435325in}{4.643626in}}%
\pgfpathlineto{\pgfqpoint{4.435325in}{4.628627in}}%
\pgfpathlineto{\pgfqpoint{4.435325in}{4.613628in}}%
\pgfpathlineto{\pgfqpoint{4.435325in}{4.598629in}}%
\pgfpathlineto{\pgfqpoint{4.435325in}{4.583630in}}%
\pgfpathlineto{\pgfqpoint{4.435325in}{4.568631in}}%
\pgfpathlineto{\pgfqpoint{4.435325in}{4.553632in}}%
\pgfpathlineto{\pgfqpoint{4.435325in}{4.538633in}}%
\pgfpathlineto{\pgfqpoint{4.435325in}{4.523634in}}%
\pgfpathlineto{\pgfqpoint{4.435325in}{4.508635in}}%
\pgfpathlineto{\pgfqpoint{4.435325in}{4.493636in}}%
\pgfpathlineto{\pgfqpoint{4.435325in}{4.478637in}}%
\pgfpathlineto{\pgfqpoint{4.435327in}{4.463638in}}%
\pgfpathlineto{\pgfqpoint{4.435333in}{4.448639in}}%
\pgfpathlineto{\pgfqpoint{4.435342in}{4.433640in}}%
\pgfpathlineto{\pgfqpoint{4.435341in}{4.418641in}}%
\pgfpathlineto{\pgfqpoint{4.435331in}{4.403642in}}%
\pgfpathlineto{\pgfqpoint{4.435327in}{4.388643in}}%
\pgfpathlineto{\pgfqpoint{4.435329in}{4.373644in}}%
\pgfpathlineto{\pgfqpoint{4.435340in}{4.358645in}}%
\pgfpathlineto{\pgfqpoint{4.435359in}{4.343646in}}%
\pgfpathlineto{\pgfqpoint{4.435367in}{4.328647in}}%
\pgfpathlineto{\pgfqpoint{4.435356in}{4.313648in}}%
\pgfpathlineto{\pgfqpoint{4.435344in}{4.298649in}}%
\pgfpathlineto{\pgfqpoint{4.435337in}{4.283650in}}%
\pgfpathlineto{\pgfqpoint{4.435334in}{4.268651in}}%
\pgfpathlineto{\pgfqpoint{4.435341in}{4.253652in}}%
\pgfpathlineto{\pgfqpoint{4.435356in}{4.238653in}}%
\pgfpathlineto{\pgfqpoint{4.435363in}{4.223654in}}%
\pgfpathlineto{\pgfqpoint{4.435362in}{4.208655in}}%
\pgfpathlineto{\pgfqpoint{4.435373in}{4.193656in}}%
\pgfpathlineto{\pgfqpoint{4.435398in}{4.178657in}}%
\pgfpathlineto{\pgfqpoint{4.435411in}{4.163658in}}%
\pgfpathlineto{\pgfqpoint{4.435392in}{4.148659in}}%
\pgfpathlineto{\pgfqpoint{4.435359in}{4.133660in}}%
\pgfpathlineto{\pgfqpoint{4.435339in}{4.118661in}}%
\pgfpathlineto{\pgfqpoint{4.435340in}{4.103662in}}%
\pgfpathlineto{\pgfqpoint{4.435346in}{4.088663in}}%
\pgfpathlineto{\pgfqpoint{4.435353in}{4.073664in}}%
\pgfpathlineto{\pgfqpoint{4.435400in}{4.058665in}}%
\pgfpathlineto{\pgfqpoint{4.435499in}{4.043666in}}%
\pgfpathlineto{\pgfqpoint{4.435516in}{4.028667in}}%
\pgfpathlineto{\pgfqpoint{4.435424in}{4.013668in}}%
\pgfpathlineto{\pgfqpoint{4.435370in}{3.998669in}}%
\pgfpathlineto{\pgfqpoint{4.435395in}{3.983670in}}%
\pgfpathlineto{\pgfqpoint{4.435467in}{3.968671in}}%
\pgfpathlineto{\pgfqpoint{4.435549in}{3.953672in}}%
\pgfpathlineto{\pgfqpoint{4.435809in}{3.938673in}}%
\pgfpathlineto{\pgfqpoint{4.436368in}{3.923674in}}%
\pgfpathlineto{\pgfqpoint{4.436569in}{3.908675in}}%
\pgfpathlineto{\pgfqpoint{4.436090in}{3.893676in}}%
\pgfpathlineto{\pgfqpoint{4.435605in}{3.878677in}}%
\pgfpathlineto{\pgfqpoint{4.435438in}{3.863678in}}%
\pgfpathlineto{\pgfqpoint{4.435425in}{3.848679in}}%
\pgfpathlineto{\pgfqpoint{4.435466in}{3.833680in}}%
\pgfpathlineto{\pgfqpoint{4.435527in}{3.818681in}}%
\pgfpathlineto{\pgfqpoint{4.435556in}{3.803682in}}%
\pgfpathlineto{\pgfqpoint{4.435599in}{3.788683in}}%
\pgfpathlineto{\pgfqpoint{4.435834in}{3.773684in}}%
\pgfpathlineto{\pgfqpoint{4.436600in}{3.758685in}}%
\pgfpathlineto{\pgfqpoint{4.437697in}{3.743686in}}%
\pgfpathlineto{\pgfqpoint{4.437902in}{3.728687in}}%
\pgfpathlineto{\pgfqpoint{4.437018in}{3.713688in}}%
\pgfpathlineto{\pgfqpoint{4.436304in}{3.698689in}}%
\pgfpathlineto{\pgfqpoint{4.436064in}{3.683690in}}%
\pgfpathlineto{\pgfqpoint{4.435922in}{3.668691in}}%
\pgfpathlineto{\pgfqpoint{4.435869in}{3.653692in}}%
\pgfpathlineto{\pgfqpoint{4.435911in}{3.638693in}}%
\pgfpathlineto{\pgfqpoint{4.435945in}{3.623694in}}%
\pgfpathlineto{\pgfqpoint{4.436055in}{3.608695in}}%
\pgfpathlineto{\pgfqpoint{4.436729in}{3.593696in}}%
\pgfpathlineto{\pgfqpoint{4.438410in}{3.578697in}}%
\pgfpathlineto{\pgfqpoint{4.440404in}{3.563698in}}%
\pgfpathlineto{\pgfqpoint{4.440844in}{3.548699in}}%
\pgfpathlineto{\pgfqpoint{4.439781in}{3.533700in}}%
\pgfpathlineto{\pgfqpoint{4.439050in}{3.518701in}}%
\pgfpathlineto{\pgfqpoint{4.438583in}{3.503702in}}%
\pgfpathlineto{\pgfqpoint{4.437822in}{3.488703in}}%
\pgfpathlineto{\pgfqpoint{4.437564in}{3.473704in}}%
\pgfpathlineto{\pgfqpoint{4.438003in}{3.458705in}}%
\pgfpathlineto{\pgfqpoint{4.438669in}{3.443706in}}%
\pgfpathlineto{\pgfqpoint{4.439432in}{3.428707in}}%
\pgfpathlineto{\pgfqpoint{4.440379in}{3.413708in}}%
\pgfpathlineto{\pgfqpoint{4.442634in}{3.398709in}}%
\pgfpathlineto{\pgfqpoint{4.449643in}{3.383710in}}%
\pgfpathlineto{\pgfqpoint{4.459334in}{3.368711in}}%
\pgfpathlineto{\pgfqpoint{4.459448in}{3.353712in}}%
\pgfpathlineto{\pgfqpoint{4.452135in}{3.338713in}}%
\pgfpathlineto{\pgfqpoint{4.449312in}{3.323714in}}%
\pgfpathlineto{\pgfqpoint{4.450642in}{3.308715in}}%
\pgfpathlineto{\pgfqpoint{4.452966in}{3.293716in}}%
\pgfpathlineto{\pgfqpoint{4.457621in}{3.278717in}}%
\pgfpathlineto{\pgfqpoint{4.466107in}{3.263718in}}%
\pgfpathlineto{\pgfqpoint{4.472335in}{3.248719in}}%
\pgfpathlineto{\pgfqpoint{4.474953in}{3.233720in}}%
\pgfpathlineto{\pgfqpoint{4.484848in}{3.218721in}}%
\pgfpathlineto{\pgfqpoint{4.498185in}{3.203722in}}%
\pgfpathlineto{\pgfqpoint{4.497387in}{3.188723in}}%
\pgfpathlineto{\pgfqpoint{4.483977in}{3.173724in}}%
\pgfpathlineto{\pgfqpoint{4.474204in}{3.158725in}}%
\pgfpathlineto{\pgfqpoint{4.471255in}{3.143726in}}%
\pgfpathlineto{\pgfqpoint{4.473820in}{3.128727in}}%
\pgfpathlineto{\pgfqpoint{4.485322in}{3.113728in}}%
\pgfpathlineto{\pgfqpoint{4.506096in}{3.098729in}}%
\pgfpathlineto{\pgfqpoint{4.518784in}{3.083730in}}%
\pgfpathlineto{\pgfqpoint{4.508502in}{3.068731in}}%
\pgfpathlineto{\pgfqpoint{4.493529in}{3.053732in}}%
\pgfpathlineto{\pgfqpoint{4.486945in}{3.038733in}}%
\pgfpathlineto{\pgfqpoint{4.483384in}{3.023734in}}%
\pgfpathlineto{\pgfqpoint{4.480731in}{3.008735in}}%
\pgfpathlineto{\pgfqpoint{4.479356in}{2.993736in}}%
\pgfpathlineto{\pgfqpoint{4.477790in}{2.978737in}}%
\pgfpathlineto{\pgfqpoint{4.475691in}{2.963738in}}%
\pgfpathlineto{\pgfqpoint{4.474912in}{2.948739in}}%
\pgfpathlineto{\pgfqpoint{4.474951in}{2.933740in}}%
\pgfpathlineto{\pgfqpoint{4.473631in}{2.918741in}}%
\pgfpathlineto{\pgfqpoint{4.470875in}{2.903742in}}%
\pgfpathlineto{\pgfqpoint{4.468803in}{2.888743in}}%
\pgfpathlineto{\pgfqpoint{4.468107in}{2.873744in}}%
\pgfpathlineto{\pgfqpoint{4.469488in}{2.858745in}}%
\pgfpathlineto{\pgfqpoint{4.473624in}{2.843746in}}%
\pgfpathlineto{\pgfqpoint{4.478878in}{2.828747in}}%
\pgfpathlineto{\pgfqpoint{4.484797in}{2.813748in}}%
\pgfpathlineto{\pgfqpoint{4.494506in}{2.798749in}}%
\pgfpathlineto{\pgfqpoint{4.510524in}{2.783750in}}%
\pgfpathlineto{\pgfqpoint{4.530697in}{2.768751in}}%
\pgfpathlineto{\pgfqpoint{4.552171in}{2.753752in}}%
\pgfpathlineto{\pgfqpoint{4.573723in}{2.738753in}}%
\pgfpathlineto{\pgfqpoint{4.597154in}{2.723754in}}%
\pgfpathlineto{\pgfqpoint{4.629460in}{2.708755in}}%
\pgfpathlineto{\pgfqpoint{4.653468in}{2.693756in}}%
\pgfpathlineto{\pgfqpoint{4.651543in}{2.678757in}}%
\pgfpathlineto{\pgfqpoint{4.654972in}{2.663758in}}%
\pgfpathlineto{\pgfqpoint{4.683192in}{2.648759in}}%
\pgfpathlineto{\pgfqpoint{4.728416in}{2.633760in}}%
\pgfpathlineto{\pgfqpoint{4.782625in}{2.618761in}}%
\pgfpathlineto{\pgfqpoint{4.846692in}{2.603762in}}%
\pgfpathlineto{\pgfqpoint{4.939329in}{2.588763in}}%
\pgfpathlineto{\pgfqpoint{5.029409in}{2.573764in}}%
\pgfpathlineto{\pgfqpoint{4.999376in}{2.558765in}}%
\pgfpathlineto{\pgfqpoint{4.860809in}{2.543766in}}%
\pgfpathlineto{\pgfqpoint{4.737089in}{2.528767in}}%
\pgfpathlineto{\pgfqpoint{4.666351in}{2.513768in}}%
\pgfpathlineto{\pgfqpoint{4.635452in}{2.498769in}}%
\pgfpathlineto{\pgfqpoint{4.642051in}{2.483770in}}%
\pgfpathlineto{\pgfqpoint{4.664114in}{2.468771in}}%
\pgfpathlineto{\pgfqpoint{4.653209in}{2.453772in}}%
\pgfpathlineto{\pgfqpoint{4.611542in}{2.438773in}}%
\pgfpathlineto{\pgfqpoint{4.592385in}{2.423774in}}%
\pgfpathlineto{\pgfqpoint{4.617541in}{2.408775in}}%
\pgfpathlineto{\pgfqpoint{4.647001in}{2.393776in}}%
\pgfpathlineto{\pgfqpoint{4.628476in}{2.378777in}}%
\pgfpathlineto{\pgfqpoint{4.578668in}{2.363778in}}%
\pgfpathlineto{\pgfqpoint{4.546995in}{2.348779in}}%
\pgfpathlineto{\pgfqpoint{4.547711in}{2.333780in}}%
\pgfpathlineto{\pgfqpoint{4.552949in}{2.318781in}}%
\pgfpathlineto{\pgfqpoint{4.543559in}{2.303782in}}%
\pgfpathlineto{\pgfqpoint{4.524157in}{2.288783in}}%
\pgfpathlineto{\pgfqpoint{4.491386in}{2.273784in}}%
\pgfpathlineto{\pgfqpoint{4.461319in}{2.258785in}}%
\pgfpathlineto{\pgfqpoint{4.446077in}{2.243786in}}%
\pgfpathlineto{\pgfqpoint{4.439991in}{2.228787in}}%
\pgfpathlineto{\pgfqpoint{4.438079in}{2.213788in}}%
\pgfpathlineto{\pgfqpoint{4.437540in}{2.198789in}}%
\pgfpathlineto{\pgfqpoint{4.436962in}{2.183790in}}%
\pgfpathlineto{\pgfqpoint{4.436240in}{2.168791in}}%
\pgfpathlineto{\pgfqpoint{4.435747in}{2.153792in}}%
\pgfpathlineto{\pgfqpoint{4.435647in}{2.138793in}}%
\pgfpathlineto{\pgfqpoint{4.435734in}{2.123794in}}%
\pgfpathlineto{\pgfqpoint{4.435736in}{2.108795in}}%
\pgfpathlineto{\pgfqpoint{4.435676in}{2.093796in}}%
\pgfpathlineto{\pgfqpoint{4.435577in}{2.078797in}}%
\pgfpathlineto{\pgfqpoint{4.435434in}{2.063798in}}%
\pgfpathlineto{\pgfqpoint{4.435356in}{2.048799in}}%
\pgfpathlineto{\pgfqpoint{4.435334in}{2.033800in}}%
\pgfpathlineto{\pgfqpoint{4.435328in}{2.018801in}}%
\pgfpathlineto{\pgfqpoint{4.435326in}{2.003802in}}%
\pgfpathlineto{\pgfqpoint{4.435325in}{1.988803in}}%
\pgfpathclose%
\pgfusepath{stroke,fill}%
}%
\begin{pgfscope}%
\pgfsys@transformshift{0.000000in}{0.000000in}%
\pgfsys@useobject{currentmarker}{}%
\end{pgfscope}%
\end{pgfscope}%
\begin{pgfscope}%
\pgfsetbuttcap%
\pgfsetroundjoin%
\definecolor{currentfill}{rgb}{0.000000,0.000000,0.000000}%
\pgfsetfillcolor{currentfill}%
\pgfsetlinewidth{0.803000pt}%
\definecolor{currentstroke}{rgb}{0.000000,0.000000,0.000000}%
\pgfsetstrokecolor{currentstroke}%
\pgfsetdash{}{0pt}%
\pgfsys@defobject{currentmarker}{\pgfqpoint{-0.048611in}{0.000000in}}{\pgfqpoint{0.000000in}{0.000000in}}{%
\pgfpathmoveto{\pgfqpoint{0.000000in}{0.000000in}}%
\pgfpathlineto{\pgfqpoint{-0.048611in}{0.000000in}}%
\pgfusepath{stroke,fill}%
}%
\begin{pgfscope}%
\pgfsys@transformshift{4.435325in}{0.499691in}%
\pgfsys@useobject{currentmarker}{}%
\end{pgfscope}%
\end{pgfscope}%
\begin{pgfscope}%
\pgfsetbuttcap%
\pgfsetroundjoin%
\definecolor{currentfill}{rgb}{0.000000,0.000000,0.000000}%
\pgfsetfillcolor{currentfill}%
\pgfsetlinewidth{0.803000pt}%
\definecolor{currentstroke}{rgb}{0.000000,0.000000,0.000000}%
\pgfsetstrokecolor{currentstroke}%
\pgfsetdash{}{0pt}%
\pgfsys@defobject{currentmarker}{\pgfqpoint{-0.048611in}{0.000000in}}{\pgfqpoint{0.000000in}{0.000000in}}{%
\pgfpathmoveto{\pgfqpoint{0.000000in}{0.000000in}}%
\pgfpathlineto{\pgfqpoint{-0.048611in}{0.000000in}}%
\pgfusepath{stroke,fill}%
}%
\begin{pgfscope}%
\pgfsys@transformshift{4.435325in}{0.957511in}%
\pgfsys@useobject{currentmarker}{}%
\end{pgfscope}%
\end{pgfscope}%
\begin{pgfscope}%
\pgfsetbuttcap%
\pgfsetroundjoin%
\definecolor{currentfill}{rgb}{0.000000,0.000000,0.000000}%
\pgfsetfillcolor{currentfill}%
\pgfsetlinewidth{0.803000pt}%
\definecolor{currentstroke}{rgb}{0.000000,0.000000,0.000000}%
\pgfsetstrokecolor{currentstroke}%
\pgfsetdash{}{0pt}%
\pgfsys@defobject{currentmarker}{\pgfqpoint{-0.048611in}{0.000000in}}{\pgfqpoint{0.000000in}{0.000000in}}{%
\pgfpathmoveto{\pgfqpoint{0.000000in}{0.000000in}}%
\pgfpathlineto{\pgfqpoint{-0.048611in}{0.000000in}}%
\pgfusepath{stroke,fill}%
}%
\begin{pgfscope}%
\pgfsys@transformshift{4.435325in}{1.415331in}%
\pgfsys@useobject{currentmarker}{}%
\end{pgfscope}%
\end{pgfscope}%
\begin{pgfscope}%
\pgfsetbuttcap%
\pgfsetroundjoin%
\definecolor{currentfill}{rgb}{0.000000,0.000000,0.000000}%
\pgfsetfillcolor{currentfill}%
\pgfsetlinewidth{0.803000pt}%
\definecolor{currentstroke}{rgb}{0.000000,0.000000,0.000000}%
\pgfsetstrokecolor{currentstroke}%
\pgfsetdash{}{0pt}%
\pgfsys@defobject{currentmarker}{\pgfqpoint{-0.048611in}{0.000000in}}{\pgfqpoint{0.000000in}{0.000000in}}{%
\pgfpathmoveto{\pgfqpoint{0.000000in}{0.000000in}}%
\pgfpathlineto{\pgfqpoint{-0.048611in}{0.000000in}}%
\pgfusepath{stroke,fill}%
}%
\begin{pgfscope}%
\pgfsys@transformshift{4.435325in}{1.873151in}%
\pgfsys@useobject{currentmarker}{}%
\end{pgfscope}%
\end{pgfscope}%
\begin{pgfscope}%
\pgfsetbuttcap%
\pgfsetroundjoin%
\definecolor{currentfill}{rgb}{0.000000,0.000000,0.000000}%
\pgfsetfillcolor{currentfill}%
\pgfsetlinewidth{0.803000pt}%
\definecolor{currentstroke}{rgb}{0.000000,0.000000,0.000000}%
\pgfsetstrokecolor{currentstroke}%
\pgfsetdash{}{0pt}%
\pgfsys@defobject{currentmarker}{\pgfqpoint{-0.048611in}{0.000000in}}{\pgfqpoint{0.000000in}{0.000000in}}{%
\pgfpathmoveto{\pgfqpoint{0.000000in}{0.000000in}}%
\pgfpathlineto{\pgfqpoint{-0.048611in}{0.000000in}}%
\pgfusepath{stroke,fill}%
}%
\begin{pgfscope}%
\pgfsys@transformshift{4.435325in}{2.330971in}%
\pgfsys@useobject{currentmarker}{}%
\end{pgfscope}%
\end{pgfscope}%
\begin{pgfscope}%
\pgfsetbuttcap%
\pgfsetroundjoin%
\definecolor{currentfill}{rgb}{0.000000,0.000000,0.000000}%
\pgfsetfillcolor{currentfill}%
\pgfsetlinewidth{0.803000pt}%
\definecolor{currentstroke}{rgb}{0.000000,0.000000,0.000000}%
\pgfsetstrokecolor{currentstroke}%
\pgfsetdash{}{0pt}%
\pgfsys@defobject{currentmarker}{\pgfqpoint{-0.048611in}{0.000000in}}{\pgfqpoint{0.000000in}{0.000000in}}{%
\pgfpathmoveto{\pgfqpoint{0.000000in}{0.000000in}}%
\pgfpathlineto{\pgfqpoint{-0.048611in}{0.000000in}}%
\pgfusepath{stroke,fill}%
}%
\begin{pgfscope}%
\pgfsys@transformshift{4.435325in}{2.788791in}%
\pgfsys@useobject{currentmarker}{}%
\end{pgfscope}%
\end{pgfscope}%
\begin{pgfscope}%
\pgfsetbuttcap%
\pgfsetroundjoin%
\definecolor{currentfill}{rgb}{0.000000,0.000000,0.000000}%
\pgfsetfillcolor{currentfill}%
\pgfsetlinewidth{0.803000pt}%
\definecolor{currentstroke}{rgb}{0.000000,0.000000,0.000000}%
\pgfsetstrokecolor{currentstroke}%
\pgfsetdash{}{0pt}%
\pgfsys@defobject{currentmarker}{\pgfqpoint{-0.048611in}{0.000000in}}{\pgfqpoint{0.000000in}{0.000000in}}{%
\pgfpathmoveto{\pgfqpoint{0.000000in}{0.000000in}}%
\pgfpathlineto{\pgfqpoint{-0.048611in}{0.000000in}}%
\pgfusepath{stroke,fill}%
}%
\begin{pgfscope}%
\pgfsys@transformshift{4.435325in}{3.246611in}%
\pgfsys@useobject{currentmarker}{}%
\end{pgfscope}%
\end{pgfscope}%
\begin{pgfscope}%
\pgfsetbuttcap%
\pgfsetroundjoin%
\definecolor{currentfill}{rgb}{0.000000,0.000000,0.000000}%
\pgfsetfillcolor{currentfill}%
\pgfsetlinewidth{0.803000pt}%
\definecolor{currentstroke}{rgb}{0.000000,0.000000,0.000000}%
\pgfsetstrokecolor{currentstroke}%
\pgfsetdash{}{0pt}%
\pgfsys@defobject{currentmarker}{\pgfqpoint{-0.048611in}{0.000000in}}{\pgfqpoint{0.000000in}{0.000000in}}{%
\pgfpathmoveto{\pgfqpoint{0.000000in}{0.000000in}}%
\pgfpathlineto{\pgfqpoint{-0.048611in}{0.000000in}}%
\pgfusepath{stroke,fill}%
}%
\begin{pgfscope}%
\pgfsys@transformshift{4.435325in}{3.704431in}%
\pgfsys@useobject{currentmarker}{}%
\end{pgfscope}%
\end{pgfscope}%
\begin{pgfscope}%
\pgfsetbuttcap%
\pgfsetroundjoin%
\definecolor{currentfill}{rgb}{0.000000,0.000000,0.000000}%
\pgfsetfillcolor{currentfill}%
\pgfsetlinewidth{0.803000pt}%
\definecolor{currentstroke}{rgb}{0.000000,0.000000,0.000000}%
\pgfsetstrokecolor{currentstroke}%
\pgfsetdash{}{0pt}%
\pgfsys@defobject{currentmarker}{\pgfqpoint{-0.048611in}{0.000000in}}{\pgfqpoint{0.000000in}{0.000000in}}{%
\pgfpathmoveto{\pgfqpoint{0.000000in}{0.000000in}}%
\pgfpathlineto{\pgfqpoint{-0.048611in}{0.000000in}}%
\pgfusepath{stroke,fill}%
}%
\begin{pgfscope}%
\pgfsys@transformshift{4.435325in}{4.162251in}%
\pgfsys@useobject{currentmarker}{}%
\end{pgfscope}%
\end{pgfscope}%
\begin{pgfscope}%
\pgfsetrectcap%
\pgfsetmiterjoin%
\pgfsetlinewidth{0.803000pt}%
\definecolor{currentstroke}{rgb}{0.000000,0.000000,0.000000}%
\pgfsetstrokecolor{currentstroke}%
\pgfsetdash{}{0pt}%
\pgfpathmoveto{\pgfqpoint{4.435325in}{0.499691in}}%
\pgfpathlineto{\pgfqpoint{4.435325in}{4.162251in}}%
\pgfusepath{stroke}%
\end{pgfscope}%
\end{pgfpicture}%
\makeatother%
\endgroup%

    \caption{Histogram \small x of verified \acrshortpl{sc} on Ethereum}
\end{figure}
