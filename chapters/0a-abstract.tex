\chapter*{Abstract}

Vulnerability detection and security of \acrlongpl{sc} are of paramount importance because of their immutable nature. A \acrlong{sc} is a program stored on a blockchain that runs when some predetermined conditions are met. While smart contracts have enabled a variety of applications on the blockchain, they may pose a significant security risk. Once a smart contract is deployed to a blockchain, it cannot be changed. It is therefore imperative that all bugs and errors are pruned out before deployment. With the increase in studies on \acrlong{sc} vulnerability detection tools and methods, it is important to systematically review the state-of-the-art tools and methods. This, to classify the existing solutions, as well as identify gaps and challenges for future research. In this \acrfull{slr}, a total of 125 papers on \acrlong{sc} vulnerability analysis and detection methods and tools were retrieved. These were then filtered based on predefined inclusion and exclusion criteria. Snowballing was then applied. A total of 40 relevant papers were selected and analyzed. The vulnerability detection tools and methods were classified into six categories: Symbolic execution, Syntax analysis, Abstract interpretation, Data flow analysis, Fuzzing test, and Machine learning. This \acrshort{slr} provides a broader scope than just Ethereum. Thus, the cross-chain applicability of the tools and methods were also evaluated. Cross-chain vulnerability detection is in this \acrshort{slr} defined as a method for detecting vulnerabilities in \acrlong{sc} code that can be applied for multiple blockchains. The results of this study show that there are many highly accurate tools and methods available for \acrfull{sc} vulnerability detection. Especially \acrlong{ml} has in recent years drawn much attention from the research community. However, little effort has been invested in \acrlong{sc} vulnerability detection on other chains.
